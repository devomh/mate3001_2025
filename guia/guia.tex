\documentclass{article}
\usepackage[utf8]{inputenc}
\usepackage[spanish]{babel}
\usepackage{longtable}
\usepackage{geometry}
\geometry{legalpaper, margin=1in}

\begin{document}

\begin{center}
    \textbf{Universidad de Puerto Rico en Humacao} \\
    \textbf{Departamento de Matemáticas} \\
    \textbf{Guía del estudiante MATE 3001} \\
    \textbf{Primer Semestre 2025-2026}
\end{center}

\vspace{0.25cm}

\noindent
\begin{tabular}{@{}ll@{}}
	\textbf{Profesor:} & Ollantay Medina \\
	\textbf{Correo electrónico:} & ollantay.medina@upr.edu \\
	\textbf{Oficina:} & CNO 176 \\
	\textbf{Horas de Oficina:} & Ma, Ju: 8:00-10:30am, Vi: 8:00-9:00am \\
\end{tabular}

\vspace{0.25cm}

\noindent \textbf{Texto:} Fundamentos y aplicaciones del álgebra; Rolando Castro; La Editorial, UPR

\vspace{0.25cm}

\begin{longtable}{|p{0.15\textwidth}|p{0.55\textwidth}|p{0.2\textwidth}|}
    \hline
    \textbf{Sección} & \textbf{Tema} & \textbf{Problemas Asignados (impares)} \\
    \hline
    \endfirsthead
    \hline
    \multicolumn{3}{|c|}{Continuación} \\
    \hline
    \textbf{Sección} & \textbf{Tema} & \textbf{Problemas Asignados (impares)} \\
    \hline
    \endhead
    \hline
    \endfoot
    \hline
    \endlastfoot

    1.1 & Definición. Operaciones entre conjuntos. Números reales. & Pág. 11-13 \\
    \hline
    1.2 & La recta numérica & Pág. 17 \\
    \hline
    1.3 & Leyes de los signos & Pág. 22-23 \\
    \hline
    1.4 & Propiedades de los números reales. & Pág. 27 \\
    \hline
    Ap. B & Operaciones entre racionales. DCM y MCM, suma y resta de fracciones (Apéndice B). & Pág. 470-471 \\
    \hline
    1.5 & Exponentes y orden de las operaciones & Pág. 32 \\
    \hline
    1.6 & Expresiones algebraicas & Pág. 36-37 \\
    \hline
    2.1 & Leyes de los exponentes & Pág. 59 \\
    \hline
    2.2 & Exponentes enteros no positivos & Pág. 64-65 \\
    \hline
    2.3 & Polinomios & Pág. 70-71 \\
    \hline
    2.4 & Multiplicación de polinomios & Pág. 75 \\
    \hline
    \multicolumn{3}{|c|}{\textbf{Examen Parcial I}} \\
    \multicolumn{3}{|c|}{\textbf{Fecha sugerida: semana del 2-5 de septiembre}} \\
    \hline
    2.6 & Factorización de polinomios & Pág. 89-90 \\
    \hline
    2.7 & Diferencia de cuadrados & Pág. 95 \\
    \hline
    2.8 & Suma y diferencia de cubos & Pág. 98 \\
    \hline
    2.9 & Trinomios cuadráticos & Pág. 102-103 \\
    \hline
    2.10 & Repaso General de factorización (opcional) & Pág. 140-141 \\
    \hline
    3.1 & Ecuaciones. Definición y términos importantes. & Pág. 123-124 \\
    \hline
    3.2 & Ecuaciones lineales en una variable, ecuaciones equivalentes, solución de ecuaciones. & Pág. 130 \\
    \hline
    3.3 & Ecuaciones que contienen fracciones & Pág. 136-137 \\
    \hline
    3.4 & Solución de ecuaciones por factorización & Pág. 140-141 \\
    \hline
    \multicolumn{3}{|c|}{\textbf{Examen Parcial II}} \\
    \multicolumn{3}{|c|}{\textbf{Fecha sugerida: semana del 29 de septiembre al 3 de octubre}} \\
    \hline
    3.5 & Fórmulas & Pág. 143 \\
    \hline
    3.8 & Ecuaciones cuadráticas & Pág. 168 \\
    \hline
    3.9 & La fórmula cuadrática & Pág. 171 \\
    \hline
    4.1 & Traducción de frases a expresiones algebraicas & Pág. 188-189 \\
    \hline
    4.2 & Problemas verbales & Pág. 194-195 \\
    \hline
    5.1 & Sistemas de coordenadas cartesianas & Pág. 251-252 \\
    \hline
    5.2 & Ecuaciones lineales con dos variables & Pág. 261-262 \\
    \hline
    6.1 & Sistemas de ecuaciones, solución de un sistema de ecuaciones & Pág. 310-311 \\
    \hline
    6.2 & Método de sustitución & Pág. 315-316 \\
    \hline
    6.3 & Método de la suma. Punto de equilibrio. & Pág. 323-324 \\
    \hline
    9.1 & Desigualdades lineales & Pág. 430-431 \\
    \hline
    \multicolumn{3}{|c|}{\textbf{Examen Parcial III}} \\
    \multicolumn{3}{|c|}{\textbf{Fecha sugerida: semana del 27 al 31 de octubre}} \\
    \hline
    7.1 & Simplificación de expresiones racionales & Pág. 354-355 \\
    \hline
    7.2 & Multiplicación y división de expresiones racionales & Pág. 360 \\
    \hline
    7.3 & Suma y resta de expresiones racionales & Pág. 368-369 \\
    \hline
    7.4 & Ecuaciones racionales & Pág. 374 \\
    \hline
    8.1 & Raíces y radicales & Pág. 392-393 \\
    \hline
        & Simplificación de expresiones con radicales & \\
    8.2 & Multiplicación y simplificación de radicales & Pág. 397-398 \\
    \hline
    8.3 & División y racionalización de radicales & Pág. 401-402 \\
    \hline
    8.4 & Suma y resta de radicales & Pág. 406 \\
    \hline
    8.5 & Radicales y notación exponencial & Pág. 413 \\
    \hline
     & Funciones. Definición y gráficas de funciones simples & \\
    \hline
    \multicolumn{3}{|c|}{\textbf{Examen Parcial IV (opcional)}} \\
    \multicolumn{3}{|c|}{\textbf{Fecha sugerida: semana del 24 al 26 de noviembre}} \\
    \hline
    \multicolumn{3}{|c|}{\textbf{Examen Final Comprensivo (Fecha anunciada por el Registrador)}} \\
\end{longtable}

\vspace{1cm}
\pagebreak

\section*{Evaluación}
\begin{tabular}{p{0.8\textwidth}r}
    3 exámenes parciales departamentales (100 pts. c/u) & 60\% \\
    1 Examen final comprensivo departamental (no se elimina) (100 pts.) & 20\% \\
    Otras evaluaciones: (100 pts.) & 20\% \\
\end{tabular}

\vspace{0.5cm}

Se utilizará la siguiente curva para adjudicar la nota final en el curso:
\begin{center}
    100 – 90 (A), 89 – 80 (B), 79 – 70 (C), 69 – 60 (D), 59 – 0 (F).
\end{center}

\section*{Normas generales}
\begin{enumerate}
    \item El curso de MATE 3001 está catalogado como uno presencial. La asistencia a todas las actividades planificadas sincrónicas es compulsoria.
    \item Se utilizará la plataforma de Moodle Institucional como recurso de enseñanza complementario y en caso de que surja la necesidad, la plataforma de ZOOM para ofrecer las conferencias.
    \item No se permite ingerir comidas o bebidas durante la hora de clase (a menos que el(la) estudiante tenga alguna condición médica y lo haya advertido previamente al profesor).
    \item Se permite el uso de calculadora científica; NO el teléfono celular. Durante evaluaciones, el profesor puede dar instrucciones particulares sobre el uso de calculadora.
    \item No se permite el uso de teléfonos celulares durante la clase (Cert. \# 1994-95-42 de la Junta Académica de la UPRH). Se permiten las siguientes excepciones:
    \subitem - Registrar asistencia.
    \subitem - Revisar documentos o recursos estrictamente referentes a la clase
    \item La falta de integridad académica por parte de un(a) estudiante conllevará sanciones disciplinarias.
    \item Los exámenes se ofrecerán de forma presencial en la medida que la propagación del COVID - 19 lo permita. A estos efectos se tomarán en consideración todas la medidas establecidas en las guías para evitar la propagación del virus.
    \item Las fechas de los exámenes y el contenido de los exámenes están sujetos a cambio(s).
    \item Asignaciones y exámenes presenciales NO se reponen. De surgir imprevistos justificados, cada situación individual se evalúa por separado en consulta directa con el profesor.
    \item El cuarto examen es OPCIONAL. Esta nota se puede utilizar para eliminar la peor de las cuatro notas. Si no lo toma, NO elimina ninguna de las notas parciales.
\end{enumerate}

\vspace{1cm}

\begin{center}
    Revisión de Contenido: Profa. B. Santiago-Figueroa, Coordinadora Mate 3001, 7 de agosto de 2024
\end{center}

\end{document}