\documentclass[aspectratio=169]{beamer}
\usetheme{Madrid}
\usecolortheme{seahorse}

\usepackage[spanish]{babel}
\usepackage[utf8]{inputenc}
\usepackage{amsmath,amssymb,amsthm}
\usepackage{tikz}
\usetikzlibrary{arrows.meta,positioning,shapes.geometric,decorations.pathreplacing}
\usepackage{pgfplots}
\pgfplotsset{compat=1.18}

% Custom colors
\definecolor{myblue}{RGB}{0,102,204}
\definecolor{myred}{RGB}{204,0,0}
\definecolor{mygreen}{RGB}{0,153,76}
\definecolor{mypurple}{RGB}{153,0,153}

\title[Ecuaciones Cuadráticas]{Sección 3.8: Ecuaciones Cuadráticas}
\subtitle{Métodos de Solución y Análisis}
\author{MATE 3001}
\date{}

\begin{document}

% Title slide
\begin{frame}
\titlepage
\end{frame}

% Table of contents
\begin{frame}{Contenido}
\tableofcontents
\end{frame}

\section{Definición y Ejemplos}

\begin{frame}{¿Qué es una Ecuación Cuadrática?}
\begin{block}{Definición}
Una \textbf{ecuación cuadrática} es una ecuación que puede escribirse como:
\[
\boxed{ax^2 + bx + c = 0, \quad a\neq 0,\ a,b,c\in\mathbb{R}}
\]
\end{block}

\pause
\vspace{0.5cm}
\begin{columns}[T]
\column{0.5\textwidth}
\textbf{Características clave:}
\begin{itemize}
\item Grado máximo: 2
\item Coeficiente $a \neq 0$
\item Forma estándar
\end{itemize}

\column{0.5\textwidth}
\begin{center}
\begin{tikzpicture}[scale=0.6]
\begin{axis}[
    axis lines=center,
    xlabel={$x$},
    ylabel={$y$},
    ymin=-1, ymax=5,
    xmin=-3, xmax=3,
    xtick={-2,-1,0,1,2},
    ytick={0,2,4},
    width=5cm,
    height=4cm
]
\addplot[domain=-2.5:2.5, samples=100, thick, myblue] {x^2};
\node[myblue] at (axis cs:1.5,3.5) {$y=x^2$};
\end{axis}
\end{tikzpicture}
\end{center}
\end{columns}
\end{frame}

\begin{frame}{Ejemplos de Ecuaciones Cuadráticas}
\begin{enumerate}
\item<1-> $x^2 + 6x = -9$ \quad {\color{mygreen}$\Rightarrow x^2 + 6x + 9 = 0$}
\item<2-> $x(x+5) = -4$ \quad {\color{mygreen}$\Rightarrow x^2 + 5x + 4 = 0$}
\item<3-> $x^2 - 3x = 0$ \quad {\color{mygreen}Ya en forma estándar}
\item<4-> $5x^2 = x^2$ \quad {\color{mygreen}$\Rightarrow 4x^2 = 0$}
\end{enumerate}

\onslide<5->{
\begin{alertblock}{Nota Importante}
Siempre escribir en \textbf{forma estándar} antes de resolver.
\end{alertblock}
}
\end{frame}

\begin{frame}{Soluciones de Ecuaciones Cuadráticas}
\begin{definition}
$x=c$ es \textbf{solución} si, al sustituir en la ecuación, produce una identidad.
\end{definition}

\pause
\vspace{0.5cm}
\begin{center}
\begin{tikzpicture}
\begin{axis}[
    axis lines=center,
    xlabel={$x$},
    ylabel={$y$},
    ymin=-2, ymax=4,
    xmin=-4, xmax=4,
    width=10cm,
    height=5cm,
    legend pos=north west
]
% Two real solutions
\addplot[domain=-3.5:2.5, samples=100, thick, myblue] {x^2 + x - 2};
\addplot[only marks, mark=*, mark size=3pt, myred] coordinates {(-2,0) (1,0)};
\legend{$y=x^2+x-2$, Soluciones}
\end{axis}
\end{tikzpicture}
\end{center}
\end{frame}

\section{Métodos de Solución}

\begin{frame}{Tres Métodos Principales}
\begin{center}
\begin{tikzpicture}[
    box/.style={rectangle, draw, fill=blue!20, text width=3.5cm,
                text centered, rounded corners, minimum height=1.2cm},
    arrow/.style={-Stealth, thick}
]
\node[box, fill=myblue!30] (eq) at (0,3) {\textbf{Ecuación Cuadrática}\\$ax^2+bx+c=0$};

\node[box, fill=mygreen!30] (factor) at (-4,0) {\textbf{Factorización}\\Más rápido};
\node[box, fill=mypurple!30] (complete) at (0,0) {\textbf{Completar Cuadrado}\\Útil teóricamente};
\node[box, fill=myred!30] (formula) at (4,0) {\textbf{Fórmula Cuadrática}\\Siempre funciona};

\draw[arrow] (eq) -- (factor);
\draw[arrow] (eq) -- (complete);
\draw[arrow] (eq) -- (formula);
\end{tikzpicture}
\end{center}
\end{frame}

\subsection{Factorización}

\begin{frame}{Método 1: Factorización}
\begin{block}{Teorema del Cero como Factor}
Si $p$ y $q$ son expresiones algebraicas, entonces
\[
pq=0 \iff p=0 \text{ o bien } q=0
\]
\end{block}

\pause
\textbf{Pasos:}
\begin{enumerate}
\item<2-> Escribir en forma estándar: $ax^2 + bx + c = 0$
\item<3-> Factorizar: $(rx+p)(sx+q)=0$ con $rs=a$ y $pq=c$
\item<4-> Igualar cada factor a cero
\item<5-> Resolver las ecuaciones lineales resultantes
\end{enumerate}
\end{frame}

\begin{frame}{Ejemplo A: Factorización}
\textbf{Resolver:} $6x^2 + x - 12 = 0$

\pause
\vspace{0.3cm}
\textbf{Paso 1:} Factorizar el trinomio
\[
6x^2 + x - 12 = {\color{myblue}(3x - 4)(2x + 3)} = 0
\]

\pause
\textbf{Paso 2:} Igualar cada factor a cero
\begin{align*}
{\color{myred}3x - 4 = 0} &\quad \text{o bien} \quad {\color{mygreen}2x + 3 = 0}
\end{align*}

\pause
\textbf{Paso 3:} Resolver
\begin{align*}
{\color{myred}x = \frac{4}{3}} &\quad \text{o bien} \quad {\color{mygreen}x = -\frac{3}{2}}
\end{align*}

\pause
\begin{alertblock}{Soluciones}
$x = \dfrac{4}{3}$ \quad o \quad $x = -\dfrac{3}{2}$
\end{alertblock}
\end{frame}

\begin{frame}{Ejemplo B: Factorización con Forma No Estándar}
\textbf{Resolver:} $2x(4x+15)=27$

\pause
\textbf{Paso 1:} Expandir y escribir en forma estándar
\begin{align*}
8x^2 + 30x &= 27\\
{\color{myblue}8x^2 + 30x - 27} &{\color{myblue}= 0}
\end{align*}

\pause
\textbf{Paso 2:} Factorizar
\[
(4x - 3)(2x + 9) = 0
\]

\pause
\textbf{Paso 3:} Resolver
\begin{align*}
4x - 3 = 0 \quad &\Rightarrow \quad {\color{myred}x = \frac{3}{4}}\\
2x + 9 = 0 \quad &\Rightarrow \quad {\color{mygreen}x = -\frac{9}{2}}
\end{align*}
\end{frame}

\subsection{Fórmula Cuadrática}

\begin{frame}{Método 2: La Fórmula Cuadrática}
\begin{block}{Fórmula Cuadrática}
Para la ecuación $ax^2 + bx + c = 0$ con $a\neq 0$:
\[
\boxed{x = \frac{-b \pm \sqrt{b^2 - 4ac}}{2a}}
\]
\end{block}

\pause
\vspace{0.3cm}
\begin{center}
\begin{tikzpicture}[scale=0.8]
% Drawing components of the formula
\node[fill=myblue!20, rounded corners, text width=2.5cm, align=center] (b) at (0,0) {$-b$\\Opuesto de $b$};
\node[fill=mygreen!20, rounded corners, text width=3cm, align=center] (disc) at (4,0) {$\sqrt{b^2-4ac}$\\Discriminante};
\node[fill=myred!20, rounded corners, text width=2cm, align=center] (denom) at (2,-2) {$2a$\\Denominador};

\draw[-Stealth, thick] (b) -- (1,1.5);
\draw[-Stealth, thick] (disc) -- (3,1.5);
\draw[-Stealth, thick] (denom) -- (2,1.5);
\node at (2,2) {\Large $x = \dfrac{-b \pm \sqrt{b^2-4ac}}{2a}$};
\end{tikzpicture}
\end{center}
\end{frame}

\begin{frame}{El Discriminante}
\begin{definition}
El \textbf{discriminante} es $\Delta = b^2 - 4ac$
\end{definition}

\vspace{0.5cm}
\begin{columns}
\column{0.5\textwidth}
\begin{itemize}
\item<2-> \textcolor{mygreen}{\textbf{$\Delta > 0$:}} Dos raíces reales distintas
\item<3-> \textcolor{myblue}{\textbf{$\Delta = 0$:}} Una raíz real (multiplicidad 2)
\item<4-> \textcolor{myred}{\textbf{$\Delta < 0$:}} No hay raíces reales
\end{itemize}

\column{0.5\textwidth}
\begin{center}
\begin{tikzpicture}[scale=0.5]
\begin{axis}[
    axis lines=center,
    xlabel={$x$},
    ymin=-2, ymax=3,
    xmin=-3, xmax=3,
    width=5cm,
    height=4cm,
    title={$\Delta > 0$}
]
\addplot[domain=-2.5:1.5, samples=50, thick, mygreen] {x^2 + x - 2};
\addplot[only marks, mark=*, mygreen] coordinates {(-2,0) (1,0)};
\end{axis}
\end{tikzpicture}
\end{center}
\end{columns}
\end{frame}

\begin{frame}{Visualización del Discriminante}
\begin{center}
\begin{tikzpicture}[scale=0.7]
% Delta > 0
\begin{axis}[
    name=plot1,
    axis lines=center,
    ymin=-2, ymax=3,
    xmin=-3, xmax=2,
    width=4.5cm,
    height=3.5cm,
    title={\color{mygreen}$\Delta > 0$},
    title style={at={(0.5,1.1)}},
    xtick=\empty,
    ytick=\empty
]
\addplot[domain=-2.5:1.5, samples=50, thick, mygreen] {x^2 + x - 2};
\addplot[only marks, mark=*, mark size=2pt, mygreen] coordinates {(-2,0) (1,0)};
\end{axis}

% Delta = 0
\begin{axis}[
    name=plot2,
    at={(plot1.right of south east)},
    anchor=left of south west,
    xshift=1cm,
    axis lines=center,
    ymin=-1, ymax=4,
    xmin=-3, xmax=3,
    width=4.5cm,
    height=3.5cm,
    title={\color{myblue}$\Delta = 0$},
    title style={at={(0.5,1.1)}},
    xtick=\empty,
    ytick=\empty
]
\addplot[domain=-2.5:2.5, samples=50, thick, myblue] {(x-1)^2};
\addplot[only marks, mark=*, mark size=2pt, myblue] coordinates {(1,0)};
\end{axis}

% Delta < 0
\begin{axis}[
    name=plot3,
    at={(plot2.right of south east)},
    anchor=left of south west,
    xshift=1cm,
    axis lines=center,
    ymin=-1, ymax=4,
    xmin=-3, xmax=3,
    width=4.5cm,
    height=3.5cm,
    title={\color{myred}$\Delta < 0$},
    title style={at={(0.5,1.1)}},
    xtick=\empty,
    ytick=\empty
]
\addplot[domain=-2.5:2.5, samples=50, thick, myred] {x^2 + 1};
\end{axis}
\end{tikzpicture}
\end{center}

\begin{itemize}
\item<2-> {\color{mygreen}Dos puntos de intersección} con el eje $x$
\item<3-> {\color{myblue}Toca tangencialmente} el eje $x$
\item<4-> {\color{myred}No interseca} el eje $x$
\end{itemize}
\end{frame}

\begin{frame}{Ejemplo C: Usando la Fórmula Cuadrática}
\textbf{Resolver:} $x^2 - x - 2 = 0$

\pause
\textbf{Paso 1:} Identificar coeficientes
\[
a = 1, \quad b = -1, \quad c = -2
\]

\pause
\textbf{Paso 2:} Calcular el discriminante
\begin{align*}
\Delta &= b^2 - 4ac\\
&= (-1)^2 - 4(1)(-2)\\
&= 1 + 8 = {\color{mygreen}9 > 0}
\end{align*}
{\color{mygreen}$\Rightarrow$ Dos raíces reales distintas}

\end{frame}

\begin{frame}{Ejemplo C: Continuación}
\textbf{Paso 3:} Aplicar la fórmula
\begin{align*}
x &= \frac{-b \pm \sqrt{\Delta}}{2a}\\
&= \frac{-(-1) \pm \sqrt{9}}{2(1)}\\
&= \frac{1 \pm 3}{2}
\end{align*}

\pause
\textbf{Paso 4:} Obtener las dos soluciones
\begin{align*}
x_1 &= \frac{1 + 3}{2} = \frac{4}{2} = {\color{myred}2}\\
x_2 &= \frac{1 - 3}{2} = \frac{-2}{2} = {\color{myblue}-1}
\end{align*}

\pause
\begin{alertblock}{Soluciones}
$x = 2$ \quad o \quad $x = -1$
\end{alertblock}
\end{frame}

\begin{frame}{Ejemplo D: Ecuación sin Soluciones Reales}
\textbf{Resolver:} $\dfrac{5x}{x^2 + 9} = -1$

\pause
\textbf{Paso 1:} Multiplicar por $x^2 + 9$ (siempre positivo, no crea soluciones extrañas)
\[
5x = -(x^2 + 9)
\]

\pause
\textbf{Paso 2:} Forma estándar
\[
x^2 + 5x + 9 = 0
\]

\pause
\textbf{Paso 3:} Calcular discriminante
\begin{align*}
\Delta &= 5^2 - 4(1)(9)\\
&= 25 - 36 = {\color{myred}-11 < 0}
\end{align*}

\pause
\begin{alertblock}{Conclusión}
{\color{myred}No hay soluciones reales}
\end{alertblock}
\end{frame}

\subsection{Completar el Cuadrado}

\begin{frame}{Método 3: Completar el Cuadrado}
\textbf{Idea:} Transformar $ax^2 + bx + c = 0$ en $(x+p)^2 = q$

\pause
\vspace{0.3cm}
\begin{center}
\begin{tikzpicture}[scale=0.8]
% Visual representation
\draw[fill=myblue!30] (0,0) rectangle (2,2);
\draw[fill=mygreen!30] (2,0) rectangle (3,2);
\draw[fill=mygreen!30] (0,2) rectangle (2,3);
\draw[fill=myred!30] (2,2) rectangle (3,3);

\node at (1,1) {$x^2$};
\node at (2.5,1) {$\frac{b}{2}x$};
\node at (1,2.5) {$\frac{b}{2}x$};
\node at (2.5,2.5) {$\left(\frac{b}{2}\right)^2$};

\draw[<->, thick] (0,-0.3) -- (2,-0.3);
\node at (1,-0.6) {$x$};
\draw[<->, thick] (2,-0.3) -- (3,-0.3);
\node at (2.5,-0.6) {$\frac{b}{2}$};

\draw[<->, thick] (-0.3,0) -- (-0.3,2);
\node at (-0.8,1) {$x$};
\draw[<->, thick] (-0.3,2) -- (-0.3,3);
\node at (-0.8,2.5) {$\frac{b}{2}$};

\draw[decorate, decoration={brace, amplitude=10pt}, thick] (3.2,0) -- (3.2,3);
\node at (5,1.5) {$\left(x + \frac{b}{2}\right)^2$};
\end{tikzpicture}
\end{center}

\onslide<3->{
\textbf{Fórmula clave:} Para obtener un cuadrado perfecto de $x^2 + bx$, añadir $\left(\dfrac{b}{2}\right)^2$
}
\end{frame}

\begin{frame}{Ejemplo: Completar el Cuadrado}
\textbf{Resolver:} $x^2 + 6x + 5 = 0$

\pause
\textbf{Paso 1:} Mover el término constante
\[
x^2 + 6x = -5
\]

\pause
\textbf{Paso 2:} Completar el cuadrado (añadir $\left(\frac{6}{2}\right)^2 = 9$ a ambos lados)
\[
x^2 + 6x + 9 = -5 + 9
\]

\pause
\textbf{Paso 3:} Factorizar el lado izquierdo
\[
(x + 3)^2 = 4
\]

\pause
\textbf{Paso 4:} Tomar raíz cuadrada
\[
x + 3 = \pm 2
\]

\pause
\textbf{Paso 5:} Resolver
\[
x = -3 \pm 2 \quad \Rightarrow \quad {\color{myred}x = -1} \text{ o } {\color{myblue}x = -5}
\]
\end{frame}

\section{Estrategia de Solución}

\begin{frame}{¿Qué Método Usar?}
\begin{center}
\begin{tikzpicture}[
    node distance=1.5cm,
    decision/.style={diamond, draw, fill=yellow!30, text width=4cm,
                     text centered, inner sep=0pt},
    block/.style={rectangle, draw, fill=blue!20, text width=3.5cm,
                  text centered, rounded corners, minimum height=1cm},
    result/.style={rectangle, draw, fill=green!30, text width=3cm,
                   text centered, rounded corners},
    arrow/.style={-Stealth, thick}
]

\node[block, fill=mypurple!30] (start) {Ecuación Cuadrática\\$ax^2+bx+c=0$};
\node[decision, below of=start, yshift=-0.5cm] (factor) {¿Se factoriza\\fácilmente?};
\node[result, below of=factor, yshift=-1.2cm, xshift=-3cm] (factormethod) {\textbf{Factorización}\\Más rápido};
\node[decision, below of=factor, yshift=-1.2cm, xshift=3cm] (choose) {¿Qué método\\prefieres?};
\node[result, below of=choose, yshift=-1.2cm, xshift=-2cm] (formula) {\textbf{Fórmula}\\Confiable};
\node[result, below of=choose, yshift=-1.2cm, xshift=2cm] (complete) {\textbf{Completar}\\Teórico};

\draw[arrow] (start) -- (factor);
\draw[arrow] (factor) -- node[left] {Sí} (factormethod);
\draw[arrow] (factor) -- node[above] {No} (choose);
\draw[arrow] (choose) -- (formula);
\draw[arrow] (choose) -- (complete);

\end{tikzpicture}
\end{center}
\end{frame}

\begin{frame}{Resumen de Métodos}
\begin{table}
\centering
\begin{tabular}{|l|c|c|c|}
\hline
\textbf{Método} & \textbf{Velocidad} & \textbf{Confiabilidad} & \textbf{Cuándo usar} \\
\hline
Factorización & \color{mygreen}Rápida & Media & Factores obvios \\
\hline
Fórmula & Media & \color{mygreen}Alta & Siempre funciona \\
\hline
Completar & Lenta & \color{mygreen}Alta & Derivar fórmula \\
\hline
\end{tabular}
\end{table}

\pause
\vspace{0.5cm}
\begin{block}{Recomendación General}
\begin{enumerate}
\item Intenta \textbf{factorizar} primero
\item Si no es obvio, usa la \textbf{fórmula cuadrática}
\item Siempre verifica el \textbf{discriminante} para anticipar el tipo de soluciones
\end{enumerate}
\end{block}
\end{frame}

\section{Práctica}

\begin{frame}{Ejercicios de Práctica}
\textbf{Resuelve las siguientes ecuaciones cuadráticas:}

\begin{enumerate}
\item $x^2 - 5x + 6 = 0$
\vspace{0.3cm}
\item $2x^2 + 7x - 15 = 0$
\vspace{0.3cm}
\item $x^2 - 4x + 5 = 0$
\vspace{0.3cm}
\item $3x^2 = 12$
\vspace{0.3cm}
\item $x(x - 8) = -15$
\end{enumerate}

\vspace{0.5cm}
\textbf{Sugerencias:}
\begin{itemize}
\item Identifica el método más apropiado
\item Calcula el discriminante cuando uses la fórmula
\item Verifica tus respuestas sustituyendo
\end{itemize}
\end{frame}

\begin{frame}{Soluciones de Práctica}
\begin{enumerate}
\item $x^2 - 5x + 6 = 0$ \\
{\color{myblue}$(x-2)(x-3)=0 \Rightarrow x=2$ o $x=3$}
\vspace{0.2cm}

\item $2x^2 + 7x - 15 = 0$ \\
{\color{myblue}$(2x-3)(x+5)=0 \Rightarrow x=\frac{3}{2}$ o $x=-5$}
\vspace{0.2cm}

\item $x^2 - 4x + 5 = 0$ \\
{\color{myred}$\Delta = 16-20 = -4 < 0 \Rightarrow$ No hay soluciones reales}
\vspace{0.2cm}

\item $3x^2 = 12$ \\
{\color{myblue}$x^2=4 \Rightarrow x=\pm 2$}
\vspace{0.2cm}

\item $x(x - 8) = -15$ \\
{\color{myblue}$x^2-8x+15=0 \Rightarrow (x-3)(x-5)=0 \Rightarrow x=3$ o $x=5$}
\end{enumerate}
\end{frame}

\begin{frame}{Puntos Clave para Recordar}
\begin{itemize}
\item<1-> Una ecuación cuadrática tiene la forma $ax^2 + bx + c = 0$ con $a \neq 0$
\vspace{0.3cm}
\item<2-> Tres métodos principales: factorización, fórmula cuadrática, completar el cuadrado
\vspace{0.3cm}
\item<3-> El discriminante $\Delta = b^2 - 4ac$ determina el número y tipo de soluciones
\vspace{0.3cm}
\item<4-> La fórmula cuadrática \textbf{siempre funciona}: $x = \dfrac{-b \pm \sqrt{b^2-4ac}}{2a}$
\vspace{0.3cm}
\item<5-> Siempre verifica tus soluciones sustituyendo en la ecuación original
\end{itemize}
\end{frame}

\begin{frame}
\begin{center}
\Huge ¿Preguntas?
\end{center}
\end{frame}

\end{document}