\documentclass[12pt]{article}

%========================================
% PACKAGES AND CONFIGURATION
%========================================
\usepackage[utf8]{inputenc}
\usepackage[spanish]{babel}              % Spanish language support
\decimalpoint                                % Force decimal point instead of comma
\usepackage[margin=1in]{geometry}
\setlength{\headheight}{15pt}
\usepackage{amsmath, amsthm, amssymb}
\usepackage{mdframed}
\usepackage{xcolor}
\usepackage{enumitem}
\usepackage{fancyhdr}
\usepackage{graphicx}
\usepackage{tikz}                        % For LaTeX-generated diagrams
\usetikzlibrary{arrows.meta}           % For arrow styles
\usetikzlibrary{shapes}                % For diamond and other shapes
\usetikzlibrary{decorations.pathreplacing} % For braces and decorations
\usetikzlibrary{calc}                  % For coordinate calculations
\usepackage{comment}                     % For conditional content exclusion

% Fix Spanish babel conflicts with TikZ
\usetikzlibrary{babel}

%========================================
% COURSE CUSTOMIZATION SECTION
%========================================
% MODIFY THESE FOR EACH COURSE:
\newcommand{\coursecode}{MATE 3001}      % Course code
\newcommand{\coursename}{Matemática Elemental}     % Course name
\newcommand{\institution}{UPR-Humacao}   % Institution name
\newcommand{\lessontitle}{Ecuaciones Lineales en Dos Variables}    % Will be overridden per lesson

%========================================
% COLOR SCHEME DEFINITIONS
%========================================
\definecolor{defcolor}{RGB}{240,248,255}     % Light blue for definitions
\definecolor{examplecolor}{RGB}{245,255,245} % Light green for examples
\definecolor{exercisecolor}{RGB}{255,248,240} % Light orange for exercises
\definecolor{theoremcolor}{RGB}{255,250,240}  % Light orange for theorems

%========================================
% CUSTOM ENVIRONMENTS
%========================================
% Definition Environment (Blue)
\newmdenv[
    backgroundcolor=defcolor,
    linecolor=blue!50,
    linewidth=2pt,
    leftmargin=10pt,
    rightmargin=10pt,
    innertopmargin=10pt,
    innerbottommargin=10pt,
    frametitle={\textbf{Definición}},
    frametitlealignment=\raggedright
]{definition}

% Example Environment (Green)
\newmdenv[
    backgroundcolor=examplecolor,
    linecolor=green!50,
    linewidth=2pt,
    leftmargin=10pt,
    rightmargin=10pt,
    innertopmargin=10pt,
    innerbottommargin=10pt,
    frametitle={\textbf{Ejemplo}},
    frametitlealignment=\raggedright
]{example}

% Exercise Environment (Orange)
\newcounter{exercise}[section]
\newcounter{problem}[exercise]
\newmdenv[
    backgroundcolor=exercisecolor,
    linecolor=orange!50,
    linewidth=2pt,
    leftmargin=10pt,
    rightmargin=10pt,
    innertopmargin=10pt,
    innerbottommargin=10pt,
    frametitle={\stepcounter{exercise}\textbf{Ejercicio \theexercise}},
    frametitlealignment=\raggedright
]{exercise}

% Theorem Environment (Orange variant)
\newmdenv[
    backgroundcolor=theoremcolor,
    linecolor=orange!50,
    linewidth=2pt,
    leftmargin=10pt,
    rightmargin=10pt,
    innertopmargin=10pt,
    innerbottommargin=10pt
]{theorem}

%========================================
% HEADER AND FOOTER CONFIGURATION
%========================================
\pagestyle{fancy}
\fancyhf{}
\rhead{\coursecode\ - \coursename}
\lhead{\lessontitle}
\cfoot{\thepage}

%========================================
% CUSTOM COMMANDS
%========================================
\newcommand{\lesson}[1]{\renewcommand{\lessontitle}{#1}\section{#1}}
\newcommand{\subsectiontitle}[1]{\subsection{#1}}

% Exercise numbering commands
\newcommand{\problem}{\stepcounter{problem}\textbf{\theproblem.} }
\newcommand{\solution}{\textbf{Solución:} }

% Custom environment for exercise lists
\newenvironment{exerciselist}
    {\begin{enumerate}[label=\textbf{\alph*.}]}
    {\end{enumerate}}

% Solution environment with conditional display
\newif\ifshowsolutions
% \showsolutionstrue  % Uncomment for instructor version
\showsolutionsfalse   % Default: student version

\ifshowsolutions
    \newenvironment{solucion}[1][Solución]
      {\par\medskip\noindent\textbf{#1:}\par\nopagebreak}
      {\par\medskip}
\else
    \excludecomment{solucion}
\fi

%========================================
% GRAPHICS CONFIGURATION
%========================================
\graphicspath{{../images/}{../images/shared/}{../images/diagrams/}{../images/11_ecuaciones_lineales_dos_variables/}}

%========================================
% DOCUMENT CONTENT
%========================================
\begin{document}

% Title page
\title{\lessontitle}
\author{\coursecode\ - \coursename}
\date{}
\maketitle

% Set section counter to lesson number
\setcounter{section}{10}

% Modular content inclusion
%========================================
% LESSON CONTENT: Ecuaciones Lineales en Dos Variables
%========================================

\lesson{Ecuaciones Lineales en Dos Variables}

%========================================
% SECTION 11.1: Introducción
%========================================
\subsectiontitle{Introducción a las Ecuaciones Lineales en Dos Variables}

\begin{definition}
\textbf{Ecuación Lineal en Dos Variables:}

Una ecuación lineal en dos variables es una ecuación que se puede escribir en la forma:

$$\boxed{Ax + By = C}$$

donde:
\begin{itemize}
    \item $A$, $B$ y $C$ son constantes (números reales)
    \item $A$ y $B$ no son ambos cero
    \item $x$ e $y$ son las variables
\end{itemize}

Esta forma se conoce como \textbf{forma estándar} de una ecuación lineal.
\end{definition}

\textbf{Ejemplos de ecuaciones lineales en dos variables:}
\begin{itemize}
    \item $2x + 3y = 6$ \quad (donde $A = 2$, $B = 3$, $C = 6$)
    \item $x - 4y = 8$ \quad (donde $A = 1$, $B = -4$, $C = 8$)
    \item $5x = 10$ \quad (donde $A = 5$, $B = 0$, $C = 10$)
    \item $-3y = 9$ \quad (donde $A = 0$, $B = -3$, $C = 9$)
\end{itemize}

\textbf{Observación importante:} A diferencia de las ecuaciones lineales con una variable (que tienen una solución), las ecuaciones lineales con dos variables tienen un \textbf{número infinito de soluciones}.

\newpage
%========================================
% SECTION 11.2: Soluciones y Gráficas
%========================================
\subsectiontitle{Soluciones y Gráficas de Ecuaciones Lineales}

\begin{definition}
\textbf{Solución de una Ecuación Lineal en Dos Variables:}

Una solución de una ecuación lineal en dos variables es un \textbf{par ordenado} $(x, y)$ que satisface la ecuación cuando los valores se sustituyen en ella.

Estas soluciones se pueden representar como \textbf{puntos} en el sistema de coordenadas rectangulares (visto en la Lección 10).
\end{definition}

\subsubsection*{Método de Tabulación}

Para encontrar soluciones de una ecuación lineal:
\begin{enumerate}
    \item Elija un valor para una de las variables (usualmente $x$)
    \item Sustituya ese valor en la ecuación
    \item Resuelva para la otra variable ($y$)
    \item El par ordenado $(x, y)$ es una solución
    \item Repita el proceso para obtener más soluciones
\end{enumerate}

\begin{example}
\textbf{Encontrar soluciones de una ecuación lineal}

Dada la ecuación $2x + 3y = 6$, encuentre al menos tres soluciones.

\textbf{Solución:}

\textbf{Para $x = 0$:}
\begin{align*}
2(0) + 3y &= 6 \\
3y &= 6 \\
y &= 2
\end{align*}
Solución: $(0, 2)$

\textbf{Para $x = 3$:}
\begin{align*}
2(3) + 3y &= 6 \\
6 + 3y &= 6 \\
3y &= 0 \\
y &= 0
\end{align*}
Solución: $(3, 0)$

\textbf{Para $x = -3$:}
\begin{align*}
2(-3) + 3y &= 6 \\
-6 + 3y &= 6 \\
3y &= 12 \\
y &= 4
\end{align*}
Solución: $(-3, 4)$

\textbf{Tabla de valores:}

\begin{center}
\begin{tabular}{|c|c|c|}
\hline
$x$ & $y$ & Punto $(x, y)$ \\
\hline
$-3$ & $4$ & $(-3, 4)$ \\
\hline
$0$ & $2$ & $(0, 2)$ \\
\hline
$3$ & $0$ & $(3, 0)$ \\
\hline
$6$ & $-2$ & $(6, -2)$ \\
\hline
$9$ & $-4$ & $(9, -4)$ \\
\hline
\end{tabular}
\end{center}
\end{example}

\newpage

\subsubsection*{Gráfica de una Ecuación Lineal}

\begin{theorem}
\textbf{Propiedad Fundamental:}

La gráfica de cualquier ecuación lineal en dos variables es una \textbf{línea recta}.

Para graficar una ecuación lineal:
\begin{enumerate}
    \item Encuentre al menos dos puntos (soluciones) de la ecuación
    \item Grafique estos puntos en el plano de coordenadas
    \item Trace una línea recta que pase por todos los puntos
\end{enumerate}

\textbf{Nota:} Aunque solo se necesitan dos puntos para definir una recta, es recomendable encontrar un tercer punto como verificación.
\end{theorem}

% Graphical representation of 2x + 3y = 6
\begin{center}
\begin{tikzpicture}[scale=0.8]
    % Grid
    \draw[gray!20, very thin] (-4,-5) grid (10,5);

    % Axes
    \draw[->, thick] (-4,0) -- (10,0) node[right] {$x$};
    \draw[->, thick] (0,-5) -- (0,5) node[above] {$y$};

    % Axis labels
    \foreach \x in {-3,3,6,9}
        \draw (\x,0.15) -- (\x,-0.15) node[below, font=\small] {\x};
    \foreach \y in {-4,-2,2,4}
        \draw (0.15,\y) -- (-0.15,\y) node[left, font=\small] {\y};

    % Plot the line 2x + 3y = 6
    \draw[blue, very thick, domain=-3.5:9.5] plot (\x, {(6-2*\x)/3});

    % Plot the points from table
    \filldraw[red] (-3,4) circle (3pt);
    \node[red, above left] at (-3,4) {$(-3,4)$};

    \filldraw[red] (0,2) circle (3pt);
    \node[red, above right] at (0,2) {$(0,2)$};

    \filldraw[red] (3,0) circle (3pt);
    \node[red, below right] at (3,0) {$(3,0)$};

    \filldraw[red] (6,-2) circle (3pt);
    \node[red, below] at (6,-2) {$(6,-2)$};

    \filldraw[red] (9,-4) circle (3pt);
    \node[red, below] at (9,-4) {$(9,-4)$};

    % Equation label
    \node[blue, above] at (5,3.5) {\Large $2x + 3y = 6$};
\end{tikzpicture}
\end{center}

\newpage
%========================================
% SECTION 11.3: Intersecciones con los Ejes
%========================================
\subsectiontitle{Intersecciones con los Ejes}

Las intersecciones con los ejes son puntos especiales que facilitan la gráfica de ecuaciones lineales.

\begin{definition}
\textbf{Intersecciones (Interceptos):}

\begin{itemize}
    \item \textbf{Intersección con el eje $x$ (x-intercept):} Las coordenadas $x$ de los puntos donde la gráfica interseca al eje $x$.
    \begin{itemize}
        \item Para encontrarla: Haga $y = 0$ y despeje $x$
        \item El punto tiene la forma $(a, 0)$
    \end{itemize}

    \item \textbf{Intersección con el eje $y$ (y-intercept):} Las coordenadas $y$ de los puntos donde la gráfica interseca al eje $y$.
    \begin{itemize}
        \item Para encontrarla: Haga $x = 0$ y despeje $y$
        \item El punto tiene la forma $(0, b)$
    \end{itemize}
\end{itemize}
\end{definition}

% Visual representation of intercepts
\begin{center}
\begin{tikzpicture}[scale=1.0]
    % Grid
    \draw[gray!20, very thin] (-2,-2) grid (6,4);

    % Axes
    \draw[->, thick] (-2,0) -- (6,0) node[right] {$x$};
    \draw[->, thick] (0,-2) -- (0,4) node[above] {$y$};

    % Axis labels
    \foreach \x in {2,4}
        \draw (\x,0.1) -- (\x,-0.1) node[below, font=\small] {\x};
    \foreach \y in {2}
        \draw (0.1,\y) -- (-0.1,\y) node[left, font=\small] {\y};

    % Plot a line
    \draw[blue, very thick] (-1,3) -- (5,-1);

    % x-intercept
    \filldraw[red] (3.667,0) circle (3pt);
    \node[red, below] at (3.667,-0.3) {\textbf{Intersección $x$}};
    \node[red, below] at (3.667,-0.7) {$(a, 0)$};
    \draw[red, dashed, thick] (3.667,0) -- (3.667,-1.5);

    % y-intercept
    \filldraw[green!70!black] (0,2.333) circle (3pt);
    \node[green!70!black, left] at (-0.3,2.333) {\textbf{Intersección $y$}};
    \node[green!70!black, left] at (-0.3,1.9) {$(0, b)$};
    \draw[green!70!black, dashed, thick] (0,2.333) -- (-1.5,2.333);

    % Origin
    \node[below left] at (0,0) {$O$};
\end{tikzpicture}
\end{center}

\newpage
\begin{example}
\textbf{Encontrar las intersecciones con los ejes}

Para la ecuación $3x - 4y = 12$, encuentre las intersecciones con ambos ejes.

\textbf{Solución:}

\textbf{Intersección con el eje $x$ ($y = 0$):}
\begin{align*}
3x - 4(0) &= 12 \\
3x &= 12 \\
x &= 4
\end{align*}
Intersección con el eje $x$: $(4, 0)$

\textbf{Intersección con el eje $y$ ($x = 0$):}
\begin{align*}
3(0) - 4y &= 12 \\
-4y &= 12 \\
y &= -3
\end{align*}
Intersección con el eje $y$: $(0, -3)$

\textbf{Gráfica usando las intersecciones:}

\begin{center}
\begin{tikzpicture}[scale=0.8]
    % Grid
    \draw[gray!20, very thin] (-2,-4) grid (6,2);

    % Axes
    \draw[->, thick] (-2,0) -- (6,0) node[right] {$x$};
    \draw[->, thick] (0,-4) -- (0,2) node[above] {$y$};

    % Axis labels
    \foreach \x in {2,4}
        \draw (\x,0.1) -- (\x,-0.1) node[below, font=\small] {\x};
    \foreach \y in {-3,1}
        \draw (0.1,\y) -- (-0.1,\y) node[left, font=\small] {\y};

    % Plot the line 3x - 4y = 12
    \draw[blue, very thick, domain=-1:5.5] plot (\x, {(3*\x - 12)/4});

    % x-intercept
    \filldraw[red] (4,0) circle (3pt);
    \node[red, above right] at (4,0) {$(4,0)$};

    % y-intercept
    \filldraw[green!70!black] (0,-3) circle (3pt);
    \node[green!70!black, left] at (0,-3) {$(0,-3)$};

    % Equation label
    \node[blue, above] at (3,1.3) {$3x - 4y = 12$};
\end{tikzpicture}
\end{center}
\end{example}


%========================================
% EXERCISES: Ecuaciones Lineales en Dos Variables
%========================================

\section{Ejercicios}

%========================================
% Exercise 1: Finding Solutions
%========================================
\begin{exercise}
\textbf{Encontrar Soluciones}

Para cada ecuación, complete la tabla de valores y encuentre las soluciones indicadas:

\problem $x + y = 5$

Complete la tabla:
\begin{center}
\begin{tabular}{|c|c|c|}
\hline
$x$ & $y$ & $(x, y)$ \\
\hline
0 & & \\
\hline
2 & & \\
\hline
& 0 & \\
\hline
-1 & & \\
\hline
\end{tabular}
\end{center}

\begin{solucion}
\begin{tabular}{|c|c|c|}
\hline
$x$ & $y$ & $(x, y)$ \\
\hline
0 & 5 & $(0, 5)$ \\
\hline
2 & 3 & $(2, 3)$ \\
\hline
5 & 0 & $(5, 0)$ \\
\hline
-1 & 6 & $(-1, 6)$ \\
\hline
\end{tabular}
\end{solucion}

\problem $2x - y = 4$

Complete la tabla:
\begin{center}
\begin{tabular}{|c|c|c|}
\hline
$x$ & $y$ & $(x, y)$ \\
\hline
0 & & \\
\hline
2 & & \\
\hline
& 0 & \\
\hline
-2 & & \\
\hline
\end{tabular}
\end{center}

\begin{solucion}
\begin{tabular}{|c|c|c|}
\hline
$x$ & $y$ & $(x, y)$ \\
\hline
0 & -4 & $(0, -4)$ \\
\hline
2 & 0 & $(2, 0)$ \\
\hline
2 & 0 & $(2, 0)$ \\
\hline
-2 & -8 & $(-2, -8)$ \\
\hline
\end{tabular}
\end{solucion}

\problem $3x + 2y = 12$

Complete la tabla:
\begin{center}
\begin{tabular}{|c|c|c|}
\hline
$x$ & $y$ & $(x, y)$ \\
\hline
0 & & \\
\hline
& 0 & \\
\hline
2 & & \\
\hline
& 3 & \\
\hline
\end{tabular}
\end{center}

\begin{solucion}
\begin{tabular}{|c|c|c|}
\hline
$x$ & $y$ & $(x, y)$ \\
\hline
0 & 6 & $(0, 6)$ \\
\hline
4 & 0 & $(4, 0)$ \\
\hline
2 & 3 & $(2, 3)$ \\
\hline
2 & 3 & $(2, 3)$ \\
\hline
\end{tabular}
\end{solucion}
\end{exercise}

%========================================
% Exercise 2: Finding Intercepts
%========================================
\begin{exercise}
\textbf{Encontrar Intersecciones con los Ejes}

Para cada ecuación, encuentre las intersecciones con el eje $x$ y el eje $y$:

\problem $2x + 3y = 6$

\begin{solucion}
Intersección con eje $x$ (hacer $y = 0$): $2x = 6 \implies x = 3$. Punto: $(3, 0)$

Intersección con eje $y$ (hacer $x = 0$): $3y = 6 \implies y = 2$. Punto: $(0, 2)$
\end{solucion}

\problem $4x - y = 8$

\begin{solucion}
Intersección con eje $x$ (hacer $y = 0$): $4x = 8 \implies x = 2$. Punto: $(2, 0)$

Intersección con eje $y$ (hacer $x = 0$): $-y = 8 \implies y = -8$. Punto: $(0, -8)$
\end{solucion}

\problem $x + 2y = 10$

\begin{solucion}
Intersección con eje $x$ (hacer $y = 0$): $x = 10$. Punto: $(10, 0)$

Intersección con eje $y$ (hacer $x = 0$): $2y = 10 \implies y = 5$. Punto: $(0, 5)$
\end{solucion}

\problem $5x - 2y = 20$

\begin{solucion}
Intersección con eje $x$ (hacer $y = 0$): $5x = 20 \implies x = 4$. Punto: $(4, 0)$

Intersección con eje $y$ (hacer $x = 0$): $-2y = 20 \implies y = -10$. Punto: $(0, -10)$
\end{solucion}

\problem $3x + 4y = 24$

\begin{solucion}
Intersección con eje $x$ (hacer $y = 0$): $3x = 24 \implies x = 8$. Punto: $(8, 0)$

Intersección con eje $y$ (hacer $x = 0$): $4y = 24 \implies y = 6$. Punto: $(0, 6)$
\end{solucion}

\problem $-2x + 5y = 10$

\begin{solucion}
Intersección con eje $x$ (hacer $y = 0$): $-2x = 10 \implies x = -5$. Punto: $(-5, 0)$

Intersección con eje $y$ (hacer $x = 0$): $5y = 10 \implies y = 2$. Punto: $(0, 2)$
\end{solucion}
\end{exercise}

%========================================
% Exercise 3: Calculating Slope
%========================================
\begin{exercise}
\textbf{Calcular la Pendiente}

Calcule la pendiente de la recta que pasa por cada par de puntos:

\problem $(1, 2)$ y $(3, 6)$

\begin{solucion}
$m = \dfrac{6 - 2}{3 - 1} = \dfrac{4}{2} = 2$
\end{solucion}

\problem $(0, 5)$ y $(4, 1)$

\begin{solucion}
$m = \dfrac{1 - 5}{4 - 0} = \dfrac{-4}{4} = -1$
\end{solucion}

\problem $(-2, 3)$ y $(4, -1)$

\begin{solucion}
$m = \dfrac{-1 - 3}{4 - (-2)} = \dfrac{-4}{6} = -\dfrac{2}{3}$
\end{solucion}

\problem $(5, 2)$ y $(5, 7)$

\begin{solucion}
$m = \dfrac{7 - 2}{5 - 5} = \dfrac{5}{0}$ = indefinida (recta vertical)
\end{solucion}

\problem $(-3, 4)$ y $(2, 4)$

\begin{solucion}
$m = \dfrac{4 - 4}{2 - (-3)} = \dfrac{0}{5} = 0$ (recta horizontal)
\end{solucion}

\problem $(-1, -2)$ y $(3, 10)$

\begin{solucion}
$m = \dfrac{10 - (-2)}{3 - (-1)} = \dfrac{12}{4} = 3$
\end{solucion}

\problem $(6, 1)$ y $(2, 5)$

\begin{solucion}
$m = \dfrac{5 - 1}{2 - 6} = \dfrac{4}{-4} = -1$
\end{solucion}

\problem $(0, 0)$ y $(4, 8)$

\begin{solucion}
$m = \dfrac{8 - 0}{4 - 0} = \dfrac{8}{4} = 2$
\end{solucion}
\end{exercise}

%========================================
% Exercise 4: Identifying Slope Type
%========================================
\begin{exercise}
\textbf{Identificar el Tipo de Pendiente}

Para cada ecuación o descripción, indique si la pendiente es positiva, negativa, cero o indefinida:

\problem Una recta que sube de izquierda a derecha.

\begin{solucion}
Pendiente positiva
\end{solucion}

\problem $y = -3$

\begin{solucion}
Pendiente cero (recta horizontal)
\end{solucion}

\problem $x = 5$

\begin{solucion}
Pendiente indefinida (recta vertical)
\end{solucion}

\problem Una recta que pasa por $(1, 5)$ y $(4, 2)$.

\begin{solucion}
$m = \frac{2-5}{4-1} = \frac{-3}{3} = -1$. Pendiente negativa
\end{solucion}

\problem $y = 7x + 2$

\begin{solucion}
$m = 7$. Pendiente positiva
\end{solucion}

\problem $y = -\dfrac{1}{2}x + 3$

\begin{solucion}
$m = -\frac{1}{2}$. Pendiente negativa
\end{solucion}
\end{exercise}

%========================================
% Exercise 5: Slope-Intercept Form
%========================================
\begin{exercise}
\textbf{Forma Pendiente-Intersecto}

Para cada ecuación en forma pendiente-intersecto, identifique la pendiente y la intersección con el eje $y$:

\problem $y = 3x + 5$

\begin{solucion}
Pendiente: $m = 3$; Intersección $y$: $b = 5$
\end{solucion}

\problem $y = -2x - 1$

\begin{solucion}
Pendiente: $m = -2$; Intersección $y$: $b = -1$
\end{solucion}

\problem $y = \dfrac{1}{2}x + 4$

\begin{solucion}
Pendiente: $m = \frac{1}{2}$; Intersección $y$: $b = 4$
\end{solucion}

\problem $y = -x + 7$

\begin{solucion}
Pendiente: $m = -1$; Intersección $y$: $b = 7$
\end{solucion}

\problem $y = 4x$

\begin{solucion}
Pendiente: $m = 4$; Intersección $y$: $b = 0$
\end{solucion}

\problem $y = -\dfrac{3}{4}x - 2$

\begin{solucion}
Pendiente: $m = -\frac{3}{4}$; Intersección $y$: $b = -2$
\end{solucion}
\end{exercise}

%========================================
% Exercise 6: Converting to Slope-Intercept Form
%========================================
\begin{exercise}
\textbf{Convertir a Forma Pendiente-Intersecto}

Convierta cada ecuación a la forma $y = mx + b$:

\problem $2x + y = 6$

\begin{solucion}
$y = -2x + 6$

Pendiente: $m = -2$; Intersección $y$: $b = 6$
\end{solucion}

\problem $3x - y = 9$

\begin{solucion}
$-y = -3x + 9 \implies y = 3x - 9$

Pendiente: $m = 3$; Intersección $y$: $b = -9$
\end{solucion}

\problem $4x + 2y = 8$

\begin{solucion}
$2y = -4x + 8 \implies y = -2x + 4$

Pendiente: $m = -2$; Intersección $y$: $b = 4$
\end{solucion}

\problem $x - 2y = 10$

\begin{solucion}
$-2y = -x + 10 \implies y = \frac{1}{2}x - 5$

Pendiente: $m = \frac{1}{2}$; Intersección $y$: $b = -5$
\end{solucion}

\problem $6x + 3y = 12$

\begin{solucion}
$3y = -6x + 12 \implies y = -2x + 4$

Pendiente: $m = -2$; Intersección $y$: $b = 4$
\end{solucion}

\problem $5x - 2y = 20$

\begin{solucion}
$-2y = -5x + 20 \implies y = \frac{5}{2}x - 10$

Pendiente: $m = \frac{5}{2}$; Intersección $y$: $b = -10$
\end{solucion}
\end{exercise}

%========================================
% Exercise 7: Writing Equations (Slope and Point)
%========================================
\begin{exercise}
\textbf{Escribir Ecuaciones dada la Pendiente y un Punto}

Escriba la ecuación de la recta en forma pendiente-intersecto:

\problem Pendiente $m = 2$, pasa por $(1, 3)$

\begin{solucion}
Usando forma punto-pendiente: $y - 3 = 2(x - 1)$

$y - 3 = 2x - 2$

$y = 2x + 1$
\end{solucion}

\problem Pendiente $m = -3$, pasa por $(2, 5)$

\begin{solucion}
$y - 5 = -3(x - 2)$

$y - 5 = -3x + 6$

$y = -3x + 11$
\end{solucion}

\problem Pendiente $m = \dfrac{1}{2}$, pasa por $(4, 1)$

\begin{solucion}
$y - 1 = \frac{1}{2}(x - 4)$

$y - 1 = \frac{1}{2}x - 2$

$y = \frac{1}{2}x - 1$
\end{solucion}

\problem Pendiente $m = -\dfrac{2}{3}$, pasa por $(-3, 4)$

\begin{solucion}
$y - 4 = -\frac{2}{3}(x - (-3))$

$y - 4 = -\frac{2}{3}(x + 3)$

$y - 4 = -\frac{2}{3}x - 2$

$y = -\frac{2}{3}x + 2$
\end{solucion}

\problem Pendiente $m = 0$, pasa por $(5, -2)$

\begin{solucion}
Recta horizontal: $y = -2$
\end{solucion}

\problem Pendiente indefinida, pasa por $(3, 7)$

\begin{solucion}
Recta vertical: $x = 3$
\end{solucion}
\end{exercise}

%========================================
% Exercise 8: Writing Equations (Two Points)
%========================================
\begin{exercise}
\textbf{Escribir Ecuaciones dados Dos Puntos}

Encuentre la ecuación de la recta que pasa por los puntos dados. Escriba la respuesta en forma pendiente-intersecto:

\problem $(1, 2)$ y $(3, 8)$

\begin{solucion}
Pendiente: $m = \frac{8-2}{3-1} = \frac{6}{2} = 3$

Usando $(1, 2)$: $y - 2 = 3(x - 1)$

$y = 3x - 1$
\end{solucion}

\problem $(0, 4)$ y $(2, 0)$

\begin{solucion}
Pendiente: $m = \frac{0-4}{2-0} = \frac{-4}{2} = -2$

Usando $(0, 4)$: $y - 4 = -2(x - 0)$

$y = -2x + 4$
\end{solucion}

\problem $(-1, 3)$ y $(2, -6)$

\begin{solucion}
Pendiente: $m = \frac{-6-3}{2-(-1)} = \frac{-9}{3} = -3$

Usando $(-1, 3)$: $y - 3 = -3(x + 1)$

$y - 3 = -3x - 3$

$y = -3x$
\end{solucion}

\problem $(4, 1)$ y $(6, 5)$

\begin{solucion}
Pendiente: $m = \frac{5-1}{6-4} = \frac{4}{2} = 2$

Usando $(4, 1)$: $y - 1 = 2(x - 4)$

$y - 1 = 2x - 8$

$y = 2x - 7$
\end{solucion}

\problem $(-2, -1)$ y $(3, -1)$

\begin{solucion}
Pendiente: $m = \frac{-1-(-1)}{3-(-2)} = \frac{0}{5} = 0$

Recta horizontal: $y = -1$
\end{solucion}

\problem $(5, 2)$ y $(5, 8)$

\begin{solucion}
Pendiente: $m = \frac{8-2}{5-5} = \frac{6}{0}$ = indefinida

Recta vertical: $x = 5$
\end{solucion}
\end{exercise}

%========================================
% Exercise 9: Parallel and Perpendicular Lines
%========================================
\begin{exercise}
\textbf{Rectas Paralelas y Perpendiculares}

\problem Encuentre la ecuación de la recta que pasa por $(2, 3)$ y es paralela a $y = 4x - 1$.

\begin{solucion}
Recta paralela tiene la misma pendiente: $m = 4$

$y - 3 = 4(x - 2)$

$y - 3 = 4x - 8$

$y = 4x - 5$
\end{solucion}

\problem Encuentre la ecuación de la recta que pasa por $(1, 5)$ y es perpendicular a $y = 2x + 3$.

\begin{solucion}
Pendiente original: $m_1 = 2$

Pendiente perpendicular: $m_2 = -\frac{1}{2}$

$y - 5 = -\frac{1}{2}(x - 1)$

$y - 5 = -\frac{1}{2}x + \frac{1}{2}$

$y = -\frac{1}{2}x + \frac{11}{2}$
\end{solucion}

\problem Determine si las rectas $y = 3x + 2$ y $y = 3x - 7$ son paralelas, perpendiculares o ninguna.

\begin{solucion}
Ambas tienen pendiente $m = 3$.

Como $m_1 = m_2$, las rectas son \textbf{paralelas}.
\end{solucion}

\problem Determine si las rectas $y = \frac{2}{3}x + 1$ y $y = -\frac{3}{2}x + 4$ son paralelas, perpendiculares o ninguna.

\begin{solucion}
$m_1 = \frac{2}{3}$ y $m_2 = -\frac{3}{2}$

Producto: $m_1 \cdot m_2 = \frac{2}{3} \cdot (-\frac{3}{2}) = -1$

Las rectas son \textbf{perpendiculares}.
\end{solucion}

\problem Encuentre la ecuación de la recta que pasa por $(-1, 2)$ y es perpendicular a $3x + 2y = 6$.

\begin{solucion}
Convertir a forma pendiente-intersecto: $2y = -3x + 6 \implies y = -\frac{3}{2}x + 3$

Pendiente original: $m_1 = -\frac{3}{2}$

Pendiente perpendicular: $m_2 = \frac{2}{3}$

$y - 2 = \frac{2}{3}(x + 1)$

$y - 2 = \frac{2}{3}x + \frac{2}{3}$

$y = \frac{2}{3}x + \frac{8}{3}$
\end{solucion}
\end{exercise}

%========================================
% Exercise 10: Application Problems
%========================================
\begin{exercise}
\textbf{Problemas de Aplicación}

\problem Un taxista cobra una tarifa base de \$3 más \$0.50 por cada milla recorrida. Escriba una ecuación que relacione el costo total $C$ con el número de millas $m$ recorridas.

\begin{solucion}
$C = 0.50m + 3$

Donde $m = 0.50$ (costo por milla) y $b = 3$ (tarifa base).
\end{solucion}

\problem La temperatura en grados Fahrenheit ($F$) y Celsius ($C$) están relacionadas por la ecuación $F = \frac{9}{5}C + 32$. ¿Cuál es la temperatura en Fahrenheit cuando la temperatura es 20°C?

\begin{solucion}
$F = \frac{9}{5}(20) + 32 = 36 + 32 = 68$

La temperatura es 68°F.
\end{solucion}

\problem Un depósito de agua tiene 500 galones inicialmente. El agua se drena a una tasa de 25 galones por hora. Escriba una ecuación que exprese la cantidad de agua $A$ en el depósito después de $t$ horas.

\begin{solucion}
$A = -25t + 500$

Donde $m = -25$ (tasa de drenaje, negativa porque disminuye) y $b = 500$ (cantidad inicial).
\end{solucion}

\problem Una planta de alquiler de autos cobra \$40 por día más \$0.25 por cada kilómetro recorrido. Si el costo total por un día fue de \$65, ¿cuántos kilómetros se recorrieron?

\begin{solucion}
Ecuación: $C = 0.25k + 40$

Sustituir $C = 65$:

$65 = 0.25k + 40$

$25 = 0.25k$

$k = 100$

Se recorrieron 100 kilómetros.
\end{solucion}
\end{exercise}


% Conditional solution inclusion
\ifshowsolutions
    \newpage
    \section*{Soluciones}
    %========================================
% DETAILED SOLUTIONS: Ecuaciones Lineales en Dos Variables
%========================================

\subsection*{Ejercicio 1: Encontrar Soluciones}

\textbf{Problema 1:} $x + y = 5$

Para $x = 0$: $0 + y = 5 \implies y = 5$. Punto: $(0, 5)$

Para $x = 2$: $2 + y = 5 \implies y = 3$. Punto: $(2, 3)$

Para $y = 0$: $x + 0 = 5 \implies x = 5$. Punto: $(5, 0)$

Para $x = -1$: $-1 + y = 5 \implies y = 6$. Punto: $(-1, 6)$

\medskip

\textbf{Problema 2:} $2x - y = 4$

Para $x = 0$: $2(0) - y = 4 \implies -y = 4 \implies y = -4$. Punto: $(0, -4)$

Para $x = 2$: $2(2) - y = 4 \implies 4 - y = 4 \implies y = 0$. Punto: $(2, 0)$

Para $y = 0$: $2x - 0 = 4 \implies 2x = 4 \implies x = 2$. Punto: $(2, 0)$

Para $x = -2$: $2(-2) - y = 4 \implies -4 - y = 4 \implies y = -8$. Punto: $(-2, -8)$

\medskip

\textbf{Problema 3:} $3x + 2y = 12$

Para $x = 0$: $3(0) + 2y = 12 \implies 2y = 12 \implies y = 6$. Punto: $(0, 6)$

Para $y = 0$: $3x + 2(0) = 12 \implies 3x = 12 \implies x = 4$. Punto: $(4, 0)$

Para $x = 2$: $3(2) + 2y = 12 \implies 6 + 2y = 12 \implies 2y = 6 \implies y = 3$. Punto: $(2, 3)$

Para $y = 3$: $3x + 2(3) = 12 \implies 3x + 6 = 12 \implies 3x = 6 \implies x = 2$. Punto: $(2, 3)$

\newpage

\subsection*{Ejercicio 2: Encontrar Intersecciones con los Ejes}

\textbf{Problema 1:} $2x + 3y = 6$

\textbf{Intersección con el eje $x$ (hacer $y = 0$):}
\begin{align*}
2x + 3(0) &= 6 \\
2x &= 6 \\
x &= 3
\end{align*}
Punto de intersección con el eje $x$: $(3, 0)$

\textbf{Intersección con el eje $y$ (hacer $x = 0$):}
\begin{align*}
2(0) + 3y &= 6 \\
3y &= 6 \\
y &= 2
\end{align*}
Punto de intersección con el eje $y$: $(0, 2)$

\medskip

\textbf{Problema 2:} $4x - y = 8$

\textbf{Intersección con el eje $x$:}
$$4x - 0 = 8 \implies x = 2$$
Punto: $(2, 0)$

\textbf{Intersección con el eje $y$:}
$$4(0) - y = 8 \implies y = -8$$
Punto: $(0, -8)$

\medskip

\textbf{Problema 3:} $x + 2y = 10$

\textbf{Intersección con el eje $x$:}
$$x + 2(0) = 10 \implies x = 10$$
Punto: $(10, 0)$

\textbf{Intersección con el eje $y$:}
$$0 + 2y = 10 \implies y = 5$$
Punto: $(0, 5)$

\medskip

\textbf{Problema 4:} $5x - 2y = 20$

\textbf{Intersección con el eje $x$:}
$$5x - 0 = 20 \implies x = 4$$
Punto: $(4, 0)$

\textbf{Intersección con el eje $y$:}
$$0 - 2y = 20 \implies y = -10$$
Punto: $(0, -10)$

\medskip

\textbf{Problema 5:} $3x + 4y = 24$

\textbf{Intersección con el eje $x$:}
$$3x + 0 = 24 \implies x = 8$$
Punto: $(8, 0)$

\textbf{Intersección con el eje $y$:}
$$0 + 4y = 24 \implies y = 6$$
Punto: $(0, 6)$

\medskip

\textbf{Problema 6:} $-2x + 5y = 10$

\textbf{Intersección con el eje $x$:}
$$-2x + 0 = 10 \implies x = -5$$
Punto: $(-5, 0)$

\textbf{Intersección con el eje $y$:}
$$0 + 5y = 10 \implies y = 2$$
Punto: $(0, 2)$

\newpage

\subsection*{Ejercicio 3: Calcular la Pendiente}

\textbf{Problema 1:} Pendiente entre $(1, 2)$ y $(3, 6)$

$$m = \frac{y_2 - y_1}{x_2 - x_1} = \frac{6 - 2}{3 - 1} = \frac{4}{2} = 2$$

\medskip

\textbf{Problema 2:} Pendiente entre $(0, 5)$ y $(4, 1)$

$$m = \frac{1 - 5}{4 - 0} = \frac{-4}{4} = -1$$

\medskip

\textbf{Problema 3:} Pendiente entre $(-2, 3)$ y $(4, -1)$

$$m = \frac{-1 - 3}{4 - (-2)} = \frac{-4}{6} = -\frac{2}{3}$$

\medskip

\textbf{Problema 4:} Pendiente entre $(5, 2)$ y $(5, 7)$

$$m = \frac{7 - 2}{5 - 5} = \frac{5}{0} = \text{indefinida}$$

Los puntos tienen la misma coordenada $x$, por lo que forman una \textbf{recta vertical} con pendiente indefinida.

\medskip

\textbf{Problema 5:} Pendiente entre $(-3, 4)$ y $(2, 4)$

$$m = \frac{4 - 4}{2 - (-3)} = \frac{0}{5} = 0$$

Los puntos tienen la misma coordenada $y$, por lo que forman una \textbf{recta horizontal} con pendiente 0.

\medskip

\textbf{Problema 6:} Pendiente entre $(-1, -2)$ y $(3, 10)$

$$m = \frac{10 - (-2)}{3 - (-1)} = \frac{12}{4} = 3$$

\medskip

\textbf{Problema 7:} Pendiente entre $(6, 1)$ y $(2, 5)$

$$m = \frac{5 - 1}{2 - 6} = \frac{4}{-4} = -1$$

\medskip

\textbf{Problema 8:} Pendiente entre $(0, 0)$ y $(4, 8)$

$$m = \frac{8 - 0}{4 - 0} = \frac{8}{4} = 2$$

\newpage

\subsection*{Ejercicio 4: Identificar el Tipo de Pendiente}

\textbf{Problema 1:} Una recta que sube de izquierda a derecha.

Cuando una recta sube de izquierda a derecha, el cambio vertical y el cambio horizontal tienen el mismo signo (ambos positivos o ambos negativos). Por lo tanto, la pendiente es \textbf{positiva}.

\medskip

\textbf{Problema 2:} $y = -3$

Esta es la ecuación de una recta horizontal (todos los puntos tienen $y = -3$). Las rectas horizontales tienen \textbf{pendiente cero}.

\medskip

\textbf{Problema 3:} $x = 5$

Esta es la ecuación de una recta vertical (todos los puntos tienen $x = 5$). Las rectas verticales tienen \textbf{pendiente indefinida}.

\medskip

\textbf{Problema 4:} Una recta que pasa por $(1, 5)$ y $(4, 2)$.

Calculamos la pendiente:
$$m = \frac{2 - 5}{4 - 1} = \frac{-3}{3} = -1$$

La pendiente es \textbf{negativa}. La recta baja de izquierda a derecha.

\medskip

\textbf{Problema 5:} $y = 7x + 2$

Esta ecuación está en forma pendiente-intersecto donde $m = 7$. La pendiente es \textbf{positiva}.

\medskip

\textbf{Problema 6:} $y = -\frac{1}{2}x + 3$

Esta ecuación está en forma pendiente-intersecto donde $m = -\frac{1}{2}$. La pendiente es \textbf{negativa}.

\newpage

\subsection*{Ejercicio 5: Forma Pendiente-Intersecto}

\textbf{Problema 1:} $y = 3x + 5$

Comparando con $y = mx + b$:
\begin{itemize}
    \item Pendiente: $m = 3$
    \item Intersección con el eje $y$: $b = 5$
    \item La recta pasa por el punto $(0, 5)$
\end{itemize}

\medskip

\textbf{Problema 2:} $y = -2x - 1$

\begin{itemize}
    \item Pendiente: $m = -2$
    \item Intersección con el eje $y$: $b = -1$
    \item La recta pasa por el punto $(0, -1)$
\end{itemize}

\medskip

\textbf{Problema 3:} $y = \frac{1}{2}x + 4$

\begin{itemize}
    \item Pendiente: $m = \frac{1}{2}$
    \item Intersección con el eje $y$: $b = 4$
    \item La recta pasa por el punto $(0, 4)$
\end{itemize}

\medskip

\textbf{Problema 4:} $y = -x + 7$

\begin{itemize}
    \item Pendiente: $m = -1$ (el coeficiente implícito de $x$)
    \item Intersección con el eje $y$: $b = 7$
    \item La recta pasa por el punto $(0, 7)$
\end{itemize}

\medskip

\textbf{Problema 5:} $y = 4x$

Esta ecuación se puede escribir como $y = 4x + 0$:
\begin{itemize}
    \item Pendiente: $m = 4$
    \item Intersección con el eje $y$: $b = 0$
    \item La recta pasa por el origen $(0, 0)$
\end{itemize}

\medskip

\textbf{Problema 6:} $y = -\frac{3}{4}x - 2$

\begin{itemize}
    \item Pendiente: $m = -\frac{3}{4}$
    \item Intersección con el eje $y$: $b = -2$
    \item La recta pasa por el punto $(0, -2)$
\end{itemize}

\newpage

\subsection*{Ejercicio 6: Convertir a Forma Pendiente-Intersecto}

\textbf{Problema 1:} $2x + y = 6$

Despejamos $y$:
\begin{align*}
y &= -2x + 6
\end{align*}

Pendiente: $m = -2$; Intersección $y$: $b = 6$

\medskip

\textbf{Problema 2:} $3x - y = 9$

\begin{align*}
-y &= -3x + 9 \\
y &= 3x - 9
\end{align*}

Pendiente: $m = 3$; Intersección $y$: $b = -9$

\medskip

\textbf{Problema 3:} $4x + 2y = 8$

\begin{align*}
2y &= -4x + 8 \\
y &= \frac{-4x + 8}{2} \\
y &= -2x + 4
\end{align*}

Pendiente: $m = -2$; Intersección $y$: $b = 4$

\medskip

\textbf{Problema 4:} $x - 2y = 10$

\begin{align*}
-2y &= -x + 10 \\
y &= \frac{-x + 10}{-2} \\
y &= \frac{x}{2} - 5 \\
y &= \frac{1}{2}x - 5
\end{align*}

Pendiente: $m = \frac{1}{2}$; Intersección $y$: $b = -5$

\medskip

\textbf{Problema 5:} $6x + 3y = 12$

\begin{align*}
3y &= -6x + 12 \\
y &= \frac{-6x + 12}{3} \\
y &= -2x + 4
\end{align*}

Pendiente: $m = -2$; Intersección $y$: $b = 4$

\medskip

\textbf{Problema 6:} $5x - 2y = 20$

\begin{align*}
-2y &= -5x + 20 \\
y &= \frac{-5x + 20}{-2} \\
y &= \frac{5x}{2} - 10 \\
y &= \frac{5}{2}x - 10
\end{align*}

Pendiente: $m = \frac{5}{2}$; Intersección $y$: $b = -10$

\newpage

\subsection*{Ejercicio 7: Escribir Ecuaciones dada la Pendiente y un Punto}

\textbf{Problema 1:} Pendiente $m = 2$, pasa por $(1, 3)$

Usamos la forma punto-pendiente:
\begin{align*}
y - y_1 &= m(x - x_1) \\
y - 3 &= 2(x - 1) \\
y - 3 &= 2x - 2 \\
y &= 2x + 1
\end{align*}

\textbf{Respuesta:} $y = 2x + 1$

\medskip

\textbf{Problema 2:} Pendiente $m = -3$, pasa por $(2, 5)$

\begin{align*}
y - 5 &= -3(x - 2) \\
y - 5 &= -3x + 6 \\
y &= -3x + 11
\end{align*}

\textbf{Respuesta:} $y = -3x + 11$

\medskip

\textbf{Problema 3:} Pendiente $m = \frac{1}{2}$, pasa por $(4, 1)$

\begin{align*}
y - 1 &= \frac{1}{2}(x - 4) \\
y - 1 &= \frac{1}{2}x - 2 \\
y &= \frac{1}{2}x - 1
\end{align*}

\textbf{Respuesta:} $y = \frac{1}{2}x - 1$

\medskip

\textbf{Problema 4:} Pendiente $m = -\frac{2}{3}$, pasa por $(-3, 4)$

\begin{align*}
y - 4 &= -\frac{2}{3}(x - (-3)) \\
y - 4 &= -\frac{2}{3}(x + 3) \\
y - 4 &= -\frac{2}{3}x - 2 \\
y &= -\frac{2}{3}x + 2
\end{align*}

\textbf{Respuesta:} $y = -\frac{2}{3}x + 2$

\medskip

\textbf{Problema 5:} Pendiente $m = 0$, pasa por $(5, -2)$

Una recta con pendiente 0 es horizontal. Todos los puntos tienen la misma coordenada $y$.

\textbf{Respuesta:} $y = -2$

\medskip

\textbf{Problema 6:} Pendiente indefinida, pasa por $(3, 7)$

Una recta con pendiente indefinida es vertical. Todos los puntos tienen la misma coordenada $x$.

\textbf{Respuesta:} $x = 3$

\newpage

\subsection*{Ejercicio 8: Escribir Ecuaciones dados Dos Puntos}

\textbf{Problema 1:} $(1, 2)$ y $(3, 8)$

\textbf{Paso 1:} Calcular la pendiente
$$m = \frac{8 - 2}{3 - 1} = \frac{6}{2} = 3$$

\textbf{Paso 2:} Usar forma punto-pendiente con $(1, 2)$
\begin{align*}
y - 2 &= 3(x - 1) \\
y - 2 &= 3x - 3 \\
y &= 3x - 1
\end{align*}

\textbf{Respuesta:} $y = 3x - 1$

\medskip

\textbf{Problema 2:} $(0, 4)$ y $(2, 0)$

\textbf{Paso 1:} Calcular la pendiente
$$m = \frac{0 - 4}{2 - 0} = \frac{-4}{2} = -2$$

\textbf{Paso 2:} Usar forma punto-pendiente con $(0, 4)$
\begin{align*}
y - 4 &= -2(x - 0) \\
y - 4 &= -2x \\
y &= -2x + 4
\end{align*}

\textbf{Nota:} Como uno de los puntos es $(0, 4)$, podemos identificar directamente que $b = 4$.

\textbf{Respuesta:} $y = -2x + 4$

\medskip

\textbf{Problema 3:} $(-1, 3)$ y $(2, -6)$

\textbf{Paso 1:} Calcular la pendiente
$$m = \frac{-6 - 3}{2 - (-1)} = \frac{-9}{3} = -3$$

\textbf{Paso 2:} Usar forma punto-pendiente con $(-1, 3)$
\begin{align*}
y - 3 &= -3(x - (-1)) \\
y - 3 &= -3(x + 1) \\
y - 3 &= -3x - 3 \\
y &= -3x
\end{align*}

\textbf{Respuesta:} $y = -3x$

\medskip

\textbf{Problema 4:} $(4, 1)$ y $(6, 5)$

\textbf{Paso 1:} Calcular la pendiente
$$m = \frac{5 - 1}{6 - 4} = \frac{4}{2} = 2$$

\textbf{Paso 2:} Usar forma punto-pendiente con $(4, 1)$
\begin{align*}
y - 1 &= 2(x - 4) \\
y - 1 &= 2x - 8 \\
y &= 2x - 7
\end{align*}

\textbf{Respuesta:} $y = 2x - 7$

\medskip

\textbf{Problema 5:} $(-2, -1)$ y $(3, -1)$

\textbf{Paso 1:} Calcular la pendiente
$$m = \frac{-1 - (-1)}{3 - (-2)} = \frac{0}{5} = 0$$

Como la pendiente es 0, la recta es horizontal. Ambos puntos tienen $y = -1$.

\textbf{Respuesta:} $y = -1$

\medskip

\textbf{Problema 6:} $(5, 2)$ y $(5, 8)$

\textbf{Paso 1:} Calcular la pendiente
$$m = \frac{8 - 2}{5 - 5} = \frac{6}{0} = \text{indefinida}$$

Como la pendiente es indefinida, la recta es vertical. Ambos puntos tienen $x = 5$.

\textbf{Respuesta:} $x = 5$

\newpage

\subsection*{Ejercicio 9: Rectas Paralelas y Perpendiculares}

\textbf{Problema 1:} Ecuación de la recta que pasa por $(2, 3)$ y es paralela a $y = 4x - 1$

Las rectas paralelas tienen la misma pendiente. La pendiente de $y = 4x - 1$ es $m = 4$.

Usando forma punto-pendiente con $(2, 3)$ y $m = 4$:
\begin{align*}
y - 3 &= 4(x - 2) \\
y - 3 &= 4x - 8 \\
y &= 4x - 5
\end{align*}

\textbf{Respuesta:} $y = 4x - 5$

\medskip

\textbf{Problema 2:} Ecuación de la recta que pasa por $(1, 5)$ y es perpendicular a $y = 2x + 3$

Las rectas perpendiculares tienen pendientes que son recíprocas negativas.

Pendiente de la recta dada: $m_1 = 2$

Pendiente de la recta perpendicular: $m_2 = -\frac{1}{2}$

Usando forma punto-pendiente con $(1, 5)$ y $m = -\frac{1}{2}$:
\begin{align*}
y - 5 &= -\frac{1}{2}(x - 1) \\
y - 5 &= -\frac{1}{2}x + \frac{1}{2} \\
y &= -\frac{1}{2}x + \frac{1}{2} + 5 \\
y &= -\frac{1}{2}x + \frac{11}{2}
\end{align*}

\textbf{Respuesta:} $y = -\frac{1}{2}x + \frac{11}{2}$

\medskip

\textbf{Problema 3:} ¿Son $y = 3x + 2$ y $y = 3x - 7$ paralelas, perpendiculares o ninguna?

Pendientes:
\begin{itemize}
    \item Primera recta: $m_1 = 3$
    \item Segunda recta: $m_2 = 3$
\end{itemize}

Como $m_1 = m_2$, las rectas son \textbf{paralelas}.

\medskip

\textbf{Problema 4:} ¿Son $y = \frac{2}{3}x + 1$ y $y = -\frac{3}{2}x + 4$ paralelas, perpendiculares o ninguna?

Pendientes:
\begin{itemize}
    \item Primera recta: $m_1 = \frac{2}{3}$
    \item Segunda recta: $m_2 = -\frac{3}{2}$
\end{itemize}

Verificamos el producto:
$$m_1 \cdot m_2 = \frac{2}{3} \cdot \left(-\frac{3}{2}\right) = -\frac{6}{6} = -1$$

Como el producto de las pendientes es $-1$, las rectas son \textbf{perpendiculares}.

\medskip

\textbf{Problema 5:} Ecuación de la recta que pasa por $(-1, 2)$ y es perpendicular a $3x + 2y = 6$

\textbf{Paso 1:} Convertir la ecuación dada a forma pendiente-intersecto para encontrar su pendiente
\begin{align*}
3x + 2y &= 6 \\
2y &= -3x + 6 \\
y &= -\frac{3}{2}x + 3
\end{align*}

Pendiente de la recta dada: $m_1 = -\frac{3}{2}$

\textbf{Paso 2:} Calcular la pendiente perpendicular
$$m_2 = -\frac{1}{m_1} = -\frac{1}{-\frac{3}{2}} = \frac{2}{3}$$

\textbf{Paso 3:} Usar forma punto-pendiente con $(-1, 2)$ y $m = \frac{2}{3}$
\begin{align*}
y - 2 &= \frac{2}{3}(x - (-1)) \\
y - 2 &= \frac{2}{3}(x + 1) \\
y - 2 &= \frac{2}{3}x + \frac{2}{3} \\
y &= \frac{2}{3}x + \frac{2}{3} + 2 \\
y &= \frac{2}{3}x + \frac{8}{3}
\end{align*}

\textbf{Respuesta:} $y = \frac{2}{3}x + \frac{8}{3}$

\newpage

\subsection*{Ejercicio 10: Problemas de Aplicación}

\textbf{Problema 1:} Tarifa de taxi

El costo total $C$ incluye una tarifa base de \$3 más \$0.50 por cada milla.

Identificamos:
\begin{itemize}
    \item Pendiente (costo por milla): $m = 0.50$
    \item Intersección $y$ (tarifa base): $b = 3$
\end{itemize}

\textbf{Ecuación:} $C = 0.50m + 3$ o $C = 0.5m + 3$

donde $C$ = costo total en dólares y $m$ = número de millas.

\medskip

\textbf{Problema 2:} Conversión de temperatura

Dado $F = \frac{9}{5}C + 32$ y $C = 20$, encontramos $F$:

\begin{align*}
F &= \frac{9}{5}(20) + 32 \\
F &= \frac{180}{5} + 32 \\
F &= 36 + 32 \\
F &= 68
\end{align*}

\textbf{Respuesta:} La temperatura es 68°F.

\medskip

\textbf{Problema 3:} Drenaje del depósito de agua

\begin{itemize}
    \item Cantidad inicial: 500 galones
    \item Tasa de drenaje: 25 galones por hora (negativa porque disminuye)
    \item Pendiente: $m = -25$
    \item Intersección $y$: $b = 500$
\end{itemize}

\textbf{Ecuación:} $A = -25t + 500$

donde $A$ = cantidad de agua (galones) y $t$ = tiempo (horas).

\textbf{Verificación:}
\begin{itemize}
    \item Cuando $t = 0$: $A = -25(0) + 500 = 500$ galones \checkmark
    \item Cuando $t = 10$: $A = -25(10) + 500 = 250$ galones
    \item Cuando $t = 20$: $A = -25(20) + 500 = 0$ galones (depósito vacío)
\end{itemize}

\medskip

\textbf{Problema 4:} Alquiler de autos

\textbf{Ecuación del costo:} $C = 0.25k + 40$

donde $C$ = costo total, $k$ = kilómetros recorridos.

\textbf{Dado:} $C = 65$. Encontrar $k$.

\begin{align*}
65 &= 0.25k + 40 \\
65 - 40 &= 0.25k \\
25 &= 0.25k \\
k &= \frac{25}{0.25} \\
k &= 100
\end{align*}

\textbf{Respuesta:} Se recorrieron 100 kilómetros.

\textbf{Verificación:} $C = 0.25(100) + 40 = 25 + 40 = 65$ \checkmark

\fi

\end{document}
