\documentclass[12pt]{article}

%========================================
% PACKAGES AND CONFIGURATION
%========================================
\usepackage[utf8]{inputenc}
\usepackage[spanish]{babel}              % Spanish language support
\decimalpoint                                % Force decimal point instead of comma
\usepackage[margin=1in]{geometry}
\setlength{\headheight}{15pt}
\usepackage{amsmath, amsthm, amssymb}
\usepackage{mdframed}
\usepackage{xcolor}
\usepackage{enumitem}
\usepackage{fancyhdr}
\usepackage{graphicx}
\usepackage{tikz}                        % For LaTeX-generated diagrams
\usetikzlibrary{arrows.meta}           % For arrow styles
\usetikzlibrary{shapes}                % For diamond and other shapes
\usetikzlibrary{decorations.pathreplacing} % For braces and decorations
\usetikzlibrary{calc}                  % For coordinate calculations
\usepackage{comment}                     % For conditional content exclusion

% Fix Spanish babel conflicts with TikZ
\usetikzlibrary{babel}

%========================================
% COURSE CUSTOMIZATION SECTION
%========================================
% MODIFY THESE FOR EACH COURSE:
\newcommand{\coursecode}{MATE 3001}      % Course code
\newcommand{\coursename}{Matemática Elemental}     % Course name
\newcommand{\institution}{UPR-Humacao}   % Institution name
\newcommand{\lessontitle}{Desigualdades Lineales}    % Will be overridden per lesson

%========================================
% COLOR SCHEME DEFINITIONS
%========================================
\definecolor{defcolor}{RGB}{240,248,255}     % Light blue for definitions
\definecolor{examplecolor}{RGB}{245,255,245} % Light green for examples
\definecolor{exercisecolor}{RGB}{255,248,240} % Light orange for exercises
\definecolor{theoremcolor}{RGB}{255,250,240}  % Light orange for theorems
\definecolor{warningcolor}{RGB}{255,240,240}  % Light red for warnings

%========================================
% CUSTOM ENVIRONMENTS
%========================================
% Definition Environment (Blue)
\newmdenv[
    backgroundcolor=defcolor,
    linecolor=blue!50,
    linewidth=2pt,
    leftmargin=10pt,
    rightmargin=10pt,
    innertopmargin=10pt,
    innerbottommargin=10pt,
    frametitle={\textbf{Definición}},
    frametitlealignment=\raggedright
]{definition}

% Example Environment (Green)
\newmdenv[
    backgroundcolor=examplecolor,
    linecolor=green!50,
    linewidth=2pt,
    leftmargin=10pt,
    rightmargin=10pt,
    innertopmargin=10pt,
    innerbottommargin=10pt,
    frametitle={\textbf{Ejemplo}},
    frametitlealignment=\raggedright
]{example}

% Exercise Environment (Orange)
\newcounter{exercise}[section]
\newcounter{problem}[exercise]
\newmdenv[
    backgroundcolor=exercisecolor,
    linecolor=orange!50,
    linewidth=2pt,
    leftmargin=10pt,
    rightmargin=10pt,
    innertopmargin=10pt,
    innerbottommargin=10pt,
    frametitle={\stepcounter{exercise}\textbf{Ejercicio \theexercise}},
    frametitlealignment=\raggedright
]{exercise}

% Theorem Environment (Orange variant)
\newmdenv[
    backgroundcolor=theoremcolor,
    linecolor=orange!50,
    linewidth=2pt,
    leftmargin=10pt,
    rightmargin=10pt,
    innertopmargin=10pt,
    innerbottommargin=10pt
]{theorem}

% Warning Environment (Red)
\newmdenv[
    backgroundcolor=warningcolor,
    linecolor=red!50,
    linewidth=2pt,
    leftmargin=10pt,
    rightmargin=10pt,
    innertopmargin=10pt,
    innerbottommargin=10pt,
    frametitle={\textbf{¡Atención!}},
    frametitlealignment=\raggedright
]{warning}

%========================================
% HEADER AND FOOTER CONFIGURATION
%========================================
\pagestyle{fancy}
\fancyhf{}
\rhead{\coursecode\ - \coursename}
\lhead{\lessontitle}
\cfoot{\thepage}

%========================================
% CUSTOM COMMANDS
%========================================
\newcommand{\lesson}[1]{\renewcommand{\lessontitle}{#1}\section{#1}}
\newcommand{\subsectiontitle}[1]{\subsection{#1}}

% Exercise numbering commands
\newcommand{\problem}{\par\noindent\stepcounter{problem}\textbf{\theproblem.} }
\newcommand{\solution}{\textbf{Solución:} }

% Custom environment for exercise lists
\newenvironment{exerciselist}
    {\begin{enumerate}[label=\textbf{\alph*.}]}
    {\end{enumerate}}

% Solution environment with conditional display
\newif\ifshowsolutions
% \showsolutionstrue  % Uncomment for instructor version
\showsolutionsfalse   % Default: student version

\ifshowsolutions
    \newenvironment{solucion}[1][Solución]
      {\par\medskip\noindent\textbf{#1:}\par\nopagebreak}
      {\par\medskip}
\else
    \excludecomment{solucion}
\fi

%========================================
% GRAPHICS CONFIGURATION
%========================================
\graphicspath{{../images/}{../images/shared/}{../images/diagrams/}{../images/13_desigualdades/}}

%========================================
% DOCUMENT CONTENT
%========================================
\begin{document}

% Title page
\title{\lessontitle}
\author{\coursecode\ - \coursename}
\date{}
\maketitle

% Set section counter to lesson number
\setcounter{section}{12}

% Modular content inclusion
%========================================
% LESSON CONTENT: Desigualdades Lineales
%========================================

\lesson{Desigualdades Lineales}

En esta lección introducimos las desigualdades lineales en una variable, sus reglas de manipulación, la resolución paso a paso y la representación del conjunto solución con notación de intervalos y en la recta real. También vemos desigualdades compuestas del tipo ``y'' (intersección) y ``o'' (unión), y practicamos con ejercicios.

\subsectiontitle{Conceptos básicos}

\begin{definition}
Una \textbf{desigualdad} es una expresión matemática que compara dos cantidades usando los símbolos:
\begin{itemize}
    \item $<$ (menor que)
    \item $>$ (mayor que)
    \item $\le$ (menor o igual que)
    \item $\ge$ (mayor o igual que)
\end{itemize}

Una \textbf{solución} de una desigualdad es todo valor de la variable que hace verdadera la desigualdad.

El \textbf{conjunto solución} de una desigualdad generalmente es un intervalo o unión de intervalos en $\mathbb{R}$.
\end{definition}

\begin{example}
\textbf{Diferencia entre ecuación y desigualdad:}

\begin{itemize}
    \item \textbf{Ecuación:} $4x - 7 = 19$ tiene solución única $x = \dfrac{26}{4} = \dfrac{13}{2}$.

    \item \textbf{Desigualdad:} $4x - 7 \le 19$ tiene soluciones $x \le \dfrac{26}{4} = \dfrac{13}{2}$, que es un intervalo $\left(-\infty, \dfrac{13}{2}\right]$.
\end{itemize}

\textbf{Representación gráfica:}

\begin{center}
\begin{tikzpicture}[scale=0.8]
    % Equation solution (single point)
    \draw[thick, <->] (-2,2) -- (8,2);
    \foreach \x in {0,2,4,6}
        \draw (\x,2.1) -- (\x,1.9) node[below] {\x};
    \node at (-2.5,2) [left] {Ecuación:};
    \filldraw[blue] (6.5,2) circle (3pt) node[above] {$x = \frac{13}{2}$};

    % Inequality solution (interval)
    \draw[thick, <->] (-2,0) -- (8,0);
    \foreach \x in {0,2,4,6}
        \draw (\x,0.1) -- (\x,-0.1) node[below] {\x};
    \node at (-2.5,0) [left] {Desigualdad:};
    \draw[blue, line width=3pt] (-1.9,0) -- (6.5,0);
    \filldraw[blue] (6.5,0) circle (3pt) node[above] {$\frac{13}{2}$};
    \draw[blue, <-] (-2,0) -- (-1.5,0);
\end{tikzpicture}
\end{center}

\vspace{0.3cm}
\textbf{Notación de intervalos:}

La solución $x \le \dfrac{13}{2}$ se escribe en notación de intervalos como $\left(-\infty, \dfrac{13}{2}\right]$.

\begin{itemize}
    \item Paréntesis ( o ) indica que el extremo \textbf{no está incluido} (desigualdad estricta: $<$ o $>$)
    \item Corchete [ o ] indica que el extremo \textbf{está incluido} (desigualdad no estricta: $\le$ o $\ge$)
    \item El infinito ($\infty$ o $-\infty$) siempre lleva paréntesis
\end{itemize}

\textbf{Tabla de referencia:}

\begin{center}
\begin{tabular}{|c|c|c|}
\hline
\textbf{Desigualdad} & \textbf{Notación de intervalos} & \textbf{Representación gráfica} \\
\hline
$x < a$ & $(-\infty, a)$ & \begin{tikzpicture}[scale=0.4, baseline=0.1cm]
    \draw[<->] (-2,0) -- (2,0);
    \draw[blue, line width=2pt] (-2,0) -- (1,0);
    \draw[blue] (1,0) circle (2pt);
    \filldraw (1,0.1) -- (1,-0.1) node[below] {\tiny $a$};
\end{tikzpicture} \\
\hline
$x \le a$ & $(-\infty, a]$ & \begin{tikzpicture}[scale=0.4, baseline=0.5cm]
    \draw[<->] (-2,0) -- (2,0);
    \draw[blue, line width=2pt] (-2,0) -- (1,0);
    \filldraw[blue] (1,0) circle (2pt);
    \filldraw (1,0.1) -- (1,-0.1) node[below] {\tiny $a$};
\end{tikzpicture} \\
\hline
$x > a$ & $(a, \infty)$ & \begin{tikzpicture}[scale=0.4, baseline=0.5cm]
    \draw[<->] (-2,0) -- (2,0);
    \draw[blue, line width=2pt] (1,0) -- (2,0);
    \draw[blue] (1,0) circle (2pt);
    \filldraw (1,0.1) -- (1,-0.1) node[below] {\tiny $a$};
\end{tikzpicture} \\
\hline
$x \ge a$ & $[a, \infty)$ & \begin{tikzpicture}[scale=0.4, baseline=0.5cm]
    \draw[<->] (-2,0) -- (2,0);
    \draw[blue, line width=2pt] (1,0) -- (2,0);
    \filldraw[blue] (1,0) circle (2pt);
    \filldraw (1,0.1) -- (1,-0.1) node[below] {\tiny $a$};
\end{tikzpicture} \\
\hline
$a < x < b$ & $(a, b)$ & \begin{tikzpicture}[scale=0.4, baseline=0.5cm]
    \draw[<->] (-2,0) -- (2,0);
    \draw[blue, line width=2pt] (-0.5,0) -- (1.5,0);
    \draw[blue] (-0.5,0) circle (2pt);
    \draw[blue] (1.5,0) circle (2pt);
    \filldraw (-0.5,0.1) -- (-0.5,-0.1) node[below] {\tiny $a$};
    \filldraw (1.5,0.1) -- (1.5,-0.1) node[below] {\tiny $b$};
\end{tikzpicture} \\
\hline
\end{tabular}
\end{center}
\end{example}

\newpage

\subsectiontitle{Reglas para desigualdades}

\begin{theorem}
Sea $A, B, C \in \mathbb{R}$.

\textbf{Regla 1 (Suma):} Si $A \le B$, entonces $A + C \le B + C$.

\textbf{Regla 2 (Resta):} Si $A \le B$, entonces $A - C \le B - C$.

\textbf{Regla 3 (Multiplicación/División por positivo):} Si $C > 0$ y $A \le B$, entonces
\[CA \le CB \quad \text{y} \quad \dfrac{A}{C} \le \dfrac{B}{C}.\]

\textbf{Regla 4 (Multiplicación/División por negativo):} Si $C < 0$ y $A \le B$, entonces
\[CA \ge CB \quad \text{y} \quad \dfrac{A}{C} \ge \dfrac{B}{C}.\]

\footnotesize
\begin{center}
\textcolor{red}{\textbf{¡Se invierte la desigualdad!}}
\end{center}
\normalsize

\textbf{Regla 5 (Recíprocos positivos):} Si $0 < A \le B$, entonces
\[\dfrac{1}{A} \ge \dfrac{1}{B}.\]

\footnotesize
\begin{center}
	\textcolor{red}{\textbf{¡Se invierte la desigualdad!}}
\end{center}
\normalsize


\textbf{Regla 6 (Suma de desigualdades):} Si $A \le B$ y $C \le D$, entonces
\[A + C \le B + D.\]
\end{theorem}

\begin{warning}
\textbf{Punto clave:} Al multiplicar o dividir ambos lados de una desigualdad por un número negativo, se debe invertir la dirección de la desigualdad.

\textbf{Ejemplo:}
\begin{align*}
-2x &> 6 \\
x &< -3 \quad \text{(dividimos entre $-2 < 0$ e invertimos la desigualdad)}
\end{align*}

Este es el error más común al resolver desigualdades. ¡Tenga cuidado!
\end{warning}

\newpage

\subsectiontitle{Resolución de desigualdades lineales (una variable)}

La idea es aislar la variable aplicando las reglas anteriores, cuidando el cambio de sentido al multiplicar o dividir por números negativos.

\begin{example}
\textbf{Ejemplo 1: Desigualdad lineal sencilla}

Resuelva y grafique el conjunto solución de $4x - 7 \le 19$.

\solution
\begin{align*}
4x - 7 &\le 19 \\
4x &\le 26 \quad \text{(sumamos 7 a ambos lados)} \\
x &\le \dfrac{26}{4} \quad \text{(dividimos entre 4 > 0, sin cambiar el sentido)} \\
x &\le \dfrac{13}{2}
\end{align*}

\textbf{Solución en notación de intervalos:} $\left(-\infty, \dfrac{13}{2}\right]$

\textbf{Representación gráfica:}

\begin{center}
\begin{tikzpicture}[scale=1.2]
    \draw[thick, <->] (-2,0) -- (8,0);
    \foreach \x in {-1,0,1,2,3,4,5,6,7}
        \draw (\x,0.1) -- (\x,-0.1) node[below] {\small $\x$};

    % Solution: x <= 13/2 = 6.5
    \draw[blue, line width=4pt] (-1.9,0) -- (6.5,0);
    \filldraw[blue] (6.5,0) circle (3pt) node[above=5pt] {$\frac{13}{2} = 6.5$};

\end{tikzpicture}
\end{center}

\textbf{Interpretación:} El círculo relleno en $\dfrac{13}{2}$ indica que este valor está incluido en la solución ($\le$). La línea azul hacia la izquierda representa todos los números menores o iguales a $\dfrac{13}{2}$.
\end{example}

\newpage
\begin{example}
\textbf{Ejemplo 2: Desigualdad con coeficiente negativo}

Resuelva $-3x + 5 > 11$ y exprese en notación de intervalos.

\solution
\begin{align*}
-3x + 5 &> 11 \\
-3x &> 6 \quad \text{(restamos 5 de ambos lados)} \\
x &< -2 \quad \textcolor{red}{\textbf{(dividimos entre $-3 < 0$ e invertimos la desigualdad)}}
\end{align*}

\textbf{Solución en notación de intervalos:} $(-\infty, -2)$

\textbf{Representación gráfica:}

\begin{center}
\begin{tikzpicture}[scale=1.2]
    \draw[thick, <->] (-6,0) -- (2,0);
    \foreach \x in {-5,-4,-3,-2,-1,0,1}
        \draw (\x,0.1) -- (\x,-0.1) node[below] {\small $\x$};

    % Solution: x < -2
    \draw[blue, line width=4pt] (-5.9,0) -- (-2,0);
    \draw[blue] (-2,0) circle (3pt);
    \filldraw[white] (-2,0) circle (2.5pt);
    \draw[blue, <-] (-6,0) -- (-5.5,0);
    \node at (-2,0.6) {$-2$};
\end{tikzpicture}
\end{center}

\textbf{Interpretación:} El círculo abierto en $-2$ indica que este valor \textbf{no está incluido} en la solución ($<$). La línea azul representa todos los números menores que $-2$.
\end{example}


\begin{example}
\textbf{Ejemplo 3: Desigualdad con fracciones}

Resuelva $\dfrac{2x - 3}{4} \ge \dfrac{x + 1}{2}$.

\solution

Multiplicamos ambos lados por 4 (que es positivo, por lo que no cambia el sentido):
\begin{align*}
\dfrac{2x - 3}{4} &\ge \dfrac{x + 1}{2} \\
4 \cdot \dfrac{2x - 3}{4} &\ge 4 \cdot \dfrac{x + 1}{2} \\
2x - 3 &\ge 2(x + 1) \\
2x - 3 &\ge 2x + 2 \\
-3 &\ge 2 \quad \text{(falso para todo $x$)}
\end{align*}

\textbf{Solución:} No hay solución. El conjunto solución es $\varnothing$ (conjunto vacío).

\textbf{Representación gráfica:} No hay puntos en la recta real que satisfagan la desigualdad.
\end{example}

\begin{example}
\textbf{Ejemplo 4: Multiplicación por número negativo}

Considere la desigualdad: $-6x + 4 < 0$

\solution
\begin{align*}
-6x + 4 &< 0 \\
-6x &< -4 \quad \text{(restamos 4)} \\
x &> \dfrac{-4}{-6} \quad \textcolor{red}{\textbf{(dividimos entre $-6 < 0$ e invertimos)}} \\
x &> \dfrac{2}{3}
\end{align*}

\textbf{Solución en notación de intervalos:} $\left(\dfrac{2}{3}, \infty\right)$

\textbf{Representación gráfica:}

\begin{center}
\begin{tikzpicture}[scale=1.2]
    \draw[thick, <->] (-2,0) -- (6,0);
    \foreach \x in {-1,0,1,2,3,4,5}
        \draw (\x,0.1) -- (\x,-0.1) node[below] {\small $\x$};

    % Solution: x > 2/3 ≈ 0.67
    \draw[blue, line width=4pt] (0.67,0) -- (5.9,0);
    \draw[blue] (0.67,0) circle (3pt);
    \filldraw[white] (0.67,0) circle (2.5pt);
    \draw[blue, ->] (5.5,0) -- (6,0);
    \node at (0.67,0.6) {$\frac{2}{3}$};
\end{tikzpicture}
\end{center}
\end{example}

\newpage

\subsectiontitle{Desigualdades compuestas}

Hay dos formas comunes de desigualdades compuestas:

\begin{definition}
\textbf{Tipo ``y'' (intersección):} $a < \text{expresión} \le b$

Se resuelve aplicando operaciones a los tres lados a la vez. La solución es la intersección de ambas condiciones.

\textbf{Tipo ``o'' (unión):} $\text{expresión} < a \;\;\text{o}\;\; \text{expresión} \ge b$

Se resuelve cada parte por separado y se toma la unión de las soluciones.
\end{definition}

\begin{example}
\textbf{Ejemplo 5: Desigualdad compuesta tipo ``y'' (intersección)}

Resuelva $4 \le 3x - 2 < 13$.

\solution

Aplicamos las operaciones a las tres partes:
\begin{align*}
4 &\le 3x - 2 < 13 \\
4 + 2 &\le 3x - 2 + 2 < 13 + 2 \quad \text{(sumamos 2 a las tres partes)} \\
6 &\le 3x < 15 \\
\dfrac{6}{3} &\le \dfrac{3x}{3} < \dfrac{15}{3} \quad \text{(dividimos entre 3 > 0)} \\
2 &\le x < 5
\end{align*}

\textbf{Solución en notación de intervalos:} $[2, 5)$

\textbf{Representación gráfica:}

\begin{center}
\begin{tikzpicture}[scale=1.2]
    \draw[thick, <->] (-1,0) -- (7,0);
    \foreach \x in {0,1,2,3,4,5,6}
        \draw (\x,0.1) -- (\x,-0.1) node[below] {\small $\x$};

    % Solution: 2 <= x < 5
    \draw[blue, line width=4pt] (2,0) -- (5,0);
    \filldraw[blue] (2,0) circle (3pt) node[above=5pt] {$2$};
    \draw[blue] (5,0) circle (3pt);
    \filldraw[white] (5,0) circle (2.5pt);
    \node at (5,0.6) {$5$};
\end{tikzpicture}
\end{center}

\textbf{Interpretación:} El círculo relleno en 2 indica que está incluido ($\le$). El círculo abierto en 5 indica que no está incluido ($<$).
\end{example}

\newpage

\begin{example}
\textbf{Ejemplo 6: Desigualdad compuesta tipo ``o'' (unión)}

Resuelva $2x + 1 < -3 \;\;\text{o}\;\; 2x + 1 \ge 7$.

\solution

Resolvemos cada desigualdad por separado:

\textbf{Primera parte:}
\begin{align*}
2x + 1 &< -3 \\
2x &< -4 \\
x &< -2
\end{align*}

\textbf{Segunda parte:}
\begin{align*}
2x + 1 &\ge 7 \\
2x &\ge 6 \\
x &\ge 3
\end{align*}

La solución es la \textbf{unión} de ambas: $x < -2$ \textbf{o} $x \ge 3$.

\textbf{Solución en notación de intervalos:} $(-\infty, -2) \cup [3, \infty)$

\textbf{Representación gráfica:}

\begin{center}
\begin{tikzpicture}[scale=1.2]
    \draw[thick, <->] (-5,0) -- (6,0);
    \foreach \x in {-4,-3,-2,-1,0,1,2,3,4,5}
        \draw (\x,0.1) -- (\x,-0.1) node[below] {\small $\x$};

    % Solution: x < -2 or x >= 3
    \draw[blue, line width=4pt] (-5,0) -- (-2,0);
    \draw[blue] (-2,0) circle (3pt);
    \filldraw[white] (-2,0) circle (2.5pt);
    \node at (-2,0.6) {$-2$};

    \draw[blue, line width=4pt] (3,0) -- (6,0);
    \filldraw[blue] (3,0) circle (3pt) node[above=5pt] {$3$};

    \draw[blue, <-] (-5,0) -- (-4.5,0);
    \draw[blue, ->] (5.5,0) -- (6,0);
\end{tikzpicture}
\end{center}

\textbf{Interpretación:} La solución consiste en dos regiones separadas: todos los números menores que $-2$ (círculo abierto) y todos los números mayores o iguales a 3 (círculo relleno).
\end{example}

\newpage

\subsectiontitle{Modelado con desigualdades}

Las desigualdades se usan para modelar situaciones donde hay restricciones o límites.

\begin{example}
\textbf{Ejemplo 7: Modelado con palabras}

Un boleto de cine y un refrigerio cuestan a lo más \$18. El refrigerio cuesta \$5. Si $x$ representa el costo del boleto, ¿qué valores de $x$ son posibles?

\solution

\textbf{Traducción:} ``a lo más \$18'' significa ``menor o igual que \$18''.

La desigualdad es:
\[x + 5 \le 18\]

Resolviendo:
\begin{align*}
x + 5 &\le 18 \\
x &\le 13
\end{align*}

\textbf{Solución:} El boleto puede costar a lo más \$13. En notación de intervalos: $(0, 13]$ (asumiendo que el precio debe ser positivo).

\textbf{Representación gráfica:}

\begin{center}
\begin{tikzpicture}[scale=0.5]
    \draw[thick, ->] (-1,0) -- (16,0);
    \foreach \x in {0,2,4,6,8,10,12,14}
        \draw (\x,0.1) -- (\x,-0.1) node[below] {\small $\x$};

    % Solution: 0 < x <= 13
    \draw[blue, line width=4pt] (0,0) -- (13,0);
    \draw[blue] (0,0) circle (3pt);
    \filldraw[white] (0,0) circle (2.5pt);
    \filldraw[blue] (13,0) circle (3pt) node[above=5pt] {$13$};
\end{tikzpicture}
\end{center}
\end{example}

\vspace{1cm}

\textbf{Palabras clave en problemas de desigualdades:}

\begin{center}
\begin{tabular}{|l|c|}
\hline
\textbf{Frase} & \textbf{Símbolo} \\
\hline
``a lo más'', ``como máximo'', ``no más de'' & $\le$ \\
``al menos'', ``como mínimo'', ``no menos de'' & $\ge$ \\
``más que'', ``mayor que'' & $>$ \\
``menos que'', ``menor que'' & $<$ \\
``entre $a$ y $b$'' (incluyendo ambos) & $a \le x \le b$ \\
``entre $a$ y $b$'' (sin incluir extremos) & $a < x < b$ \\
\hline
\end{tabular}
\end{center}

%========================================
% EXERCISES: Desigualdades Lineales
%========================================
\newpage
\subsectiontitle{Ejercicios}

Resuelva cada desigualdad, exprese la solución en notación de intervalos y describa cómo representarla en la recta real.

\vspace{-0.2cm}

\subsection*{Lineales simples}

\begin{exercise}
\problem $2x - 7 \le 11$

\begin{solucion}
\begin{align*}
2x - 7 &\le 11 \\
2x &\le 18 \\
x &\le 9
\end{align*}

\textbf{Notación de intervalos:} $(-\infty, 9]$

\textbf{Representación gráfica:} En la recta real, se dibuja un círculo relleno en $x = 9$ (porque está incluido, $\le$) y se sombrea hacia la izquierda hasta $-\infty$.
\end{solucion}

\problem $4x > 10$

\begin{solucion}
\begin{align*}
4x &> 10 \\
x &> \dfrac{10}{4} \\
x &> \dfrac{5}{2}
\end{align*}

\textbf{Notación de intervalos:} $\left(\dfrac{5}{2}, \infty\right)$

\textbf{Representación gráfica:} En la recta real, se dibuja un círculo abierto en $x = \dfrac{5}{2} = 2.5$ (porque no está incluido, $>$) y se sombrea hacia la derecha hasta $\infty$.
\end{solucion}

\problem $3x - 5 \ge 11$

\begin{solucion}
\begin{align*}
3x - 5 &\ge 11 \\
3x &\ge 16 \\
x &\ge \dfrac{16}{3}
\end{align*}

\textbf{Notación de intervalos:} $\left[\dfrac{16}{3}, \infty\right)$

\textbf{Representación gráfica:} En la recta real, se dibuja un círculo relleno en $x = \dfrac{16}{3} \approx 5.33$ (porque está incluido, $\ge$) y se sombrea hacia la derecha hasta $\infty$.
\end{solucion}

\problem $5 - 3x < 16$

\begin{solucion}
\begin{align*}
5 - 3x &< 16 \\
-3x &< 11 \\
x &> -\dfrac{11}{3} \quad \textcolor{red}{\text{(dividimos entre $-3 < 0$ e invertimos)}}
\end{align*}

\textbf{Notación de intervalos:} $\left(-\dfrac{11}{3}, \infty\right)$

\textbf{Representación gráfica:} En la recta real, se dibuja un círculo abierto en $x = -\dfrac{11}{3} \approx -3.67$ (porque no está incluido, $>$) y se sombrea hacia la derecha hasta $\infty$.
\end{solucion}

\problem $\dfrac{x}{3} + 2 \le \dfrac{5}{6}$

\begin{solucion}
\begin{align*}
\dfrac{x}{3} + 2 &\le \dfrac{5}{6} \\
\dfrac{x}{3} &\le \dfrac{5}{6} - 2 \\
\dfrac{x}{3} &\le \dfrac{5}{6} - \dfrac{12}{6} \\
\dfrac{x}{3} &\le -\dfrac{7}{6} \\
x &\le -\dfrac{7}{2} \quad \text{(multiplicamos por 3 > 0)}
\end{align*}

\textbf{Notación de intervalos:} $\left(-\infty, -\dfrac{7}{2}\right]$

\textbf{Representación gráfica:} En la recta real, se dibuja un círculo relleno en $x = -\dfrac{7}{2} = -3.5$ (porque está incluido, $\le$) y se sombrea hacia la izquierda hasta $-\infty$.
\end{solucion}

\problem $\dfrac{2 - x}{4} > -\dfrac{1}{2}$

\begin{solucion}
\begin{align*}
\dfrac{2 - x}{4} &> -\dfrac{1}{2} \\
2 - x &> -2 \quad \text{(multiplicamos por 4 > 0)} \\
-x &> -4 \\
x &< 4 \quad \textcolor{red}{\text{(multiplicamos por $-1 < 0$ e invertimos)}}
\end{align*}

\textbf{Notación de intervalos:} $(-\infty, 4)$

\textbf{Representación gráfica:} En la recta real, se dibuja un círculo abierto en $x = 4$ (porque no está incluido, $<$) y se sombrea hacia la izquierda hasta $-\infty$.
\end{solucion}

\problem $-6x + 9 \ge -3$

\begin{solucion}
\begin{align*}
-6x + 9 &\ge -3 \\
-6x &\ge -12 \\
x &\le 2 \quad \textcolor{red}{\text{(dividimos entre $-6 < 0$ e invertimos)}}
\end{align*}

\textbf{Notación de intervalos:} $(-\infty, 2]$

\textbf{Representación gráfica:} En la recta real, se dibuja un círculo relleno en $x = 2$ (porque está incluido, $\le$) y se sombrea hacia la izquierda hasta $-\infty$.
\end{solucion}

\problem $7 - 2x > 1$

\begin{solucion}
\begin{align*}
7 - 2x &> 1 \\
-2x &> -6 \\
x &< 3 \quad \textcolor{red}{\text{(dividimos entre $-2 < 0$ e invertimos)}}
\end{align*}

\textbf{Notación de intervalos:} $(-\infty, 3)$

\textbf{Representación gráfica:} En la recta real, se dibuja un círculo abierto en $x = 3$ (porque no está incluido, $<$) y se sombrea hacia la izquierda hasta $-\infty$.
\end{solucion}

\problem $\dfrac{5x - 1}{2} \le 3x + 4$

\begin{solucion}
\begin{align*}
\dfrac{5x - 1}{2} &\le 3x + 4 \\
5x - 1 &\le 6x + 8 \quad \text{(multiplicamos por 2 > 0)} \\
-1 &\le x + 8 \\
-9 &\le x \\
x &\ge -9
\end{align*}

\textbf{Notación de intervalos:} $[-9, \infty)$

\textbf{Representación gráfica:} En la recta real, se dibuja un círculo relleno en $x = -9$ (porque está incluido, $\ge$) y se sombrea hacia la derecha hasta $\infty$.
\end{solucion}

\problem $\dfrac{x - 4}{3} \ge \dfrac{2x + 1}{6}$

\begin{solucion}
\begin{align*}
\dfrac{x - 4}{3} &\ge \dfrac{2x + 1}{6} \\
2(x - 4) &\ge 2x + 1 \quad \text{(multiplicamos por 6 > 0)} \\
2x - 8 &\ge 2x + 1 \\
-8 &\ge 1 \quad \text{(falso)}
\end{align*}

\textbf{Solución:} No hay solución. El conjunto solución es $\varnothing$ (conjunto vacío).

\textbf{Representación gráfica:} No hay ningún punto en la recta real que satisfaga esta desigualdad.
\end{solucion}
\end{exercise}

\subsection*{Compuestas}

\begin{exercise}
\problem $-1 \le 2x + 3 < 7$

\begin{solucion}
\begin{align*}
-1 &\le 2x + 3 < 7 \\
-4 &\le 2x < 4 \quad \text{(restamos 3 de las tres partes)} \\
-2 &\le x < 2 \quad \text{(dividimos entre 2 > 0)}
\end{align*}

\textbf{Notación de intervalos:} $[-2, 2)$

\textbf{Representación gráfica:} En la recta real, se dibuja un círculo relleno en $x = -2$ (incluido, $\le$), un círculo abierto en $x = 2$ (no incluido, $<$), y se sombrea la región entre ambos puntos.
\end{solucion}

\problem $1 < \dfrac{3x - 2}{2} \le 5$

\begin{solucion}
\begin{align*}
1 &< \dfrac{3x - 2}{2} \le 5 \\
2 &< 3x - 2 \le 10 \quad \text{(multiplicamos por 2 > 0)} \\
4 &< 3x \le 12 \quad \text{(sumamos 2)} \\
\dfrac{4}{3} &< x \le 4 \quad \text{(dividimos entre 3 > 0)}
\end{align*}

\textbf{Notación de intervalos:} $\left(\dfrac{4}{3}, 4\right]$

\textbf{Representación gráfica:} En la recta real, se dibuja un círculo abierto en $x = \dfrac{4}{3} \approx 1.33$ (no incluido, $>$), un círculo relleno en $x = 4$ (incluido, $\le$), y se sombrea la región entre ambos puntos.
\end{solucion}

\problem $-4 \le \dfrac{x - 1}{2} < 2$

\begin{solucion}
\begin{align*}
-4 &\le \dfrac{x - 1}{2} < 2 \\
-8 &\le x - 1 < 4 \quad \text{(multiplicamos por 2 > 0)} \\
-7 &\le x < 5 \quad \text{(sumamos 1)}
\end{align*}

\textbf{Notación de intervalos:} $[-7, 5)$

\textbf{Representación gráfica:} En la recta real, se dibuja un círculo relleno en $x = -7$ (incluido, $\le$), un círculo abierto en $x = 5$ (no incluido, $<$), y se sombrea la región entre ambos puntos.
\end{solucion}

\problem $3x - 5 < -2 \;\;\text{o}\;\; 3x - 5 \ge 7$

\begin{solucion}
\textbf{Primera parte:}
\begin{align*}
3x - 5 &< -2 \\
3x &< 3 \\
x &< 1
\end{align*}

\textbf{Segunda parte:}
\begin{align*}
3x - 5 &\ge 7 \\
3x &\ge 12 \\
x &\ge 4
\end{align*}

\textbf{Notación de intervalos:} $(-\infty, 1) \cup [4, \infty)$

\textbf{Representación gráfica:} En la recta real, se dibujan dos regiones separadas:
\begin{itemize}
    \item Círculo abierto en $x = 1$ (no incluido) con sombra hacia la izquierda hasta $-\infty$
    \item Círculo relleno en $x = 4$ (incluido) con sombra hacia la derecha hasta $\infty$
\end{itemize}
\end{solucion}

\problem $\dfrac{x + 2}{4} \le -1 \;\;\text{o}\;\; \dfrac{x + 2}{4} > 3$

\begin{solucion}
\textbf{Primera parte:}
\begin{align*}
\dfrac{x + 2}{4} &\le -1 \\
x + 2 &\le -4 \quad \text{(multiplicamos por 4 > 0)} \\
x &\le -6
\end{align*}

\textbf{Segunda parte:}
\begin{align*}
\dfrac{x + 2}{4} &> 3 \\
x + 2 &> 12 \quad \text{(multiplicamos por 4 > 0)} \\
x &> 10
\end{align*}

\textbf{Notación de intervalos:} $(-\infty, -6] \cup (10, \infty)$

\textbf{Representación gráfica:} En la recta real, se dibujan dos regiones separadas:
\begin{itemize}
    \item Círculo relleno en $x = -6$ (incluido) con sombra hacia la izquierda hasta $-\infty$
    \item Círculo abierto en $x = 10$ (no incluido) con sombra hacia la derecha hasta $\infty$
\end{itemize}
\end{solucion}
\end{exercise}

\subsection*{Modelado}

\begin{exercise}
\problem Un artículo con precio $p$ recibe un descuento de \$6 y debe costar al menos \$19 después del descuento. Escriba y resuelva la desigualdad para $p$.

\begin{solucion}
\textbf{Traducción:} ``al menos \$19'' significa ``mayor o igual que \$19''.

El precio después del descuento es $p - 6$, entonces:
\begin{align*}
p - 6 &\ge 19 \\
p &\ge 25
\end{align*}

\textbf{Interpretación:} El precio original del artículo debe ser al menos \$25.

\textbf{Notación de intervalos:} $[25, \infty)$

\textbf{Representación gráfica:} En la recta real, se dibuja un círculo relleno en $p = 25$ (incluido, $\ge$) y se sombrea hacia la derecha hasta $\infty$.
\end{solucion}

\problem En un examen de 100 puntos, cada pregunta correcta vale 4 puntos y cada incorrecta resta 1 punto. Si se contestan exactamente 30 preguntas y se necesita un puntaje de al menos 80 puntos, ¿cuántas preguntas correctas ($x$) se necesitan como mínimo?

\begin{solucion}
\textbf{Variables:}
\begin{itemize}
    \item $x$ = número de preguntas correctas
    \item $30 - x$ = número de preguntas incorrectas (porque se contestan 30 en total)
\end{itemize}

\textbf{Puntaje:}
\[\text{Puntaje} = 4x - 1(30 - x) = 4x - 30 + x = 5x - 30\]

\textbf{Desigualdad:} ``al menos 80 puntos'' significa $\ge 80$
\begin{align*}
5x - 30 &\ge 80 \\
5x &\ge 110 \\
x &\ge 22
\end{align*}

\textbf{Interpretación:} Se necesitan al menos 22 preguntas correctas (de 30) para obtener un puntaje de al menos 80 puntos.

\textbf{Notación de intervalos:} Como $x$ debe ser un número entero entre 0 y 30, la solución es $\{22, 23, 24, 25, 26, 27, 28, 29, 30\}$ o en notación continua: $[22, 30]$.

\textbf{Representación gráfica:} En la recta real de 0 a 30, se dibuja un círculo relleno en $x = 22$ (incluido) y se sombrea desde 22 hasta 30 (también incluido).
\end{solucion}
\end{exercise}


% Conditional solution inclusion
\ifshowsolutions
    \newpage
    \section*{Soluciones}
    %========================================
% INSTRUCTOR SOLUTIONS AND NOTES: Desigualdades Lineales
%========================================

\subsection*{Notas Pedagógicas para el Instructor}

\subsubsection*{Objetivos de Aprendizaje}

Al completar esta lección, los estudiantes deberán ser capaces de:

\begin{enumerate}
    \item Distinguir entre ecuaciones y desigualdades lineales
    \item Aplicar correctamente las reglas para manipular desigualdades
    \item \textbf{Identificar cuándo invertir el sentido de la desigualdad} (al multiplicar/dividir por negativos)
    \item Resolver desigualdades lineales en una variable
    \item Expresar soluciones en notación de intervalos
    \item Representar soluciones gráficamente en la recta real
    \item Resolver desigualdades compuestas (tipo ``y'' y tipo ``o'')
    \item Modelar situaciones del mundo real con desigualdades
\end{enumerate}

\subsubsection*{Errores Comunes de los Estudiantes}

\begin{enumerate}
    \item \textbf{ERROR \#1: No invertir la desigualdad al multiplicar/dividir por negativos}

    \textbf{Ejemplo incorrecto:}
    \begin{align*}
        -2x &> 6 \\
        x &> -3 \quad \textcolor{red}{\text{¡INCORRECTO!}}
    \end{align*}

    \textbf{Corrección:} Al dividir entre $-2 < 0$, se debe invertir:
    \begin{align*}
        -2x &> 6 \\
        x &< -3 \quad \textcolor{green}{\text{¡CORRECTO!}}
    \end{align*}

    \textbf{Estrategia de enseñanza:} Pida a los estudiantes que verifiquen su respuesta sustituyendo un valor del conjunto solución. Por ejemplo, si $x < -3$, pruebe $x = -4$: ¿$-2(-4) > 6$? Sí, $8 > 6$ ✓

    \item \textbf{ERROR \#2: Confundir círculos abiertos y cerrados}

    Los estudiantes frecuentemente dibujan círculos rellenos cuando deberían ser abiertos y viceversa.

    \textbf{Mnemotecnia:}
    \begin{itemize}
        \item \textbf{Círculo relleno (●):} El valor \textbf{está incluido} ($\le$ o $\ge$)
        \item \textbf{Círculo abierto (○):} El valor \textbf{no está incluido} ($<$ o $>$)
    \end{itemize}

    \item \textbf{ERROR \#3: Confundir paréntesis y corchetes en notación de intervalos}

    \textbf{Regla nemotécnica:}
    \begin{itemize}
        \item \textbf{Paréntesis ( ):} ``abierto'' = no incluido
        \item \textbf{Corchete [ ]:} ``cerrado'' = incluido
        \item El infinito ($\infty$) \textbf{siempre} lleva paréntesis
    \end{itemize}

    \item \textbf{ERROR \#4: En desigualdades compuestas tipo ``o'', intentar combinarlas incorrectamente}

    \textbf{Ejemplo incorrecto:}
    \[x < -2 \text{ o } x \ge 3 \quad \Rightarrow \quad -2 < x \ge 3 \quad \textcolor{red}{\text{¡INCORRECTO!}}\]

    \textbf{Corrección:} Las desigualdades tipo ``o'' representan la \textbf{unión} de dos conjuntos disjuntos, no se pueden combinar en una sola expresión. La notación correcta es:
    \[(-\infty, -2) \cup [3, \infty)\]

    \item \textbf{ERROR \#5: Olvidar que algunas desigualdades no tienen solución}

    Como en el Ejercicio 10, donde se obtiene $-8 \ge 1$ (falso), el conjunto solución es $\varnothing$.
\end{enumerate}

\subsubsection*{Sugerencias de Enseñanza}

\begin{enumerate}
    \item \textbf{Enfatizar la verificación de soluciones}

    Anime a los estudiantes a verificar sus respuestas sustituyendo valores del conjunto solución en la desigualdad original. Esto refuerza la comprensión y detecta errores.

    \item \textbf{Uso de la recta numérica física}

    Considere usar una recta numérica grande en el pizarrón o en el suelo donde los estudiantes puedan pararse físicamente en diferentes puntos para visualizar las desigualdades.

    \item \textbf{Conexión con ecuaciones}

    Comience cada tema resolviendo una ecuación similar antes de pasar a la desigualdad. Por ejemplo:
    \begin{itemize}
        \item Ecuación: $-2x = 6 \Rightarrow x = -3$ (un punto)
        \item Desigualdad: $-2x > 6 \Rightarrow x < -3$ (un intervalo)
    \end{itemize}

    \item \textbf{Práctica progresiva}

    Orden sugerido de dificultad:
    \begin{enumerate}
        \item Desigualdades simples con coeficientes positivos
        \item Desigualdades con coeficientes negativos (enfatizar inversión)
        \item Desigualdades con fracciones
        \item Desigualdades compuestas tipo ``y''
        \item Desigualdades compuestas tipo ``o''
        \item Problemas de aplicación
    \end{enumerate}

    \item \textbf{Uso de color en el pizarrón}

    \begin{itemize}
        \item Use color rojo cuando se invierte una desigualdad
        \item Use azul para resaltar las soluciones finales
        \item Use verde para verificaciones correctas
    \end{itemize}
\end{enumerate}

\subsubsection*{Actividades de Clase Sugeridas}

\begin{enumerate}
    \item \textbf{Actividad 1: ¿Verdadero o Falso?}

    Presente una lista de valores y pida a los estudiantes que determinen si satisfacen una desigualdad dada. Ejemplo:
    \begin{itemize}
        \item Desigualdad: $2x - 5 < 7$
        \item ¿$x = 0$ es solución? (Sí, porque $-5 < 7$)
        \item ¿$x = 6$ es solución? (No, porque $7 \not< 7$)
        \item ¿$x = -10$ es solución? (Sí, porque $-25 < 7$)
    \end{itemize}

    \item \textbf{Actividad 2: Matching Game}

    Prepare tarjetas con desigualdades, notaciones de intervalos y gráficas. Los estudiantes deben emparejar las tres representaciones de la misma solución.

    \item \textbf{Actividad 3: Error Analysis}

    Presente soluciones incorrectas y pida a los estudiantes que identifiquen y corrijan los errores. Esto desarrolla pensamiento crítico.

    \item \textbf{Actividad 4: Problemas del Mundo Real}

    Pida a los estudiantes que creen sus propios problemas verbales que requieran desigualdades, luego intercambien con compañeros para resolver.
\end{enumerate}

\subsubsection*{Preguntas de Discusión}

\begin{enumerate}
    \item ¿Por qué multiplicar por un número negativo invierte la desigualdad? (Relacionar con la recta numérica y reflexión)

    \item ¿Cuál es la diferencia entre ``y'' e ``o'' en desigualdades compuestas? ¿Cómo se relaciona con intersección y unión de conjuntos?

    \item ¿Por qué el infinito siempre lleva paréntesis en notación de intervalos? (Porque no es un número alcanzable)

    \item ¿Cómo podemos verificar si nuestra solución es correcta? (Sustitución de valores de prueba)

    \item ¿En qué situaciones del mundo real usamos desigualdades en lugar de ecuaciones? (Presupuestos, límites de velocidad, restricciones de edad, etc.)
\end{enumerate}

\subsubsection*{Extensiones para Estudiantes Avanzados}

\begin{enumerate}
    \item Desigualdades con valor absoluto: $|x - 3| < 5$

    \item Sistemas de desigualdades en dos variables (representación gráfica en el plano)

    \item Desigualdades cuadráticas: $x^2 - 5x + 6 \le 0$

    \item Aplicaciones de optimización con restricciones
\end{enumerate}

\subsubsection*{Recursos Adicionales}

\begin{itemize}
    \item \textbf{Videos:} Khan Academy - Linear Inequalities
    \item \textbf{Software:} GeoGebra para visualizar desigualdades gráficamente
    \item \textbf{Práctica adicional:} IXL Math - Solve linear inequalities
    \item \textbf{Manipulativos:} Rectas numéricas físicas, fichas para marcar intervalos
\end{itemize}

\subsubsection*{Evaluación Formativa}

\textbf{Pregunta rápida de verificación (Exit Ticket):}

Resuelva: $-3x + 7 \le 1$

\textbf{Respuesta correcta:} $x \ge 2$ o $[2, \infty)$

Si los estudiantes responden $x \le 2$, necesitan refuerzo sobre la inversión de desigualdades.

\subsubsection*{Conexiones con Otros Temas}

\begin{itemize}
    \item \textbf{Lección anterior (Sistema de ecuaciones):} Las desigualdades lineales son la base para sistemas de desigualdades
    \item \textbf{Próxima lección:} Desigualdades con valor absoluto requieren comprender desigualdades compuestas
    \item \textbf{Aplicaciones futuras:} Programación lineal, optimización, cálculo (derivadas para encontrar máximos/mínimos)
\end{itemize}

\subsubsection*{Notas Específicas por Ejercicio}

\textbf{Ejercicios 1-3:} Buenos problemas para comenzar. Coeficientes positivos, sin complicaciones.

\textbf{Ejercicios 4, 6-8:} Requieren inversión de desigualdad. Monitoree cuidadosamente que los estudiantes inviertan correctamente.

\textbf{Ejercicio 5:} Práctica con fracciones. Algunos estudiantes pueden necesitar revisar operaciones con fracciones.

\textbf{Ejercicio 9:} Requiere reorganizar términos. Buena práctica de álgebra.

\textbf{Ejercicio 10:} No tiene solución ($\varnothing$). Asegúrese de que los estudiantes reconozcan declaraciones falsas.

\textbf{Ejercicios 11-13:} Desigualdades compuestas tipo ``y''. Enfatice que se trabaja con las tres partes simultáneamente.

\textbf{Ejercicios 14-15:} Desigualdades tipo ``o''. Enfatice la unión ($\cup$) y que las regiones están separadas.

\textbf{Ejercicio 16:} Problema verbal directo. Buena introducción al modelado.

\textbf{Ejercicio 17:} Problema más complejo que requiere plantear la expresión del puntaje. Excelente para pensamiento crítico. Note que la solución debe ser un entero entre 0 y 30.

\subsubsection*{Tiempo Estimado}

\begin{itemize}
    \item \textbf{Teoría y ejemplos:} 45-60 minutos
    \item \textbf{Práctica guiada:} 30-45 minutos
    \item \textbf{Práctica independiente:} 45-60 minutos
    \item \textbf{Total:} 2-3 sesiones de clase (50 minutos cada una)
\end{itemize}

\subsubsection*{Diferenciación}

\textbf{Para estudiantes que necesitan apoyo adicional:}
\begin{itemize}
    \item Proporcione rectas numéricas pre-dibujadas
    \item Use manipulativos físicos
    \item Comience con más ejemplos de coeficientes positivos
    \item Proporcione una tarjeta de referencia con las reglas
\end{itemize}

\textbf{Para estudiantes avanzados:}
\begin{itemize}
    \item Introduzca desigualdades con valor absoluto
    \item Pida que creen sus propios problemas
    \item Explore desigualdades en contextos más complejos
    \item Investigue sistemas de desigualdades
\end{itemize}

\vspace{1cm}

\hrule

\vspace{0.5cm}

\textbf{Nota final:} El concepto de invertir la desigualdad al multiplicar/dividir por negativos es contraintuitivo para muchos estudiantes. Dedique tiempo adicional a este concepto, use múltiples ejemplos, y verifique frecuentemente la comprensión. La visualización en la recta numérica es especialmente útil para desarrollar intuición.

\fi

\end{document}
