%========================================
% DETAILED SOLUTIONS: Propiedades de los Números Reales y Exponentes
%========================================

\subsection*{Ejercicio 1}

\textbf{Problema:} Identificar propiedades de los números reales.

\textbf{Soluciones detalladas:}

\textbf{a)} $5x + 2y = 2y + 5x$ - \textbf{Conmutativa de la suma}

Esta expresión muestra que el orden de los sumandos no afecta el resultado. La propiedad conmutativa de la suma establece que $a + b = b + a$ para cualesquiera números reales $a$ y $b$.

\textbf{b)} $4(5x + 1) = 20x + 4$ - \textbf{Distributiva}

Aquí se aplica la propiedad distributiva: $a(b + c) = ab + ac$. El factor 4 se distribuye a cada término dentro del paréntesis: $4 \cdot 5x + 4 \cdot 1 = 20x + 4$.

\textbf{c)} $(ab)c = a(bc)$ - \textbf{Asociativa de la multiplicación}

La forma en que se agrupan los factores no cambia el producto. Esta es la esencia de la propiedad asociativa de la multiplicación.

\textbf{d)} $(x + y)(x-y) = (x+y)x + (x+y)(-y)$ - \textbf{Distributiva}

Se aplica la propiedad distributiva donde $(x+y)$ se distribuye a ambos términos: $x$ y $(-y)$.

\textbf{e)} $3(x + y + z) = 3x + 3y + 3z$ - \textbf{Distributiva}

El factor 3 se distribuye a cada uno de los tres términos dentro del paréntesis.

\subsection*{Ejercicio 2}

\textbf{Problema:} Clasificar números en los diferentes conjuntos numéricos.

\textbf{Análisis detallado:}

La jerarquía de los conjuntos numéricos es: $\mathbb{N} \subset \mathbb{Z} \subset \mathbb{Q} \subset \mathbb{R}$

\textbf{a)} $7$ - Natural, Entero, Racional, Real
\begin{itemize}
    \item Es natural porque $7 \in \{1, 2, 3, \ldots\}$
    \item Es entero porque todo natural es entero
    \item Es racional porque $7 = \frac{7}{1}$
    \item Es real porque todo racional es real
\end{itemize}

\textbf{b)} $-3$ - Entero, Racional, Real
\begin{itemize}
    \item No es natural (los naturales son positivos)
    \item Es entero porque $-3 \in \mathbb{Z}$
    \item Es racional porque $-3 = \frac{-3}{1}$
    \item Es real porque todo racional es real
\end{itemize}

\textbf{f)} $\sqrt{7}$ - Irracional, Real
\begin{itemize}
    \item No es racional porque $\sqrt{7}$ no puede expresarse como $\frac{p}{q}$ con $p, q$ enteros
    \item Su expansión decimal es infinita no periódica: $\sqrt{7} \approx 2.645751311\ldots$
    \item Es real porque $\mathbb{R} = \mathbb{Q} \cup \text{Irracionales}$
\end{itemize}

\textbf{i)} $0.\overline{3}$ - Racional, Real

Para convertir a fracción:
\begin{align}
x &= 0.\overline{3} = 0.333\ldots\\
10x &= 3.333\ldots\\
10x - x &= 3.333\ldots - 0.333\ldots\\
9x &= 3\\
x &= \frac{3}{9} = \frac{1}{3}
\end{align}

\subsection*{Ejercicio 3}

\textbf{Problema:} Ubicar números en la recta numérica.

\textbf{Procedimiento:}

La recta numérica es una representación visual donde cada punto corresponde a un único número real.

\textbf{Valores a ubicar:}
\begin{itemize}
    \item $-2.5$: Exactamente a la mitad entre $-3$ y $-2$
    \item $\frac{3}{2} = 1.5$: Exactamente a la mitad entre $1$ y $2$
    \item $-\sqrt{4} = -2$: Exactamente en la marca $-2$
    \item $\pi \approx 3.14159$: Entre $3$ y $4$, muy cerca de $3$
    \item $0$: En el origen de la recta
\end{itemize}

\subsection*{Ejercicio 4}

\textbf{Problema:} Aplicar orden de operaciones (PEMDSR).

\textbf{Soluciones paso a paso:}

\textbf{a)} $8 + 2 \times 3^2$
\begin{align}
8 + 2 \times 3^2 &= 8 + 2 \times 9 && \text{(Exponentes primero)}\\
&= 8 + 18 && \text{(Multiplicación)}\\
&= 26 && \text{(Suma)}
\end{align}

\textbf{d)} $2^3 + 4(5 - 2)$
\begin{align}
2^3 + 4(5 - 2) &= 8 + 4(3) && \text{(Exponentes y paréntesis)}\\
&= 8 + 12 && \text{(Multiplicación)}\\
&= 20 && \text{(Suma)}
\end{align}

\textbf{e)} $\frac{12 + 8}{4} - 2^2$
\begin{align}
\frac{12 + 8}{4} - 2^2 &= \frac{20}{4} - 4 && \text{(Paréntesis implícitos y exponentes)}\\
&= 5 - 4 && \text{(División)}\\
&= 1 && \text{(Resta)}
\end{align}

\subsection*{Ejercicio 5}

\textbf{Problema:} Operaciones con fracciones.

\textbf{Procedimientos detallados:}

\textbf{a)} $\frac{2}{3} + \frac{1}{4}$

Para sumar fracciones con diferentes denominadores, encontramos el mínimo común múltiplo:
\begin{align}
\text{MCM}(3,4) &= 12\\
\frac{2}{3} + \frac{1}{4} &= \frac{2 \times 4}{3 \times 4} + \frac{1 \times 3}{4 \times 3}\\
&= \frac{8}{12} + \frac{3}{12}\\
&= \frac{11}{12}
\end{align}

\textbf{d)} $\frac{7}{8} \div \frac{3}{4}$

Para dividir fracciones, multiplicamos por el recíproco del divisor:
\begin{align}
\frac{7}{8} \div \frac{3}{4} &= \frac{7}{8} \times \frac{4}{3}\\
&= \frac{7 \times 4}{8 \times 3}\\
&= \frac{28}{24}\\
&= \frac{7}{6} && \text{(Simplificando por 4)}
\end{align}

\textbf{f)} $\frac{4}{7} \times \frac{14}{8}$

Antes de multiplicar, podemos simplificar:
\begin{align}
\frac{4}{7} \times \frac{14}{8} &= \frac{4 \times 14}{7 \times 8}\\
&= \frac{56}{56}\\
&= 1
\end{align}

Alternativamente: $\frac{4}{7} \times \frac{14}{8} = \frac{4}{8} \times \frac{14}{7} = \frac{1}{2} \times 2 = 1$

\subsection*{Ejercicio 6}

\textbf{Problema:} Aplicar propiedades de los números reales.

\textbf{a)} $3x + 7x = (3 + 7)x = 10x$

Se aplica la propiedad distributiva en reversa: $ac + bc = (a + b)c$

\textbf{d)} $2(3x + 4y - 5) = 2 \cdot 3x + 2 \cdot 4y + 2 \cdot (-5) = 6x + 8y - 10$

La propiedad distributiva se extiende a múltiples términos: $a(b + c + d) = ab + ac + ad$

\textbf{f)} $x \cdot 0 + y \cdot 1 = 0 + y = y$

Se aplican las propiedades del elemento neutro:
\begin{itemize}
    \item $x \cdot 0 = 0$ (propiedad del cero)
    \item $y \cdot 1 = y$ (neutro multiplicativo)
    \item $0 + y = y$ (neutro aditivo)
\end{itemize}

\subsection*{Ejercicio 7}

\textbf{Problema:} Encontrar inversos aditivos y multiplicativos.

\textbf{Conceptos importantes:}
\begin{itemize}
    \item \textbf{Inverso aditivo} de $a$ es $-a$ tal que $a + (-a) = 0$
    \item \textbf{Inverso multiplicativo} de $a$ es $\frac{1}{a}$ tal que $a \cdot \frac{1}{a} = 1$ (para $a \neq 0$)
\end{itemize}

\textbf{c)} $\frac{2}{7}$
\begin{itemize}
    \item Inverso aditivo: $-\frac{2}{7}$ porque $\frac{2}{7} + \left(-\frac{2}{7}\right) = 0$
    \item Inverso multiplicativo: $\frac{7}{2}$ porque $\frac{2}{7} \times \frac{7}{2} = \frac{14}{14} = 1$
\end{itemize}

\textbf{d)} $0$
\begin{itemize}
    \item Inverso aditivo: $0$ porque $0 + 0 = 0$
    \item Inverso multiplicativo: \textbf{No existe} porque la división por cero no está definida
\end{itemize}

\subsection*{Ejercicio 8}

\textbf{Problema:} Evaluar potencias.

\textbf{Conceptos clave:}
\begin{itemize}
    \item $a^n = a \times a \times \cdots \times a$ ($n$ factores)
    \item El signo del resultado depende de la base y si el exponente es par o impar
\end{itemize}

\textbf{b)} $(-3)^3 = (-3) \times (-3) \times (-3)$
\begin{align}
(-3)^3 &= (-3) \times (-3) \times (-3)\\
&= 9 \times (-3) && \text{(Exponente impar $\Rightarrow$ resultado negativo)}\\
&= -27
\end{align}

\textbf{c)} $(-2)^4 = (-2) \times (-2) \times (-2) \times (-2)$
\begin{align}
(-2)^4 &= (-2) \times (-2) \times (-2) \times (-2)\\
&= 4 \times 4 && \text{(Exponente par $\Rightarrow$ resultado positivo)}\\
&= 16
\end{align}

\textbf{h)} $\left(-\frac{2}{3}\right)^2$
\begin{align}
\left(-\frac{2}{3}\right)^2 &= \left(-\frac{2}{3}\right) \times \left(-\frac{2}{3}\right)\\
&= \frac{(-2) \times (-2)}{3 \times 3}\\
&= \frac{4}{9}
\end{align}

\subsection*{Ejercicio 9}

\textbf{Problema:} Convertir decimales a fracciones.

\textbf{Para decimales finitos:}
Colocamos el decimal sobre una potencia de 10 y simplificamos.

\textbf{Para decimales periódicos:}
Usamos álgebra para encontrar la fracción equivalente.

\textbf{d)} $0.\overline{6}$
\begin{align}
x &= 0.\overline{6} = 0.666\ldots\\
10x &= 6.666\ldots\\
10x - x &= 6.666\ldots - 0.666\ldots\\
9x &= 6\\
x &= \frac{6}{9} = \frac{2}{3}
\end{align}

\textbf{f)} $0.\overline{45}$
\begin{align}
x &= 0.\overline{45} = 0.454545\ldots\\
100x &= 45.454545\ldots && \text{(Multiplicamos por 100 porque el periodo tiene 2 dígitos)}\\
100x - x &= 45.454545\ldots - 0.454545\ldots\\
99x &= 45\\
x &= \frac{45}{99} = \frac{5}{11} && \text{(Simplificando por 9)}
\end{align}

\subsection*{Ejercicio 10}

\textbf{Problema:} Problemas de aplicación.

\textbf{a)} Área del rectángulo
\begin{align}
\text{Área} &= \text{largo} \times \text{ancho}\\
&= (3x + 2) \times 2\\
&= 2(3x + 2)\\
&= 2 \cdot 3x + 2 \cdot 2\\
&= 6x + 4
\end{align}

\textbf{c)} Pizza restante
\begin{align}
\text{Pizza restante} &= \text{Pizza inicial} - \text{Pizza consumida}\\
&= \frac{3}{4} - \frac{1}{3}\\
&= \frac{3 \times 3}{4 \times 3} - \frac{1 \times 4}{3 \times 4}\\
&= \frac{9}{12} - \frac{4}{12}\\
&= \frac{5}{12}
\end{align}

\textbf{d)} Cambio total de temperatura
\begin{align}
\text{Cambio total} &= (-5) + (+8) + (-3)\\
&= -5 + 8 - 3\\
&= 3 - 3\\
&= 0°C
\end{align}

La temperatura regresó a su valor inicial después de todos los cambios.