%========================================
% EXERCISES: Valor Absoluto y Leyes de Signos
%========================================

\subsection{Ejercicios}

\begin{exercise}
\problem Evalúe las siguientes expresiones de valor absoluto:

\begin{exerciselist}
    \item $|7| = $ \underline{\hspace{3cm}}
    \item $|-12| = $ \underline{\hspace{3cm}}
    \item $|0| = $ \underline{\hspace{3cm}}
    \item $|-(-8)| = $ \underline{\hspace{3cm}}
    \item $|15 - 20| = $ \underline{\hspace{3cm}}
\end{exerciselist}

\begin{solucion}
\begin{exerciselist}
    \item $|7| = 7$
    \item $|-12| = 12$
    \item $|0| = 0$
    \item $|-(-8)| = |8| = 8$
    \item $|15 - 20| = |-5| = 5$
\end{exerciselist}
\end{solucion}
\end{exercise}

\begin{exercise}
\problem Simplifique las siguientes expresiones usando las propiedades de valor absoluto:

\begin{exerciselist}
    \item $|8| + |-6| = $ \underline{\hspace{4cm}}
    \item $|-9| - |3| = $ \underline{\hspace{4cm}}
    \item $|4| \cdot |-7| = $ \underline{\hspace{4cm}}
    \item $\frac{|-20|}{|4|} = $ \underline{\hspace{4cm}}
    \item $|-3 + 8| - |2 - 10| = $ \underline{\hspace{4cm}}
\end{exerciselist}

\begin{solucion}
\begin{exerciselist}
    \item $|8| + |-6| = 8 + 6 = 14$
    \item $|-9| - |3| = 9 - 3 = 6$
    \item $|4| \cdot |-7| = 4 \cdot 7 = 28$
    \item $\frac{|-20|}{|4|} = \frac{20}{4} = 5$
    \item $|-3 + 8| - |2 - 10| = |5| - |-8| = 5 - 8 = -3$
\end{exerciselist}
\end{solucion}
\end{exercise}

\begin{exercise}
\problem Aplique las propiedades de signos para simplificar:

\begin{exerciselist}
    \item $(-1) \cdot 15 = $ \underline{\hspace{3cm}}
    \item $-(-9) = $ \underline{\hspace{3cm}}
    \item $(-4) \cdot 6 = $ \underline{\hspace{3cm}}
    \item $(-5) \cdot (-8) = $ \underline{\hspace{3cm}}
    \item $-(7 + 2) = $ \underline{\hspace{3cm}}
    \item $-(10 - 3) = $ \underline{\hspace{3cm}}
\end{exerciselist}

\begin{solucion}
\begin{exerciselist}
    \item $(-1) \cdot 15 = -15$
    \item $-(-9) = 9$
    \item $(-4) \cdot 6 = -24$
    \item $(-5) \cdot (-8) = 40$
    \item $-(7 + 2) = -7 - 2 = -9$
    \item $-(10 - 3) = 3 - 10 = -7$
\end{exerciselist}
\end{solucion}
\end{exercise}

\begin{exercise}
\problem Resuelva las siguientes operaciones aplicando las reglas de signos:

\begin{exerciselist}
    \item $(-6) + (-4) = $ \underline{\hspace{3cm}}
    \item $(-8) + 12 = $ \underline{\hspace{3cm}}
    \item $5 - (-7) = $ \underline{\hspace{3cm}}
    \item $(-3) \times (-9) = $ \underline{\hspace{3cm}}
    \item $(-24) \div 8 = $ \underline{\hspace{3cm}}
    \item $(-45) \div (-5) = $ \underline{\hspace{3cm}}
\end{exerciselist}

\begin{solucion}
\begin{exerciselist}
    \item $(-6) + (-4) = -10$ (mismo signo: suma y conserva el signo)
    \item $(-8) + 12 = 4$ (signos diferentes: resta y conserva el signo del mayor)
    \item $5 - (-7) = 5 + 7 = 12$ (restar es sumar el opuesto)
    \item $(-3) \times (-9) = 27$ (negativo por negativo es positivo)
    \item $(-24) \div 8 = -3$ (negativo entre positivo es negativo)
    \item $(-45) \div (-5) = 9$ (negativo entre negativo es positivo)
\end{exerciselist}
\end{solucion}
\end{exercise}

\begin{exercise}
\problem Simplifique las siguientes expresiones paso a paso:

\begin{exerciselist}
    \item $|3 - 7| + |-2|$
    \item $(-5) \cdot 4 - (-8)$
    \item $\frac{|-18|}{|-3|} + |1 - 6|$
    \item $-(2 - 9) + |4|$
\end{exerciselist}

\begin{solucion}
\begin{exerciselist}
    \item $|3 - 7| + |-2| = |-4| + 2 = 4 + 2 = 6$
    \item $(-5) \cdot 4 - (-8) = -20 + 8 = -12$
    \item $\frac{|-18|}{|-3|} + |1 - 6| = \frac{18}{3} + |-5| = 6 + 5 = 11$
    \item $-(2 - 9) + |4| = -(-7) + 4 = 7 + 4 = 11$
\end{exerciselist}
\end{solucion}
\end{exercise}

\begin{exercise}
\problem Determine si las siguientes afirmaciones son verdaderas (V) o falsas (F):

\begin{exerciselist}
    \item $|a| = |-a|$ para cualquier número real $a$ \hspace{2cm} \underline{\hspace{2cm}}
    \item $|a| \geq 0$ para cualquier número real $a$ \hspace{2cm} \underline{\hspace{2cm}}
    \item $(-a)(-b) = -ab$ para cualquier $a, b$ real \hspace{2cm} \underline{\hspace{2cm}}
    \item $-(a + b) = -a + b$ para cualquier $a, b$ real \hspace{2cm} \underline{\hspace{2cm}}
    \item $|a + b| = |a| + |b|$ para cualquier $a, b$ real \hspace{2cm} \underline{\hspace{2cm}}
\end{exerciselist}

\begin{solucion}
\begin{exerciselist}
    \item V (propiedad fundamental del valor absoluto)
    \item V (el valor absoluto siempre es no negativo)
    \item F (debería ser $(-a)(-b) = ab$)
    \item F (debería ser $-(a + b) = -a - b$)
    \item F (contraejemplo: $|2 + (-3)| = 1$ pero $|2| + |-3| = 5$)
\end{exerciselist}
\end{solucion}
\end{exercise}

\begin{exercise}
\problem Resuelva las siguientes expresiones complejas:

\begin{exerciselist}
    \item $|(-3) \cdot 4| - |(-2)^2|$
    \item $(-1) \cdot |5 - 8| + |-4| \cdot 2$
    \item $\frac{|(-6) \cdot (-2)|}{|-3|} - |1 - 7|$
    \item $-(|-5| - |3|) + |(-2) + 6|$
\end{exerciselist}

\begin{solucion}
\begin{exerciselist}
    \item $|(-3) \cdot 4| - |(-2)^2| = |-12| - |4| = 12 - 4 = 8$
    \item $(-1) \cdot |5 - 8| + |-4| \cdot 2 = (-1) \cdot 3 + 4 \cdot 2 = -3 + 8 = 5$
    \item $\frac{|(-6) \cdot (-2)|}{|-3|} - |1 - 7| = \frac{12}{3} - 6 = 4 - 6 = -2$
    \item $-(|-5| - |3|) + |(-2) + 6| = -(5 - 3) + |4| = -2 + 4 = 2$
\end{exerciselist}
\end{solucion}
\end{exercise}

\begin{exercise}
\problem \textbf{Problema de aplicación:} La temperatura en una ciudad varía según la fórmula $T = 20 - |h - 12|$, donde $T$ es la temperatura en grados Celsius y $h$ es la hora del día (en formato de 24 horas).

\begin{exerciselist}
    \item ¿Cuál es la temperatura a las 8:00 AM ($h = 8$)?
    \item ¿Cuál es la temperatura a las 3:00 PM ($h = 15$)?
    \item ¿A qué hora del día la temperatura es máxima?
    \item ¿Cuál es la temperatura máxima?
\end{exerciselist}

\begin{solucion}
\begin{exerciselist}
    \item $T = 20 - |8 - 12| = 20 - |-4| = 20 - 4 = 16°C$
    \item $T = 20 - |15 - 12| = 20 - |3| = 20 - 3 = 17°C$
    \item La temperatura es máxima cuando $|h - 12| = 0$, es decir, cuando $h = 12$ (mediodía)
    \item La temperatura máxima es $T = 20 - |12 - 12| = 20 - 0 = 20°C$
\end{exerciselist}
\end{solucion}
\end{exercise}