%========================================
% EXERCISES: Ecuaciones Lineales en Dos Variables
%========================================

\section{Ejercicios}

%========================================
% Exercise 1: Finding Solutions
%========================================
\begin{exercise}
\textbf{Encontrar Soluciones}

Para cada ecuación, complete la tabla de valores y encuentre las soluciones indicadas:

\problem $x + y = 5$

Complete la tabla:
\begin{center}
\begin{tabular}{|c|c|c|}
\hline
$x$ & $y$ & $(x, y)$ \\
\hline
0 & & \\
\hline
2 & & \\
\hline
& 0 & \\
\hline
-1 & & \\
\hline
\end{tabular}
\end{center}

\begin{solucion}
\begin{tabular}{|c|c|c|}
\hline
$x$ & $y$ & $(x, y)$ \\
\hline
0 & 5 & $(0, 5)$ \\
\hline
2 & 3 & $(2, 3)$ \\
\hline
5 & 0 & $(5, 0)$ \\
\hline
-1 & 6 & $(-1, 6)$ \\
\hline
\end{tabular}
\end{solucion}

\problem $2x - y = 4$

Complete la tabla:
\begin{center}
\begin{tabular}{|c|c|c|}
\hline
$x$ & $y$ & $(x, y)$ \\
\hline
0 & & \\
\hline
2 & & \\
\hline
& 0 & \\
\hline
-2 & & \\
\hline
\end{tabular}
\end{center}

\begin{solucion}
\begin{tabular}{|c|c|c|}
\hline
$x$ & $y$ & $(x, y)$ \\
\hline
0 & -4 & $(0, -4)$ \\
\hline
2 & 0 & $(2, 0)$ \\
\hline
2 & 0 & $(2, 0)$ \\
\hline
-2 & -8 & $(-2, -8)$ \\
\hline
\end{tabular}
\end{solucion}

\problem $3x + 2y = 12$

Complete la tabla:
\begin{center}
\begin{tabular}{|c|c|c|}
\hline
$x$ & $y$ & $(x, y)$ \\
\hline
0 & & \\
\hline
& 0 & \\
\hline
2 & & \\
\hline
& 3 & \\
\hline
\end{tabular}
\end{center}

\begin{solucion}
\begin{tabular}{|c|c|c|}
\hline
$x$ & $y$ & $(x, y)$ \\
\hline
0 & 6 & $(0, 6)$ \\
\hline
4 & 0 & $(4, 0)$ \\
\hline
2 & 3 & $(2, 3)$ \\
\hline
2 & 3 & $(2, 3)$ \\
\hline
\end{tabular}
\end{solucion}
\end{exercise}

%========================================
% Exercise 2: Finding Intercepts
%========================================
\begin{exercise}
\textbf{Encontrar Intersecciones con los Ejes}

Para cada ecuación, encuentre las intersecciones con el eje $x$ y el eje $y$:

\problem $2x + 3y = 6$

\begin{solucion}
Intersección con eje $x$ (hacer $y = 0$): $2x = 6 \implies x = 3$. Punto: $(3, 0)$

Intersección con eje $y$ (hacer $x = 0$): $3y = 6 \implies y = 2$. Punto: $(0, 2)$
\end{solucion}

\problem $4x - y = 8$

\begin{solucion}
Intersección con eje $x$ (hacer $y = 0$): $4x = 8 \implies x = 2$. Punto: $(2, 0)$

Intersección con eje $y$ (hacer $x = 0$): $-y = 8 \implies y = -8$. Punto: $(0, -8)$
\end{solucion}

\problem $x + 2y = 10$

\begin{solucion}
Intersección con eje $x$ (hacer $y = 0$): $x = 10$. Punto: $(10, 0)$

Intersección con eje $y$ (hacer $x = 0$): $2y = 10 \implies y = 5$. Punto: $(0, 5)$
\end{solucion}

\problem $5x - 2y = 20$

\begin{solucion}
Intersección con eje $x$ (hacer $y = 0$): $5x = 20 \implies x = 4$. Punto: $(4, 0)$

Intersección con eje $y$ (hacer $x = 0$): $-2y = 20 \implies y = -10$. Punto: $(0, -10)$
\end{solucion}

\problem $3x + 4y = 24$

\begin{solucion}
Intersección con eje $x$ (hacer $y = 0$): $3x = 24 \implies x = 8$. Punto: $(8, 0)$

Intersección con eje $y$ (hacer $x = 0$): $4y = 24 \implies y = 6$. Punto: $(0, 6)$
\end{solucion}

\problem $-2x + 5y = 10$

\begin{solucion}
Intersección con eje $x$ (hacer $y = 0$): $-2x = 10 \implies x = -5$. Punto: $(-5, 0)$

Intersección con eje $y$ (hacer $x = 0$): $5y = 10 \implies y = 2$. Punto: $(0, 2)$
\end{solucion}
\end{exercise}

%========================================
% Exercise 3: Verifying Solutions
%========================================
\begin{exercise}
\textbf{Verificar Soluciones}

Determine si cada punto dado es solución de la ecuación indicada. Muestre su trabajo sustituyendo los valores en la ecuación.

\problem ¿Es $(2, 3)$ una solución de $2x + y = 7$?

\begin{solucion}
Sustituir $x = 2$ y $y = 3$:
$$2(2) + 3 = 4 + 3 = 7$$

Como $7 = 7$ es verdadero, \textbf{SÍ} es una solución.
\end{solucion}

\problem ¿Es $(-1, 5)$ una solución de $3x - 2y = -13$?

\begin{solucion}
Sustituir $x = -1$ y $y = 5$:
$$3(-1) - 2(5) = -3 - 10 = -13$$

Como $-13 = -13$ es verdadero, \textbf{SÍ} es una solución.
\end{solucion}

\problem ¿Es $(4, -2)$ una solución de $x + 3y = 2$?

\begin{solucion}
Sustituir $x = 4$ y $y = -2$:
$$4 + 3(-2) = 4 - 6 = -2$$

Como $-2 \neq 2$, \textbf{NO} es una solución.
\end{solucion}

\problem ¿Es $(0, 0)$ una solución de $5x - 4y = 0$?

\begin{solucion}
Sustituir $x = 0$ y $y = 0$:
$$5(0) - 4(0) = 0 - 0 = 0$$

Como $0 = 0$ es verdadero, \textbf{SÍ} es una solución.
\end{solucion}

\problem ¿Es $(3, -1)$ una solución de $2x + 4y = 2$?

\begin{solucion}
Sustituir $x = 3$ y $y = -1$:
$$2(3) + 4(-1) = 6 - 4 = 2$$

Como $2 = 2$ es verdadero, \textbf{SÍ} es una solución.
\end{solucion}

\problem ¿Es $(-2, -3)$ una solución de $x - y = 5$?

\begin{solucion}
Sustituir $x = -2$ y $y = -3$:
$$-2 - (-3) = -2 + 3 = 1$$

Como $1 \neq 5$, \textbf{NO} es una solución.
\end{solucion}

\problem ¿Es $(5, 4)$ una solución de $y = 4$?

\begin{solucion}
Sustituir $y = 4$ en la ecuación:
$$4 = 4$$

Como esto es verdadero (la coordenada $y$ es 4), \textbf{SÍ} es una solución.

\textbf{Nota:} Cualquier punto con $y = 4$ es solución, sin importar el valor de $x$.
\end{solucion}

\problem ¿Es $(-3, 2)$ una solución de $x = -3$?

\begin{solucion}
Sustituir $x = -3$ en la ecuación:
$$-3 = -3$$

Como esto es verdadero (la coordenada $x$ es $-3$), \textbf{SÍ} es una solución.

\textbf{Nota:} Cualquier punto con $x = -3$ es solución, sin importar el valor de $y$.
\end{solucion}
\end{exercise}

%========================================
% Exercise 4: Horizontal and Vertical Lines
%========================================
\begin{exercise}
\textbf{Rectas Horizontales y Verticales}

\textbf{Parte A: Identificar el Tipo de Recta}

Para cada ecuación, indique si representa una recta horizontal o vertical:

\problem $y = 5$

\begin{solucion}
Recta \textbf{horizontal} (todos los puntos tienen $y = 5$)
\end{solucion}

\problem $x = -2$

\begin{solucion}
Recta \textbf{vertical} (todos los puntos tienen $x = -2$)
\end{solucion}

\problem $y = 0$

\begin{solucion}
Recta \textbf{horizontal} (es el eje $x$, donde todos los puntos tienen $y = 0$)
\end{solucion}

\problem $x = 7$

\begin{solucion}
Recta \textbf{vertical} (todos los puntos tienen $x = 7$)
\end{solucion}

\vspace{0.5cm}

\textbf{Parte B: Encontrar Intersecciones}

Para cada ecuación, encuentre las intersecciones con los ejes (si existen):

\problem $y = 3$

\begin{solucion}
\textbf{Intersección con eje $x$:} No existe (la recta nunca toca el eje $x$)

\textbf{Intersección con eje $y$:} $(0, 3)$
\end{solucion}

\problem $x = -4$

\begin{solucion}
\textbf{Intersección con eje $x$:} $(-4, 0)$

\textbf{Intersección con eje $y$:} No existe (la recta nunca toca el eje $y$)
\end{solucion}

\problem $y = -1$

\begin{solucion}
\textbf{Intersección con eje $x$:} No existe (la recta nunca toca el eje $x$)

\textbf{Intersección con eje $y$:} $(0, -1)$
\end{solucion}

\problem $x = 0$

\begin{solucion}
Esta ecuación representa el eje $y$.

\textbf{Intersección con eje $x$:} $(0, 0)$ (el origen)

\textbf{Intersección con eje $y$:} Todos los puntos de la recta están en el eje $y$
\end{solucion}

\vspace{0.5cm}

\textbf{Parte C: Escribir Ecuaciones}

Escriba la ecuación de la recta que satisface las condiciones dadas:

\problem Una recta horizontal que pasa por el punto $(5, -2)$

\begin{solucion}
$y = -2$

\textbf{Explicación:} Una recta horizontal tiene todos sus puntos con la misma coordenada $y$.
\end{solucion}

\problem Una recta vertical que pasa por el punto $(-3, 7)$

\begin{solucion}
$x = -3$

\textbf{Explicación:} Una recta vertical tiene todos sus puntos con la misma coordenada $x$.
\end{solucion}

\problem Una recta horizontal que pasa por el punto $(0, 4)$

\begin{solucion}
$y = 4$

\textbf{Explicación:} Como pasa por $(0, 4)$, la coordenada $y$ es 4 para todos los puntos.
\end{solucion}
\end{exercise}

%========================================
% Exercise 5: Graphing Linear Equations
%========================================
\begin{exercise}
\textbf{Graficar Ecuaciones Lineales}

Para cada ecuación, encuentre las intersecciones con los ejes y trace la gráfica de la ecuación.

\problem $2x + y = 6$

\begin{solucion}
\textbf{Paso 1:} Encontrar la intersección con el eje $x$ (hacer $y = 0$):
$$2x + 0 = 6 \implies x = 3$$
Punto: $(3, 0)$

\textbf{Paso 2:} Encontrar la intersección con el eje $y$ (hacer $x = 0$):
$$2(0) + y = 6 \implies y = 6$$
Punto: $(0, 6)$

\textbf{Paso 3:} Graficar los puntos $(3, 0)$ y $(0, 6)$ en el plano de coordenadas.

\textbf{Paso 4:} Trazar una línea recta que pase por ambos puntos.
\end{solucion}

\problem $x - 2y = 4$

\begin{solucion}
\textbf{Paso 1:} Encontrar la intersección con el eje $x$ (hacer $y = 0$):
$$x - 2(0) = 4 \implies x = 4$$
Punto: $(4, 0)$

\textbf{Paso 2:} Encontrar la intersección con el eje $y$ (hacer $x = 0$):
$$0 - 2y = 4 \implies y = -2$$
Punto: $(0, -2)$

\textbf{Paso 3:} Graficar los puntos $(4, 0)$ y $(0, -2)$ en el plano de coordenadas.

\textbf{Paso 4:} Trazar una línea recta que pase por ambos puntos.
\end{solucion}

\problem $3x + 4y = 12$

\begin{solucion}
\textbf{Paso 1:} Encontrar la intersección con el eje $x$ (hacer $y = 0$):
$$3x + 4(0) = 12 \implies x = 4$$
Punto: $(4, 0)$

\textbf{Paso 2:} Encontrar la intersección con el eje $y$ (hacer $x = 0$):
$$3(0) + 4y = 12 \implies y = 3$$
Punto: $(0, 3)$

\textbf{Paso 3:} Graficar los puntos $(4, 0)$ y $(0, 3)$ en el plano de coordenadas.

\textbf{Paso 4:} Trazar una línea recta que pase por ambos puntos.
\end{solucion}

\problem $-x + 3y = 6$

\begin{solucion}
\textbf{Paso 1:} Encontrar la intersección con el eje $x$ (hacer $y = 0$):
$$-x + 3(0) = 6 \implies x = -6$$
Punto: $(-6, 0)$

\textbf{Paso 2:} Encontrar la intersección con el eje $y$ (hacer $x = 0$):
$$-0 + 3y = 6 \implies y = 2$$
Punto: $(0, 2)$

\textbf{Paso 3:} Graficar los puntos $(-6, 0)$ y $(0, 2)$ en el plano de coordenadas.

\textbf{Paso 4:} Trazar una línea recta que pase por ambos puntos.
\end{solucion}

\problem $y = -3$

\begin{solucion}
Esta es una \textbf{recta horizontal}.

\textbf{Intersección con el eje $x$:} No existe (la recta no cruza el eje $x$)

\textbf{Intersección con el eje $y$:} $(0, -3)$

\textbf{Para graficar:} Trace una línea horizontal que pase por todos los puntos con coordenada $y = -3$. La línea es paralela al eje $x$ y pasa por el punto $(0, -3)$.
\end{solucion}

\problem $x = 2$

\begin{solucion}
Esta es una \textbf{recta vertical}.

\textbf{Intersección con el eje $x$:} $(2, 0)$

\textbf{Intersección con el eje $y$:} No existe (la recta no cruza el eje $y$)

\textbf{Para graficar:} Trace una línea vertical que pase por todos los puntos con coordenada $x = 2$. La línea es paralela al eje $y$ y pasa por el punto $(2, 0)$.
\end{solucion}
\end{exercise}
