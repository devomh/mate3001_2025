%========================================
% EXERCISES: Ecuaciones Lineales en Dos Variables
%========================================

\section{Ejercicios}

%========================================
% Exercise 1: Finding Solutions
%========================================
\begin{exercise}
\textbf{Encontrar Soluciones}

Para cada ecuación, complete la tabla de valores y encuentre las soluciones indicadas:

\problem $x + y = 5$

Complete la tabla:
\begin{center}
\begin{tabular}{|c|c|c|}
\hline
$x$ & $y$ & $(x, y)$ \\
\hline
0 & & \\
\hline
2 & & \\
\hline
& 0 & \\
\hline
-1 & & \\
\hline
\end{tabular}
\end{center}

\begin{solucion}
\begin{tabular}{|c|c|c|}
\hline
$x$ & $y$ & $(x, y)$ \\
\hline
0 & 5 & $(0, 5)$ \\
\hline
2 & 3 & $(2, 3)$ \\
\hline
5 & 0 & $(5, 0)$ \\
\hline
-1 & 6 & $(-1, 6)$ \\
\hline
\end{tabular}
\end{solucion}

\problem $2x - y = 4$

Complete la tabla:
\begin{center}
\begin{tabular}{|c|c|c|}
\hline
$x$ & $y$ & $(x, y)$ \\
\hline
0 & & \\
\hline
2 & & \\
\hline
& 0 & \\
\hline
-2 & & \\
\hline
\end{tabular}
\end{center}

\begin{solucion}
\begin{tabular}{|c|c|c|}
\hline
$x$ & $y$ & $(x, y)$ \\
\hline
0 & -4 & $(0, -4)$ \\
\hline
2 & 0 & $(2, 0)$ \\
\hline
2 & 0 & $(2, 0)$ \\
\hline
-2 & -8 & $(-2, -8)$ \\
\hline
\end{tabular}
\end{solucion}

\problem $3x + 2y = 12$

Complete la tabla:
\begin{center}
\begin{tabular}{|c|c|c|}
\hline
$x$ & $y$ & $(x, y)$ \\
\hline
0 & & \\
\hline
& 0 & \\
\hline
2 & & \\
\hline
& 3 & \\
\hline
\end{tabular}
\end{center}

\begin{solucion}
\begin{tabular}{|c|c|c|}
\hline
$x$ & $y$ & $(x, y)$ \\
\hline
0 & 6 & $(0, 6)$ \\
\hline
4 & 0 & $(4, 0)$ \\
\hline
2 & 3 & $(2, 3)$ \\
\hline
2 & 3 & $(2, 3)$ \\
\hline
\end{tabular}
\end{solucion}
\end{exercise}

%========================================
% Exercise 2: Finding Intercepts
%========================================
\begin{exercise}
\textbf{Encontrar Intersecciones con los Ejes}

Para cada ecuación, encuentre las intersecciones con el eje $x$ y el eje $y$:

\problem $2x + 3y = 6$

\begin{solucion}
Intersección con eje $x$ (hacer $y = 0$): $2x = 6 \implies x = 3$. Punto: $(3, 0)$

Intersección con eje $y$ (hacer $x = 0$): $3y = 6 \implies y = 2$. Punto: $(0, 2)$
\end{solucion}

\problem $4x - y = 8$

\begin{solucion}
Intersección con eje $x$ (hacer $y = 0$): $4x = 8 \implies x = 2$. Punto: $(2, 0)$

Intersección con eje $y$ (hacer $x = 0$): $-y = 8 \implies y = -8$. Punto: $(0, -8)$
\end{solucion}

\problem $x + 2y = 10$

\begin{solucion}
Intersección con eje $x$ (hacer $y = 0$): $x = 10$. Punto: $(10, 0)$

Intersección con eje $y$ (hacer $x = 0$): $2y = 10 \implies y = 5$. Punto: $(0, 5)$
\end{solucion}

\problem $5x - 2y = 20$

\begin{solucion}
Intersección con eje $x$ (hacer $y = 0$): $5x = 20 \implies x = 4$. Punto: $(4, 0)$

Intersección con eje $y$ (hacer $x = 0$): $-2y = 20 \implies y = -10$. Punto: $(0, -10)$
\end{solucion}

\problem $3x + 4y = 24$

\begin{solucion}
Intersección con eje $x$ (hacer $y = 0$): $3x = 24 \implies x = 8$. Punto: $(8, 0)$

Intersección con eje $y$ (hacer $x = 0$): $4y = 24 \implies y = 6$. Punto: $(0, 6)$
\end{solucion}

\problem $-2x + 5y = 10$

\begin{solucion}
Intersección con eje $x$ (hacer $y = 0$): $-2x = 10 \implies x = -5$. Punto: $(-5, 0)$

Intersección con eje $y$ (hacer $x = 0$): $5y = 10 \implies y = 2$. Punto: $(0, 2)$
\end{solucion}
\end{exercise}

%========================================
% Exercise 3: Calculating Slope
%========================================
\begin{exercise}
\textbf{Calcular la Pendiente}

Calcule la pendiente de la recta que pasa por cada par de puntos:

\problem $(1, 2)$ y $(3, 6)$

\begin{solucion}
$m = \dfrac{6 - 2}{3 - 1} = \dfrac{4}{2} = 2$
\end{solucion}

\problem $(0, 5)$ y $(4, 1)$

\begin{solucion}
$m = \dfrac{1 - 5}{4 - 0} = \dfrac{-4}{4} = -1$
\end{solucion}

\problem $(-2, 3)$ y $(4, -1)$

\begin{solucion}
$m = \dfrac{-1 - 3}{4 - (-2)} = \dfrac{-4}{6} = -\dfrac{2}{3}$
\end{solucion}

\problem $(5, 2)$ y $(5, 7)$

\begin{solucion}
$m = \dfrac{7 - 2}{5 - 5} = \dfrac{5}{0}$ = indefinida (recta vertical)
\end{solucion}

\problem $(-3, 4)$ y $(2, 4)$

\begin{solucion}
$m = \dfrac{4 - 4}{2 - (-3)} = \dfrac{0}{5} = 0$ (recta horizontal)
\end{solucion}

\problem $(-1, -2)$ y $(3, 10)$

\begin{solucion}
$m = \dfrac{10 - (-2)}{3 - (-1)} = \dfrac{12}{4} = 3$
\end{solucion}

\problem $(6, 1)$ y $(2, 5)$

\begin{solucion}
$m = \dfrac{5 - 1}{2 - 6} = \dfrac{4}{-4} = -1$
\end{solucion}

\problem $(0, 0)$ y $(4, 8)$

\begin{solucion}
$m = \dfrac{8 - 0}{4 - 0} = \dfrac{8}{4} = 2$
\end{solucion}
\end{exercise}

%========================================
% Exercise 4: Identifying Slope Type
%========================================
\begin{exercise}
\textbf{Identificar el Tipo de Pendiente}

Para cada ecuación o descripción, indique si la pendiente es positiva, negativa, cero o indefinida:

\problem Una recta que sube de izquierda a derecha.

\begin{solucion}
Pendiente positiva
\end{solucion}

\problem $y = -3$

\begin{solucion}
Pendiente cero (recta horizontal)
\end{solucion}

\problem $x = 5$

\begin{solucion}
Pendiente indefinida (recta vertical)
\end{solucion}

\problem Una recta que pasa por $(1, 5)$ y $(4, 2)$.

\begin{solucion}
$m = \frac{2-5}{4-1} = \frac{-3}{3} = -1$. Pendiente negativa
\end{solucion}

\problem $y = 7x + 2$

\begin{solucion}
$m = 7$. Pendiente positiva
\end{solucion}

\problem $y = -\dfrac{1}{2}x + 3$

\begin{solucion}
$m = -\frac{1}{2}$. Pendiente negativa
\end{solucion}
\end{exercise}

%========================================
% Exercise 5: Slope-Intercept Form
%========================================
\begin{exercise}
\textbf{Forma Pendiente-Intersecto}

Para cada ecuación en forma pendiente-intersecto, identifique la pendiente y la intersección con el eje $y$:

\problem $y = 3x + 5$

\begin{solucion}
Pendiente: $m = 3$; Intersección $y$: $b = 5$
\end{solucion}

\problem $y = -2x - 1$

\begin{solucion}
Pendiente: $m = -2$; Intersección $y$: $b = -1$
\end{solucion}

\problem $y = \dfrac{1}{2}x + 4$

\begin{solucion}
Pendiente: $m = \frac{1}{2}$; Intersección $y$: $b = 4$
\end{solucion}

\problem $y = -x + 7$

\begin{solucion}
Pendiente: $m = -1$; Intersección $y$: $b = 7$
\end{solucion}

\problem $y = 4x$

\begin{solucion}
Pendiente: $m = 4$; Intersección $y$: $b = 0$
\end{solucion}

\problem $y = -\dfrac{3}{4}x - 2$

\begin{solucion}
Pendiente: $m = -\frac{3}{4}$; Intersección $y$: $b = -2$
\end{solucion}
\end{exercise}

%========================================
% Exercise 6: Converting to Slope-Intercept Form
%========================================
\begin{exercise}
\textbf{Convertir a Forma Pendiente-Intersecto}

Convierta cada ecuación a la forma $y = mx + b$:

\problem $2x + y = 6$

\begin{solucion}
$y = -2x + 6$

Pendiente: $m = -2$; Intersección $y$: $b = 6$
\end{solucion}

\problem $3x - y = 9$

\begin{solucion}
$-y = -3x + 9 \implies y = 3x - 9$

Pendiente: $m = 3$; Intersección $y$: $b = -9$
\end{solucion}

\problem $4x + 2y = 8$

\begin{solucion}
$2y = -4x + 8 \implies y = -2x + 4$

Pendiente: $m = -2$; Intersección $y$: $b = 4$
\end{solucion}

\problem $x - 2y = 10$

\begin{solucion}
$-2y = -x + 10 \implies y = \frac{1}{2}x - 5$

Pendiente: $m = \frac{1}{2}$; Intersección $y$: $b = -5$
\end{solucion}

\problem $6x + 3y = 12$

\begin{solucion}
$3y = -6x + 12 \implies y = -2x + 4$

Pendiente: $m = -2$; Intersección $y$: $b = 4$
\end{solucion}

\problem $5x - 2y = 20$

\begin{solucion}
$-2y = -5x + 20 \implies y = \frac{5}{2}x - 10$

Pendiente: $m = \frac{5}{2}$; Intersección $y$: $b = -10$
\end{solucion}
\end{exercise}

%========================================
% Exercise 7: Writing Equations (Slope and Point)
%========================================
\begin{exercise}
\textbf{Escribir Ecuaciones dada la Pendiente y un Punto}

Escriba la ecuación de la recta en forma pendiente-intersecto:

\problem Pendiente $m = 2$, pasa por $(1, 3)$

\begin{solucion}
Usando forma punto-pendiente: $y - 3 = 2(x - 1)$

$y - 3 = 2x - 2$

$y = 2x + 1$
\end{solucion}

\problem Pendiente $m = -3$, pasa por $(2, 5)$

\begin{solucion}
$y - 5 = -3(x - 2)$

$y - 5 = -3x + 6$

$y = -3x + 11$
\end{solucion}

\problem Pendiente $m = \dfrac{1}{2}$, pasa por $(4, 1)$

\begin{solucion}
$y - 1 = \frac{1}{2}(x - 4)$

$y - 1 = \frac{1}{2}x - 2$

$y = \frac{1}{2}x - 1$
\end{solucion}

\problem Pendiente $m = -\dfrac{2}{3}$, pasa por $(-3, 4)$

\begin{solucion}
$y - 4 = -\frac{2}{3}(x - (-3))$

$y - 4 = -\frac{2}{3}(x + 3)$

$y - 4 = -\frac{2}{3}x - 2$

$y = -\frac{2}{3}x + 2$
\end{solucion}

\problem Pendiente $m = 0$, pasa por $(5, -2)$

\begin{solucion}
Recta horizontal: $y = -2$
\end{solucion}

\problem Pendiente indefinida, pasa por $(3, 7)$

\begin{solucion}
Recta vertical: $x = 3$
\end{solucion}
\end{exercise}

%========================================
% Exercise 8: Writing Equations (Two Points)
%========================================
\begin{exercise}
\textbf{Escribir Ecuaciones dados Dos Puntos}

Encuentre la ecuación de la recta que pasa por los puntos dados. Escriba la respuesta en forma pendiente-intersecto:

\problem $(1, 2)$ y $(3, 8)$

\begin{solucion}
Pendiente: $m = \frac{8-2}{3-1} = \frac{6}{2} = 3$

Usando $(1, 2)$: $y - 2 = 3(x - 1)$

$y = 3x - 1$
\end{solucion}

\problem $(0, 4)$ y $(2, 0)$

\begin{solucion}
Pendiente: $m = \frac{0-4}{2-0} = \frac{-4}{2} = -2$

Usando $(0, 4)$: $y - 4 = -2(x - 0)$

$y = -2x + 4$
\end{solucion}

\problem $(-1, 3)$ y $(2, -6)$

\begin{solucion}
Pendiente: $m = \frac{-6-3}{2-(-1)} = \frac{-9}{3} = -3$

Usando $(-1, 3)$: $y - 3 = -3(x + 1)$

$y - 3 = -3x - 3$

$y = -3x$
\end{solucion}

\problem $(4, 1)$ y $(6, 5)$

\begin{solucion}
Pendiente: $m = \frac{5-1}{6-4} = \frac{4}{2} = 2$

Usando $(4, 1)$: $y - 1 = 2(x - 4)$

$y - 1 = 2x - 8$

$y = 2x - 7$
\end{solucion}

\problem $(-2, -1)$ y $(3, -1)$

\begin{solucion}
Pendiente: $m = \frac{-1-(-1)}{3-(-2)} = \frac{0}{5} = 0$

Recta horizontal: $y = -1$
\end{solucion}

\problem $(5, 2)$ y $(5, 8)$

\begin{solucion}
Pendiente: $m = \frac{8-2}{5-5} = \frac{6}{0}$ = indefinida

Recta vertical: $x = 5$
\end{solucion}
\end{exercise}

%========================================
% Exercise 9: Parallel and Perpendicular Lines
%========================================
\begin{exercise}
\textbf{Rectas Paralelas y Perpendiculares}

\problem Encuentre la ecuación de la recta que pasa por $(2, 3)$ y es paralela a $y = 4x - 1$.

\begin{solucion}
Recta paralela tiene la misma pendiente: $m = 4$

$y - 3 = 4(x - 2)$

$y - 3 = 4x - 8$

$y = 4x - 5$
\end{solucion}

\problem Encuentre la ecuación de la recta que pasa por $(1, 5)$ y es perpendicular a $y = 2x + 3$.

\begin{solucion}
Pendiente original: $m_1 = 2$

Pendiente perpendicular: $m_2 = -\frac{1}{2}$

$y - 5 = -\frac{1}{2}(x - 1)$

$y - 5 = -\frac{1}{2}x + \frac{1}{2}$

$y = -\frac{1}{2}x + \frac{11}{2}$
\end{solucion}

\problem Determine si las rectas $y = 3x + 2$ y $y = 3x - 7$ son paralelas, perpendiculares o ninguna.

\begin{solucion}
Ambas tienen pendiente $m = 3$.

Como $m_1 = m_2$, las rectas son \textbf{paralelas}.
\end{solucion}

\problem Determine si las rectas $y = \frac{2}{3}x + 1$ y $y = -\frac{3}{2}x + 4$ son paralelas, perpendiculares o ninguna.

\begin{solucion}
$m_1 = \frac{2}{3}$ y $m_2 = -\frac{3}{2}$

Producto: $m_1 \cdot m_2 = \frac{2}{3} \cdot (-\frac{3}{2}) = -1$

Las rectas son \textbf{perpendiculares}.
\end{solucion}

\problem Encuentre la ecuación de la recta que pasa por $(-1, 2)$ y es perpendicular a $3x + 2y = 6$.

\begin{solucion}
Convertir a forma pendiente-intersecto: $2y = -3x + 6 \implies y = -\frac{3}{2}x + 3$

Pendiente original: $m_1 = -\frac{3}{2}$

Pendiente perpendicular: $m_2 = \frac{2}{3}$

$y - 2 = \frac{2}{3}(x + 1)$

$y - 2 = \frac{2}{3}x + \frac{2}{3}$

$y = \frac{2}{3}x + \frac{8}{3}$
\end{solucion}
\end{exercise}

%========================================
% Exercise 10: Application Problems
%========================================
\begin{exercise}
\textbf{Problemas de Aplicación}

\problem Un taxista cobra una tarifa base de \$3 más \$0.50 por cada milla recorrida. Escriba una ecuación que relacione el costo total $C$ con el número de millas $m$ recorridas.

\begin{solucion}
$C = 0.50m + 3$

Donde $m = 0.50$ (costo por milla) y $b = 3$ (tarifa base).
\end{solucion}

\problem La temperatura en grados Fahrenheit ($F$) y Celsius ($C$) están relacionadas por la ecuación $F = \frac{9}{5}C + 32$. ¿Cuál es la temperatura en Fahrenheit cuando la temperatura es 20°C?

\begin{solucion}
$F = \frac{9}{5}(20) + 32 = 36 + 32 = 68$

La temperatura es 68°F.
\end{solucion}

\problem Un depósito de agua tiene 500 galones inicialmente. El agua se drena a una tasa de 25 galones por hora. Escriba una ecuación que exprese la cantidad de agua $A$ en el depósito después de $t$ horas.

\begin{solucion}
$A = -25t + 500$

Donde $m = -25$ (tasa de drenaje, negativa porque disminuye) y $b = 500$ (cantidad inicial).
\end{solucion}

\problem Una planta de alquiler de autos cobra \$40 por día más \$0.25 por cada kilómetro recorrido. Si el costo total por un día fue de \$65, ¿cuántos kilómetros se recorrieron?

\begin{solucion}
Ecuación: $C = 0.25k + 40$

Sustituir $C = 65$:

$65 = 0.25k + 40$

$25 = 0.25k$

$k = 100$

Se recorrieron 100 kilómetros.
\end{solucion}
\end{exercise}
