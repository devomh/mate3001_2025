%========================================
% DETAILED SOLUTIONS: Expresiones Algebraicas y Polinomios
%========================================

\subsection*{Ejercicio 1}

\textbf{Problema:} Identificar componentes de expresiones algebraicas.

\textbf{Soluciones detalladas:}

\textbf{a)} $5x^3 - 2x^2 + 7x - 9$

Para identificar los componentes de una expresión algebraica, debemos separar cada elemento:

\begin{itemize}
    \item \textbf{Términos:} Son las partes separadas por signos + o -
    \begin{itemize}
        \item Primer término: $5x^3$
        \item Segundo término: $-2x^2$ 
        \item Tercer término: $7x$ (equivale a $7x^1$)
        \item Cuarto término: $-9$ (equivale a $-9x^0$)
    \end{itemize}
    
    \item \textbf{Coeficientes:} Los números que multiplican a las variables
    \begin{itemize}
        \item En $5x^3$: coeficiente = 5
        \item En $-2x^2$: coeficiente = -2
        \item En $7x$: coeficiente = 7
        \item En $-9$: coeficiente = -9
    \end{itemize}
    
    \item \textbf{Variable:} La letra que representa valores desconocidos = $x$
    
    \item \textbf{Exponentes:} Las potencias a las que están elevadas las variables
    \begin{itemize}
        \item En $5x^3$: exponente = 3
        \item En $-2x^2$: exponente = 2  
        \item En $7x$: exponente = 1 (implícito)
        \item En $-9$: exponente = 0 (constante)
    \end{itemize}
\end{itemize}

\textbf{b)} $-3a^2b + 4ab^2 - b^3$

Esta expresión tiene dos variables ($a$ y $b$):

\begin{itemize}
    \item \textbf{Términos:} $-3a^2b$, $4ab^2$, $-b^3$
    \item \textbf{Coeficientes:} $-3$, $4$, $-1$ (en $-b^3$, el coeficiente implícito es -1)
    \item \textbf{Variables:} $a$ y $b$
    \item \textbf{Exponentes:}
    \begin{itemize}
        \item Variable $a$: exponentes 2, 1, 0 respectivamente
        \item Variable $b$: exponentes 1, 2, 3 respectivamente
    \end{itemize}
\end{itemize}

\subsection*{Ejercicio 2}

\textbf{Problema:} Determinar dominios de expresiones algebraicas.

\textbf{Conceptos fundamentales:}
\begin{itemize}
    \item El dominio incluye todos los valores reales excepto aquellos que hacen la expresión indefinida
    \item Principales restricciones: división por cero, raíces pares de números negativos
\end{itemize}

\textbf{a)} $f(x) = \frac{3x + 2}{x - 5}$

El denominador no puede ser cero: $x - 5 \neq 0 \Rightarrow x \neq 5$

\textbf{Dominio:} $\{x \in \mathbb{R} \mid x \neq 5\}$ o en notación de intervalos: $(-\infty, 5) \cup (5, \infty)$

\textbf{b)} $g(x) = \frac{x^2 - 1}{x^2 - 4}$

El denominador no puede ser cero: $x^2 - 4 \neq 0$

Resolviendo: $x^2 - 4 = 0 \Rightarrow x^2 = 4 \Rightarrow x = \pm 2$

\textbf{Dominio:} $\{x \in \mathbb{R} \mid x \neq 2, x \neq -2\}$ o $(-\infty, -2) \cup (-2, 2) \cup (2, \infty)$

\textbf{c)} $h(x) = \sqrt{2x - 6}$

Para que la raíz cuadrada esté definida en los reales: $2x - 6 \geq 0$

Resolviendo: $2x \geq 6 \Rightarrow x \geq 3$

\textbf{Dominio:} $[3, \infty)$

\textbf{e)} $m(x) = \frac{1}{\sqrt{x - 4}}$

Aquí necesitamos que el denominador sea positivo (no puede ser cero ni negativo):
$x - 4 > 0 \Rightarrow x > 4$

\textbf{Dominio:} $(4, \infty)$

\subsection*{Ejercicio 3}

\textbf{Problema:} Clasificar polinomios.

\textbf{Recordatorio:} El grado de un polinomio es el mayor exponente de la variable.

\textbf{a)} $P(x) = 7x^2 - 3x + 1$

\begin{itemize}
    \item \textbf{Grado:} 2 (el mayor exponente es 2) → Polinomio cuadrático
    \item \textbf{Número de términos:} 3 → Trinomio
    \item \textbf{Coeficiente principal:} 7 (coeficiente del término de mayor grado)
    \item \textbf{Término constante:} 1 (término sin variable)
\end{itemize}

\textbf{b)} $Q(x) = 4x^5 - x^3 + 2x^2 - x + 8$

\begin{itemize}
    \item \textbf{Grado:} 5 → Polinomio de grado 5
    \item \textbf{Número de términos:} 5 → Polinomio (más de 3 términos)
    \item \textbf{Coeficiente principal:} 4
    \item \textbf{Término constante:} 8
\end{itemize}

\textbf{c)} $R(x) = -2x^4 + 5x^2$

\begin{itemize}
    \item \textbf{Grado:} 4 → Polinomio cuártico
    \item \textbf{Número de términos:} 2 → Binomio
    \item \textbf{Coeficiente principal:} -2
    \item \textbf{Término constante:} 0 (no hay término independiente)
\end{itemize}

\subsection*{Ejercicio 4}

\textbf{Problema:} Suma y resta de polinomios.

\textbf{Procedimiento:} Combinar términos semejantes (mismo exponente de la variable).

\textbf{a)} $(3x^2 - 5x + 2) + (2x^2 + 7x - 4)$

Agrupando términos semejantes:
\begin{align}
&= (3x^2 + 2x^2) + (-5x + 7x) + (2 - 4)\\
&= 5x^2 + 2x - 2
\end{align}

\textbf{b)} $(4x^3 - 2x^2 + x - 3) - (x^3 + 3x^2 - 2x + 1)$

Distribuyendo el signo negativo y agrupando:
\begin{align}
&= 4x^3 - 2x^2 + x - 3 - x^3 - 3x^2 + 2x - 1\\
&= (4x^3 - x^3) + (-2x^2 - 3x^2) + (x + 2x) + (-3 - 1)\\
&= 3x^3 - 5x^2 + 3x - 4
\end{align}

\subsection*{Ejercicio 5}

\textbf{Problema:} Aplicar leyes de exponentes.

\textbf{Leyes fundamentales utilizadas:}
\begin{itemize}
    \item Producto: $a^m \cdot a^n = a^{m+n}$
    \item Cociente: $\frac{a^m}{a^n} = a^{m-n}$
    \item Potencia de potencia: $(a^m)^n = a^{mn}$
    \item Potencia de producto: $(ab)^n = a^n b^n$
\end{itemize}

\textbf{d)} $(3x^2)^4$

Aplicando potencia de producto:
\begin{align}
(3x^2)^4 &= 3^4 \cdot (x^2)^4\\
&= 81 \cdot x^{2 \cdot 4}\\
&= 81x^8
\end{align}

\textbf{e)} $\frac{12x^6y^4}{4x^2y^7}$

Separando coeficientes y variables:
\begin{align}
\frac{12x^6y^4}{4x^2y^7} &= \frac{12}{4} \cdot \frac{x^6}{x^2} \cdot \frac{y^4}{y^7}\\
&= 3 \cdot x^{6-2} \cdot y^{4-7}\\
&= 3x^4y^{-3}\\
&= \frac{3x^4}{y^3}
\end{align}

\textbf{f)} $(2x^{-3}y^2)^{-2}$

Aplicando potencia de producto y potencia de potencia:
\begin{align}
(2x^{-3}y^2)^{-2} &= 2^{-2} \cdot (x^{-3})^{-2} \cdot (y^2)^{-2}\\
&= \frac{1}{4} \cdot x^{(-3)(-2)} \cdot y^{2(-2)}\\
&= \frac{1}{4} \cdot x^6 \cdot y^{-4}\\
&= \frac{x^6}{4y^4}
\end{align}

\subsection*{Ejercicio 6}

\textbf{Problema:} Multiplicación de polinomios.

\textbf{b)} $(x + 5)(x - 3)$

Método FOIL:
\begin{itemize}
    \item \textbf{F}irst: $x \cdot x = x^2$
    \item \textbf{O}uter: $x \cdot (-3) = -3x$
    \item \textbf{I}nner: $5 \cdot x = 5x$
    \item \textbf{L}ast: $5 \cdot (-3) = -15$
\end{itemize}

Resultado: $x^2 - 3x + 5x - 15 = x^2 + 2x - 15$

\textbf{d)} $(x^2 + 2x - 1)(x + 2)$

Distribuyendo cada término del segundo polinomio:
\begin{align}
&= x^2(x + 2) + 2x(x + 2) - 1(x + 2)\\
&= x^3 + 2x^2 + 2x^2 + 4x - x - 2\\
&= x^3 + 4x^2 + 3x - 2
\end{align}

\textbf{e)} $(a + b)(a^2 - ab + b^2)$

Este es el patrón para suma de cubos. Desarrollando:
\begin{align}
&= a(a^2 - ab + b^2) + b(a^2 - ab + b^2)\\
&= a^3 - a^2b + ab^2 + a^2b - ab^2 + b^3\\
&= a^3 + b^3
\end{align}

Observe cómo los términos intermedios se cancelan, confirmando la fórmula $a^3 + b^3 = (a + b)(a^2 - ab + b^2)$.

\subsection*{Ejercicio 7}

\textbf{Problema:} Productos notables.

\textbf{a)} $(x + 4)^2$

Usando la fórmula $(a + b)^2 = a^2 + 2ab + b^2$:
\begin{align}
(x + 4)^2 &= x^2 + 2(x)(4) + 4^2\\
&= x^2 + 8x + 16
\end{align}

\textbf{c)} $(x + 6)(x - 6)$

Diferencia de cuadrados: $(a + b)(a - b) = a^2 - b^2$
\begin{align}
(x + 6)(x - 6) &= x^2 - 6^2\\
&= x^2 - 36
\end{align}

\textbf{e)} $(x + 3)^3$

Usando la fórmula $(a + b)^3 = a^3 + 3a^2b + 3ab^2 + b^3$:
\begin{align}
(x + 3)^3 &= x^3 + 3x^2(3) + 3x(3^2) + 3^3\\
&= x^3 + 9x^2 + 27x + 27
\end{align}

\subsection*{Ejercicio 8}

\textbf{Problema:} Factorización de polinomios.

\textbf{Estrategia general:}
1. Buscar factor común
2. Identificar patrones especiales (diferencia de cuadrados, trinomio cuadrático perfecto, etc.)
3. Usar agrupación si es necesario
4. Verificar multiplicando

\textbf{b)} $x^2 + 7x + 12$

Para factorizar $x^2 + 7x + 12$, buscamos dos números que:
\begin{itemize}
    \item Se multipliquen para dar 12
    \item Se sumen para dar 7
\end{itemize}

Factores de 12: $1 \times 12$, $2 \times 6$, $3 \times 4$

Probando sumas: $1 + 12 = 13$, $2 + 6 = 8$, $3 + 4 = 7$ ✓

Por tanto: $x^2 + 7x + 12 = (x + 3)(x + 4)$

\textbf{Verificación:} $(x + 3)(x + 4) = x^2 + 4x + 3x + 12 = x^2 + 7x + 12$ ✓

\textbf{f)} $2x^3 - 8x$

Paso 1: Factor común
$2x^3 - 8x = 2x(x^2 - 4)$

Paso 2: Diferencia de cuadrados en $(x^2 - 4)$
$x^2 - 4 = x^2 - 2^2 = (x + 2)(x - 2)$

\textbf{Resultado:} $2x^3 - 8x = 2x(x + 2)(x - 2)$

\textbf{g)} $x^4 - 16$

Reconocemos esto como diferencia de cuadrados:
$x^4 - 16 = (x^2)^2 - 4^2 = (x^2 + 4)(x^2 - 4)$

Pero $x^2 - 4$ es también diferencia de cuadrados:
$x^2 - 4 = (x + 2)(x - 2)$

\textbf{Resultado:} $x^4 - 16 = (x^2 + 4)(x + 2)(x - 2)$

Nota: $x^2 + 4$ no se puede factorizar más en los números reales.

\subsection*{Ejercicio 9}

\textbf{Problema:} Evaluación de polinomios.

\textbf{a)} $P(x) = 2x^3 - 3x^2 + x - 4$; evaluar $P(2)$

Sustituyendo $x = 2$:
\begin{align}
P(2) &= 2(2)^3 - 3(2)^2 + 2 - 4\\
&= 2(8) - 3(4) + 2 - 4\\
&= 16 - 12 + 2 - 4\\
&= 2
\end{align}

\textbf{c)} $R(x) = 3x^2 - 5x + 2$; evaluar $R(0)$ y $R(1)$

Para $R(0)$:
\begin{align}
R(0) &= 3(0)^2 - 5(0) + 2\\
&= 0 - 0 + 2\\
&= 2
\end{align}

Para $R(1)$:
\begin{align}
R(1) &= 3(1)^2 - 5(1) + 2\\
&= 3 - 5 + 2\\
&= 0
\end{align}

Observe que $R(1) = 0$, lo que significa que $(x - 1)$ es un factor de $R(x)$.

\subsection*{Ejercicio 10}

\textbf{Problema:} Aplicación del Teorema del Factor.

\textbf{Teorema del Factor:} $(x - a)$ es factor de $P(x)$ si y solo si $P(a) = 0$.

\textbf{a)} ¿Es $(x - 2)$ factor de $x^3 - 3x^2 + 4x - 12$?

Evaluamos $P(2)$:
\begin{align}
P(2) &= 2^3 - 3(2^2) + 4(2) - 12\\
&= 8 - 3(4) + 8 - 12\\
&= 8 - 12 + 8 - 12\\
&= -8 \neq 0
\end{align}

Como $P(2) \neq 0$, $(x - 2)$ \textbf{no es} factor de $P(x)$.

\textbf{d)} ¿Es $(2x - 1)$ factor de $2x^3 - 3x^2 + 1$?

Para $(2x - 1)$, necesitamos evaluar en $x = \frac{1}{2}$ (donde $2x - 1 = 0$):

\begin{align}
P\left(\frac{1}{2}\right) &= 2\left(\frac{1}{2}\right)^3 - 3\left(\frac{1}{2}\right)^2 + 1\\
&= 2 \cdot \frac{1}{8} - 3 \cdot \frac{1}{4} + 1\\
&= \frac{1}{4} - \frac{3}{4} + 1\\
&= \frac{1 - 3 + 4}{4}\\
&= \frac{2}{4} = \frac{1}{2} \neq 0
\end{align}

Como $P(\frac{1}{2}) \neq 0$, $(2x - 1)$ \textbf{no es} factor de $P(x)$.

\subsection*{Ejercicio 11}

\textbf{Problema:} Problemas de aplicación.

\textbf{a)} Área del rectángulo

Área = largo × ancho = $(2x + 3)(x - 1)$

Usando FOIL:
\begin{align}
\text{Área} &= (2x + 3)(x - 1)\\
&= 2x \cdot x + 2x \cdot (-1) + 3 \cdot x + 3 \cdot (-1)\\
&= 2x^2 - 2x + 3x - 3\\
&= 2x^2 + x - 3
\end{align}

\textbf{c)} Problema del proyectil

$h(t) = -5t^2 + 20t + 25$

\textbf{Altura inicial:} $h(0) = -5(0)^2 + 20(0) + 25 = 25$ metros

\textbf{Altura después de 2 segundos:}
\begin{align}
h(2) &= -5(2)^2 + 20(2) + 25\\
&= -5(4) + 40 + 25\\
&= -20 + 40 + 25\\
&= 45 \text{ metros}
\end{align}

\textbf{d)} Problema de costos

$C(x) = 2x^2 + 15x + 100$

\textbf{Costo fijo:} Es el costo cuando no se produce nada ($x = 0$)
$C(0) = 2(0)^2 + 15(0) + 100 = 100$

\textbf{Costo de producir 10 artículos:}
\begin{align}
C(10) &= 2(10)^2 + 15(10) + 100\\
&= 2(100) + 150 + 100\\
&= 200 + 150 + 100\\
&= 450
\end{align}