%========================================
% DETAILED SOLUTIONS: Fórmulas
%========================================

\subsection*{Ejercicio 1}

\textbf{Problema:} Evaluación de fórmulas básicas

\textbf{Estrategia general:} Para evaluar una fórmula, sustituimos los valores dados directamente en la ecuación y realizamos las operaciones aritméticas en el orden correcto (PEMDSR).

\textbf{Ejercicio 1.1 - Fórmula de interés simple $I = Prt$:}

\textbf{a)} $P = 1000$, $r = 0.05$, $t = 3$
\begin{align}
I &= Prt\\
&= 1000 \times 0.05 \times 3\\
&= 50 \times 3\\
&= 150
\end{align}
El interés ganado es $\$150$.

\textbf{b)} $P = 2500$, $r = 0.04$, $t = 2$
\begin{align}
I &= 2500 \times 0.04 \times 2\\
&= 100 \times 2\\
&= 200
\end{align}

\textbf{Ejercicio 1.2 - Fórmula de monto total $A = P + Prt$:}

\textbf{Método alternativo usando factorización:}
\begin{align}
A &= P + Prt\\
&= P(1 + rt)
\end{align}

\textbf{a)} $P = 500$, $r = 0.08$, $t = 2$
\begin{align}
A &= 500(1 + 0.08 \times 2)\\
&= 500(1 + 0.16)\\
&= 500(1.16)\\
&= 580
\end{align}

\textbf{b)} $P = 1500$, $r = 0.05$, $t = 3$
\begin{align}
A &= 1500(1 + 0.05 \times 3)\\
&= 1500(1 + 0.15)\\
&= 1500(1.15)\\
&= 1725
\end{align}

\textbf{Ejercicio 1.3 - Conversión de temperatura:}

\textbf{Nota importante:} La fórmula $C = \frac{5}{9}(F-32)$ requiere seguir el orden de operaciones: primero paréntesis, luego multiplicación/división.

\textbf{a)} $F = 32°$ (punto de congelación del agua)
\begin{align}
C &= \frac{5}{9}(32-32)\\
&= \frac{5}{9}(0)\\
&= 0°C
\end{align}

\textbf{b)} $F = 212°$ (punto de ebullición del agua)
\begin{align}
C &= \frac{5}{9}(212-32)\\
&= \frac{5}{9}(180)\\
&= \frac{900}{9}\\
&= 100°C
\end{align}

\subsection*{Ejercicio 2}

\textbf{Problema:} Encontrar variables desconocidas en fórmulas

\textbf{Estrategia general:} Cuando conocemos todos los valores excepto uno, sustituimos los valores conocidos y resolvemos la ecuación resultante para la variable desconocida.

\textbf{Ejercicio 2.1 - Usando $I = Prt$:}

\textbf{a)} Encontrar $P$ si $I = 240$, $r = 0.06$, $t = 4$
\begin{align}
I &= Prt\\
240 &= P \times 0.06 \times 4\\
240 &= P \times 0.24\\
P &= \frac{240}{0.24}\\
P &= 1000
\end{align}

\textbf{Verificación:} $I = 1000 \times 0.06 \times 4 = 240$ ✓

\textbf{b)} Encontrar $r$ si $I = 180$, $P = 1500$, $t = 2$
\begin{align}
180 &= 1500 \times r \times 2\\
180 &= 3000r\\
r &= \frac{180}{3000} = \frac{18}{300} = \frac{3}{50} = 0.06
\end{align}

La tasa de interés es $6\%$ anual.

\textbf{Ejercicio 2.2 - Usando $A = P + Prt$:}

\textbf{a)} Encontrar $P$ si $A = 1320$, $r = 0.08$, $t = 2$

\textbf{Método 1 - Factorización:}
\begin{align}
A &= P + Prt\\
A &= P(1 + rt)\\
1320 &= P(1 + 0.08 \times 2)\\
1320 &= P(1 + 0.16)\\
1320 &= 1.16P\\
P &= \frac{1320}{1.16} = 1137.93
\end{align}

\textbf{b)} Encontrar $t$ si $A = 1150$, $P = 1000$, $r = 0.05$
\begin{align}
A &= P + Prt\\
1150 &= 1000 + 1000 \times 0.05 \times t\\
1150 &= 1000(1 + 0.05t)\\
\frac{1150}{1000} &= 1 + 0.05t\\
1.15 &= 1 + 0.05t\\
0.15 &= 0.05t\\
t &= \frac{0.15}{0.05} = 3
\end{align}

\subsection*{Ejercicio 3}

\textbf{Problema:} Despeje de expresiones algebraicas que requieren factorización

\textbf{Estrategia general:} Cuando la variable deseada aparece en múltiples términos, debemos factorizarla como factor común antes de despejarla.

\textbf{Ejercicio 3.1 - Fórmulas geométricas:}

\textbf{a)} Despejar $l$ de $V = lwh$:
\begin{align}
V &= lwh\\
\frac{V}{wh} &= l \quad \text{(dividir por $wh$)}
\end{align}
Resultado: $l = \frac{V}{wh}$

\textbf{b)} Despejar $w$ de $P = 2l + 2w$:
\begin{align}
P &= 2l + 2w\\
P - 2l &= 2w \quad \text{(restar $2l$)}\\
\frac{P - 2l}{2} &= w \quad \text{(dividir por 2)}
\end{align}
Resultado: $w = \frac{P - 2l}{2}$

\textbf{c)} Despejar $h$ de $A = \frac{1}{2}bh$:
\begin{align}
A &= \frac{1}{2}bh\\
2A &= bh \quad \text{(multiplicar por 2)}\\
\frac{2A}{b} &= h \quad \text{(dividir por $b$)}
\end{align}
Resultado: $h = \frac{2A}{b}$

\textbf{d)} Despejar $h$ de $S = 2\pi r^2 + 2\pi rh$:
\begin{align}
S &= 2\pi r^2 + 2\pi rh\\
S - 2\pi r^2 &= 2\pi rh \quad \text{(restar $2\pi r^2$)}\\
\frac{S - 2\pi r^2}{2\pi r} &= h \quad \text{(dividir por $2\pi r$)}
\end{align}
Resultado: $h = \frac{S - 2\pi r^2}{2\pi r}$

\textbf{Ejercicio 3.2 - Expresiones algebraicas que requieren factorización:}

\textbf{a)} Despejar $x$ de $ax + b = c$ (donde $a \neq 0$):
\begin{align}
ax + b &= c\\
ax &= c - b\\
x &= \frac{c - b}{a}
\end{align}

\textbf{b)} Despejar $x$ de $ax + bx = c$:
\begin{align}
ax + bx &= c\\
x(a + b) &= c \quad \text{(factorizar $x$)}\\
x &= \frac{c}{a + b}
\end{align}

\textbf{c)} Despejar $x$ de $ax + b = cx + d$ (donde $a \neq c$):
\begin{align}
ax + b &= cx + d\\
ax - cx &= d - b \quad \text{(reunir términos con $x$)}\\
x(a - c) &= d - b \quad \text{(factorizar $x$)}\\
x &= \frac{d - b}{a - c}
\end{align}

\textbf{d)} Despejar $x$ de $a(x + b) = c(x + d)$ (donde $a \neq c$):
\begin{align}
a(x + b) &= c(x + d)\\
ax + ab &= cx + cd \quad \text{(expandir)}\\
ax - cx &= cd - ab \quad \text{(reunir términos con $x$)}\\
x(a - c) &= cd - ab \quad \text{(factorizar $x$)}\\
x &= \frac{cd - ab}{a - c}
\end{align}

\textbf{e)} Despejar $x$ de $ax + ay = bx + by$ (donde $a \neq b$):
\begin{align}
ax + bx &= c\\
x(a + b) &= c \quad \text{(factorizar $x$)}\\
x &= \frac{c}{a + b}
\end{align}

\textbf{f)} Despejar $x$ de $ax + ay = bx + by$ (donde $a \neq b$):
\begin{align}
ax + ay &= bx + by\\
ax - bx &= by - ay \quad \text{(reunir términos similares)}\\
x(a - b) &= y(b - a) \quad \text{(factorizar)}\\
x(a - b) &= -y(a - b) \quad \text{(factor -1)}\\
x &= -y
\end{align}

\textbf{f)} Despejar $x$ de $a(x + y) + b(x + z) = c$:
\begin{align}
a(x + y) + b(x + z) &= c\\
ax + ay + bx + bz &= c \quad \text{(expandir)}\\
ax + bx &= c - ay - bz \quad \text{(reunir términos con $x$)}\\
x(a + b) &= c - ay - bz \quad \text{(factorizar $x$)}\\
x &= \frac{c - ay - bz}{a + b}
\end{align}

\textbf{g)} Despejar $x$ de $\frac{x}{a} + \frac{x}{b} = c$ (donde $a, b \neq 0$):
\begin{align}
\frac{x}{a} + \frac{x}{b} &= c\\
x\left(\frac{1}{a} + \frac{1}{b}\right) &= c \quad \text{(factorizar $x$)}\\
x\left(\frac{b + a}{ab}\right) &= c \quad \text{(común denominador)}\\
x &= \frac{c \cdot ab}{a + b} = \frac{abc}{a + b}
\end{align}

\subsection*{Ejercicio 4}

\textbf{Problema:} Problemas de despeje aplicado

\textbf{a)} Área del rectángulo: $A = lw$
\textbf{Datos:} $A = 48$, $w = 6$
\begin{align}
48 &= l \times 6\\
l &= \frac{48}{6} = 8
\end{align}

\textbf{b)} Volumen del cilindro: $V = \pi r^2 h$
\textbf{Datos:} $V = 500\pi$, $r = 5$
\begin{align}
500\pi &= \pi (5)^2 h\\
500\pi &= 25\pi h\\
h &= \frac{500\pi}{25\pi} = \frac{500}{25} = 20
\end{align}

\textbf{c)} Movimiento uniforme: $d = vt$
\textbf{Datos:} $d = 240$ km, $t = 3$ horas
\begin{align}
240 &= v \times 3\\
v &= \frac{240}{3} = 80 \text{ km/h}
\end{align}

\textbf{Notas importantes para el estudiante:}
\begin{enumerate}
\item Cuando la variable a despejar aparece en múltiples términos, siempre factorízala primero
\item El despeje de expresiones algebraicas requiere las mismas técnicas que resolver ecuaciones lineales
\item Siempre verifica que no estés dividiendo por cero (restricciones del dominio)
\item La factorización es fundamental para resolver ecuaciones más complejas
\item Cuando obtienes una ecuación cuadrática factorizada, usa el teorema del factor cero
\end{enumerate}