%========================================
% DETAILED SOLUTIONS: Ecuaciones Racionales
%========================================

\subsection*{Práctica Rápida}

%========================================
% PROBLEM 1
%========================================
\textbf{Problema 1:} \(\displaystyle \frac{1}{x+1} + \frac{1}{x-1} = 1\)

\textbf{Solución Detallada:}

\textbf{Paso 1: Identificar restricciones}

Los denominadores son \(x+1\) y \(x-1\). Estos se anulan cuando:
\begin{align*}
x + 1 &= 0 \quad \Rightarrow \quad x = -1\\
x - 1 &= 0 \quad \Rightarrow \quad x = 1
\end{align*}

Por lo tanto, las restricciones son: \(x \neq -1\) y \(x \neq 1\)

\textbf{Paso 2: Determinar el MCD}

El MCD de \((x+1)\) y \((x-1)\) es: \((x+1)(x-1)\)

\textbf{Paso 3: Multiplicar ambos lados por el MCD}

\[\left(\frac{1}{x+1} + \frac{1}{x-1}\right) \cdot (x+1)(x-1) = 1 \cdot (x+1)(x-1)\]

Distribuyendo:
\[\frac{1}{\cancel{x+1}} \cdot (\cancel{x+1})(x-1) + \frac{1}{\cancel{x-1}} \cdot (x+1)(\cancel{x-1}) = (x+1)(x-1)\]

\[(x-1) + (x+1) = (x+1)(x-1)\]

\textbf{Paso 4: Simplificar el lado izquierdo}

\[x - 1 + x + 1 = (x+1)(x-1)\]
\[2x = (x+1)(x-1)\]

\textbf{Paso 5: Expandir el lado derecho}

\[2x = x^2 - 1\]

\textbf{Paso 6: Llevar a forma estándar}

\[0 = x^2 - 2x - 1\]

\textbf{Paso 7: Aplicar la fórmula cuadrática}

Con \(a = 1\), \(b = -2\), \(c = -1\):

\begin{align*}
x &= \frac{-b \pm \sqrt{b^2 - 4ac}}{2a}\\
&= \frac{-(-2) \pm \sqrt{(-2)^2 - 4(1)(-1)}}{2(1)}\\
&= \frac{2 \pm \sqrt{4 + 4}}{2}\\
&= \frac{2 \pm \sqrt{8}}{2}\\
&= \frac{2 \pm 2\sqrt{2}}{2}\\
&= 1 \pm \sqrt{2}
\end{align*}

\textbf{Paso 8: Verificar restricciones}

\begin{itemize}
    \item \(x = 1 + \sqrt{2} \approx 2.414\): Es diferente de \(-1\) y \(1\). \checkmark
    \item \(x = 1 - \sqrt{2} \approx -0.414\): Es diferente de \(-1\) y \(1\). \checkmark
\end{itemize}

\textbf{Respuesta final:} \(x = 1 + \sqrt{2}\) o \(x = 1 - \sqrt{2}\)

%========================================
% PROBLEM 2
%========================================
\textbf{Problema 2:} \(\displaystyle \frac{x}{x-2} = 2\)

\textbf{Solución Detallada:}

\textbf{Paso 1: Restricción}

\[x - 2 = 0 \quad \Rightarrow \quad x \neq 2\]

\textbf{Paso 2: MCD}

El único denominador es \((x-2)\), que es también el MCD.

\textbf{Paso 3: Multiplicar por el MCD}

\[\frac{x}{\cancel{x-2}} \cdot (\cancel{x-2}) = 2(x-2)\]

\[x = 2(x-2)\]

\textbf{Paso 4: Resolver}

\begin{align*}
x &= 2x - 4\\
x - 2x &= -4\\
-x &= -4\\
x &= 4
\end{align*}

\textbf{Paso 5: Verificar}

Sustituir \(x = 4\) en la ecuación original:
\[\frac{4}{4-2} = \frac{4}{2} = 2 \quad \checkmark\]

Además, \(x = 4 \neq 2\), por lo que no viola la restricción.

\textbf{Respuesta final:} \(x = 4\)

%========================================
% PROBLEM 3
%========================================
\textbf{Problema 3:} \(\displaystyle \frac{x+1}{x} = \frac{3}{x-2}\)

\textbf{Solución Detallada:}

\textbf{Paso 1: Restricciones}

\begin{align*}
x &= 0 \quad \Rightarrow \quad x \neq 0\\
x - 2 &= 0 \quad \Rightarrow \quad x \neq 2
\end{align*}

\textbf{Paso 2: MCD}

El MCD es: \(x(x-2)\)

\textbf{Paso 3: Multiplicar por el MCD}

\[\frac{x+1}{\cancel{x}} \cdot \cancel{x}(x-2) = \frac{3}{\cancel{x-2}} \cdot x(\cancel{x-2})\]

\[(x+1)(x-2) = 3x\]

\textbf{Paso 4: Expandir el lado izquierdo}

\begin{align*}
(x+1)(x-2) &= x \cdot x + x \cdot (-2) + 1 \cdot x + 1 \cdot (-2)\\
&= x^2 - 2x + x - 2\\
&= x^2 - x - 2
\end{align*}

\textbf{Paso 5: Igualar y simplificar}

\[x^2 - x - 2 = 3x\]
\[x^2 - x - 2 - 3x = 0\]
\[x^2 - 4x - 2 = 0\]

\textbf{Paso 6: Aplicar la fórmula cuadrática}

Con \(a = 1\), \(b = -4\), \(c = -2\):

\begin{align*}
x &= \frac{4 \pm \sqrt{(-4)^2 - 4(1)(-2)}}{2(1)}\\
&= \frac{4 \pm \sqrt{16 + 8}}{2}\\
&= \frac{4 \pm \sqrt{24}}{2}\\
&= \frac{4 \pm 2\sqrt{6}}{2}\\
&= 2 \pm \sqrt{6}
\end{align*}

\textbf{Paso 7: Verificar restricciones}

\begin{itemize}
    \item \(x = 2 + \sqrt{6} \approx 4.449\): Diferente de 0 y 2. \checkmark
    \item \(x = 2 - \sqrt{6} \approx -0.449\): Diferente de 0 y 2. \checkmark
\end{itemize}

\textbf{Respuesta final:} \(x = 2 + \sqrt{6}\) o \(x = 2 - \sqrt{6}\)

%========================================
% PROBLEM 4
%========================================
\textbf{Problema 4:} \(\displaystyle \frac{1}{x-3} - \frac{1}{x+3} = \frac{1}{3}\)

\textbf{Solución Detallada:}

\textbf{Paso 1: Restricciones}

\[x \neq 3 \quad \text{y} \quad x \neq -3\]

\textbf{Paso 2: MCD}

El MCD de \((x-3)\), \((x+3)\), y \(3\) es: \(3(x-3)(x+3)\)

\textbf{Paso 3: Multiplicar por el MCD}

\[\left(\frac{1}{x-3} - \frac{1}{x+3}\right) \cdot 3(x-3)(x+3) = \frac{1}{3} \cdot 3(x-3)(x+3)\]

\[3(x+3) - 3(x-3) = (x-3)(x+3)\]

\textbf{Paso 4: Expandir y simplificar}

\begin{align*}
3x + 9 - 3x + 9 &= (x-3)(x+3)\\
18 &= x^2 - 9
\end{align*}

\textbf{Paso 5: Resolver}

\begin{align*}
18 &= x^2 - 9\\
27 &= x^2\\
x &= \pm\sqrt{27}\\
x &= \pm\sqrt{9 \cdot 3}\\
x &= \pm 3\sqrt{3}
\end{align*}

\textbf{Paso 6: Verificar restricciones}

\begin{itemize}
    \item \(x = 3\sqrt{3} \approx 5.196\): Diferente de \(\pm 3\). \checkmark
    \item \(x = -3\sqrt{3} \approx -5.196\): Diferente de \(\pm 3\). \checkmark
\end{itemize}

\textbf{Respuesta final:} \(x = 3\sqrt{3}\) o \(x = -3\sqrt{3}\)

%========================================
% EXIT TICKET
%========================================
\subsection*{Exit Ticket}

\textbf{Problema:} \(\displaystyle \frac{5}{x-2} - \frac{1}{x} = 1\)

\textbf{Solución Detallada:}

\textbf{Paso 1: Restricciones}

\[x \neq 2 \quad \text{y} \quad x \neq 0\]

\textbf{Paso 2: MCD}

El MCD es: \(x(x-2)\)

\textbf{Paso 3: Multiplicar por el MCD}

\[\frac{5}{\cancel{x-2}} \cdot x(\cancel{x-2}) - \frac{1}{\cancel{x}} \cdot \cancel{x}(x-2) = 1 \cdot x(x-2)\]

\[5x - (x-2) = x(x-2)\]

\textbf{Paso 4: Expandir y simplificar}

\begin{align*}
5x - x + 2 &= x^2 - 2x\\
4x + 2 &= x^2 - 2x\\
0 &= x^2 - 2x - 4x - 2\\
0 &= x^2 - 6x - 2
\end{align*}

\textbf{Paso 5: Aplicar la fórmula cuadrática}

Con \(a = 1\), \(b = -6\), \(c = -2\):

\begin{align*}
x &= \frac{6 \pm \sqrt{36 - 4(1)(-2)}}{2}\\
&= \frac{6 \pm \sqrt{36 + 8}}{2}\\
&= \frac{6 \pm \sqrt{44}}{2}\\
&= \frac{6 \pm 2\sqrt{11}}{2}\\
&= 3 \pm \sqrt{11}
\end{align*}

\textbf{Paso 6: Verificar restricciones}

\begin{itemize}
    \item \(x = 3 + \sqrt{11} \approx 6.317\): Diferente de 0 y 2. \checkmark
    \item \(x = 3 - \sqrt{11} \approx -0.317\): Diferente de 0 y 2. \checkmark
\end{itemize}

\textbf{Verificación (para \(x = 3 + \sqrt{11}\)):}

\[\frac{5}{(3+\sqrt{11})-2} - \frac{1}{3+\sqrt{11}} = \frac{5}{1+\sqrt{11}} - \frac{1}{3+\sqrt{11}}\]

Esto es tedioso de verificar sin calculadora, pero podemos confiar en nuestro álgebra dado que ambos valores cumplen las restricciones.

\textbf{Respuesta final:} \(x = 3 + \sqrt{11}\) o \(x = 3 - \sqrt{11}\)

%========================================
% PROPOSED EXERCISES
%========================================
\subsection*{Ejercicios Propuestos}

%========================================
% EXERCISE 1
%========================================
\textbf{Ejercicio 1:} \(\displaystyle \frac{3}{x-2} + \frac{4}{x+3} = 1\)

\textbf{Solución Completa:}

\textbf{Restricciones:} \(x \neq 2\) y \(x \neq -3\)

\textbf{MCD:} \((x-2)(x+3)\)

Multiplicando:
\[3(x+3) + 4(x-2) = (x-2)(x+3)\]

Expandiendo el lado izquierdo:
\[3x + 9 + 4x - 8 = (x-2)(x+3)\]
\[7x + 1 = (x-2)(x+3)\]

Expandiendo el lado derecho:
\[7x + 1 = x^2 + 3x - 2x - 6\]
\[7x + 1 = x^2 + x - 6\]

Llevando a forma estándar:
\[0 = x^2 + x - 6 - 7x - 1\]
\[0 = x^2 - 6x - 7\]

Factorizando:
\[0 = (x - 7)(x + 1)\]

Soluciones: \(x = 7\) o \(x = -1\)

\textbf{Verificación:}
\begin{itemize}
    \item \(x = 7\): \(\dfrac{3}{7-2} + \dfrac{4}{7+3} = \dfrac{3}{5} + \dfrac{4}{10} = \dfrac{6}{10} + \dfrac{4}{10} = \dfrac{10}{10} = 1\) \checkmark
    \item \(x = -1\): \(\dfrac{3}{-1-2} + \dfrac{4}{-1+3} = \dfrac{3}{-3} + \dfrac{4}{2} = -1 + 2 = 1\) \checkmark
\end{itemize}

\textbf{Respuesta:} \(x = 7\) o \(x = -1\)

%========================================
% EXERCISE 2
%========================================
\textbf{Ejercicio 2:} \(\displaystyle \frac{x}{x^2-4} - \frac{2}{x+2} = 0\)

\textbf{Solución Completa:}

\textbf{Factorización:} \(x^2 - 4 = (x-2)(x+2)\)

\textbf{Restricciones:} \(x \neq 2\) y \(x \neq -2\)

La ecuación se reescribe como:
\[\frac{x}{(x-2)(x+2)} - \frac{2}{x+2} = 0\]

\textbf{MCD:} \((x-2)(x+2)\)

Multiplicando:
\[\frac{x}{(x-2)(x+2)} \cdot (x-2)(x+2) - \frac{2}{x+2} \cdot (x-2)(x+2) = 0\]

\[x - 2(x-2) = 0\]

Expandiendo:
\[x - 2x + 4 = 0\]
\[-x + 4 = 0\]
\[x = 4\]

\textbf{Verificación:}
\[\frac{4}{4^2-4} - \frac{2}{4+2} = \frac{4}{16-4} - \frac{2}{6} = \frac{4}{12} - \frac{2}{6} = \frac{1}{3} - \frac{1}{3} = 0\] \checkmark

\textbf{Respuesta:} \(x = 4\)

%========================================
% EXERCISE 3
%========================================
\textbf{Ejercicio 3:} \(\displaystyle \frac{1}{x} + \frac{2}{x^2} - \frac{3}{x^3} = 0\)

\textbf{Solución Completa:}

\textbf{Restricción:} \(x \neq 0\)

\textbf{MCD:} \(x^3\)

Multiplicando cada término por \(x^3\):
\[\frac{1}{x} \cdot x^3 + \frac{2}{x^2} \cdot x^3 - \frac{3}{x^3} \cdot x^3 = 0\]

\[x^2 + 2x - 3 = 0\]

Factorizando:
\[(x + 3)(x - 1) = 0\]

Soluciones: \(x = -3\) o \(x = 1\)

\textbf{Verificación para \(x = -3\):}
\[\frac{1}{-3} + \frac{2}{(-3)^2} - \frac{3}{(-3)^3} = -\frac{1}{3} + \frac{2}{9} - \frac{3}{-27}\]
\[= -\frac{1}{3} + \frac{2}{9} + \frac{1}{9} = -\frac{3}{9} + \frac{3}{9} = 0\] \checkmark

\textbf{Verificación para \(x = 1\):}
\[\frac{1}{1} + \frac{2}{1^2} - \frac{3}{1^3} = 1 + 2 - 3 = 0\] \checkmark

\textbf{Respuesta:} \(x = -3\) o \(x = 1\)

%========================================
% EXERCISE 4
%========================================
\textbf{Ejercicio 4:} \(\displaystyle \frac{2x}{x^2-1} + \frac{3}{x-1} - \frac{1}{x+1} = 0\)

\textbf{Solución Completa:}

\textbf{Factorización:} \(x^2 - 1 = (x-1)(x+1)\)

\textbf{Restricciones:} \(x \neq 1\) y \(x \neq -1\)

La ecuación se reescribe:
\[\frac{2x}{(x-1)(x+1)} + \frac{3}{x-1} - \frac{1}{x+1} = 0\]

\textbf{MCD:} \((x-1)(x+1)\)

Multiplicando:
\[2x + 3(x+1) - (x-1) = 0\]

Expandiendo:
\[2x + 3x + 3 - x + 1 = 0\]
\[4x + 4 = 0\]
\[4x = -4\]
\[x = -1\]

\textbf{Verificación de restricción:}

¡La solución \(x = -1\) viola la restricción \(x \neq -1\)!

Esta es una \textbf{solución extraña} y debe descartarse.

\textbf{Respuesta:} No hay solución (la única solución potencial es extraña).

%========================================
% EXERCISE 5
%========================================
\textbf{Ejercicio 5:} \(\displaystyle \frac{x+1}{x^2+5x+6} - \frac{x-1}{x^2+4x+3} = \frac{1}{x+3}\)

\textbf{Solución Completa:}

\textbf{Factorización:}
\begin{align*}
x^2 + 5x + 6 &= (x+2)(x+3)\\
x^2 + 4x + 3 &= (x+1)(x+3)
\end{align*}

\textbf{Restricciones:} \(x \neq -1, -2, -3\)

La ecuación se reescribe:
\[\frac{x+1}{(x+2)(x+3)} - \frac{x-1}{(x+1)(x+3)} = \frac{1}{x+3}\]

\textbf{MCD:} \((x+1)(x+2)(x+3)\)

Multiplicando cada término:
\[(x+1)^2 - (x-1)(x+2) = (x+1)(x+2)\]

Expandiendo el lado izquierdo:
\begin{align*}
(x+1)^2 &= x^2 + 2x + 1\\
(x-1)(x+2) &= x^2 + 2x - x - 2 = x^2 + x - 2
\end{align*}

\[(x^2 + 2x + 1) - (x^2 + x - 2) = (x+1)(x+2)\]
\[x^2 + 2x + 1 - x^2 - x + 2 = (x+1)(x+2)\]
\[x + 3 = (x+1)(x+2)\]

Expandiendo el lado derecho:
\[x + 3 = x^2 + 2x + x + 2\]
\[x + 3 = x^2 + 3x + 2\]

Llevando a forma estándar:
\[0 = x^2 + 3x + 2 - x - 3\]
\[0 = x^2 + 2x - 1\]

Aplicando la fórmula cuadrática:
\begin{align*}
x &= \frac{-2 \pm \sqrt{4 - 4(1)(-1)}}{2}\\
&= \frac{-2 \pm \sqrt{4 + 4}}{2}\\
&= \frac{-2 \pm \sqrt{8}}{2}\\
&= \frac{-2 \pm 2\sqrt{2}}{2}\\
&= -1 \pm \sqrt{2}
\end{align*}

\textbf{Verificación de restricciones:}
\begin{itemize}
    \item \(x = -1 + \sqrt{2} \approx 0.414\): No es \(-1, -2,\) o \(-3\). \checkmark
    \item \(x = -1 - \sqrt{2} \approx -2.414\): No es \(-1, -2,\) o \(-3\). \checkmark
\end{itemize}

\textbf{Respuesta:} \(x = -1 + \sqrt{2}\) o \(x = -1 - \sqrt{2}\)

%========================================
% CHALLENGE PROBLEMS
%========================================
\subsection*{Problemas de Desafío}

\textbf{Problema 1:} Una ecuación racional tiene la forma \(\displaystyle \frac{a}{x-1} + \frac{b}{x+1} = 1\). Si \(x = 2\) es una solución, encuentre una relación entre \(a\) y \(b\).

\textbf{Solución:}

Sustituir \(x = 2\) en la ecuación:
\[\frac{a}{2-1} + \frac{b}{2+1} = 1\]
\[\frac{a}{1} + \frac{b}{3} = 1\]
\[a + \frac{b}{3} = 1\]

Multiplicando por 3:
\[3a + b = 3\]

Por lo tanto: \(b = 3 - 3a\)

\vspace{1em}

\textbf{Problema 2:} Analice la ecuación \(\displaystyle \frac{x}{x-1} = \frac{x-1}{x}\)

\textbf{Análisis:}

\textbf{Restricciones:} \(x \neq 0\) y \(x \neq 1\)

\textbf{MCD:} \(x(x-1)\)

Multiplicando:
\[x^2 = (x-1)^2\]

Expandiendo:
\[x^2 = x^2 - 2x + 1\]
\[0 = -2x + 1\]
\[x = \frac{1}{2}\]

\textbf{Verificación:}
\[\frac{1/2}{1/2 - 1} = \frac{1/2}{-1/2} = -1\]
\[\frac{1/2 - 1}{1/2} = \frac{-1/2}{1/2} = -1\]

Ambos lados dan \(-1\), por lo que \(x = \dfrac{1}{2}\) es la única solución.

La ecuación tiene exactamente \textbf{una} solución real, no dos.
