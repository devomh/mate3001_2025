%========================================
% DETAILED SOLUTIONS: Ecuaciones Cuadráticas
%========================================

\subsection*{Ejercicio 1: Identificación de Coeficientes}

\textbf{Problema 1a:} $x^2 + 8x + 15 = 0$

La ecuación ya está en forma estándar. Identificamos directamente:
\begin{itemize}
\item $a = 1$ (coeficiente de $x^2$)
\item $b = 8$ (coeficiente de $x$)
\item $c = 15$ (término constante)
\end{itemize}

\textbf{Problema 1b:} $3x^2 - 7x + 2 = 0$

Ya está en forma estándar:
\begin{itemize}
\item $a = 3$
\item $b = -7$ (notar el signo negativo)
\item $c = 2$
\end{itemize}

\textbf{Problema 1c:} $x^2 - 16 = 0$

Esta ecuación no tiene término lineal. Ya está en forma estándar:
\begin{itemize}
\item $a = 1$
\item $b = 0$ (el coeficiente de $x$ es cero)
\item $c = -16$
\end{itemize}

\textbf{Problema 1d:} $5x^2 + 2x = 3$

Primero escribimos en forma estándar restando 3 de ambos lados:
$$5x^2 + 2x - 3 = 0$$
Por tanto:
\begin{itemize}
\item $a = 5$
\item $b = 2$
\item $c = -3$
\end{itemize}

\textbf{Problema 1e:} $-2x^2 = 8x - 10$

Movemos todos los términos al lado izquierdo:
\begin{align}
-2x^2 - 8x + 10 &= 0
\end{align}
Por tanto: $a = -2$, $b = -8$, $c = 10$

Alternativamente, podríamos multiplicar por $-1$:
$$2x^2 + 8x - 10 = 0$$
Dando: $a = 2$, $b = 8$, $c = -10$ (ambas formas son correctas)

\textbf{Problema 1f:} $x(x + 5) = 6$

Primero expandimos el producto:
\begin{align}
x^2 + 5x &= 6\\
x^2 + 5x - 6 &= 0
\end{align}
Por tanto: $a = 1$, $b = 5$, $c = -6$

%========================================

\subsection*{Ejercicio 2: Factorización Simple}

\textbf{Problema 2a:} $x^2 + 9x + 20 = 0$

Buscamos dos números que sumen 9 y multipliquen 20. Probamos factores de 20:
\begin{itemize}
\item $1 \times 20 = 20$, pero $1 + 20 = 21$ ✗
\item $2 \times 10 = 20$, pero $2 + 10 = 12$ ✗
\item $4 \times 5 = 20$, y $4 + 5 = 9$ $\checkmark$
\end{itemize}

Factorizamos:
$$(x + 4)(x + 5) = 0$$

Aplicamos el principio del producto cero:
\begin{align}
x + 4 &= 0 \quad \Rightarrow \quad x = -4\\
x + 5 &= 0 \quad \Rightarrow \quad x = -5
\end{align}

\textbf{Verificación para $x = -4$:}
$$(-4)^2 + 9(-4) + 20 = 16 - 36 + 20 = 0$$ $\checkmark$

\textbf{Respuesta:} $x = -4$ o $x = -5$

\textbf{Problema 2b:} $x^2 - x - 12 = 0$

Buscamos dos números que sumen $-1$ y multipliquen $-12$:
\begin{itemize}
\item $3 \times (-4) = -12$, y $3 + (-4) = -1$ $\checkmark$
\end{itemize}

Factorizamos:
$$(x + 3)(x - 4) = 0$$

Soluciones:
$$x = -3 \quad \text{o} \quad x = 4$$

\textbf{Problema 2e:} $x^2 + 6x + 9 = 0$

Reconocemos un trinomio cuadrado perfecto:
$$x^2 + 6x + 9 = (x + 3)^2 = 0$$

Esto da una raíz doble:
$$(x + 3)^2 = 0 \quad \Rightarrow \quad x + 3 = 0 \quad \Rightarrow \quad x = -3$$

\textbf{Nota:} Cuando hay una raíz doble, la parábola toca el eje $x$ en un solo punto (el vértice).

\textbf{Problema 2f:} $x^2 - 49 = 0$

Esta es una diferencia de cuadrados: $x^2 - 7^2$

Factorizamos usando la fórmula $a^2 - b^2 = (a+b)(a-b)$:
$$(x + 7)(x - 7) = 0$$

Soluciones:
$$x = -7 \quad \text{o} \quad x = 7$$

\textbf{Problema 2g:} $4x^2 - 25 = 0$

Otra diferencia de cuadrados: $(2x)^2 - 5^2$

Factorizamos:
$$(2x + 5)(2x - 5) = 0$$

Resolvemos:
\begin{align}
2x + 5 &= 0 \quad \Rightarrow \quad x = -\frac{5}{2}\\
2x - 5 &= 0 \quad \Rightarrow \quad x = \frac{5}{2}
\end{align}

\textbf{Problema 2h:} $x^2 + 8x = 0$

Factorizamos el término común $x$:
$$x(x + 8) = 0$$

Soluciones:
$$x = 0 \quad \text{o} \quad x = -8$$

\textbf{Nota importante:} No dividir por $x$. Siempre factorizar para no perder la solución $x = 0$.

%========================================

\subsection*{Ejercicio 3: Método AC}

\textbf{Problema 3a:} $2x^2 + 7x + 3 = 0$

\textbf{Paso 1:} Calculamos $ac = 2 \times 3 = 6$

\textbf{Paso 2:} Buscamos dos números que sumen $b = 7$ y multipliquen $ac = 6$:
\begin{itemize}
\item Factores de 6: 1 y 6
\item Verificamos: $1 + 6 = 7$ $\checkmark$ y $1 \times 6 = 6$ $\checkmark$
\end{itemize}

\textbf{Paso 3:} Reescribimos $7x$ como $6x + x$:
$$2x^2 + 6x + x + 3 = 0$$

\textbf{Paso 4:} Agrupamos y factorizamos:
\begin{align}
(2x^2 + 6x) + (x + 3) &= 0\\
2x(x + 3) + 1(x + 3) &= 0\\
(x + 3)(2x + 1) &= 0
\end{align}

\textbf{Paso 5:} Resolvemos:
\begin{align}
x + 3 &= 0 \quad \Rightarrow \quad x = -3\\
2x + 1 &= 0 \quad \Rightarrow \quad x = -\frac{1}{2}
\end{align}

\textbf{Problema 3b:} $3x^2 - 10x + 8 = 0$

$ac = 3 \times 8 = 24$

Buscamos dos números que sumen $-10$ y multipliquen $24$:
\begin{itemize}
\item $-6 \times (-4) = 24$ y $-6 + (-4) = -10$ $\checkmark$
\end{itemize}

Reescribimos y factorizamos:
\begin{align}
3x^2 - 6x - 4x + 8 &= 0\\
3x(x - 2) - 4(x - 2) &= 0\\
(x - 2)(3x - 4) &= 0
\end{align}

Soluciones: $x = 2$ o $x = \frac{4}{3}$

\textbf{Problema 3d:} $4x^2 - 4x - 15 = 0$

$ac = 4 \times (-15) = -60$

Buscamos dos números que sumen $-4$ y multipliquen $-60$:
\begin{itemize}
\item $6 \times (-10) = -60$ y $6 + (-10) = -4$ $\checkmark$
\end{itemize}

\begin{align}
4x^2 + 6x - 10x - 15 &= 0\\
2x(2x + 3) - 5(2x + 3) &= 0\\
(2x + 3)(2x - 5) &= 0
\end{align}

Soluciones: $x = -\frac{3}{2}$ o $x = \frac{5}{2}$

%========================================

\subsection*{Ejercicio 4: Completar el Cuadrado}

\textbf{Problema 4a:} $x^2 + 4x - 5 = 0$

\textbf{Paso 1:} Movemos el término constante:
$$x^2 + 4x = 5$$

\textbf{Paso 2:} Calculamos $\left(\frac{b}{2}\right)^2 = \left(\frac{4}{2}\right)^2 = 4$

\textbf{Paso 3:} Sumamos 4 a ambos lados:
$$x^2 + 4x + 4 = 5 + 4$$

\textbf{Paso 4:} Factorizamos el lado izquierdo:
$$(x + 2)^2 = 9$$

\textbf{Paso 5:} Tomamos raíz cuadrada:
$$x + 2 = \pm 3$$

\textbf{Paso 6:} Resolvemos:
\begin{align}
x + 2 &= 3 \quad \Rightarrow \quad x = 1\\
x + 2 &= -3 \quad \Rightarrow \quad x = -5
\end{align}

\textbf{Respuesta:} $x = 1$ o $x = -5$

\textbf{Problema 4e:} $2x^2 + 12x + 10 = 0$

\textbf{Paso 1:} Dividimos por 2 para que $a = 1$:
$$x^2 + 6x + 5 = 0$$

\textbf{Paso 2:} Movemos el término constante:
$$x^2 + 6x = -5$$

\textbf{Paso 3:} Calculamos $\left(\frac{6}{2}\right)^2 = 9$ y sumamos:
$$x^2 + 6x + 9 = -5 + 9 = 4$$

\textbf{Paso 4:} Factorizamos:
$$(x + 3)^2 = 4$$

\textbf{Paso 5:} Tomamos raíz cuadrada:
$$x + 3 = \pm 2$$

\textbf{Paso 6:} Resolvemos:
$$x = -3 + 2 = -1 \quad \text{o} \quad x = -3 - 2 = -5$$

%========================================

\subsection*{Ejercicio 5: Fórmula Cuadrática}

\textbf{Problema 5a:} $x^2 + 3x - 10 = 0$

Identificamos: $a = 1$, $b = 3$, $c = -10$

Aplicamos la fórmula:
$$x = \frac{-b \pm \sqrt{b^2 - 4ac}}{2a}$$

Sustituimos:
\begin{align}
x &= \frac{-3 \pm \sqrt{3^2 - 4(1)(-10)}}{2(1)}\\
x &= \frac{-3 \pm \sqrt{9 + 40}}{2}\\
x &= \frac{-3 \pm \sqrt{49}}{2}\\
x &= \frac{-3 \pm 7}{2}
\end{align}

Calculamos ambas soluciones:
\begin{align}
x_1 &= \frac{-3 + 7}{2} = \frac{4}{2} = 2\\
x_2 &= \frac{-3 - 7}{2} = \frac{-10}{2} = -5
\end{align}

\textbf{Verificación para $x = 2$:}
$$2^2 + 3(2) - 10 = 4 + 6 - 10 = 0$$ $\checkmark$

\textbf{Problema 5e:} $x^2 + 4x + 1 = 0$

Identificamos: $a = 1$, $b = 4$, $c = 1$

\begin{align}
x &= \frac{-4 \pm \sqrt{16 - 4}}{2}\\
x &= \frac{-4 \pm \sqrt{12}}{2}\\
x &= \frac{-4 \pm 2\sqrt{3}}{2}\\
x &= \frac{2(-2 \pm \sqrt{3})}{2}\\
x &= -2 \pm \sqrt{3}
\end{align}

\textbf{Respuesta:} $x = -2 + \sqrt{3} \approx -0.268$ o $x = -2 - \sqrt{3} \approx -3.732$

\textbf{Nota:} Estas son soluciones irracionales exactas. Es importante simplificar los radicales completamente.

\textbf{Problema 5g:} $x^2 - 2x - 5 = 0$

Identificamos: $a = 1$, $b = -2$, $c = -5$

\begin{align}
x &= \frac{-(-2) \pm \sqrt{(-2)^2 - 4(1)(-5)}}{2(1)}\\
x &= \frac{2 \pm \sqrt{4 + 20}}{2}\\
x &= \frac{2 \pm \sqrt{24}}{2}\\
x &= \frac{2 \pm 2\sqrt{6}}{2}\\
x &= 1 \pm \sqrt{6}
\end{align}

%========================================

\subsection*{Ejercicio 6: El Discriminante}

\textbf{Problema 6a:} $x^2 + 5x + 6 = 0$

Calculamos el discriminante:
$$D = b^2 - 4ac = 5^2 - 4(1)(6) = 25 - 24 = 1$$

Como $D = 1 > 0$, la ecuación tiene \textbf{dos raíces reales distintas}.

Podemos confirmar factorizando: $(x+2)(x+3) = 0$, dando $x = -2$ y $x = -3$.

\textbf{Problema 6b:} $x^2 - 8x + 16 = 0$

$$D = (-8)^2 - 4(1)(16) = 64 - 64 = 0$$

Como $D = 0$, la ecuación tiene \textbf{una raíz real doble}.

Confirmación: $x^2 - 8x + 16 = (x-4)^2 = 0$, dando $x = 4$ (doble).

\textbf{Problema 6c:} $x^2 + 3x + 5 = 0$

$$D = 3^2 - 4(1)(5) = 9 - 20 = -11$$

Como $D = -11 < 0$, la ecuación \textbf{no tiene raíces reales}.

La parábola $y = x^2 + 3x + 5$ no interseca el eje $x$.

\textbf{Problema 6e:} $9x^2 + 6x + 1 = 0$

$$D = 6^2 - 4(9)(1) = 36 - 36 = 0$$

Raíz doble. Confirmación: $9x^2 + 6x + 1 = (3x+1)^2 = 0$, dando $x = -\frac{1}{3}$ (doble).

%========================================

\subsection*{Ejercicio 7: Forma Estándar}

\textbf{Problema 7c:} $(x + 2)(x - 3) = 6$

\textbf{Error común:} No igualar cada factor a 6. Primero debemos expandir.

\textbf{Paso 1:} Expandir:
$$x^2 - 3x + 2x - 6 = 6$$
$$x^2 - x - 6 = 6$$

\textbf{Paso 2:} Forma estándar:
$$x^2 - x - 12 = 0$$

\textbf{Paso 3:} Factorizar:
$$(x - 4)(x + 3) = 0$$

Soluciones: $x = 4$ o $x = -3$

\textbf{Problema 7d:} $x^2 = 4(x - 3)$

\textbf{Paso 1:} Expandir:
$$x^2 = 4x - 12$$

\textbf{Paso 2:} Forma estándar:
$$x^2 - 4x + 12 = 0$$

\textbf{Paso 3:} Calcular discriminante:
$$D = (-4)^2 - 4(1)(12) = 16 - 48 = -32 < 0$$

\textbf{Conclusión:} No hay soluciones reales.

%========================================

\subsection*{Ejercicio 8: Aplicaciones - Soluciones Detalladas}

\textbf{Problema 8.1:} Rectángulo con largo 4 cm mayor que ancho, área 60 cm$^2$

\textbf{Paso 1 - Definir variable:}
\begin{itemize}
\item Sea $x$ = ancho del rectángulo (cm)
\item Entonces $x + 4$ = largo del rectángulo (cm)
\end{itemize}

\textbf{Paso 2 - Escribir ecuación:}
$$\text{Área} = \text{largo} \times \text{ancho}$$
$$60 = (x+4) \cdot x$$
$$60 = x^2 + 4x$$

\textbf{Paso 3 - Forma estándar:}
$$x^2 + 4x - 60 = 0$$

\textbf{Paso 4 - Factorizar:}

Buscamos dos números que sumen 4 y multipliquen $-60$: son 10 y $-6$
$$(x + 10)(x - 6) = 0$$

\textbf{Paso 5 - Resolver:}
$$x = -10 \quad \text{o} \quad x = 6$$

\textbf{Paso 6 - Interpretar:}

Como $x$ representa una longitud física, debe ser positiva. Descartamos $x = -10$.

Por tanto: $x = 6$ cm (ancho) y $x + 4 = 10$ cm (largo)

\textbf{Verificación:}
$$\text{Área} = 6 \times 10 = 60 \text{ cm}^2$$ $\checkmark$

\textbf{Problema 8.3:} Dos números positivos con diferencia 3, suma de cuadrados 89

\textbf{Paso 1 - Variables:}
\begin{itemize}
\item $x$ = número mayor
\item $x - 3$ = número menor
\end{itemize}

\textbf{Paso 2 - Ecuación:}
$$x^2 + (x-3)^2 = 89$$

\textbf{Paso 3 - Expandir:}
$$x^2 + x^2 - 6x + 9 = 89$$
$$2x^2 - 6x + 9 = 89$$

\textbf{Paso 4 - Forma estándar:}
$$2x^2 - 6x - 80 = 0$$

Dividir por 2:
$$x^2 - 3x - 40 = 0$$

\textbf{Paso 5 - Factorizar:}

Buscamos dos números que sumen $-3$ y multipliquen $-40$: son 5 y $-8$
$$(x + 5)(x - 8) = 0$$

\textbf{Paso 6 - Soluciones:}
$$x = -5 \quad \text{o} \quad x = 8$$

Como necesitamos números positivos, $x = 8$.

\textbf{Respuesta:} Los números son 8 y 5

\textbf{Verificación:}
$$8^2 + 5^2 = 64 + 25 = 89$$ $\checkmark$

\textbf{Problema 8.5:} Objeto lanzado verticalmente, $h = -16t^2 + 48t$, altura 32 pies

\textbf{Interpretación física:}

El objeto alcanza 32 pies dos veces: una vez subiendo y otra vez bajando.

\textbf{Paso 1 - Sustituir:}
$$32 = -16t^2 + 48t$$

\textbf{Paso 2 - Forma estándar:}
$$-16t^2 + 48t - 32 = 0$$

\textbf{Paso 3 - Simplificar (dividir por $-16$):}
$$t^2 - 3t + 2 = 0$$

\textbf{Paso 4 - Factorizar:}
$$(t - 1)(t - 2) = 0$$

\textbf{Paso 5 - Soluciones:}
$$t = 1 \text{ segundo} \quad \text{o} \quad t = 2 \text{ segundos}$$

\textbf{Interpretación:}
\begin{itemize}
\item En $t = 1$ s: el objeto está a 32 pies \textbf{subiendo}
\item En $t = 2$ s: el objeto está a 32 pies \textbf{bajando}
\end{itemize}

El objeto alcanza su altura máxima en $t = 1.5$ s (punto medio).

%========================================

\subsection*{Ejercicio 9: Problemas Desafiantes}

\textbf{Problema 9c:} Encuentre $k$ para que $x^2 + kx + 9 = 0$ tenga una raíz doble

\textbf{Concepto:} Una ecuación tiene raíz doble cuando su discriminante es cero.

\textbf{Solución:}

Para $ax^2 + bx + c = 0$: $a = 1$, $b = k$, $c = 9$

Discriminante:
$$D = b^2 - 4ac = k^2 - 4(1)(9) = k^2 - 36$$

Para raíz doble, $D = 0$:
$$k^2 - 36 = 0$$
$$k^2 = 36$$
$$k = \pm 6$$

\textbf{Verificación con $k = 6$:}
$$x^2 + 6x + 9 = (x+3)^2 = 0$$ $\checkmark$ (raíz doble en $x = -3$)

\textbf{Verificación con $k = -6$:}
$$x^2 - 6x + 9 = (x-3)^2 = 0$$ $\checkmark$ (raíz doble en $x = 3$)

\textbf{Respuesta:} $k = 6$ o $k = -6$

\textbf{Problema 9d:} Encuentre $k$ para que $kx^2 - 6x + 2 = 0$ tenga dos raíces reales distintas

\textbf{Concepto:} Dos raíces reales distintas requieren $D > 0$ y $a \neq 0$.

\textbf{Solución:}

Para $kx^2 - 6x + 2 = 0$: $a = k$, $b = -6$, $c = 2$

Discriminante:
$$D = (-6)^2 - 4(k)(2) = 36 - 8k$$

Para dos raíces reales distintas:
$$D > 0$$
$$36 - 8k > 0$$
$$36 > 8k$$
$$k < \frac{36}{8} = 4.5$$

Además, para que sea ecuación cuadrática: $k \neq 0$

\textbf{Respuesta:} $k < 4.5$ y $k \neq 0$

En notación de intervalos: $k \in (-\infty, 0) \cup (0, 4.5)$
