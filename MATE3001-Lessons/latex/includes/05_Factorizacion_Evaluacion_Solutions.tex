%========================================
% DETAILED SOLUTIONS: Factorización y Evaluación de Polinomios
%========================================

\subsection*{Ejercicio 1}

\textbf{Problema:} Factorización de polinomios.

\textbf{Estrategia general:}
1. Buscar factor común
2. Identificar patrones especiales (diferencia de cuadrados, trinomio cuadrático perfecto, etc.)
3. Usar agrupación si es necesario
4. Verificar multiplicando

\textbf{b)} $x^2 + 7x + 12$

Para factorizar $x^2 + 7x + 12$, buscamos dos números que:
\begin{itemize}
    \item Se multipliquen para dar 12
    \item Se sumen para dar 7
\end{itemize}

Factores de 12: $1 \times 12$, $2 \times 6$, $3 \times 4$

Probando sumas: $1 + 12 = 13$, $2 + 6 = 8$, $3 + 4 = 7$ ✓

Por tanto: $x^2 + 7x + 12 = (x + 3)(x + 4)$

\textbf{Verificación:} $(x + 3)(x + 4) = x^2 + 4x + 3x + 12 = x^2 + 7x + 12$ ✓

\textbf{f)} $2x^3 - 8x$

Paso 1: Factor común
$2x^3 - 8x = 2x(x^2 - 4)$

Paso 2: Diferencia de cuadrados en $(x^2 - 4)$
$x^2 - 4 = x^2 - 2^2 = (x + 2)(x - 2)$

\textbf{Resultado:} $2x^3 - 8x = 2x(x + 2)(x - 2)$

\textbf{g)} $x^4 - 16$

Reconocemos esto como diferencia de cuadrados:
$x^4 - 16 = (x^2)^2 - 4^2 = (x^2 + 4)(x^2 - 4)$

Pero $x^2 - 4$ es también diferencia de cuadrados:
$x^2 - 4 = (x + 2)(x - 2)$

\textbf{Resultado:} $x^4 - 16 = (x^2 + 4)(x + 2)(x - 2)$

Nota: $x^2 + 4$ no se puede factorizar más en los números reales.

\subsection*{Ejercicio 2}

\textbf{Problema:} Evaluación de polinomios.

\textbf{a)} $P(x) = 2x^3 - 3x^2 + x - 4$; evaluar $P(2)$

Sustituyendo $x = 2$:
\begin{align}
P(2) &= 2(2)^3 - 3(2)^2 + 2 - 4\\
&= 2(8) - 3(4) + 2 - 4\\
&= 16 - 12 + 2 - 4\\
&= 2
\end{align}

\textbf{c)} $R(x) = 3x^2 - 5x + 2$; evaluar $R(0)$ y $R(1)$

Para $R(0)$:
\begin{align}
R(0) &= 3(0)^2 - 5(0) + 2\\
&= 0 - 0 + 2\\
&= 2
\end{align}

Para $R(1)$:
\begin{align}
R(1) &= 3(1)^2 - 5(1) + 2\\
&= 3 - 5 + 2\\
&= 0
\end{align}

Observe que $R(1) = 0$, lo que significa que $(x - 1)$ es un factor de $R(x)$.

\subsection*{Ejercicio 3}

\textbf{Problema:} Aplicación del Teorema del Factor.

\textbf{Teorema del Factor:} $(x - a)$ es factor de $P(x)$ si y solo si $P(a) = 0$.

\textbf{a)} ¿Es $(x - 2)$ factor de $x^3 - 3x^2 + 4x - 12$?

Evaluamos $P(2)$:
\begin{align}
P(2) &= 2^3 - 3(2^2) + 4(2) - 12\\
&= 8 - 3(4) + 8 - 12\\
&= 8 - 12 + 8 - 12\\
&= -8 \neq 0
\end{align}

Como $P(2) \neq 0$, $(x - 2)$ \textbf{no es} factor de $P(x)$.

\textbf{d)} ¿Es $(2x - 1)$ factor de $2x^3 - 3x^2 + 1$?

Para $(2x - 1)$, necesitamos evaluar en $x = \frac{1}{2}$ (donde $2x - 1 = 0$):

\begin{align}
P\left(\frac{1}{2}\right) &= 2\left(\frac{1}{2}\right)^3 - 3\left(\frac{1}{2}\right)^2 + 1\\
&= 2 \cdot \frac{1}{8} - 3 \cdot \frac{1}{4} + 1\\
&= \frac{1}{4} - \frac{3}{4} + 1\\
&= \frac{1 - 3 + 4}{4}\\
&= \frac{2}{4} = \frac{1}{2} \neq 0
\end{align}

Como $P(\frac{1}{2}) \neq 0$, $(2x - 1)$ \textbf{no es} factor de $P(x)$.

\subsection*{Ejercicio 4}

\textbf{Problema:} Problemas de aplicación.

\textbf{a)} Área del rectángulo

Área = largo × ancho = $(2x + 3)(x - 1)$

Usando FOIL:
\begin{align}
\text{Área} &= (2x + 3)(x - 1)\\
&= 2x \cdot x + 2x \cdot (-1) + 3 \cdot x + 3 \cdot (-1)\\
&= 2x^2 - 2x + 3x - 3\\
&= 2x^2 + x - 3
\end{align}

\textbf{c)} Problema del proyectil

$h(t) = -5t^2 + 20t + 25$

\textbf{Altura inicial:} $h(0) = -5(0)^2 + 20(0) + 25 = 25$ metros

\textbf{Altura después de 2 segundos:}
\begin{align}
h(2) &= -5(2)^2 + 20(2) + 25\\
&= -5(4) + 40 + 25\\
&= -20 + 40 + 25\\
&= 45 \text{ metros}
\end{align}

\textbf{d)} Problema de costos

$C(x) = 2x^2 + 15x + 100$

\textbf{Costo fijo:} Es el costo cuando no se produce nada ($x = 0$)
$C(0) = 2(0)^2 + 15(0) + 100 = 100$

\textbf{Costo de producir 10 artículos:}
\begin{align}
C(10) &= 2(10)^2 + 15(10) + 100\\
&= 2(100) + 150 + 100\\
&= 200 + 150 + 100\\
&= 450
\end{align}