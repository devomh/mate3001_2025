%========================================
% EXERCISES: Conjuntos y Operaciones con Conjuntos
%========================================

\section{Ejercicios}

\begin{exercise}
	\problem Escriba en forma descriptiva los siguientes conjuntos:
	
	\begin{exerciselist}
		\item $A = \{2, 4, 6, 8, 10\}$
		\item $B = \{1, 4, 9, 16, 25\}$
		\item $C = \{a, e, i, o, u\}$
	\end{exerciselist}
	
	\begin{solucion}
		\begin{exerciselist}
			\item $A = \{x \mid x \text{ es un número par natural menor o igual a 10}\}$
			\item $B = \{x^2 \mid x \text{ es un número natural del 1 al 5}\}$
			\item $C = \{x \mid x \text{ es una vocal del alfabeto español}\}$
		\end{exerciselist}
	\end{solucion}
\end{exercise}

\begin{exercise}
\problem Escriba en forma de enumeración los siguientes conjuntos:

\begin{exerciselist}
    \item $A = \{x \mid x \text{ es un número natural menor que 6}\}$
    \item $B = \{x \mid x \text{ es una vocal de la palabra ``matemática''}\}$
    \item $C = \{x \mid x \text{ es un número par mayor que 10 y menor que 20}\}$
\end{exerciselist}

\begin{solucion}
\begin{exerciselist}
    \item $A = \{1, 2, 3, 4, 5\}$
    \item $B = \{a, e, i\}$ (sin repetición)
    \item $C = \{12, 14, 16, 18\}$
\end{exerciselist}
\end{solucion}
\end{exercise}

\begin{exercise}
\problem Determine si las siguientes afirmaciones son verdaderas (V) o falsas (F):

\begin{exerciselist}
    \item $3 \in \{1, 2, 3, 4, 5\}$ \hspace{2cm} \underline{\hspace{2cm}}
    \item $0 \in \mathbb{N}$ \hspace{2cm} \underline{\hspace{2cm}}
    \item $-5 \in \mathbb{Z}$ \hspace{2cm} \underline{\hspace{2cm}}
    \item $\frac{2}{3} \in \mathbb{Q}$ \hspace{2cm} \underline{\hspace{2cm}}
    \item $\sqrt{2} \in \mathbb{Q}$ \hspace{2cm} \underline{\hspace{2cm}}
\end{exerciselist}

\begin{solucion}
\begin{exerciselist}
    \item V (3 pertenece al conjunto)
    \item F (0 no está en los números naturales)
    \item V (-5 es un número entero)
    \item V ($\frac{2}{3}$ es un número racional)
    \item F ($\sqrt{2}$ es irracional)
\end{exerciselist}
\end{solucion}
\end{exercise}

\begin{exercise}
\problem Sean $A = \{1, 2, 3, 4, 5\}$, $B = \{3, 4, 5, 6, 7\}$ y $C = \{5, 6, 7, 8, 9\}$. Encuentre:

\begin{exerciselist}
    \item $A \cup B = $ \underline{\hspace{5cm}}
    \item $A \cap B = $ \underline{\hspace{5cm}}
    \item $B \cup C = $ \underline{\hspace{5cm}}
    \item $B \cap C = $ \underline{\hspace{5cm}}
    \item $A \cap B \cap C = $ \underline{\hspace{5cm}}
\end{exerciselist}

\begin{solucion}
\begin{exerciselist}
    \item $A \cup B = \{1, 2, 3, 4, 5, 6, 7\}$
    \item $A \cap B = \{3, 4, 5\}$
    \item $B \cup C = \{3, 4, 5, 6, 7, 8, 9\}$
    \item $B \cap C = \{5, 6, 7\}$
    \item $A \cap B \cap C = \{5\}$
\end{exerciselist}
\end{solucion}
\end{exercise}

\begin{exercise}
\problem Determine cuáles de las siguientes relaciones de subconjuntos son verdaderas:

\begin{exerciselist}
    \item $\{1, 2\} \subseteq \{1, 2, 3, 4\}$ \hspace{2cm} \underline{\hspace{2cm}}
    \item $\{a, b, c\} \subseteq \{c, b, a\}$ \hspace{2cm} \underline{\hspace{2cm}}
    \item $\mathbb{N} \subseteq \mathbb{Z}$ \hspace{2cm} \underline{\hspace{2cm}}
    \item $\mathbb{Q} \subseteq \mathbb{N}$ \hspace{2cm} \underline{\hspace{2cm}}
    \item $\emptyset \subseteq \{1, 2, 3\}$ \hspace{2cm} \underline{\hspace{2cm}}
\end{exerciselist}

\begin{solucion}
\begin{exerciselist}
    \item V (todos los elementos de $\{1, 2\}$ están en $\{1, 2, 3, 4\}$)
    \item V (ambos conjuntos tienen los mismos elementos)
    \item V (todos los naturales son enteros)
    \item F (no todos los racionales son naturales, ej: $\frac{1}{2}$)
    \item V (el conjunto vacío es subconjunto de cualquier conjunto)
\end{exerciselist}
\end{solucion}
\end{exercise}

\begin{exercise}
\problem Clasifique los siguientes números en las categorías apropiadas (puede pertenecer a más de una):

Números: $-3$, $0$, $\frac{1}{2}$, $\sqrt{9}$, $\sqrt{7}$, $\pi$, $2.75$

\begin{center}
\begin{tabular}{|c|c|c|c|c|}
\hline
\textbf{Número} & \textbf{Naturales} & \textbf{Enteros} & \textbf{Racionales} & \textbf{Irracionales} \\
\hline
$-3$ & & & & \\
\hline
$0$ & & & & \\
\hline
$\frac{1}{2}$ & & & & \\
\hline
$\sqrt{9}$ & & & & \\
\hline
$\sqrt{7}$ & & & & \\
\hline
$\pi$ & & & & \\
\hline
$2.75$ & & & & \\
\hline
\end{tabular}
\end{center}

\begin{solucion}
\begin{center}
\begin{tabular}{|c|c|c|c|c|}
\hline
\textbf{Número} & \textbf{Naturales} & \textbf{Enteros} & \textbf{Racionales} & \textbf{Irracionales} \\
\hline
$-3$ & & Sí & Sí & \\
\hline
$0$ & & Sí & Sí & \\
\hline
$\frac{1}{2}$ & & & Sí & \\
\hline
$\sqrt{9} = 3$ & Sí & Sí & Sí & \\
\hline
$\sqrt{7}$ & & & & Sí \\
\hline
$\pi$ & & & & Sí \\
\hline
$2.75 = \frac{11}{4}$ & & & Sí & \\
\hline
\end{tabular}
\end{center}
\end{solucion}
\end{exercise}

\begin{exercise}
\problem Determine si los siguientes números son primos o compuestos. Si son compuestos, encuentre su factorización prima:

\begin{exerciselist}
    \item $13$ \hspace{3cm} \underline{\hspace{3cm}}
    \item $15$ \hspace{3cm} \underline{\hspace{3cm}}
    \item $17$ \hspace{3cm} \underline{\hspace{3cm}}
    \item $24$ \hspace{3cm} \underline{\hspace{3cm}}
    \item $29$ \hspace{3cm} \underline{\hspace{3cm}}
\end{exerciselist}

\begin{solucion}
\begin{exerciselist}
    \item $13$ es primo (solo divisible por 1 y 13)
    \item $15$ es compuesto: $15 = 3 \times 5$
    \item $17$ es primo (solo divisible por 1 y 17)
    \item $24$ es compuesto: $24 = 2^3 \times 3 = 8 \times 3$
    \item $29$ es primo (solo divisible por 1 y 29)
\end{exerciselist}
\end{solucion}
\end{exercise}

\begin{exercise}
\problem Escriba cada decimal como fracción en su forma más simple:

\begin{exerciselist}
    \item $0.25 = $ \underline{\hspace{3cm}}
    \item $0.75 = $ \underline{\hspace{3cm}}
    \item $1.2 = $ \underline{\hspace{3cm}}
    \item $0.125 = $ \underline{\hspace{3cm}}
\end{exerciselist}

\begin{solucion}
\begin{exerciselist}
    \item $0.25 = \frac{25}{100} = \frac{1}{4}$
    \item $0.75 = \frac{75}{100} = \frac{3}{4}$
    \item $1.2 = \frac{12}{10} = \frac{6}{5}$
    \item $0.125 = \frac{125}{1000} = \frac{1}{8}$
\end{exerciselist}
\end{solucion}
\end{exercise}

\begin{exercise}
\problem \textbf{Problema de aplicación:} En una clase de 30 estudiantes, 18 estudian matemáticas, 15 estudian física, y 8 estudian ambas materias.

\begin{exerciselist}
    \item ¿Cuántos estudiantes estudian matemáticas o física (o ambas)?
    \item ¿Cuántos estudiantes estudian solo matemáticas?
    \item ¿Cuántos estudiantes estudian solo física?
    \item ¿Cuántos estudiantes no estudian ninguna de las dos materias?
\end{exerciselist}

\textit{Sugerencia: Use un diagrama de Venn para resolver este problema.}

\begin{solucion}
Sea $M$ = conjunto de estudiantes que estudian matemáticas, $F$ = conjunto de estudiantes que estudian física.

Datos: $|M| = 18$, $|F| = 15$, $|M \cap F| = 8$, Total = 30

\begin{exerciselist}
    \item $|M \cup F| = |M| + |F| - |M \cap F| = 18 + 15 - 8 = 25$ estudiantes
    \item Solo matemáticas: $|M| - |M \cap F| = 18 - 8 = 10$ estudiantes
    \item Solo física: $|F| - |M \cap F| = 15 - 8 = 7$ estudiantes
    \item Ninguna materia: $30 - 25 = 5$ estudiantes
\end{exerciselist}
\end{solucion}
\end{exercise}

