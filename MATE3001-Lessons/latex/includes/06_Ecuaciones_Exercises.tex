%========================================
% EXERCISES: Ecuaciones
%========================================

\section{Ejercicios}

%========================================
% EJERCICIOS SECCIÓN 3.1: Definición y Terminología
%========================================
\begin{exercise}
\textbf{Identificación de ecuaciones e identidades}

\problem Determine si cada enunciado es una ecuación o una identidad:
\begin{exerciselist}
    \item $5x + 3 = 18$
    \item $x^2 + 2x = x^2 + 2x$
    \item $3(x + 2) = 3x + 6$
    \item $\frac{x+1}{2} = 5$
\end{exerciselist}

\begin{solucion}
a) Ecuación (no siempre cierta)\\
b) Identidad (siempre cierta)\\
c) Identidad (siempre cierta por propiedad distributiva)\\
d) Ecuación (no siempre cierta)
\end{solucion}

\problem Verifique si $x = 4$ es solución de las siguientes ecuaciones:
\begin{exerciselist}
    \item $2x - 5 = 3$
    \item $x^2 - 3x - 4 = 0$
    \item $\frac{x}{2} + 1 = 3$
\end{exerciselist}

\begin{solucion}
a) $2(4) - 5 = 8 - 5 = 3$ $\checkmark$ (Sí es solución)\\
b) $(4)^2 - 3(4) - 4 = 16 - 12 - 4 = 0$ $\checkmark$ (Sí es solución)\\
c) $\frac{4}{2} + 1 = 2 + 1 = 3$ $\checkmark$ (Sí es solución)
\end{solucion}
\end{exercise}

%========================================
% EJERCICIOS SECCIÓN 3.2: Ecuaciones lineales
%========================================
\begin{exercise}
\textbf{Resolución de ecuaciones lineales básicas}

\problem Resuelva las siguientes ecuaciones lineales:
\begin{exerciselist}
    \item $5x - 12 = 0$
    \item $3x + 7 = 22$
    \item $-2x + 9 = 1$
    \item $4x - 6 = 2x + 8$
\end{exerciselist}

\begin{solucion}
a) $5x = 12 \Rightarrow x = \frac{12}{5}$\\
b) $3x = 15 \Rightarrow x = 5$\\
c) $-2x = -8 \Rightarrow x = 4$\\
d) $2x = 14 \Rightarrow x = 7$
\end{solucion}

\problem Resuelva y verifique:
\begin{exerciselist}
    \item $2(x + 3) = 4x - 2$
    \item $5x - (3x + 1) = 7$
    \item $3(2x - 1) - 2(x + 4) = 1$
\end{exerciselist}

\begin{solucion}
a) $2x + 6 = 4x - 2 \Rightarrow 8 = 2x \Rightarrow x = 4$\\
b) $5x - 3x - 1 = 7 \Rightarrow 2x = 8 \Rightarrow x = 4$\\
c) $6x - 3 - 2x - 8 = 1 \Rightarrow 4x = 12 \Rightarrow x = 3$
\end{solucion}
\end{exercise}

\begin{exercise}
\textbf{Ecuaciones lineales con fracciones simples}

\problem Resuelva las siguientes ecuaciones:
\begin{exerciselist}
    \item $\frac{x}{3} = 5$
    \item $\frac{2x}{5} = 8$
    \item $\frac{x-1}{4} = 3$
    \item $\frac{3x+2}{7} = 2$
\end{exerciselist}

\begin{solucion}
a) $x = 15$\\
b) $2x = 40 \Rightarrow x = 20$\\
c) $x - 1 = 12 \Rightarrow x = 13$\\
d) $3x + 2 = 14 \Rightarrow 3x = 12 \Rightarrow x = 4$
\end{solucion}

\problem Resuelva ecuaciones con fracciones en ambos lados:
\begin{exerciselist}
    \item $\frac{x}{2} = \frac{x+1}{3}$
    \item $\frac{2x-1}{5} = \frac{x+3}{4}$
    \item $\frac{x+2}{6} = \frac{2x-3}{9}$
\end{exerciselist}

\begin{solucion}
a) $3x = 2(x+1) \Rightarrow x = 2$\\
b) $4(2x-1) = 5(x+3) \Rightarrow 8x-4 = 5x+15 \Rightarrow x = \frac{19}{3}$\\
c) $3(x+2) = 2(2x-3) \Rightarrow 3x+6 = 4x-6 \Rightarrow x = 12$
\end{solucion}
\end{exercise}

%========================================
% EJERCICIOS SECCIÓN 3.3: Ecuaciones con fracciones
%========================================
\begin{exercise}
\textbf{Ecuaciones con fracciones más complejas}

\problem Resuelva las siguientes ecuaciones con denominadores numéricos:
\begin{exerciselist}
    \item $\frac{2x+1}{3} = \frac{x-2}{5}$
    \item $\frac{3x-4}{8} = \frac{x+1}{6}$
    \item $\frac{x+3}{4} - \frac{x-1}{6} = 2$
\end{exerciselist}

\begin{solucion}
a) $5(2x+1) = 3(x-2) \Rightarrow 10x+5 = 3x-6 \Rightarrow x = -\frac{11}{7}$\\
b) $6(3x-4) = 8(x+1) \Rightarrow 18x-24 = 8x+8 \Rightarrow x = \frac{16}{5}$\\
c) $3(x+3) - 2(x-1) = 24 \Rightarrow 3x+9-2x+2 = 24 \Rightarrow x = 13$
\end{solucion}

\problem Resuelva ecuaciones que pueden no tener solución o tener infinitas soluciones:
\begin{exerciselist}
    \item $\frac{2x+1}{3} = \frac{2x+1}{3}$
    \item $\frac{x+2}{4} = \frac{x+5}{4}$
    \item $\frac{3x-6}{9} = \frac{x-2}{3}$
\end{exerciselist}

\begin{solucion}
a) Identidad (infinitas soluciones para todo $x$ real)\\
b) $x+2 = x+5 \Rightarrow 2 = 5$ (sin solución)\\
c) $\frac{3(x-2)}{9} = \frac{x-2}{3} \Rightarrow \frac{x-2}{3} = \frac{x-2}{3}$ (infinitas soluciones)
\end{solucion}
\end{exercise}

\begin{exercise}
\textbf{Aplicaciones y problemas verbales}

\problem \textbf{Aplicaciones:} Plantee y resuelva la ecuación correspondiente:
\begin{exerciselist}
    \item Si el doble de un número más 5 es igual a 17, ¿cuál es el número?
    \item La tercera parte de un número menos 4 es igual a 2. Encuentre el número.
    \item La mitad de un número más la cuarta parte del mismo número es igual a 15.
\end{exerciselist}

\begin{solucion}
a) $2x + 5 = 17 \Rightarrow x = 6$\\
b) $\frac{x}{3} - 4 = 2 \Rightarrow x = 18$\\
c) $\frac{x}{2} + \frac{x}{4} = 15 \Rightarrow \frac{3x}{4} = 15 \Rightarrow x = 20$
\end{solucion}

\problem \textbf{Problemas de mezclas simples:}
\begin{exerciselist}
    \item Un tercio de la edad de Ana más un cuarto de su edad es igual a 14 años. ¿Cuál es la edad de Ana?
    \item Pedro gastó la mitad de su dinero en almuerzo y un tercio en transporte. Si le quedan \$10, ¿cuánto tenía inicialmente?
\end{exerciselist}

\begin{solucion}
a) $\frac{x}{3} + \frac{x}{4} = 14 \Rightarrow \frac{7x}{12} = 14 \Rightarrow x = 24$ años\\
b) $x - \frac{x}{2} - \frac{x}{3} = 10 \Rightarrow \frac{x}{6} = 10 \Rightarrow x = 60$ dólares
\end{solucion}
\end{exercise}

