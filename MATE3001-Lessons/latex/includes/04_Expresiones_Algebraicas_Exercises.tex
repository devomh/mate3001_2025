%========================================
% EXERCISES: Expresiones Algebraicas y Polinomios
%========================================

\section{Ejercicios}

\begin{exercise}
\problem Identifique los términos, coeficientes, variables y exponentes en las siguientes expresiones algebraicas:

\begin{exerciselist}
    \item $5x^3 - 2x^2 + 7x - 9$
    \item $-3a^2b + 4ab^2 - b^3$
    \item $\frac{2}{3}x^4 - \frac{1}{2}x + 6$
    \item $2x^2y^3 - xy + 5$
\end{exerciselist}

\begin{solucion}
\begin{exerciselist}
    \item $5x^3 - 2x^2 + 7x - 9$
    \begin{itemize}
        \item Términos: $5x^3$, $-2x^2$, $7x$, $-9$
        \item Coeficientes: $5$, $-2$, $7$, $-9$
        \item Variable: $x$
        \item Exponentes: $3$, $2$, $1$, $0$
    \end{itemize}
    
    \item $-3a^2b + 4ab^2 - b^3$
    \begin{itemize}
        \item Términos: $-3a^2b$, $4ab^2$, $-b^3$
        \item Coeficientes: $-3$, $4$, $-1$
        \item Variables: $a$, $b$
        \item Exponentes en $a$: $2$, $1$, $0$; en $b$: $1$, $2$, $3$
    \end{itemize}
    
    \item $\frac{2}{3}x^4 - \frac{1}{2}x + 6$
    \begin{itemize}
        \item Términos: $\frac{2}{3}x^4$, $-\frac{1}{2}x$, $6$
        \item Coeficientes: $\frac{2}{3}$, $-\frac{1}{2}$, $6$
        \item Variable: $x$
        \item Exponentes: $4$, $1$, $0$
    \end{itemize}
    
    \item $2x^2y^3 - xy + 5$
    \begin{itemize}
        \item Términos: $2x^2y^3$, $-xy$, $5$
        \item Coeficientes: $2$, $-1$, $5$
        \item Variables: $x$, $y$
        \item Exponentes en $x$: $2$, $1$, $0$; en $y$: $3$, $1$, $0$
    \end{itemize}
\end{exerciselist}
\end{solucion}
\end{exercise}

\begin{exercise}
\problem Determine el dominio de las siguientes expresiones algebraicas:

\begin{exerciselist}
    \item $f(x) = \frac{3x + 2}{x - 5}$
    \item $g(x) = \frac{x^2 - 1}{x^2 - 4}$
    \item $h(x) = \sqrt{2x - 6}$
    \item $k(x) = \frac{x + 3}{(x - 1)(x + 2)}$
    \item $m(x) = \frac{1}{\sqrt{x - 4}}$
\end{exerciselist}

\begin{solucion}
\begin{exerciselist}
    \item $f(x) = \frac{3x + 2}{x - 5}$: Dominio $\{x \in \mathbb{R} \mid x \neq 5\}$ o $(-\infty, 5) \cup (5, \infty)$
    
    \item $g(x) = \frac{x^2 - 1}{x^2 - 4}$: $x^2 - 4 = 0 \Rightarrow x = \pm 2$. Dominio: $\{x \in \mathbb{R} \mid x \neq 2, x \neq -2\}$ o $(-\infty, -2) \cup (-2, 2) \cup (2, \infty)$
    
    \item $h(x) = \sqrt{2x - 6}$: $2x - 6 \geq 0 \Rightarrow x \geq 3$. Dominio: $[3, \infty)$
    
    \item $k(x) = \frac{x + 3}{(x - 1)(x + 2)}$: $(x - 1)(x + 2) = 0 \Rightarrow x = 1$ o $x = -2$. Dominio: $(-\infty, -2) \cup (-2, 1) \cup (1, \infty)$
    
    \item $m(x) = \frac{1}{\sqrt{x - 4}}$: $x - 4 > 0 \Rightarrow x > 4$. Dominio: $(4, \infty)$
\end{exerciselist}
\end{solucion}
\end{exercise}

\begin{exercise}
\problem Clasifique los siguientes polinomios por grado y número de términos. Identifique el coeficiente principal y el término constante:

\begin{exerciselist}
    \item $P(x) = 7x^2 - 3x + 1$
    \item $Q(x) = 4x^5 - x^3 + 2x^2 - x + 8$
    \item $R(x) = -2x^4 + 5x^2$
    \item $S(x) = 3x + 7$
    \item $T(x) = -5$
    \item $U(x) = x^6 - 1$
\end{exerciselist}

\begin{solucion}
\begin{exerciselist}
    \item $P(x) = 7x^2 - 3x + 1$: Grado 2 (cuadrático), trinomio, coeficiente principal: 7, término constante: 1
    
    \item $Q(x) = 4x^5 - x^3 + 2x^2 - x + 8$: Grado 5, polinomio (5 términos), coeficiente principal: 4, término constante: 8
    
    \item $R(x) = -2x^4 + 5x^2$: Grado 4 (cuártico), binomio, coeficiente principal: -2, término constante: 0
    
    \item $S(x) = 3x + 7$: Grado 1 (lineal), binomio, coeficiente principal: 3, término constante: 7
    
    \item $T(x) = -5$: Grado 0 (constante), monomio, coeficiente principal: -5, término constante: -5
    
    \item $U(x) = x^6 - 1$: Grado 6, binomio, coeficiente principal: 1, término constante: -1
\end{exerciselist}
\end{solucion}
\end{exercise}

\begin{exercise}
\problem Realice las siguientes operaciones de suma y resta de polinomios:

\begin{exerciselist}
    \item $(3x^2 - 5x + 2) + (2x^2 + 7x - 4)$
    \item $(4x^3 - 2x^2 + x - 3) - (x^3 + 3x^2 - 2x + 1)$
    \item $(x^4 - 3x^2 + 5) + (-2x^4 + x^3 + 4x^2 - 7)$
    \item $(5x^2 - 3x + 1) - (2x^2 + 4x - 6)$
\end{exerciselist}

\begin{solucion}
\begin{exerciselist}
    \item $(3x^2 - 5x + 2) + (2x^2 + 7x - 4) = 5x^2 + 2x - 2$
    
    \item $(4x^3 - 2x^2 + x - 3) - (x^3 + 3x^2 - 2x + 1) = 3x^3 - 5x^2 + 3x - 4$
    
    \item $(x^4 - 3x^2 + 5) + (-2x^4 + x^3 + 4x^2 - 7) = -x^4 + x^3 + x^2 - 2$
    
    \item $(5x^2 - 3x + 1) - (2x^2 + 4x - 6) = 3x^2 - 7x + 7$
\end{exerciselist}
\end{solucion}
\end{exercise}

\begin{exercise}
\problem Aplique las leyes de los exponentes para simplificar:

\begin{exerciselist}
    \item $x^3 \cdot x^5$
    \item $\frac{a^8}{a^3}$
    \item $(y^4)^3$
    \item $(3x^2)^4$
    \item $\frac{12x^6y^4}{4x^2y^7}$
    \item $(2x^{-3}y^2)^{-2}$
\end{exerciselist}

\begin{solucion}
\begin{exerciselist}
    \item $x^3 \cdot x^5 = x^{3+5} = x^8$
    
    \item $\frac{a^8}{a^3} = a^{8-3} = a^5$
    
    \item $(y^4)^3 = y^{4 \cdot 3} = y^{12}$
    
    \item $(3x^2)^4 = 3^4 \cdot (x^2)^4 = 81x^8$
    
    \item $\frac{12x^6y^4}{4x^2y^7} = 3x^{6-2}y^{4-7} = 3x^4y^{-3} = \frac{3x^4}{y^3}$
    
    \item $(2x^{-3}y^2)^{-2} = 2^{-2} \cdot x^{(-3)(-2)} \cdot y^{2(-2)} = \frac{1}{4}x^6y^{-4} = \frac{x^6}{4y^4}$
\end{exerciselist}
\end{solucion}
\end{exercise}

\begin{exercise}
\problem Multiplique los siguientes polinomios:

\begin{exerciselist}
    \item $3x(2x^2 - 5x + 4)$
    \item $(x + 5)(x - 3)$
    \item $(2x - 3)(3x + 4)$
    \item $(x^2 + 2x - 1)(x + 2)$
    \item $(a + b)(a^2 - ab + b^2)$
\end{exerciselist}

\begin{solucion}
\begin{exerciselist}
    \item $3x(2x^2 - 5x + 4) = 6x^3 - 15x^2 + 12x$
    
    \item $(x + 5)(x - 3) = x^2 - 3x + 5x - 15 = x^2 + 2x - 15$
    
    \item $(2x - 3)(3x + 4) = 6x^2 + 8x - 9x - 12 = 6x^2 - x - 12$
    
    \item $(x^2 + 2x - 1)(x + 2) = x^3 + 2x^2 + 2x^2 + 4x - x - 2 = x^3 + 4x^2 + 3x - 2$
    
    \item $(a + b)(a^2 - ab + b^2) = a^3 - a^2b + ab^2 + a^2b - ab^2 + b^3 = a^3 + b^3$
\end{exerciselist}
\end{solucion}
\end{exercise}

\begin{exercise}
\problem Use productos notables para desarrollar:

\begin{exerciselist}
    \item $(x + 4)^2$
    \item $(3x - 2)^2$
    \item $(x + 6)(x - 6)$
    \item $(2a + 3b)(2a - 3b)$
    \item $(x + 3)^3$
    \item $(2x - 1)^3$
\end{exerciselist}

\begin{solucion}
\begin{exerciselist}
    \item $(x + 4)^2 = x^2 + 2(x)(4) + 4^2 = x^2 + 8x + 16$
    
    \item $(3x - 2)^2 = (3x)^2 - 2(3x)(2) + 2^2 = 9x^2 - 12x + 4$
    
    \item $(x + 6)(x - 6) = x^2 - 6^2 = x^2 - 36$
    
    \item $(2a + 3b)(2a - 3b) = (2a)^2 - (3b)^2 = 4a^2 - 9b^2$
    
    \item $(x + 3)^3 = x^3 + 3x^2(3) + 3x(3^2) + 3^3 = x^3 + 9x^2 + 27x + 27$
    
    \item $(2x - 1)^3 = (2x)^3 - 3(2x)^2(1) + 3(2x)(1^2) - 1^3 = 8x^3 - 12x^2 + 6x - 1$
\end{exerciselist}
\end{solucion}
\end{exercise}

\begin{exercise}
\problem Factorice completamente las siguientes expresiones:

\begin{exerciselist}
    \item $6x^2 + 12x$
    \item $x^2 + 7x + 12$
    \item $x^2 - 25$
    \item $4x^2 - 12x + 9$
    \item $x^3 - 8$
    \item $2x^3 - 8x$
    \item $x^4 - 16$
    \item $ac + ad + bc + bd$
\end{exerciselist}

\begin{solucion}
\begin{exerciselist}
    \item $6x^2 + 12x = 6x(x + 2)$
    
    \item $x^2 + 7x + 12 = (x + 3)(x + 4)$ [Buscamos números que sumen 7 y multipliquen 12: 3 y 4]
    
    \item $x^2 - 25 = x^2 - 5^2 = (x + 5)(x - 5)$ [Diferencia de cuadrados]
    
    \item $4x^2 - 12x + 9 = (2x)^2 - 2(2x)(3) + 3^2 = (2x - 3)^2$ [Trinomio cuadrático perfecto]
    
    \item $x^3 - 8 = x^3 - 2^3 = (x - 2)(x^2 + 2x + 4)$ [Diferencia de cubos]
    
    \item $2x^3 - 8x = 2x(x^2 - 4) = 2x(x + 2)(x - 2)$ [Factor común y diferencia de cuadrados]
    
    \item $x^4 - 16 = (x^2)^2 - 4^2 = (x^2 + 4)(x^2 - 4) = (x^2 + 4)(x + 2)(x - 2)$ [Diferencia de cuadrados aplicada dos veces]
    
    \item $ac + ad + bc + bd = a(c + d) + b(c + d) = (a + b)(c + d)$ [Factorización por agrupación]
\end{exerciselist}
\end{solucion}
\end{exercise}

\begin{exercise}
\problem Evalúe los siguientes polinomios en los valores dados:

\begin{exerciselist}
    \item $P(x) = 2x^3 - 3x^2 + x - 4$; evalúe $P(2)$
    \item $Q(x) = x^4 - 2x^2 + 1$; evalúe $Q(-1)$
    \item $R(x) = 3x^2 - 5x + 2$; evalúe $R(0)$ y $R(1)$
    \item $S(x) = x^3 + 2x^2 - 5x - 6$; evalúe $S(3)$ y $S(-2)$
\end{exerciselist}

\begin{solucion}
\begin{exerciselist}
    \item $P(2) = 2(2)^3 - 3(2)^2 + 2 - 4 = 2(8) - 3(4) + 2 - 4 = 16 - 12 + 2 - 4 = 2$
    
    \item $Q(-1) = (-1)^4 - 2(-1)^2 + 1 = 1 - 2(1) + 1 = 1 - 2 + 1 = 0$
    
    \item $R(0) = 3(0)^2 - 5(0) + 2 = 2$\\
    $R(1) = 3(1)^2 - 5(1) + 2 = 3 - 5 + 2 = 0$
    
    \item $S(3) = (3)^3 + 2(3)^2 - 5(3) - 6 = 27 + 18 - 15 - 6 = 24$\\
    $S(-2) = (-2)^3 + 2(-2)^2 - 5(-2) - 6 = -8 + 8 + 10 - 6 = 4$
\end{exerciselist}
\end{solucion}
\end{exercise}

\begin{exercise}
\problem Use el Teorema del Factor para determinar si el binomio dado es un factor del polinomio:

\begin{exerciselist}
    \item ¿Es $(x - 2)$ un factor de $x^3 - 3x^2 + 4x - 12$?
    \item ¿Es $(x + 1)$ un factor de $2x^3 + x^2 - 5x + 2$?
    \item ¿Es $(x - 3)$ un factor de $x^4 - 4x^3 + 6x^2 - 4x + 1$?
    \item ¿Es $(2x - 1)$ un factor de $2x^3 - 3x^2 + 1$?
\end{exerciselist}

\begin{solucion}
\begin{exerciselist}
    \item Para $(x - 2)$, evaluamos $P(2)$:\\
    $P(2) = 2^3 - 3(2^2) + 4(2) - 12 = 8 - 12 + 8 - 12 = -8 \neq 0$\\
    No es factor.
    
    \item Para $(x + 1)$, evaluamos $P(-1)$:\\
    $P(-1) = 2(-1)^3 + (-1)^2 - 5(-1) + 2 = -2 + 1 + 5 + 2 = 6 \neq 0$\\
    No es factor.
    
    \item Para $(x - 3)$, evaluamos $P(3)$:\\
    $P(3) = 3^4 - 4(3^3) + 6(3^2) - 4(3) + 1 = 81 - 108 + 54 - 12 + 1 = 16 \neq 0$\\
    No es factor.
    
    \item Para $(2x - 1)$, evaluamos $P(\frac{1}{2})$:\\
    $P(\frac{1}{2}) = 2(\frac{1}{2})^3 - 3(\frac{1}{2})^2 + 1 = 2(\frac{1}{8}) - 3(\frac{1}{4}) + 1 = \frac{1}{4} - \frac{3}{4} + 1 = \frac{1}{2} \neq 0$\\
    No es factor.
\end{exerciselist}
\end{solucion}
\end{exercise}

\begin{exercise}
\problem \textbf{Problemas de aplicación:}

\begin{exerciselist}
    \item Un rectángulo tiene largo $(2x + 3)$ y ancho $(x - 1)$. Encuentre una expresión para su área.
    
    \item El volumen de una caja rectangular es $V = x^3 + 6x^2 + 11x + 6$. Si las dimensiones son $(x + 1)$, $(x + 2)$, y $(x + 3)$, verifique que esta expresión es correcta.
    
    \item Un proyectil se lanza verticalmente hacia arriba. Su altura en metros después de $t$ segundos está dada por $h(t) = -5t^2 + 20t + 25$. ¿Cuál es la altura inicial? ¿Cuál es la altura después de 2 segundos?
    
    \item El costo total de producir $x$ artículos está dado por $C(x) = 2x^2 + 15x + 100$. ¿Cuál es el costo fijo? ¿Cuál es el costo de producir 10 artículos?
\end{exerciselist}

\begin{solucion}
\begin{exerciselist}
    \item Área $= (2x + 3)(x - 1) = 2x^2 - 2x + 3x - 3 = 2x^2 + x - 3$
    
    \item Verificación:\\
    $(x + 1)(x + 2)(x + 3)$\\
    $= (x + 1)[(x + 2)(x + 3)]$\\
    $= (x + 1)[x^2 + 5x + 6]$\\
    $= x^3 + 5x^2 + 6x + x^2 + 5x + 6$\\
    $= x^3 + 6x^2 + 11x + 6$ ✓
    
    \item Altura inicial: $h(0) = -5(0)^2 + 20(0) + 25 = 25$ metros\\
    Altura después de 2 segundos: $h(2) = -5(4) + 20(2) + 25 = -20 + 40 + 25 = 45$ metros
    
    \item Costo fijo: $C(0) = 2(0)^2 + 15(0) + 100 = 100$ (cuando $x = 0$)\\
    Costo de 10 artículos: $C(10) = 2(100) + 15(10) + 100 = 200 + 150 + 100 = 450$
\end{exerciselist}
\end{solucion}
\end{exercise}