%========================================
% DETAILED SOLUTIONS: Raíces y Radicales
%========================================

\subsection*{Ejercicio 1: Simplificación de radicales (índice par)}

\textbf{Problema 1:}
\begin{align*}
\sqrt{200x^4y^3} &= \sqrt{100\cdot 2\cdot x^4\cdot y^2\cdot y}\\
&= \sqrt{100}\cdot\sqrt{x^4}\cdot\sqrt{y^2}\cdot\sqrt{2y}\\
&= 10\cdot x^2\cdot|y|\cdot\sqrt{2y}\\
&= 10x^2|y|\sqrt{2y}
\end{align*}
Si asumimos $y\geq 0$, la respuesta es $10x^2y\sqrt{2y}$.

\textbf{Problema 2:}
\begin{align*}
\sqrt{98a^6b^5} &= \sqrt{49\cdot 2\cdot a^6\cdot b^4\cdot b}\\
&= 7\cdot a^3\cdot b^2\cdot\sqrt{2b}\\
&= 7a^3b^2\sqrt{2b}
\end{align*}
(Asumiendo $a,b\geq 0$)

\textbf{Problema 3:}
\begin{align*}
\sqrt{50x^2}+\sqrt{32x^2} &= \sqrt{25\cdot 2\cdot x^2}+\sqrt{16\cdot 2\cdot x^2}\\
&= 5|x|\sqrt{2}+4|x|\sqrt{2}\\
&= 9|x|\sqrt{2}
\end{align*}

\subsection*{Ejercicio 2: Simplificación de radicales (índice impar)}

\textbf{Problema 1:}
\begin{align*}
\sqrt[3]{54a^5b^3} &= \sqrt[3]{27\cdot 2\cdot a^3\cdot a^2\cdot b^3}\\
&= \sqrt[3]{27}\cdot\sqrt[3]{a^3}\cdot\sqrt[3]{b^3}\cdot\sqrt[3]{2a^2}\\
&= 3\cdot a\cdot b\cdot\sqrt[3]{2a^2}\\
&= 3ab\sqrt[3]{2a^2}
\end{align*}
No se necesita valor absoluto porque el índice es impar.

\textbf{Problema 2:}
\begin{align*}
\sqrt[3]{-128x^7y^4} &= \sqrt[3]{-64\cdot 2\cdot x^6\cdot x\cdot y^3\cdot y}\\
&= \sqrt[3]{-64}\cdot\sqrt[3]{x^6}\cdot\sqrt[3]{y^3}\cdot\sqrt[3]{2xy}\\
&= -4\cdot x^2\cdot y\cdot\sqrt[3]{2xy}\\
&= -4x^2y\sqrt[3]{2xy}
\end{align*}

\textbf{Problema 3:}
\begin{align*}
\sqrt[4]{162a^9b^{12}} &= \sqrt[4]{81\cdot 2\cdot a^8\cdot a\cdot b^{12}}\\
&= 3\cdot a^2\cdot b^3\cdot\sqrt[4]{2a}\\
&= 3a^2b^3\sqrt[4]{2a}
\end{align*}
(Asumiendo $a\geq 0$ porque el índice 4 es par)

\subsection*{Ejercicio 3: Multiplicación de radicales}

\textbf{Problema 1:}
\begin{align*}
(\sqrt{7a})(\sqrt{21a^3}) &= \sqrt{7a\cdot 21a^3}\\
&= \sqrt{147a^4}\\
&= \sqrt{49\cdot 3\cdot a^4}\\
&= 7a^2\sqrt{3}
\end{align*}

\textbf{Problema 2:}
\begin{align*}
\sqrt{6x^2}\cdot\sqrt{24x^3} &= \sqrt{6x^2\cdot 24x^3}\\
&= \sqrt{144x^5}\\
&= \sqrt{144\cdot x^4\cdot x}\\
&= 12x^2\sqrt{x}
\end{align*}

\textbf{Problema 3:}
\begin{align*}
\sqrt[3]{4x}\cdot\sqrt[3]{16x^2} &= \sqrt[3]{4x\cdot 16x^2}\\
&= \sqrt[3]{64x^3}\\
&= 4x
\end{align*}

\subsection*{Ejercicio 4: Racionalización (denominador monomial)}

\textbf{Problema 1:}
$$\frac{3}{\sqrt{3}}=\frac{3}{\sqrt{3}}\cdot\frac{\sqrt{3}}{\sqrt{3}}=\frac{3\sqrt{3}}{3}=\sqrt{3}$$

\textbf{Problema 2:}
$$\frac{10}{\sqrt{5}}=\frac{10}{\sqrt{5}}\cdot\frac{\sqrt{5}}{\sqrt{5}}=\frac{10\sqrt{5}}{5}=2\sqrt{5}$$

\textbf{Problema 3:}
Para eliminar $\sqrt[3]{9}=\sqrt[3]{3^2}$, necesitamos obtener $\sqrt[3]{3^3}=3$:
\begin{align*}
\frac{6}{\sqrt[3]{9}} &= \frac{6}{\sqrt[3]{3^2}}\cdot\frac{\sqrt[3]{3}}{\sqrt[3]{3}}\\
&= \frac{6\sqrt[3]{3}}{\sqrt[3]{3^3}}\\
&= \frac{6\sqrt[3]{3}}{3}\\
&= 2\sqrt[3]{3}
\end{align*}

\subsection*{Ejercicio 5: Racionalización (denominador binomial)}

\textbf{Problema 1:}
El conjugado de $\sqrt{3}-\sqrt{2}$ es $\sqrt{3}+\sqrt{2}$:
\begin{align*}
\frac{2}{\sqrt{3}-\sqrt{2}} &= \frac{2}{\sqrt{3}-\sqrt{2}}\cdot\frac{\sqrt{3}+\sqrt{2}}{\sqrt{3}+\sqrt{2}}\\
&= \frac{2(\sqrt{3}+\sqrt{2})}{(\sqrt{3})^2-(\sqrt{2})^2}\\
&= \frac{2\sqrt{3}+2\sqrt{2}}{3-2}\\
&= \frac{2\sqrt{3}+2\sqrt{2}}{1}\\
&= 2\sqrt{3}+2\sqrt{2}
\end{align*}

\textbf{Problema 2:}
El conjugado de $4-\sqrt{2}$ es $4+\sqrt{2}$:
\begin{align*}
\frac{5}{4-\sqrt{2}} &= \frac{5}{4-\sqrt{2}}\cdot\frac{4+\sqrt{2}}{4+\sqrt{2}}\\
&= \frac{5(4+\sqrt{2})}{(4)^2-(\sqrt{2})^2}\\
&= \frac{20+5\sqrt{2}}{16-2}\\
&= \frac{20+5\sqrt{2}}{14}\\
&= \frac{10}{7}+\frac{5\sqrt{2}}{14}
\end{align*}

\textbf{Problema 3:}
El conjugado de $2+\sqrt{5}$ es $2-\sqrt{5}$:
\begin{align*}
\frac{3}{2+\sqrt{5}} &= \frac{3}{2+\sqrt{5}}\cdot\frac{2-\sqrt{5}}{2-\sqrt{5}}\\
&= \frac{3(2-\sqrt{5})}{(2)^2-(\sqrt{5})^2}\\
&= \frac{6-3\sqrt{5}}{4-5}\\
&= \frac{6-3\sqrt{5}}{-1}\\
&= -6+3\sqrt{5} \text{ o } 3\sqrt{5}-6
\end{align*}

\subsection*{Ejercicio 6: Suma y resta de radicales}

\textbf{Problema 1:}
Primero simplificamos cada radical:
\begin{itemize}
    \item $\sqrt{20}=\sqrt{4\cdot 5}=2\sqrt{5}$
    \item $\sqrt{45}=\sqrt{9\cdot 5}=3\sqrt{5}$
\end{itemize}
Entonces:
\begin{align*}
4\sqrt{20}-3\sqrt{5}+\sqrt{45} &= 4(2\sqrt{5})-3\sqrt{5}+3\sqrt{5}\\
&= 8\sqrt{5}-3\sqrt{5}+3\sqrt{5}\\
&= 8\sqrt{5}
\end{align*}

\textbf{Problema 2:}
\begin{itemize}
    \item $\sqrt{12}=\sqrt{4\cdot 3}=2\sqrt{3}$
    \item $\sqrt{27}=\sqrt{9\cdot 3}=3\sqrt{3}$
\end{itemize}
Entonces:
\begin{align*}
2\sqrt{12}+\sqrt{27}-5\sqrt{3} &= 2(2\sqrt{3})+3\sqrt{3}-5\sqrt{3}\\
&= 4\sqrt{3}+3\sqrt{3}-5\sqrt{3}\\
&= 2\sqrt{3}
\end{align*}

\textbf{Problema 3:}
\begin{itemize}
    \item $\sqrt{50}=\sqrt{25\cdot 2}=5\sqrt{2}$
    \item $\sqrt{18}=\sqrt{9\cdot 2}=3\sqrt{2}$
    \item $\sqrt{8}=\sqrt{4\cdot 2}=2\sqrt{2}$
\end{itemize}
Entonces:
\begin{align*}
\sqrt{50}+\sqrt{18}-\sqrt{8} &= 5\sqrt{2}+3\sqrt{2}-2\sqrt{2}\\
&= 6\sqrt{2}
\end{align*}

\subsection*{Ejercicio 7: Notación exponencial}

\textbf{Problema 1:}
$$\sqrt[5]{x^2}=x^{2/5}$$

\textbf{Problema 2:}
$$y^{-3/4}=\frac{1}{y^{3/4}}=\frac{1}{\sqrt[4]{y^3}}$$

\textbf{Problema 3:}
Usando las leyes de exponentes:
\begin{align*}
\frac{x^{3/2}\,x^{1/2}}{x^{5/2}} &= \frac{x^{3/2+1/2}}{x^{5/2}}\\
&= \frac{x^{4/2}}{x^{5/2}}\\
&= \frac{x^2}{x^{5/2}}\\
&= x^{2-5/2}\\
&= x^{4/2-5/2}\\
&= x^{-1/2}\\
&= \frac{1}{x^{1/2}}\\
&= \frac{1}{\sqrt{x}}
\end{align*}

\subsection*{Ejercicio 8: Problemas mixtos}

\textbf{Problema 1:}
\begin{align*}
\sqrt[4]{x^6} &= x^{6/4}\\
&= x^{3/2}\\
&= x^{1+1/2}\\
&= x\cdot x^{1/2}\\
&= x\sqrt{x}
\end{align*}

\textbf{Problema 2:}
\begin{align*}
\sqrt[3]{a^5} &= a^{5/3}\\
&= a^{3/3+2/3}\\
&= a^1\cdot a^{2/3}\\
&= a\cdot\sqrt[3]{a^2}\\
&= a\sqrt[3]{a^2}
\end{align*}

\textbf{Problema 3:}
\begin{align*}
\sqrt{16}+\sqrt[3]{27}-\sqrt[4]{16} &= 4+3-2\\
&= 5
\end{align*}
Justificación:
\begin{itemize}
    \item $\sqrt{16}=4$ porque $4^2=16$
    \item $\sqrt[3]{27}=3$ porque $3^3=27$
    \item $\sqrt[4]{16}=2$ porque $2^4=16$
\end{itemize}

\subsection*{Ejercicio 9: Dominio de funciones con radicales}

\textbf{Problema 1:}
Para que $f(x)=\sqrt{x-3}$ sea real, necesitamos que el radicando sea no negativo:
$$x-3\geq 0 \Rightarrow x\geq 3$$
Por lo tanto, el dominio es $[3,\infty)$ o $\{x\in\mathbb{R}\,|\,x\geq 3\}$.

\textbf{Problema 2:}
Para que $g(x)=\sqrt{5-2x}$ sea real:
\begin{align*}
5-2x &\geq 0\\
5 &\geq 2x\\
\frac{5}{2} &\geq x\\
x &\leq \frac{5}{2}
\end{align*}
Por lo tanto, el dominio es $(-\infty,\frac{5}{2}]$ o $\{x\in\mathbb{R}\,|\,x\leq \frac{5}{2}\}$.

\textbf{Problema 3:}
Como $h(x)=\sqrt[3]{x+1}$ tiene un índice impar (3), el radicando puede ser cualquier número real (positivo, negativo o cero). Por lo tanto, el dominio es $(-\infty,\infty)$ o $\mathbb{R}$ (todos los números reales).

\subsection*{Ejercicio 10: Práctica adicional}

\textbf{a)} $\sqrt{180}=\sqrt{36\cdot 5}=6\sqrt{5}$

\textbf{b)} $\sqrt[3]{-64}=-4$ porque $(-4)^3=-64$

\textbf{c)} $\sqrt{5}\cdot\sqrt{20}=\sqrt{100}=10$

\textbf{d)} $\dfrac{1}{\sqrt{7}}=\dfrac{1}{\sqrt{7}}\cdot\dfrac{\sqrt{7}}{\sqrt{7}}=\dfrac{\sqrt{7}}{7}$

\textbf{e)} $3\sqrt{2}+5\sqrt{2}=8\sqrt{2}$ (radicales semejantes)

\textbf{f)} $\sqrt[3]{x^4}=x^{4/3}$

\textbf{g)}
\begin{align*}
(\sqrt{3}+\sqrt{2})(\sqrt{3}-\sqrt{2}) &= (\sqrt{3})^2-(\sqrt{2})^2\\
&= 3-2\\
&= 1
\end{align*}
(Diferencia de cuadrados)

\textbf{h)} Para que $f(x)=\sqrt{x^2-9}$ sea real:
\begin{align*}
x^2-9 &\geq 0\\
x^2 &\geq 9\\
|x| &\geq 3
\end{align*}
Esto significa $x\leq -3$ o $x\geq 3$.
Dominio: $(-\infty,-3]\cup[3,\infty)$
