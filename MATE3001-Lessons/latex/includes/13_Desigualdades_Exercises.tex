%========================================
% EXERCISES: Desigualdades Lineales
%========================================
\newpage
\subsectiontitle{Ejercicios}

Resuelva cada desigualdad, exprese la solución en notación de intervalos y describa cómo representarla en la recta real.

\vspace{-0.2cm}

\subsection*{Lineales simples}

\begin{exercise}
\problem $2x - 7 \le 11$

\begin{solucion}
\begin{align*}
2x - 7 &\le 11 \\
2x &\le 18 \\
x &\le 9
\end{align*}

\textbf{Notación de intervalos:} $(-\infty, 9]$

\textbf{Representación gráfica:} En la recta real, se dibuja un círculo relleno en $x = 9$ (porque está incluido, $\le$) y se sombrea hacia la izquierda hasta $-\infty$.
\end{solucion}

\problem $4x > 10$

\begin{solucion}
\begin{align*}
4x &> 10 \\
x &> \dfrac{10}{4} \\
x &> \dfrac{5}{2}
\end{align*}

\textbf{Notación de intervalos:} $\left(\dfrac{5}{2}, \infty\right)$

\textbf{Representación gráfica:} En la recta real, se dibuja un círculo abierto en $x = \dfrac{5}{2} = 2.5$ (porque no está incluido, $>$) y se sombrea hacia la derecha hasta $\infty$.
\end{solucion}

\problem $3x - 5 \ge 11$

\begin{solucion}
\begin{align*}
3x - 5 &\ge 11 \\
3x &\ge 16 \\
x &\ge \dfrac{16}{3}
\end{align*}

\textbf{Notación de intervalos:} $\left[\dfrac{16}{3}, \infty\right)$

\textbf{Representación gráfica:} En la recta real, se dibuja un círculo relleno en $x = \dfrac{16}{3} \approx 5.33$ (porque está incluido, $\ge$) y se sombrea hacia la derecha hasta $\infty$.
\end{solucion}

\problem $5 - 3x < 16$

\begin{solucion}
\begin{align*}
5 - 3x &< 16 \\
-3x &< 11 \\
x &> -\dfrac{11}{3} \quad \textcolor{red}{\text{(dividimos entre $-3 < 0$ e invertimos)}}
\end{align*}

\textbf{Notación de intervalos:} $\left(-\dfrac{11}{3}, \infty\right)$

\textbf{Representación gráfica:} En la recta real, se dibuja un círculo abierto en $x = -\dfrac{11}{3} \approx -3.67$ (porque no está incluido, $>$) y se sombrea hacia la derecha hasta $\infty$.
\end{solucion}

\problem $\dfrac{x}{3} + 2 \le \dfrac{5}{6}$

\begin{solucion}
\begin{align*}
\dfrac{x}{3} + 2 &\le \dfrac{5}{6} \\
\dfrac{x}{3} &\le \dfrac{5}{6} - 2 \\
\dfrac{x}{3} &\le \dfrac{5}{6} - \dfrac{12}{6} \\
\dfrac{x}{3} &\le -\dfrac{7}{6} \\
x &\le -\dfrac{7}{2} \quad \text{(multiplicamos por 3 > 0)}
\end{align*}

\textbf{Notación de intervalos:} $\left(-\infty, -\dfrac{7}{2}\right]$

\textbf{Representación gráfica:} En la recta real, se dibuja un círculo relleno en $x = -\dfrac{7}{2} = -3.5$ (porque está incluido, $\le$) y se sombrea hacia la izquierda hasta $-\infty$.
\end{solucion}

\problem $\dfrac{2 - x}{4} > -\dfrac{1}{2}$

\begin{solucion}
\begin{align*}
\dfrac{2 - x}{4} &> -\dfrac{1}{2} \\
2 - x &> -2 \quad \text{(multiplicamos por 4 > 0)} \\
-x &> -4 \\
x &< 4 \quad \textcolor{red}{\text{(multiplicamos por $-1 < 0$ e invertimos)}}
\end{align*}

\textbf{Notación de intervalos:} $(-\infty, 4)$

\textbf{Representación gráfica:} En la recta real, se dibuja un círculo abierto en $x = 4$ (porque no está incluido, $<$) y se sombrea hacia la izquierda hasta $-\infty$.
\end{solucion}

\problem $-6x + 9 \ge -3$

\begin{solucion}
\begin{align*}
-6x + 9 &\ge -3 \\
-6x &\ge -12 \\
x &\le 2 \quad \textcolor{red}{\text{(dividimos entre $-6 < 0$ e invertimos)}}
\end{align*}

\textbf{Notación de intervalos:} $(-\infty, 2]$

\textbf{Representación gráfica:} En la recta real, se dibuja un círculo relleno en $x = 2$ (porque está incluido, $\le$) y se sombrea hacia la izquierda hasta $-\infty$.
\end{solucion}

\problem $7 - 2x > 1$

\begin{solucion}
\begin{align*}
7 - 2x &> 1 \\
-2x &> -6 \\
x &< 3 \quad \textcolor{red}{\text{(dividimos entre $-2 < 0$ e invertimos)}}
\end{align*}

\textbf{Notación de intervalos:} $(-\infty, 3)$

\textbf{Representación gráfica:} En la recta real, se dibuja un círculo abierto en $x = 3$ (porque no está incluido, $<$) y se sombrea hacia la izquierda hasta $-\infty$.
\end{solucion}

\problem $\dfrac{5x - 1}{2} \le 3x + 4$

\begin{solucion}
\begin{align*}
\dfrac{5x - 1}{2} &\le 3x + 4 \\
5x - 1 &\le 6x + 8 \quad \text{(multiplicamos por 2 > 0)} \\
-1 &\le x + 8 \\
-9 &\le x \\
x &\ge -9
\end{align*}

\textbf{Notación de intervalos:} $[-9, \infty)$

\textbf{Representación gráfica:} En la recta real, se dibuja un círculo relleno en $x = -9$ (porque está incluido, $\ge$) y se sombrea hacia la derecha hasta $\infty$.
\end{solucion}

\problem $\dfrac{x - 4}{3} \ge \dfrac{2x + 1}{6}$

\begin{solucion}
\begin{align*}
\dfrac{x - 4}{3} &\ge \dfrac{2x + 1}{6} \\
2(x - 4) &\ge 2x + 1 \quad \text{(multiplicamos por 6 > 0)} \\
2x - 8 &\ge 2x + 1 \\
-8 &\ge 1 \quad \text{(falso)}
\end{align*}

\textbf{Solución:} No hay solución. El conjunto solución es $\varnothing$ (conjunto vacío).

\textbf{Representación gráfica:} No hay ningún punto en la recta real que satisfaga esta desigualdad.
\end{solucion}
\end{exercise}

\subsection*{Compuestas}

\begin{exercise}
\problem $-1 \le 2x + 3 < 7$

\begin{solucion}
\begin{align*}
-1 &\le 2x + 3 < 7 \\
-4 &\le 2x < 4 \quad \text{(restamos 3 de las tres partes)} \\
-2 &\le x < 2 \quad \text{(dividimos entre 2 > 0)}
\end{align*}

\textbf{Notación de intervalos:} $[-2, 2)$

\textbf{Representación gráfica:} En la recta real, se dibuja un círculo relleno en $x = -2$ (incluido, $\le$), un círculo abierto en $x = 2$ (no incluido, $<$), y se sombrea la región entre ambos puntos.
\end{solucion}

\problem $1 < \dfrac{3x - 2}{2} \le 5$

\begin{solucion}
\begin{align*}
1 &< \dfrac{3x - 2}{2} \le 5 \\
2 &< 3x - 2 \le 10 \quad \text{(multiplicamos por 2 > 0)} \\
4 &< 3x \le 12 \quad \text{(sumamos 2)} \\
\dfrac{4}{3} &< x \le 4 \quad \text{(dividimos entre 3 > 0)}
\end{align*}

\textbf{Notación de intervalos:} $\left(\dfrac{4}{3}, 4\right]$

\textbf{Representación gráfica:} En la recta real, se dibuja un círculo abierto en $x = \dfrac{4}{3} \approx 1.33$ (no incluido, $>$), un círculo relleno en $x = 4$ (incluido, $\le$), y se sombrea la región entre ambos puntos.
\end{solucion}

\problem $-4 \le \dfrac{x - 1}{2} < 2$

\begin{solucion}
\begin{align*}
-4 &\le \dfrac{x - 1}{2} < 2 \\
-8 &\le x - 1 < 4 \quad \text{(multiplicamos por 2 > 0)} \\
-7 &\le x < 5 \quad \text{(sumamos 1)}
\end{align*}

\textbf{Notación de intervalos:} $[-7, 5)$

\textbf{Representación gráfica:} En la recta real, se dibuja un círculo relleno en $x = -7$ (incluido, $\le$), un círculo abierto en $x = 5$ (no incluido, $<$), y se sombrea la región entre ambos puntos.
\end{solucion}

\problem $3x - 5 < -2 \;\;\text{o}\;\; 3x - 5 \ge 7$

\begin{solucion}
\textbf{Primera parte:}
\begin{align*}
3x - 5 &< -2 \\
3x &< 3 \\
x &< 1
\end{align*}

\textbf{Segunda parte:}
\begin{align*}
3x - 5 &\ge 7 \\
3x &\ge 12 \\
x &\ge 4
\end{align*}

\textbf{Notación de intervalos:} $(-\infty, 1) \cup [4, \infty)$

\textbf{Representación gráfica:} En la recta real, se dibujan dos regiones separadas:
\begin{itemize}
    \item Círculo abierto en $x = 1$ (no incluido) con sombra hacia la izquierda hasta $-\infty$
    \item Círculo relleno en $x = 4$ (incluido) con sombra hacia la derecha hasta $\infty$
\end{itemize}
\end{solucion}

\problem $\dfrac{x + 2}{4} \le -1 \;\;\text{o}\;\; \dfrac{x + 2}{4} > 3$

\begin{solucion}
\textbf{Primera parte:}
\begin{align*}
\dfrac{x + 2}{4} &\le -1 \\
x + 2 &\le -4 \quad \text{(multiplicamos por 4 > 0)} \\
x &\le -6
\end{align*}

\textbf{Segunda parte:}
\begin{align*}
\dfrac{x + 2}{4} &> 3 \\
x + 2 &> 12 \quad \text{(multiplicamos por 4 > 0)} \\
x &> 10
\end{align*}

\textbf{Notación de intervalos:} $(-\infty, -6] \cup (10, \infty)$

\textbf{Representación gráfica:} En la recta real, se dibujan dos regiones separadas:
\begin{itemize}
    \item Círculo relleno en $x = -6$ (incluido) con sombra hacia la izquierda hasta $-\infty$
    \item Círculo abierto en $x = 10$ (no incluido) con sombra hacia la derecha hasta $\infty$
\end{itemize}
\end{solucion}
\end{exercise}

\subsection*{Modelado}

\begin{exercise}
\problem Un artículo con precio $p$ recibe un descuento de \$6 y debe costar al menos \$19 después del descuento. Escriba y resuelva la desigualdad para $p$.

\begin{solucion}
\textbf{Traducción:} ``al menos \$19'' significa ``mayor o igual que \$19''.

El precio después del descuento es $p - 6$, entonces:
\begin{align*}
p - 6 &\ge 19 \\
p &\ge 25
\end{align*}

\textbf{Interpretación:} El precio original del artículo debe ser al menos \$25.

\textbf{Notación de intervalos:} $[25, \infty)$

\textbf{Representación gráfica:} En la recta real, se dibuja un círculo relleno en $p = 25$ (incluido, $\ge$) y se sombrea hacia la derecha hasta $\infty$.
\end{solucion}

\problem En un examen de 100 puntos, cada pregunta correcta vale 4 puntos y cada incorrecta resta 1 punto. Si se contestan exactamente 30 preguntas y se necesita un puntaje de al menos 80 puntos, ¿cuántas preguntas correctas ($x$) se necesitan como mínimo?

\begin{solucion}
\textbf{Variables:}
\begin{itemize}
    \item $x$ = número de preguntas correctas
    \item $30 - x$ = número de preguntas incorrectas (porque se contestan 30 en total)
\end{itemize}

\textbf{Puntaje:}
\[\text{Puntaje} = 4x - 1(30 - x) = 4x - 30 + x = 5x - 30\]

\textbf{Desigualdad:} ``al menos 80 puntos'' significa $\ge 80$
\begin{align*}
5x - 30 &\ge 80 \\
5x &\ge 110 \\
x &\ge 22
\end{align*}

\textbf{Interpretación:} Se necesitan al menos 22 preguntas correctas (de 30) para obtener un puntaje de al menos 80 puntos.

\textbf{Notación de intervalos:} Como $x$ debe ser un número entero entre 0 y 30, la solución es $\{22, 23, 24, 25, 26, 27, 28, 29, 30\}$ o en notación continua: $[22, 30]$.

\textbf{Representación gráfica:} En la recta real de 0 a 30, se dibuja un círculo relleno en $x = 22$ (incluido) y se sombrea desde 22 hasta 30 (también incluido).
\end{solucion}
\end{exercise}
