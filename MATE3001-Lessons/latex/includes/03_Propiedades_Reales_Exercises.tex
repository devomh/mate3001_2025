%========================================
% EXERCISES: Propiedades de los Números Reales y Exponentes
%========================================

\section{Ejercicios}

\begin{exercise}
\problem Enuncie la propiedad de los números reales que se está usando en cada expresión:

\begin{exerciselist}
    \item $5x + 2y = 2y + 5x$
    \item $4(5x + 1) = 20x + 4$
    \item $(ab)c = a(bc)$
    \item $(x + y)(x-y) = (x+y)x + (x+y)(-y)$
    \item $3(x + y + z) = 3x + 3y + 3z$
\end{exerciselist}

\begin{solucion}
\begin{exerciselist}
    \item Conmutativa de la suma
    \item Distributiva
    \item Asociativa de la multiplicación
    \item Distributiva
    \item Distributiva
\end{exerciselist}
\end{solucion}
\end{exercise}

\begin{exercise}
\problem Clasifique cada número como natural ($\mathbb{N}$), entero ($\mathbb{Z}$), racional ($\mathbb{Q}$), irracional, o real ($\mathbb{R}$). (Un número puede pertenecer a más de una categoría.)

\begin{exerciselist}
    \item $7$
    \item $-3$
    \item $\frac{2}{5}$
    \item $0$
    \item $\sqrt{16}$
    \item $\sqrt{7}$
    \item $\pi$
    \item $0.75$
    \item $0.\overline{3}$
    \item $-\frac{8}{3}$
\end{exerciselist}

\begin{solucion}
\begin{exerciselist}
    \item $7$: Natural, Entero, Racional, Real
    \item $-3$: Entero, Racional, Real
    \item $\frac{2}{5}$: Racional, Real
    \item $0$: Entero, Racional, Real
    \item $\sqrt{16} = 4$: Natural, Entero, Racional, Real
    \item $\sqrt{7}$: Irracional, Real
    \item $\pi$: Irracional, Real
    \item $0.75 = \frac{3}{4}$: Racional, Real
    \item $0.\overline{3} = \frac{1}{3}$: Racional, Real
    \item $-\frac{8}{3}$: Racional, Real
\end{exerciselist}
\end{solucion}
\end{exercise}

\begin{exercise}
\problem Ubique los siguientes números en la recta numérica:

\begin{exerciselist}
    \item $-2.5$
    \item $\frac{3}{2}$
    \item $-\sqrt{4}$
    \item $\pi$
    \item $0$
\end{exerciselist}

\begin{solucion}
En la recta numérica, de izquierda a derecha:
\begin{itemize}
    \item $-2.5$: Entre $-3$ y $-2$, en el punto medio
    \item $-\sqrt{4} = -2$: En el punto $-2$
    \item $0$: En el origen
    \item $\frac{3}{2} = 1.5$: Entre $1$ y $2$, en el punto medio
    \item $\pi \approx 3.14$: Entre $3$ y $4$, más cerca de $3$
\end{itemize}
\end{solucion}
\end{exercise}

\begin{exercise}
\problem Evalúe las siguientes expresiones usando el orden de operaciones correcto:

\begin{exerciselist}
    \item $8 + 2 \times 3^2$
    \item $(5 + 3) \times 2 - 4$
    \item $20 \div 4 + 3 \times 2$
    \item $2^3 + 4(5 - 2)$
    \item $\frac{12 + 8}{4} - 2^2$
    \item $6 + 2 \times 3^2 - (8 - 5)$
\end{exerciselist}

\begin{solucion}
\begin{exerciselist}
    \item $8 + 2 \times 3^2 = 8 + 2 \times 9 = 8 + 18 = 26$
    \item $(5 + 3) \times 2 - 4 = 8 \times 2 - 4 = 16 - 4 = 12$
    \item $20 \div 4 + 3 \times 2 = 5 + 6 = 11$
    \item $2^3 + 4(5 - 2) = 8 + 4(3) = 8 + 12 = 20$
    \item $\frac{12 + 8}{4} - 2^2 = \frac{20}{4} - 4 = 5 - 4 = 1$
    \item $6 + 2 \times 3^2 - (8 - 5) = 6 + 2 \times 9 - 3 = 6 + 18 - 3 = 21$
\end{exerciselist}
\end{solucion}
\end{exercise}

\begin{exercise}
\problem Realice las siguientes operaciones con fracciones:

\begin{exerciselist}
    \item $\frac{2}{3} + \frac{1}{4}$
    \item $\frac{5}{6} - \frac{1}{3}$
    \item $\frac{3}{4} \times \frac{2}{5}$
    \item $\frac{7}{8} \div \frac{3}{4}$
    \item $\frac{2}{5} + \frac{3}{10}$
    \item $\frac{4}{7} \times \frac{14}{8}$
\end{exerciselist}

\begin{solucion}
\begin{exerciselist}
    \item $\frac{2}{3} + \frac{1}{4} = \frac{2 \cdot 4 + 1 \cdot 3}{3 \cdot 4} = \frac{8 + 3}{12} = \frac{11}{12}$
    \item $\frac{5}{6} - \frac{1}{3} = \frac{5}{6} - \frac{2}{6} = \frac{3}{6} = \frac{1}{2}$
    \item $\frac{3}{4} \times \frac{2}{5} = \frac{3 \times 2}{4 \times 5} = \frac{6}{20} = \frac{3}{10}$
    \item $\frac{7}{8} \div \frac{3}{4} = \frac{7}{8} \times \frac{4}{3} = \frac{28}{24} = \frac{7}{6}$
    \item $\frac{2}{5} + \frac{3}{10} = \frac{4}{10} + \frac{3}{10} = \frac{7}{10}$
    \item $\frac{4}{7} \times \frac{14}{8} = \frac{4 \times 14}{7 \times 8} = \frac{56}{56} = 1$
\end{exerciselist}
\end{solucion}
\end{exercise}

\begin{exercise}
\problem Simplifique las siguientes expresiones aplicando las propiedades de los números reales:

\begin{exerciselist}
    \item $3x + 7x$
    \item $5(2y + 4)$
    \item $(a + b) + c = a + (b + c)$ (identifique la propiedad)
    \item $2(3x + 4y - 5)$
    \item $\frac{1}{2}(4a + 6b)$
    \item $x \cdot 0 + y \cdot 1$
\end{exerciselist}

\begin{solucion}
\begin{exerciselist}
    \item $3x + 7x = (3 + 7)x = 10x$ (Propiedad distributiva)
    \item $5(2y + 4) = 5 \cdot 2y + 5 \cdot 4 = 10y + 20$ (Propiedad distributiva)
    \item Esta expresión ilustra la propiedad asociativa de la suma
    \item $2(3x + 4y - 5) = 6x + 8y - 10$ (Propiedad distributiva)
    \item $\frac{1}{2}(4a + 6b) = 2a + 3b$ (Propiedad distributiva)
    \item $x \cdot 0 + y \cdot 1 = 0 + y = y$ (Propiedades del neutro multiplicativo y aditivo)
\end{exerciselist}
\end{solucion}
\end{exercise}

\begin{exercise}
\problem Encuentre el inverso aditivo y multiplicativo (si existe) de cada número:

\begin{exerciselist}
    \item $5$
    \item $-3$
    \item $\frac{2}{7}$
    \item $0$
    \item $-\frac{4}{9}$
    \item $1$
\end{exerciselist}

\begin{solucion}
\begin{exerciselist}
    \item $5$: Inverso aditivo: $-5$, Inverso multiplicativo: $\frac{1}{5}$
    \item $-3$: Inverso aditivo: $3$, Inverso multiplicativo: $-\frac{1}{3}$
    \item $\frac{2}{7}$: Inverso aditivo: $-\frac{2}{7}$, Inverso multiplicativo: $\frac{7}{2}$
    \item $0$: Inverso aditivo: $0$, Inverso multiplicativo: No existe
    \item $-\frac{4}{9}$: Inverso aditivo: $\frac{4}{9}$, Inverso multiplicativo: $-\frac{9}{4}$
    \item $1$: Inverso aditivo: $-1$, Inverso multiplicativo: $1$
\end{exerciselist}
\end{solucion}
\end{exercise}

\begin{exercise}
\problem Evalúe las siguientes potencias:

\begin{exerciselist}
    \item $2^4$
    \item $(-3)^3$
    \item $(-2)^4$
    \item $5^2$
    \item $(-1)^5$
    \item $10^3$
    \item $\left(\frac{1}{2}\right)^3$
    \item $\left(-\frac{2}{3}\right)^2$
\end{exerciselist}

\begin{solucion}
\begin{exerciselist}
    \item $2^4 = 2 \times 2 \times 2 \times 2 = 16$
    \item $(-3)^3 = (-3) \times (-3) \times (-3) = -27$
    \item $(-2)^4 = (-2) \times (-2) \times (-2) \times (-2) = 16$
    \item $5^2 = 5 \times 5 = 25$
    \item $(-1)^5 = (-1) \times (-1) \times (-1) \times (-1) \times (-1) = -1$
    \item $10^3 = 10 \times 10 \times 10 = 1000$
    \item $\left(\frac{1}{2}\right)^3 = \frac{1}{2} \times \frac{1}{2} \times \frac{1}{2} = \frac{1}{8}$
    \item $\left(-\frac{2}{3}\right)^2 = \left(-\frac{2}{3}\right) \times \left(-\frac{2}{3}\right) = \frac{4}{9}$
\end{exerciselist}
\end{solucion}
\end{exercise}

\begin{exercise}
\problem Convierta los siguientes decimales a fracciones:

\begin{exerciselist}
    \item $0.5$
    \item $0.25$
    \item $0.75$
    \item $0.\overline{6}$
    \item $0.125$
    \item $0.\overline{45}$
\end{exerciselist}

\begin{solucion}
\begin{exerciselist}
    \item $0.5 = \frac{5}{10} = \frac{1}{2}$
    \item $0.25 = \frac{25}{100} = \frac{1}{4}$
    \item $0.75 = \frac{75}{100} = \frac{3}{4}$
    \item $0.\overline{6} = \frac{6}{9} = \frac{2}{3}$
    \item $0.125 = \frac{125}{1000} = \frac{1}{8}$
    \item $0.\overline{45} = \frac{45}{99} = \frac{5}{11}$
\end{exerciselist}
\end{solucion}
\end{exercise}

\begin{exercise}
\problem \textbf{Problemas de aplicación:}

\begin{exerciselist}
    \item Un rectángulo tiene largo $3x + 2$ y ancho $2$. Use la propiedad distributiva para encontrar su área.
    \item Si $a = 3$, $b = -2$, y $c = 4$, evalúe $2a + 3b - c$.
    \item Un estudiante tiene $\frac{3}{4}$ de pizza y come $\frac{1}{3}$ de pizza. ¿Cuánta pizza le queda?
    \item La temperatura cambió $-5°C$ por la mañana, $+8°C$ al mediodía, y $-3°C$ por la noche. ¿Cuál fue el cambio total de temperatura?
\end{exerciselist}

\begin{solucion}
\begin{exerciselist}
    \item Área $= (3x + 2) \times 2 = 2(3x + 2) = 6x + 4$
    \item $2a + 3b - c = 2(3) + 3(-2) - 4 = 6 - 6 - 4 = -4$
    \item Pizza restante $= \frac{3}{4} - \frac{1}{3} = \frac{9}{12} - \frac{4}{12} = \frac{5}{12}$ de pizza
    \item Cambio total $= -5 + 8 + (-3) = 0°C$
\end{exerciselist}
\end{solucion}
\end{exercise}