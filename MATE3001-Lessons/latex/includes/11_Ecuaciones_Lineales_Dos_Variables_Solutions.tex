%========================================
% DETAILED SOLUTIONS: Ecuaciones Lineales en Dos Variables
%========================================

\subsection*{Ejercicio 1: Encontrar Soluciones}

\textbf{Problema 1:} $x + y = 5$

Para $x = 0$: $0 + y = 5 \implies y = 5$. Punto: $(0, 5)$

Para $x = 2$: $2 + y = 5 \implies y = 3$. Punto: $(2, 3)$

Para $y = 0$: $x + 0 = 5 \implies x = 5$. Punto: $(5, 0)$

Para $x = -1$: $-1 + y = 5 \implies y = 6$. Punto: $(-1, 6)$

\medskip

\textbf{Problema 2:} $2x - y = 4$

Para $x = 0$: $2(0) - y = 4 \implies -y = 4 \implies y = -4$. Punto: $(0, -4)$

Para $x = 2$: $2(2) - y = 4 \implies 4 - y = 4 \implies y = 0$. Punto: $(2, 0)$

Para $y = 0$: $2x - 0 = 4 \implies 2x = 4 \implies x = 2$. Punto: $(2, 0)$

Para $x = -2$: $2(-2) - y = 4 \implies -4 - y = 4 \implies y = -8$. Punto: $(-2, -8)$

\medskip

\textbf{Problema 3:} $3x + 2y = 12$

Para $x = 0$: $3(0) + 2y = 12 \implies 2y = 12 \implies y = 6$. Punto: $(0, 6)$

Para $y = 0$: $3x + 2(0) = 12 \implies 3x = 12 \implies x = 4$. Punto: $(4, 0)$

Para $x = 2$: $3(2) + 2y = 12 \implies 6 + 2y = 12 \implies 2y = 6 \implies y = 3$. Punto: $(2, 3)$

Para $y = 3$: $3x + 2(3) = 12 \implies 3x + 6 = 12 \implies 3x = 6 \implies x = 2$. Punto: $(2, 3)$

\newpage

\subsection*{Ejercicio 2: Encontrar Intersecciones con los Ejes}

\textbf{Problema 1:} $2x + 3y = 6$

\textbf{Intersección con el eje $x$ (hacer $y = 0$):}
\begin{align*}
2x + 3(0) &= 6 \\
2x &= 6 \\
x &= 3
\end{align*}
Punto de intersección con el eje $x$: $(3, 0)$

\textbf{Intersección con el eje $y$ (hacer $x = 0$):}
\begin{align*}
2(0) + 3y &= 6 \\
3y &= 6 \\
y &= 2
\end{align*}
Punto de intersección con el eje $y$: $(0, 2)$

\medskip

\textbf{Problema 2:} $4x - y = 8$

\textbf{Intersección con el eje $x$:}
$$4x - 0 = 8 \implies x = 2$$
Punto: $(2, 0)$

\textbf{Intersección con el eje $y$:}
$$4(0) - y = 8 \implies y = -8$$
Punto: $(0, -8)$

\medskip

\textbf{Problema 3:} $x + 2y = 10$

\textbf{Intersección con el eje $x$:}
$$x + 2(0) = 10 \implies x = 10$$
Punto: $(10, 0)$

\textbf{Intersección con el eje $y$:}
$$0 + 2y = 10 \implies y = 5$$
Punto: $(0, 5)$

\medskip

\textbf{Problema 4:} $5x - 2y = 20$

\textbf{Intersección con el eje $x$:}
$$5x - 0 = 20 \implies x = 4$$
Punto: $(4, 0)$

\textbf{Intersección con el eje $y$:}
$$0 - 2y = 20 \implies y = -10$$
Punto: $(0, -10)$

\medskip

\textbf{Problema 5:} $3x + 4y = 24$

\textbf{Intersección con el eje $x$:}
$$3x + 0 = 24 \implies x = 8$$
Punto: $(8, 0)$

\textbf{Intersección con el eje $y$:}
$$0 + 4y = 24 \implies y = 6$$
Punto: $(0, 6)$

\medskip

\textbf{Problema 6:} $-2x + 5y = 10$

\textbf{Intersección con el eje $x$:}
$$-2x + 0 = 10 \implies x = -5$$
Punto: $(-5, 0)$

\textbf{Intersección con el eje $y$:}
$$0 + 5y = 10 \implies y = 2$$
Punto: $(0, 2)$

\newpage

\subsection*{Ejercicio 3: Calcular la Pendiente}

\textbf{Problema 1:} Pendiente entre $(1, 2)$ y $(3, 6)$

$$m = \frac{y_2 - y_1}{x_2 - x_1} = \frac{6 - 2}{3 - 1} = \frac{4}{2} = 2$$

\medskip

\textbf{Problema 2:} Pendiente entre $(0, 5)$ y $(4, 1)$

$$m = \frac{1 - 5}{4 - 0} = \frac{-4}{4} = -1$$

\medskip

\textbf{Problema 3:} Pendiente entre $(-2, 3)$ y $(4, -1)$

$$m = \frac{-1 - 3}{4 - (-2)} = \frac{-4}{6} = -\frac{2}{3}$$

\medskip

\textbf{Problema 4:} Pendiente entre $(5, 2)$ y $(5, 7)$

$$m = \frac{7 - 2}{5 - 5} = \frac{5}{0} = \text{indefinida}$$

Los puntos tienen la misma coordenada $x$, por lo que forman una \textbf{recta vertical} con pendiente indefinida.

\medskip

\textbf{Problema 5:} Pendiente entre $(-3, 4)$ y $(2, 4)$

$$m = \frac{4 - 4}{2 - (-3)} = \frac{0}{5} = 0$$

Los puntos tienen la misma coordenada $y$, por lo que forman una \textbf{recta horizontal} con pendiente 0.

\medskip

\textbf{Problema 6:} Pendiente entre $(-1, -2)$ y $(3, 10)$

$$m = \frac{10 - (-2)}{3 - (-1)} = \frac{12}{4} = 3$$

\medskip

\textbf{Problema 7:} Pendiente entre $(6, 1)$ y $(2, 5)$

$$m = \frac{5 - 1}{2 - 6} = \frac{4}{-4} = -1$$

\medskip

\textbf{Problema 8:} Pendiente entre $(0, 0)$ y $(4, 8)$

$$m = \frac{8 - 0}{4 - 0} = \frac{8}{4} = 2$$

\newpage

\subsection*{Ejercicio 4: Identificar el Tipo de Pendiente}

\textbf{Problema 1:} Una recta que sube de izquierda a derecha.

Cuando una recta sube de izquierda a derecha, el cambio vertical y el cambio horizontal tienen el mismo signo (ambos positivos o ambos negativos). Por lo tanto, la pendiente es \textbf{positiva}.

\medskip

\textbf{Problema 2:} $y = -3$

Esta es la ecuación de una recta horizontal (todos los puntos tienen $y = -3$). Las rectas horizontales tienen \textbf{pendiente cero}.

\medskip

\textbf{Problema 3:} $x = 5$

Esta es la ecuación de una recta vertical (todos los puntos tienen $x = 5$). Las rectas verticales tienen \textbf{pendiente indefinida}.

\medskip

\textbf{Problema 4:} Una recta que pasa por $(1, 5)$ y $(4, 2)$.

Calculamos la pendiente:
$$m = \frac{2 - 5}{4 - 1} = \frac{-3}{3} = -1$$

La pendiente es \textbf{negativa}. La recta baja de izquierda a derecha.

\medskip

\textbf{Problema 5:} $y = 7x + 2$

Esta ecuación está en forma pendiente-intersecto donde $m = 7$. La pendiente es \textbf{positiva}.

\medskip

\textbf{Problema 6:} $y = -\frac{1}{2}x + 3$

Esta ecuación está en forma pendiente-intersecto donde $m = -\frac{1}{2}$. La pendiente es \textbf{negativa}.

\newpage

\subsection*{Ejercicio 5: Forma Pendiente-Intersecto}

\textbf{Problema 1:} $y = 3x + 5$

Comparando con $y = mx + b$:
\begin{itemize}
    \item Pendiente: $m = 3$
    \item Intersección con el eje $y$: $b = 5$
    \item La recta pasa por el punto $(0, 5)$
\end{itemize}

\medskip

\textbf{Problema 2:} $y = -2x - 1$

\begin{itemize}
    \item Pendiente: $m = -2$
    \item Intersección con el eje $y$: $b = -1$
    \item La recta pasa por el punto $(0, -1)$
\end{itemize}

\medskip

\textbf{Problema 3:} $y = \frac{1}{2}x + 4$

\begin{itemize}
    \item Pendiente: $m = \frac{1}{2}$
    \item Intersección con el eje $y$: $b = 4$
    \item La recta pasa por el punto $(0, 4)$
\end{itemize}

\medskip

\textbf{Problema 4:} $y = -x + 7$

\begin{itemize}
    \item Pendiente: $m = -1$ (el coeficiente implícito de $x$)
    \item Intersección con el eje $y$: $b = 7$
    \item La recta pasa por el punto $(0, 7)$
\end{itemize}

\medskip

\textbf{Problema 5:} $y = 4x$

Esta ecuación se puede escribir como $y = 4x + 0$:
\begin{itemize}
    \item Pendiente: $m = 4$
    \item Intersección con el eje $y$: $b = 0$
    \item La recta pasa por el origen $(0, 0)$
\end{itemize}

\medskip

\textbf{Problema 6:} $y = -\frac{3}{4}x - 2$

\begin{itemize}
    \item Pendiente: $m = -\frac{3}{4}$
    \item Intersección con el eje $y$: $b = -2$
    \item La recta pasa por el punto $(0, -2)$
\end{itemize}

\newpage

\subsection*{Ejercicio 6: Convertir a Forma Pendiente-Intersecto}

\textbf{Problema 1:} $2x + y = 6$

Despejamos $y$:
\begin{align*}
y &= -2x + 6
\end{align*}

Pendiente: $m = -2$; Intersección $y$: $b = 6$

\medskip

\textbf{Problema 2:} $3x - y = 9$

\begin{align*}
-y &= -3x + 9 \\
y &= 3x - 9
\end{align*}

Pendiente: $m = 3$; Intersección $y$: $b = -9$

\medskip

\textbf{Problema 3:} $4x + 2y = 8$

\begin{align*}
2y &= -4x + 8 \\
y &= \frac{-4x + 8}{2} \\
y &= -2x + 4
\end{align*}

Pendiente: $m = -2$; Intersección $y$: $b = 4$

\medskip

\textbf{Problema 4:} $x - 2y = 10$

\begin{align*}
-2y &= -x + 10 \\
y &= \frac{-x + 10}{-2} \\
y &= \frac{x}{2} - 5 \\
y &= \frac{1}{2}x - 5
\end{align*}

Pendiente: $m = \frac{1}{2}$; Intersección $y$: $b = -5$

\medskip

\textbf{Problema 5:} $6x + 3y = 12$

\begin{align*}
3y &= -6x + 12 \\
y &= \frac{-6x + 12}{3} \\
y &= -2x + 4
\end{align*}

Pendiente: $m = -2$; Intersección $y$: $b = 4$

\medskip

\textbf{Problema 6:} $5x - 2y = 20$

\begin{align*}
-2y &= -5x + 20 \\
y &= \frac{-5x + 20}{-2} \\
y &= \frac{5x}{2} - 10 \\
y &= \frac{5}{2}x - 10
\end{align*}

Pendiente: $m = \frac{5}{2}$; Intersección $y$: $b = -10$

\newpage

\subsection*{Ejercicio 7: Escribir Ecuaciones dada la Pendiente y un Punto}

\textbf{Problema 1:} Pendiente $m = 2$, pasa por $(1, 3)$

Usamos la forma punto-pendiente:
\begin{align*}
y - y_1 &= m(x - x_1) \\
y - 3 &= 2(x - 1) \\
y - 3 &= 2x - 2 \\
y &= 2x + 1
\end{align*}

\textbf{Respuesta:} $y = 2x + 1$

\medskip

\textbf{Problema 2:} Pendiente $m = -3$, pasa por $(2, 5)$

\begin{align*}
y - 5 &= -3(x - 2) \\
y - 5 &= -3x + 6 \\
y &= -3x + 11
\end{align*}

\textbf{Respuesta:} $y = -3x + 11$

\medskip

\textbf{Problema 3:} Pendiente $m = \frac{1}{2}$, pasa por $(4, 1)$

\begin{align*}
y - 1 &= \frac{1}{2}(x - 4) \\
y - 1 &= \frac{1}{2}x - 2 \\
y &= \frac{1}{2}x - 1
\end{align*}

\textbf{Respuesta:} $y = \frac{1}{2}x - 1$

\medskip

\textbf{Problema 4:} Pendiente $m = -\frac{2}{3}$, pasa por $(-3, 4)$

\begin{align*}
y - 4 &= -\frac{2}{3}(x - (-3)) \\
y - 4 &= -\frac{2}{3}(x + 3) \\
y - 4 &= -\frac{2}{3}x - 2 \\
y &= -\frac{2}{3}x + 2
\end{align*}

\textbf{Respuesta:} $y = -\frac{2}{3}x + 2$

\medskip

\textbf{Problema 5:} Pendiente $m = 0$, pasa por $(5, -2)$

Una recta con pendiente 0 es horizontal. Todos los puntos tienen la misma coordenada $y$.

\textbf{Respuesta:} $y = -2$

\medskip

\textbf{Problema 6:} Pendiente indefinida, pasa por $(3, 7)$

Una recta con pendiente indefinida es vertical. Todos los puntos tienen la misma coordenada $x$.

\textbf{Respuesta:} $x = 3$

\newpage

\subsection*{Ejercicio 8: Escribir Ecuaciones dados Dos Puntos}

\textbf{Problema 1:} $(1, 2)$ y $(3, 8)$

\textbf{Paso 1:} Calcular la pendiente
$$m = \frac{8 - 2}{3 - 1} = \frac{6}{2} = 3$$

\textbf{Paso 2:} Usar forma punto-pendiente con $(1, 2)$
\begin{align*}
y - 2 &= 3(x - 1) \\
y - 2 &= 3x - 3 \\
y &= 3x - 1
\end{align*}

\textbf{Respuesta:} $y = 3x - 1$

\medskip

\textbf{Problema 2:} $(0, 4)$ y $(2, 0)$

\textbf{Paso 1:} Calcular la pendiente
$$m = \frac{0 - 4}{2 - 0} = \frac{-4}{2} = -2$$

\textbf{Paso 2:} Usar forma punto-pendiente con $(0, 4)$
\begin{align*}
y - 4 &= -2(x - 0) \\
y - 4 &= -2x \\
y &= -2x + 4
\end{align*}

\textbf{Nota:} Como uno de los puntos es $(0, 4)$, podemos identificar directamente que $b = 4$.

\textbf{Respuesta:} $y = -2x + 4$

\medskip

\textbf{Problema 3:} $(-1, 3)$ y $(2, -6)$

\textbf{Paso 1:} Calcular la pendiente
$$m = \frac{-6 - 3}{2 - (-1)} = \frac{-9}{3} = -3$$

\textbf{Paso 2:} Usar forma punto-pendiente con $(-1, 3)$
\begin{align*}
y - 3 &= -3(x - (-1)) \\
y - 3 &= -3(x + 1) \\
y - 3 &= -3x - 3 \\
y &= -3x
\end{align*}

\textbf{Respuesta:} $y = -3x$

\medskip

\textbf{Problema 4:} $(4, 1)$ y $(6, 5)$

\textbf{Paso 1:} Calcular la pendiente
$$m = \frac{5 - 1}{6 - 4} = \frac{4}{2} = 2$$

\textbf{Paso 2:} Usar forma punto-pendiente con $(4, 1)$
\begin{align*}
y - 1 &= 2(x - 4) \\
y - 1 &= 2x - 8 \\
y &= 2x - 7
\end{align*}

\textbf{Respuesta:} $y = 2x - 7$

\medskip

\textbf{Problema 5:} $(-2, -1)$ y $(3, -1)$

\textbf{Paso 1:} Calcular la pendiente
$$m = \frac{-1 - (-1)}{3 - (-2)} = \frac{0}{5} = 0$$

Como la pendiente es 0, la recta es horizontal. Ambos puntos tienen $y = -1$.

\textbf{Respuesta:} $y = -1$

\medskip

\textbf{Problema 6:} $(5, 2)$ y $(5, 8)$

\textbf{Paso 1:} Calcular la pendiente
$$m = \frac{8 - 2}{5 - 5} = \frac{6}{0} = \text{indefinida}$$

Como la pendiente es indefinida, la recta es vertical. Ambos puntos tienen $x = 5$.

\textbf{Respuesta:} $x = 5$

\newpage

\subsection*{Ejercicio 9: Rectas Paralelas y Perpendiculares}

\textbf{Problema 1:} Ecuación de la recta que pasa por $(2, 3)$ y es paralela a $y = 4x - 1$

Las rectas paralelas tienen la misma pendiente. La pendiente de $y = 4x - 1$ es $m = 4$.

Usando forma punto-pendiente con $(2, 3)$ y $m = 4$:
\begin{align*}
y - 3 &= 4(x - 2) \\
y - 3 &= 4x - 8 \\
y &= 4x - 5
\end{align*}

\textbf{Respuesta:} $y = 4x - 5$

\medskip

\textbf{Problema 2:} Ecuación de la recta que pasa por $(1, 5)$ y es perpendicular a $y = 2x + 3$

Las rectas perpendiculares tienen pendientes que son recíprocas negativas.

Pendiente de la recta dada: $m_1 = 2$

Pendiente de la recta perpendicular: $m_2 = -\frac{1}{2}$

Usando forma punto-pendiente con $(1, 5)$ y $m = -\frac{1}{2}$:
\begin{align*}
y - 5 &= -\frac{1}{2}(x - 1) \\
y - 5 &= -\frac{1}{2}x + \frac{1}{2} \\
y &= -\frac{1}{2}x + \frac{1}{2} + 5 \\
y &= -\frac{1}{2}x + \frac{11}{2}
\end{align*}

\textbf{Respuesta:} $y = -\frac{1}{2}x + \frac{11}{2}$

\medskip

\textbf{Problema 3:} ¿Son $y = 3x + 2$ y $y = 3x - 7$ paralelas, perpendiculares o ninguna?

Pendientes:
\begin{itemize}
    \item Primera recta: $m_1 = 3$
    \item Segunda recta: $m_2 = 3$
\end{itemize}

Como $m_1 = m_2$, las rectas son \textbf{paralelas}.

\medskip

\textbf{Problema 4:} ¿Son $y = \frac{2}{3}x + 1$ y $y = -\frac{3}{2}x + 4$ paralelas, perpendiculares o ninguna?

Pendientes:
\begin{itemize}
    \item Primera recta: $m_1 = \frac{2}{3}$
    \item Segunda recta: $m_2 = -\frac{3}{2}$
\end{itemize}

Verificamos el producto:
$$m_1 \cdot m_2 = \frac{2}{3} \cdot \left(-\frac{3}{2}\right) = -\frac{6}{6} = -1$$

Como el producto de las pendientes es $-1$, las rectas son \textbf{perpendiculares}.

\medskip

\textbf{Problema 5:} Ecuación de la recta que pasa por $(-1, 2)$ y es perpendicular a $3x + 2y = 6$

\textbf{Paso 1:} Convertir la ecuación dada a forma pendiente-intersecto para encontrar su pendiente
\begin{align*}
3x + 2y &= 6 \\
2y &= -3x + 6 \\
y &= -\frac{3}{2}x + 3
\end{align*}

Pendiente de la recta dada: $m_1 = -\frac{3}{2}$

\textbf{Paso 2:} Calcular la pendiente perpendicular
$$m_2 = -\frac{1}{m_1} = -\frac{1}{-\frac{3}{2}} = \frac{2}{3}$$

\textbf{Paso 3:} Usar forma punto-pendiente con $(-1, 2)$ y $m = \frac{2}{3}$
\begin{align*}
y - 2 &= \frac{2}{3}(x - (-1)) \\
y - 2 &= \frac{2}{3}(x + 1) \\
y - 2 &= \frac{2}{3}x + \frac{2}{3} \\
y &= \frac{2}{3}x + \frac{2}{3} + 2 \\
y &= \frac{2}{3}x + \frac{8}{3}
\end{align*}

\textbf{Respuesta:} $y = \frac{2}{3}x + \frac{8}{3}$

\newpage

\subsection*{Ejercicio 10: Problemas de Aplicación}

\textbf{Problema 1:} Tarifa de taxi

El costo total $C$ incluye una tarifa base de \$3 más \$0.50 por cada milla.

Identificamos:
\begin{itemize}
    \item Pendiente (costo por milla): $m = 0.50$
    \item Intersección $y$ (tarifa base): $b = 3$
\end{itemize}

\textbf{Ecuación:} $C = 0.50m + 3$ o $C = 0.5m + 3$

donde $C$ = costo total en dólares y $m$ = número de millas.

\medskip

\textbf{Problema 2:} Conversión de temperatura

Dado $F = \frac{9}{5}C + 32$ y $C = 20$, encontramos $F$:

\begin{align*}
F &= \frac{9}{5}(20) + 32 \\
F &= \frac{180}{5} + 32 \\
F &= 36 + 32 \\
F &= 68
\end{align*}

\textbf{Respuesta:} La temperatura es 68°F.

\medskip

\textbf{Problema 3:} Drenaje del depósito de agua

\begin{itemize}
    \item Cantidad inicial: 500 galones
    \item Tasa de drenaje: 25 galones por hora (negativa porque disminuye)
    \item Pendiente: $m = -25$
    \item Intersección $y$: $b = 500$
\end{itemize}

\textbf{Ecuación:} $A = -25t + 500$

donde $A$ = cantidad de agua (galones) y $t$ = tiempo (horas).

\textbf{Verificación:}
\begin{itemize}
    \item Cuando $t = 0$: $A = -25(0) + 500 = 500$ galones \checkmark
    \item Cuando $t = 10$: $A = -25(10) + 500 = 250$ galones
    \item Cuando $t = 20$: $A = -25(20) + 500 = 0$ galones (depósito vacío)
\end{itemize}

\medskip

\textbf{Problema 4:} Alquiler de autos

\textbf{Ecuación del costo:} $C = 0.25k + 40$

donde $C$ = costo total, $k$ = kilómetros recorridos.

\textbf{Dado:} $C = 65$. Encontrar $k$.

\begin{align*}
65 &= 0.25k + 40 \\
65 - 40 &= 0.25k \\
25 &= 0.25k \\
k &= \frac{25}{0.25} \\
k &= 100
\end{align*}

\textbf{Respuesta:} Se recorrieron 100 kilómetros.

\textbf{Verificación:} $C = 0.25(100) + 40 = 25 + 40 = 65$ \checkmark
