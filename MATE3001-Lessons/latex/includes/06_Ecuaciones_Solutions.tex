%========================================
% DETAILED SOLUTIONS: Ecuaciones
%========================================

\subsection*{Ejercicio 1}

\textbf{Problema 1:} Identificación de ecuaciones e identidades

\textbf{a)} $5x + 3 = 18$ es una \textbf{ecuación} porque solo es cierta para valores específicos de $x$ (en este caso, $x = 3$).

\textbf{b)} $x^2 + 2x = x^2 + 2x$ es una \textbf{identidad} porque es cierta para todos los valores reales de $x$.

\textbf{c)} $3(x + 2) = 3x + 6$ es una \textbf{identidad} porque por la propiedad distributiva, $3(x + 2) = 3x + 3 \cdot 2 = 3x + 6$ para todo valor real de $x$.

\textbf{d)} $\frac{x+1}{2} = 5$ es una \textbf{ecuación} porque solo es cierta para $x = 9$.

\textbf{Problema 2:} Verificación de soluciones

\textbf{a)} Para $2x - 5 = 3$ con $x = 4$:
\begin{align}
2(4) - 5 &= 3\\
8 - 5 &= 3\\
3 &= 3 \quad \checkmark
\end{align}
Por lo tanto, $x = 4$ \textbf{sí es solución}.

\textbf{b)} Para $x^2 - 3x - 4 = 0$ con $x = 4$:
\begin{align}
(4)^2 - 3(4) - 4 &= 0\\
16 - 12 - 4 &= 0\\
0 &= 0 \quad \checkmark
\end{align}
Por lo tanto, $x = 4$ \textbf{sí es solución}.

\textbf{c)} Para $\frac{x}{2} + 1 = 3$ con $x = 4$:
\begin{align}
\frac{4}{2} + 1 &= 3\\
2 + 1 &= 3\\
3 &= 3 \quad \checkmark
\end{align}
Por lo tanto, $x = 4$ \textbf{sí es solución}.

\subsection*{Ejercicio 2}

\textbf{Problema 1:} Ecuaciones lineales básicas

\textbf{a)} $5x - 12 = 0$
\begin{align}
5x - 12 &= 0\\
5x &= 12\\
x &= \frac{12}{5}
\end{align}
\textbf{Verificación:} $5 \cdot \frac{12}{5} - 12 = 12 - 12 = 0$ $\checkmark$

\textbf{b)} $3x + 7 = 22$
\begin{align}
3x + 7 &= 22\\
3x &= 22 - 7\\
3x &= 15\\
x &= 5
\end{align}
\textbf{Verificación:} $3(5) + 7 = 15 + 7 = 22$ $\checkmark$

\textbf{c)} $-2x + 9 = 1$
\begin{align}
-2x + 9 &= 1\\
-2x &= 1 - 9\\
-2x &= -8\\
x &= 4
\end{align}
\textbf{Verificación:} $-2(4) + 9 = -8 + 9 = 1$ $\checkmark$

\textbf{d)} $4x - 6 = 2x + 8$
\begin{align}
4x - 6 &= 2x + 8\\
4x - 2x &= 8 + 6\\
2x &= 14\\
x &= 7
\end{align}
\textbf{Verificación:} $4(7) - 6 = 28 - 6 = 22$ y $2(7) + 8 = 14 + 8 = 22$ $\checkmark$

\textbf{Problema 2:} Ecuaciones con paréntesis

\textbf{a)} $2(x + 3) = 4x - 2$
\begin{align}
2(x + 3) &= 4x - 2\\
2x + 6 &= 4x - 2\\
6 + 2 &= 4x - 2x\\
8 &= 2x\\
x &= 4
\end{align}
\textbf{Verificación:} $2(4 + 3) = 2(7) = 14$ y $4(4) - 2 = 16 - 2 = 14$ $\checkmark$

\textbf{b)} $5x - (3x + 1) = 7$
\begin{align}
5x - (3x + 1) &= 7\\
5x - 3x - 1 &= 7\\
2x - 1 &= 7\\
2x &= 8\\
x &= 4
\end{align}
\textbf{Verificación:} $5(4) - (3(4) + 1) = 20 - 13 = 7$ $\checkmark$

\textbf{c)} $3(2x - 1) - 2(x + 4) = 1$
\begin{align}
3(2x - 1) - 2(x + 4) &= 1\\
6x - 3 - 2x - 8 &= 1\\
4x - 11 &= 1\\
4x &= 12\\
x &= 3
\end{align}
\textbf{Verificación:} $3(2(3) - 1) - 2(3 + 4) = 3(5) - 2(7) = 15 - 14 = 1$ $\checkmark$

\subsection*{Ejercicio 3}

\textbf{Problema 1:} Fracciones simples

\textbf{a)} $\frac{x}{3} = 5$
\begin{align}
\frac{x}{3} &= 5\\
x &= 5 \cdot 3\\
x &= 15
\end{align}

\textbf{b)} $\frac{2x}{5} = 8$
\begin{align}
\frac{2x}{5} &= 8\\
2x &= 8 \cdot 5\\
2x &= 40\\
x &= 20
\end{align}

\textbf{c)} $\frac{x-1}{4} = 3$
\begin{align}
\frac{x-1}{4} &= 3\\
x - 1 &= 3 \cdot 4\\
x - 1 &= 12\\
x &= 13
\end{align}

\textbf{d)} $\frac{3x+2}{7} = 2$
\begin{align}
\frac{3x+2}{7} &= 2\\
3x + 2 &= 2 \cdot 7\\
3x + 2 &= 14\\
3x &= 12\\
x &= 4
\end{align}

\textbf{Problema 2:} Fracciones en ambos lados

\textbf{a)} $\frac{x}{2} = \frac{x+1}{3}$

MCD$(2,3) = 6$. Multiplicamos ambos lados por 6:
\begin{align}
6 \cdot \frac{x}{2} &= 6 \cdot \frac{x+1}{3}\\
3x &= 2(x+1)\\
3x &= 2x + 2\\
x &= 2
\end{align}

\textbf{b)} $\frac{2x-1}{5} = \frac{x+3}{4}$

MCD$(5,4) = 20$. Multiplicamos ambos lados por 20:
\begin{align}
20 \cdot \frac{2x-1}{5} &= 20 \cdot \frac{x+3}{4}\\
4(2x-1) &= 5(x+3)\\
8x - 4 &= 5x + 15\\
3x &= 19\\
x &= \frac{19}{3}
\end{align}

\textbf{c)} $\frac{x+2}{6} = \frac{2x-3}{9}$

MCD$(6,9) = 18$. Multiplicamos ambos lados por 18:
\begin{align}
18 \cdot \frac{x+2}{6} &= 18 \cdot \frac{2x-3}{9}\\
3(x+2) &= 2(2x-3)\\
3x + 6 &= 4x - 6\\
12 &= x\\
x &= 12
\end{align}

\subsection*{Ejercicio 4}

\textbf{Problema 1:} Ecuaciones con denominadores numéricos

\textbf{a)} $\frac{2x+1}{3} = \frac{x-2}{5}$

Multiplicamos por el MCD$(3,5) = 15$:
\begin{align}
15 \cdot \frac{2x+1}{3} &= 15 \cdot \frac{x-2}{5}\\
5(2x+1) &= 3(x-2)\\
10x + 5 &= 3x - 6\\
7x &= -11\\
x &= -\frac{11}{7}
\end{align}

\textbf{b)} $\frac{3x-4}{8} = \frac{x+1}{6}$

Multiplicamos por el MCD$(8,6) = 24$:
\begin{align}
24 \cdot \frac{3x-4}{8} &= 24 \cdot \frac{x+1}{6}\\
3(3x-4) &= 4(x+1)\\
9x - 12 &= 4x + 4\\
5x &= 16\\
x &= \frac{16}{5}
\end{align}

\textbf{c)} $\frac{x+3}{4} - \frac{x-1}{6} = 2$

Multiplicamos por el MCD$(4,6) = 12$:
\begin{align}
12 \cdot \frac{x+3}{4} - 12 \cdot \frac{x-1}{6} &= 12 \cdot 2\\
3(x+3) - 2(x-1) &= 24\\
3x + 9 - 2x + 2 &= 24\\
x + 11 &= 24\\
x &= 13
\end{align}

\textbf{Problema 2:} Casos especiales

\textbf{a)} $\frac{2x+1}{3} = \frac{2x+1}{3}$ es una identidad válida para todo $x$ real (infinitas soluciones).

\textbf{b)} $\frac{x+2}{4} = \frac{x+5}{4}$
\begin{align}
x + 2 &= x + 5\\
2 &= 5
\end{align}
Esta ecuación es imposible, por lo tanto no tiene solución.

\textbf{c)} $\frac{3x-6}{9} = \frac{x-2}{3}$
\begin{align}
\frac{3(x-2)}{9} &= \frac{x-2}{3}\\
\frac{x-2}{3} &= \frac{x-2}{3}
\end{align}
Esta es una identidad válida para todo $x$ real (infinitas soluciones).

\subsection*{Ejercicio 5}

\textbf{Problema 1:} Aplicaciones

\textbf{a)} "El doble de un número más 5 es igual a 17"
\begin{align}
2x + 5 &= 17\\
2x &= 12\\
x &= 6
\end{align}
El número es 6.

\textbf{b)} "La tercera parte de un número menos 4 es igual a 2"
\begin{align}
\frac{x}{3} - 4 &= 2\\
\frac{x}{3} &= 6\\
x &= 18
\end{align}
El número es 18.

\textbf{c)} "La mitad de un número más la cuarta parte del mismo número es igual a 15"
\begin{align}
\frac{x}{2} + \frac{x}{4} &= 15\\
\frac{2x + x}{4} &= 15\\
\frac{3x}{4} &= 15\\
3x &= 60\\
x &= 20
\end{align}
El número es 20.

\textbf{Problema 2:} Problemas de aplicación

\textbf{a)} "Un tercio de la edad de Ana más un cuarto de su edad es igual a 14 años"
\begin{align}
\frac{x}{3} + \frac{x}{4} &= 14\\
\frac{4x + 3x}{12} &= 14\\
\frac{7x}{12} &= 14\\
7x &= 168\\
x &= 24
\end{align}
Ana tiene 24 años.

\textbf{b)} "Pedro gastó la mitad de su dinero en almuerzo y un tercio en transporte. Si le quedan \$10, ¿cuánto tenía inicialmente?"

Sea $x$ el dinero inicial:
\begin{align}
x - \frac{x}{2} - \frac{x}{3} &= 10\\
\frac{6x - 3x - 2x}{6} &= 10\\
\frac{x}{6} &= 10\\
x &= 60
\end{align}
Pedro tenía inicialmente \$60.

\subsection*{Ejercicio 6}

\textbf{Problema 1:} Análisis de errores

\textbf{Problema A:} Resolver $\frac{x+3}{4} = \frac{x-1}{2}$

\textbf{Error identificado:} No se multiplicó por el MCD correctamente. El error fue igualar directamente numeradores sin eliminar denominadores.

\textbf{Correcto:}
Multiplicamos por MCD$(4,2) = 4$:
\begin{align}
4 \cdot \frac{x+3}{4} &= 4 \cdot \frac{x-1}{2}\\
x + 3 &= 2(x - 1)\\
x + 3 &= 2x - 2\\
5 &= x\\
x &= 5
\end{align}

\textbf{Problema B:} Resolver $\frac{2x-1}{3} = \frac{x+2}{6}$

\textbf{Error identificado:} No se eliminaron los denominadores antes de resolver.

\textbf{Correcto:}
Multiplicamos por MCD$(3,6) = 6$:
\begin{align}
6 \cdot \frac{2x-1}{3} &= 6 \cdot \frac{x+2}{6}\\
2(2x-1) &= x + 2\\
4x - 2 &= x + 2\\
3x &= 4\\
x &= \frac{4}{3}
\end{align}

\textbf{Problema 2:} Sistema de ecuaciones

Resolvemos el sistema:
$$\frac{x+y}{2} = 5 \quad \text{y} \quad \frac{x-y}{3} = 1$$

De la primera ecuación, multiplicando por 2:
$$x + y = 10$$

De la segunda ecuación, multiplicando por 3:
$$x - y = 3$$

Sumando ambas ecuaciones:
\begin{align}
(x + y) + (x - y) &= 10 + 3\\
2x &= 13\\
x &= \frac{13}{2}
\end{align}

Sustituyendo en $x + y = 10$:
\begin{align}
\frac{13}{2} + y &= 10\\
y &= 10 - \frac{13}{2}\\
y &= \frac{20 - 13}{2}\\
y &= \frac{7}{2}
\end{align}

\textbf{Verificación:}
- Primera ecuación: $\frac{\frac{13}{2} + \frac{7}{2}}{2} = \frac{10}{2} = 5$ $\checkmark$
- Segunda ecuación: $\frac{\frac{13}{2} - \frac{7}{2}}{3} = \frac{3}{3} = 1$ $\checkmark$

Por lo tanto, la solución es $x = \frac{13}{2}$ y $y = \frac{7}{2}$.