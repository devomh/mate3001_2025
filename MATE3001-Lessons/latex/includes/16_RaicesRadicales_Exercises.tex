%========================================
% EXERCISES: Raíces y Radicales
%========================================

\section{Ejercicios}

\begin{exercise}
\textbf{Simplificación de radicales (índice par)}

\problem Simplifique completamente: $\sqrt{200x^4y^3}$
\begin{solucion}
\begin{align*}
\sqrt{200x^4y^3} &= \sqrt{100\cdot 2\cdot x^4\cdot y^2\cdot y}\\
&= 10x^2|y|\sqrt{2y}
\end{align*}
Si $y\geq 0$: $10x^2y\sqrt{2y}$
\end{solucion}

\problem Simplifique: $\sqrt{98a^6b^5}$
\begin{solucion}
\begin{align*}
\sqrt{98a^6b^5} &= \sqrt{49\cdot 2\cdot a^6\cdot b^4\cdot b}\\
&= 7a^3b^2\sqrt{2b}
\end{align*}
(Asumiendo $a,b\geq 0$)
\end{solucion}

\problem Simplifique: $\sqrt{50x^2}+\sqrt{32x^2}$
\begin{solucion}
\begin{align*}
\sqrt{50x^2}+\sqrt{32x^2} &= 5|x|\sqrt{2}+4|x|\sqrt{2}\\
&= 9|x|\sqrt{2}
\end{align*}
\end{solucion}
\end{exercise}

\begin{exercise}
\textbf{Simplificación de radicales (índice impar)}

\problem Simplifique: $\sqrt[3]{54a^5b^3}$
\begin{solucion}
\begin{align*}
\sqrt[3]{54a^5b^3} &= \sqrt[3]{27\cdot 2\cdot a^3\cdot a^2\cdot b^3}\\
&= 3ab\sqrt[3]{2a^2}
\end{align*}
\end{solucion}

\problem Simplifique: $\sqrt[3]{-128x^7y^4}$
\begin{solucion}
\begin{align*}
\sqrt[3]{-128x^7y^4} &= \sqrt[3]{-64\cdot 2\cdot x^6\cdot x\cdot y^3\cdot y}\\
&= -4x^2y\sqrt[3]{2xy}
\end{align*}
\end{solucion}

\problem Simplifique: $\sqrt[4]{162a^9b^{12}}$
\begin{solucion}
\begin{align*}
\sqrt[4]{162a^9b^{12}} &= \sqrt[4]{81\cdot 2\cdot a^8\cdot a\cdot b^{12}}\\
&= 3a^2b^3\sqrt[4]{2a}
\end{align*}
(Asumiendo $a\geq 0$)
\end{solucion}
\end{exercise}

\begin{exercise}
\textbf{Multiplicación de radicales}

\problem Multiplique y simplifique: $(\sqrt{7a})(\sqrt{21a^3})$
\begin{solucion}
\begin{align*}
(\sqrt{7a})(\sqrt{21a^3}) &= \sqrt{7a\cdot 21a^3}\\
&= \sqrt{147a^4}\\
&= \sqrt{49\cdot 3\cdot a^4}\\
&= 7a^2\sqrt{3}
\end{align*}
(Asumiendo $a\geq 0$)
\end{solucion}

\problem Multiplique y simplifique: $\sqrt{6x^2}\cdot\sqrt{24x^3}$
\begin{solucion}
\begin{align*}
\sqrt{6x^2}\cdot\sqrt{24x^3} &= \sqrt{144x^5}\\
&= 12x^2\sqrt{x}
\end{align*}
(Asumiendo $x\geq 0$)
\end{solucion}

\problem Multiplique y simplifique: $\sqrt[3]{4x}\cdot\sqrt[3]{16x^2}$
\begin{solucion}
\begin{align*}
\sqrt[3]{4x}\cdot\sqrt[3]{16x^2} &= \sqrt[3]{64x^3}\\
&= 4x
\end{align*}
\end{solucion}
\end{exercise}

\begin{exercise}
\textbf{Racionalización (denominador monomial)}

\problem Racionalice: $\dfrac{3}{\sqrt{3}}$
\begin{solucion}
$$\frac{3}{\sqrt{3}}=\frac{3}{\sqrt{3}}\cdot\frac{\sqrt{3}}{\sqrt{3}}=\frac{3\sqrt{3}}{3}=\sqrt{3}$$
\end{solucion}

\problem Racionalice: $\dfrac{10}{\sqrt{5}}$
\begin{solucion}
$$\frac{10}{\sqrt{5}}=\frac{10}{\sqrt{5}}\cdot\frac{\sqrt{5}}{\sqrt{5}}=\frac{10\sqrt{5}}{5}=2\sqrt{5}$$
\end{solucion}

\problem Racionalice: $\dfrac{6}{\sqrt[3]{9}}$
\begin{solucion}
Para eliminar $\sqrt[3]{9}=\sqrt[3]{3^2}$, necesitamos $\sqrt[3]{3^3}$:
\begin{align*}
\frac{6}{\sqrt[3]{9}} &= \frac{6}{\sqrt[3]{9}}\cdot\frac{\sqrt[3]{3}}{\sqrt[3]{3}}\\
&= \frac{6\sqrt[3]{3}}{\sqrt[3]{27}}\\
&= \frac{6\sqrt[3]{3}}{3}\\
&= 2\sqrt[3]{3}
\end{align*}
\end{solucion}
\end{exercise}

\begin{exercise}
\textbf{Racionalización (denominador binomial)}

\problem Racionalice: $\dfrac{2}{\sqrt{3}-\sqrt{2}}$
\begin{solucion}
\begin{align*}
\frac{2}{\sqrt{3}-\sqrt{2}} &= \frac{2}{\sqrt{3}-\sqrt{2}}\cdot\frac{\sqrt{3}+\sqrt{2}}{\sqrt{3}+\sqrt{2}}\\
&= \frac{2(\sqrt{3}+\sqrt{2})}{(\sqrt{3})^2-(\sqrt{2})^2}\\
&= \frac{2\sqrt{3}+2\sqrt{2}}{3-2}\\
&= 2\sqrt{3}+2\sqrt{2}
\end{align*}
\end{solucion}

\problem Racionalice: $\dfrac{5}{4-\sqrt{2}}$
\begin{solucion}
\begin{align*}
\frac{5}{4-\sqrt{2}} &= \frac{5}{4-\sqrt{2}}\cdot\frac{4+\sqrt{2}}{4+\sqrt{2}}\\
&= \frac{5(4+\sqrt{2})}{16-2}\\
&= \frac{20+5\sqrt{2}}{14}\\
&= \frac{10}{7}+\frac{5\sqrt{2}}{14}
\end{align*}
\end{solucion}

\problem Racionalice: $\dfrac{3}{2+\sqrt{5}}$
\begin{solucion}
\begin{align*}
\frac{3}{2+\sqrt{5}} &= \frac{3}{2+\sqrt{5}}\cdot\frac{2-\sqrt{5}}{2-\sqrt{5}}\\
&= \frac{3(2-\sqrt{5})}{4-5}\\
&= \frac{6-3\sqrt{5}}{-1}\\
&= -6+3\sqrt{5}
\end{align*}
\end{solucion}
\end{exercise}

\begin{exercise}
\textbf{Suma y resta de radicales}

\problem Simplifique: $4\sqrt{20}-3\sqrt{5}+\sqrt{45}$
\begin{solucion}
\begin{align*}
4\sqrt{20}-3\sqrt{5}+\sqrt{45} &= 4(2\sqrt{5})-3\sqrt{5}+3\sqrt{5}\\
&= 8\sqrt{5}-3\sqrt{5}+3\sqrt{5}\\
&= 8\sqrt{5}
\end{align*}
\end{solucion}

\problem Simplifique: $2\sqrt{12}+\sqrt{27}-5\sqrt{3}$
\begin{solucion}
\begin{align*}
2\sqrt{12}+\sqrt{27}-5\sqrt{3} &= 2(2\sqrt{3})+3\sqrt{3}-5\sqrt{3}\\
&= 4\sqrt{3}+3\sqrt{3}-5\sqrt{3}\\
&= 2\sqrt{3}
\end{align*}
\end{solucion}

\problem Simplifique: $\sqrt{50}+\sqrt{18}-\sqrt{8}$
\begin{solucion}
\begin{align*}
\sqrt{50}+\sqrt{18}-\sqrt{8} &= 5\sqrt{2}+3\sqrt{2}-2\sqrt{2}\\
&= 6\sqrt{2}
\end{align*}
\end{solucion}
\end{exercise}

\begin{exercise}
\textbf{Notación exponencial}

\problem Convierta a forma exponencial: $\sqrt[5]{x^2}$
\begin{solucion}
$$\sqrt[5]{x^2}=x^{2/5}$$
\end{solucion}

\problem Convierta a forma radical: $y^{-3/4}$
\begin{solucion}
$$y^{-3/4}=\frac{1}{y^{3/4}}=\frac{1}{\sqrt[4]{y^3}}$$
\end{solucion}

\problem Simplifique: $\dfrac{x^{3/2}\,x^{1/2}}{x^{5/2}}$
\begin{solucion}
\begin{align*}
\frac{x^{3/2}\,x^{1/2}}{x^{5/2}} &= x^{(3/2+1/2-5/2)}\\
&= x^{-1/2}\\
&= \frac{1}{\sqrt{x}}
\end{align*}
\end{solucion}
\end{exercise}

\begin{exercise}
\textbf{Problemas mixtos}

\problem Simplifique: $\sqrt[4]{x^6}$
\begin{solucion}
\begin{align*}
\sqrt[4]{x^6} &= x^{6/4}\\
&= x^{3/2}\\
&= x\sqrt{x}
\end{align*}
(Asumiendo $x\geq 0$)
\end{solucion}

\problem Simplifique: $\sqrt[3]{a^5}$
\begin{solucion}
\begin{align*}
\sqrt[3]{a^5} &= a^{5/3}\\
&= a^{3/3+2/3}\\
&= a\cdot a^{2/3}\\
&= a\sqrt[3]{a^2}
\end{align*}
\end{solucion}

\problem Evalúe sin calculadora: $\sqrt{16}+\sqrt[3]{27}-\sqrt[4]{16}$
\begin{solucion}
$$\sqrt{16}+\sqrt[3]{27}-\sqrt[4]{16}=4+3-2=5$$
\end{solucion}
\end{exercise}

\begin{exercise}
\textbf{Dominio de funciones con radicales}

\problem Determine el dominio de $f(x)=\sqrt{x-3}$
\begin{solucion}
Para que $\sqrt{x-3}$ sea real:
$$x-3\geq 0 \Rightarrow x\geq 3$$
Dominio: $[3,\infty)$
\end{solucion}

\problem Determine el dominio de $g(x)=\sqrt{5-2x}$
\begin{solucion}
Para que $\sqrt{5-2x}$ sea real:
\begin{align*}
5-2x &\geq 0\\
-2x &\geq -5\\
x &\leq \frac{5}{2}
\end{align*}
Dominio: $(-\infty,\frac{5}{2}]$
\end{solucion}

\problem Determine el dominio de $h(x)=\sqrt[3]{x+1}$
\begin{solucion}
Como el índice es impar (3), el radicando puede ser cualquier número real.
Dominio: $(-\infty,\infty)$ o $\mathbb{R}$
\end{solucion}
\end{exercise}

\begin{exercise}
\textbf{Práctica adicional}

\begin{exerciselist}
    \item Simplifique: $\sqrt{180}$
    \item Simplifique: $\sqrt[3]{-64}$
    \item Multiplique: $\sqrt{5}\cdot\sqrt{20}$
    \item Racionalice: $\dfrac{1}{\sqrt{7}}$
    \item Simplifique: $3\sqrt{2}+5\sqrt{2}$
    \item Convierta a exponente: $\sqrt[3]{x^4}$
    \item Simplifique: $(\sqrt{3}+\sqrt{2})(\sqrt{3}-\sqrt{2})$
    \item Determine el dominio de $f(x)=\sqrt{x^2-9}$
\end{exerciselist}
\end{exercise}
