%========================================
% EXERCISES: Sistema de Dos Ecuaciones Lineales
%========================================

\section{Ejercicios}

%========================================
% Exercise 1: Solving Systems by All Three Methods
%========================================
\begin{exercise}
\textbf{Resolver Sistemas por los Tres Métodos}

Resuelva cada sistema de ecuaciones usando:
\begin{itemize}
    \item Método Gráfico
    \item Método de Sustitución
    \item Método de Eliminación
\end{itemize}

\problem $$\begin{cases}2x - y = 4 \\ 3x + y = 6\end{cases}$$

\begin{solucion}
\textbf{Método Gráfico:}

Para $2x - y = 4$: Intersecciones $(2, 0)$ y $(0, -4)$

Para $3x + y = 6$: Intersecciones $(2, 0)$ y $(0, 6)$

(Se grafican ambas rectas y se identifica su intersección)

Solución gráfica: $(2, 0)$

\vspace{0.3cm}

\textbf{Método de Sustitución:}

Despejando $y$ de la primera ecuación: $y = 2x - 4$

Reemplazando en la segunda ecuación:
$$3x + (2x - 4) = 6$$
$$5x = 10$$
$$x = 2$$

Sustituir en el despeje: $y = 2(2) - 4 = 0$

Solución: $(x,y) = (2,0)$

\vspace{0.3cm}

\textbf{Método por Eliminación:}

Los coeficientes de $y$ en las ecuaciones ya son opuestos.

Sumar ambas ecuaciones:
$$\begin{aligned}
(2x - y) + (3x + y) &= 4 + 6 \\
5x + 0y &= 10 \\
x &= 2
\end{aligned}$$

Sustituir en la primera ecuación: $2(2) - y = 4 \implies y = 0$

Solución: $(x,y) = (2,0)$
\end{solucion}

\problem $$\begin{cases}x + y = 7 \\ 2x - 3y = -1\end{cases}$$

\begin{solucion}
\textbf{Método Gráfico:}

Para $x + y = 7$: Intersecciones $(7, 0)$ y $(0, 7)$

Para $2x - 3y = -1$: Intersecciones $(-\frac{1}{2}, 0)$ y $(0, \frac{1}{3})$

Solución gráfica: $(4, 3)$

\vspace{0.3cm}

\textbf{Método de Sustitución:}

Despejando $x$ de la primera ecuación: $x = 7 - y$

Reemplazando en la segunda ecuación:
$$2(7 - y) - 3y = -1$$
$$14 - 2y - 3y = -1$$
$$-5y = -15$$
$$y = 3$$

Sustituir: $x = 7 - 3 = 4$

Solución: $(x,y) = (4,3)$

\vspace{0.3cm}

\textbf{Método por Eliminación:}

Multiplicar la primera ecuación por $-2$:
$$-2(x + y) = -2(7) \implies -2x - 2y = -14$$

Sumar con la segunda ecuación:
$$\begin{aligned}
(-2x - 2y) + (2x - 3y) &= -14 + (-1) \\
-5y &= -15 \\
y &= 3
\end{aligned}$$

Sustituir en la primera ecuación: $x + 3 = 7 \implies x = 4$

Solución: $(x,y) = (4,3)$
\end{solucion}

\problem $$\begin{cases}2x + 5y = 15 \\ 4x + y = 21\end{cases}$$

\begin{solucion}
\textbf{Método Gráfico:}

Para $2x + 5y = 15$: Intersecciones $(\frac{15}{2}, 0)$ y $(0, 3)$

Para $4x + y = 21$: Intersecciones $(\frac{21}{4}, 0)$ y $(0, 21)$

Solución gráfica: $(5, 1)$

\vspace{0.3cm}

\textbf{Método de Sustitución:}

Despejando $y$ de la segunda ecuación: $y = 21 - 4x$

Reemplazando en la primera ecuación:
$$2x + 5(21 - 4x) = 15$$
$$2x + 105 - 20x = 15$$
$$-18x = -90$$
$$x = 5$$

Sustituir: $y = 21 - 4(5) = 1$

Solución: $(x,y) = (5,1)$

\vspace{0.3cm}

\textbf{Método por Eliminación:}

Multiplicar la segunda ecuación por $-5$:
$$-5(4x + y) = -5(21) \implies -20x - 5y = -105$$

Sumar con la primera ecuación:
$$\begin{aligned}
(2x + 5y) + (-20x - 5y) &= 15 + (-105) \\
-18x &= -90 \\
x &= 5
\end{aligned}$$

Sustituir en la segunda ecuación: $4(5) + y = 21 \implies y = 1$

Solución: $(x,y) = (5,1)$
\end{solucion}
\end{exercise}

%========================================
% Exercise 2: Identifying Number of Solutions
%========================================
\begin{exercise}
\textbf{Identificar el Número de Soluciones}

Para cada sistema, determine (sin resolverlo) si tiene una solución, ninguna solución o infinitas soluciones. Explique su razonamiento.

\problem $$\begin{cases}x + 2y = 5 \\ 2x + 4y = 10\end{cases}$$

\begin{solucion}
La segunda ecuación es $2$ veces la primera ecuación: $2(x + 2y) = 2(5)$.

Las rectas \textbf{coinciden}.

\textbf{Infinitas soluciones}
\end{solucion}

\problem $$\begin{cases}3x - y = 4 \\ 3x - y = 7\end{cases}$$

\begin{solucion}
Ambas ecuaciones tienen la misma forma ($3x - y =$) pero diferentes constantes (4 y 7).

Las rectas son \textbf{paralelas}.

\textbf{Sin solución}
\end{solucion}

\problem $$\begin{cases}x + y = 8 \\ x - y = 2\end{cases}$$

\begin{solucion}
Las rectas tienen diferentes pendientes (pendientes: $-1$ y $1$).

Las rectas se \textbf{intersectan} en un punto.

\textbf{Una solución}
\end{solucion}

\problem $$\begin{cases}2x + 3y = 6 \\ 4x + 6y = 12\end{cases}$$

\begin{solucion}
La segunda ecuación es $2$ veces la primera ecuación.

Las rectas \textbf{coinciden}.

\textbf{Infinitas soluciones}
\end{solucion}

\problem $$\begin{cases}y = 2x + 1 \\ y = 2x - 3\end{cases}$$

\begin{solucion}
Ambas ecuaciones tienen la misma pendiente ($m = 2$) pero diferentes intersecciones con el eje $y$ (1 y $-3$).

Las rectas son \textbf{paralelas}.

\textbf{Sin solución}
\end{solucion}

\problem $$\begin{cases}5x - 2y = 10 \\ x + 3y = 6\end{cases}$$

\begin{solucion}
Las rectas tienen diferentes pendientes (pendientes: $\frac{5}{2}$ y $-\frac{1}{3}$).

Las rectas se \textbf{intersectan} en un punto.

\textbf{Una solución}
\end{solucion}
\end{exercise}

%========================================
% Exercise 3: Word Problems
%========================================
\begin{exercise}
\textbf{Problemas de Aplicación}

Resuelva los siguientes problemas usando sistemas de ecuaciones lineales. Siga el proceso de 4 pasos:
\begin{enumerate}
    \item Identificar las variables
    \item Expresar las cantidades en términos de las variables
    \item Establecer el sistema de ecuaciones
    \item Resolver e interpretar los resultados
\end{enumerate}

\problem Encuentre dos números cuya suma es 34 y cuya diferencia es 10.

\begin{solucion}
\textbf{Paso 1:} Sea $x$ el número mayor y $y$ el número menor.

\textbf{Paso 2:}
\begin{itemize}
    \item Suma: $x + y = 34$
    \item Diferencia: $x - y = 10$
\end{itemize}

\textbf{Paso 3:} Sistema:
$$\begin{cases}
x + y = 34 \\
x - y = 10
\end{cases}$$

\textbf{Paso 4:} Sumando ambas ecuaciones:
$$2x = 44 \implies x = 22$$

Sustituyendo: $22 + y = 34 \implies y = 12$

\textbf{Respuesta:} Los dos números son 22 y 12.
\end{solucion}

\problem Un hombre tiene 14 monedas en su bolsillo, todas las cuales son de 10 o de 25 centavos. Si el valor total de su cambio es \$2.75, ¿cuántas monedas de 10 centavos y cuántas de 25 centavos tiene?

\begin{solucion}
\textbf{Paso 1:} Sea $x$ = número de monedas de 10 centavos, $y$ = número de monedas de 25 centavos.

\textbf{Paso 2:}
\begin{itemize}
    \item Total de monedas: $x + y = 14$
    \item Valor total (en centavos): $10x + 25y = 275$
\end{itemize}

\textbf{Paso 3:} Sistema:
$$\begin{cases}
x + y = 14 \\
10x + 25y = 275
\end{cases}$$

\textbf{Paso 4:} De la primera ecuación: $x = 14 - y$

Sustituyendo en la segunda:
$$10(14 - y) + 25y = 275$$
$$140 - 10y + 25y = 275$$
$$15y = 135$$
$$y = 9$$

Entonces: $x = 14 - 9 = 5$

\textbf{Respuesta:} Tiene 5 monedas de 10 centavos y 9 monedas de 25 centavos.
\end{solucion}

\problem Una tienda vende camisetas a \$15 cada una y pantalones a \$25 cada uno. Si en un día se vendieron 45 artículos por un total de \$825, ¿cuántas camisetas y cuántos pantalones se vendieron?

\begin{solucion}
\textbf{Paso 1:} Sea $x$ = número de camisetas, $y$ = número de pantalones.

\textbf{Paso 2:}
\begin{itemize}
    \item Total de artículos: $x + y = 45$
    \item Valor total: $15x + 25y = 825$
\end{itemize}

\textbf{Paso 3:} Sistema:
$$\begin{cases}
x + y = 45 \\
15x + 25y = 825
\end{cases}$$

\textbf{Paso 4:} De la primera ecuación: $x = 45 - y$

Sustituyendo:
$$15(45 - y) + 25y = 825$$
$$675 - 15y + 25y = 825$$
$$10y = 150$$
$$y = 15$$

Entonces: $x = 45 - 15 = 30$

\textbf{Respuesta:} Se vendieron 30 camisetas y 15 pantalones.
\end{solucion}

\problem La suma de dos números es 50. Tres veces el primer número menos el segundo número es igual a 10. Encuentre los dos números.

\begin{solucion}
\textbf{Paso 1:} Sea $x$ el primer número y $y$ el segundo número.

\textbf{Paso 2:}
\begin{itemize}
    \item Suma: $x + y = 50$
    \item Relación: $3x - y = 10$
\end{itemize}

\textbf{Paso 3:} Sistema:
$$\begin{cases}
x + y = 50 \\
3x - y = 10
\end{cases}$$

\textbf{Paso 4:} Sumando ambas ecuaciones:
$$4x = 60 \implies x = 15$$

Sustituyendo: $15 + y = 50 \implies y = 35$

\textbf{Respuesta:} Los dos números son 15 y 35.
\end{solucion}

\problem Un rectángulo tiene un perímetro de 60 metros. El largo es 6 metros más que el ancho. Encuentre las dimensiones del rectángulo.

\begin{solucion}
\textbf{Paso 1:} Sea $l$ = largo y $a$ = ancho.

\textbf{Paso 2:}
\begin{itemize}
    \item Perímetro: $2l + 2a = 60$
    \item Relación: $l = a + 6$
\end{itemize}

\textbf{Paso 3:} Sistema:
$$\begin{cases}
2l + 2a = 60 \\
l = a + 6
\end{cases}$$

\textbf{Paso 4:} Sustituyendo la segunda ecuación en la primera:
$$2(a + 6) + 2a = 60$$
$$2a + 12 + 2a = 60$$
$$4a = 48$$
$$a = 12$$

Entonces: $l = 12 + 6 = 18$

\textbf{Respuesta:} El rectángulo tiene 18 metros de largo y 12 metros de ancho.
\end{solucion}
\end{exercise}
