%========================================
% INSTRUCTOR SOLUTIONS AND NOTES: Desigualdades Lineales
%========================================

\subsection*{Notas Pedagógicas para el Instructor}

\subsubsection*{Objetivos de Aprendizaje}

Al completar esta lección, los estudiantes deberán ser capaces de:

\begin{enumerate}
    \item Distinguir entre ecuaciones y desigualdades lineales
    \item Aplicar correctamente las reglas para manipular desigualdades
    \item \textbf{Identificar cuándo invertir el sentido de la desigualdad} (al multiplicar/dividir por negativos)
    \item Resolver desigualdades lineales en una variable
    \item Expresar soluciones en notación de intervalos
    \item Representar soluciones gráficamente en la recta real
    \item Resolver desigualdades compuestas (tipo ``y'' y tipo ``o'')
    \item Modelar situaciones del mundo real con desigualdades
\end{enumerate}

\subsubsection*{Errores Comunes de los Estudiantes}

\begin{enumerate}
    \item \textbf{ERROR \#1: No invertir la desigualdad al multiplicar/dividir por negativos}

    \textbf{Ejemplo incorrecto:}
    \begin{align*}
        -2x &> 6 \\
        x &> -3 \quad \textcolor{red}{\text{¡INCORRECTO!}}
    \end{align*}

    \textbf{Corrección:} Al dividir entre $-2 < 0$, se debe invertir:
    \begin{align*}
        -2x &> 6 \\
        x &< -3 \quad \textcolor{green}{\text{¡CORRECTO!}}
    \end{align*}

    \textbf{Estrategia de enseñanza:} Pida a los estudiantes que verifiquen su respuesta sustituyendo un valor del conjunto solución. Por ejemplo, si $x < -3$, pruebe $x = -4$: ¿$-2(-4) > 6$? Sí, $8 > 6$ ✓

    \item \textbf{ERROR \#2: Confundir círculos abiertos y cerrados}

    Los estudiantes frecuentemente dibujan círculos rellenos cuando deberían ser abiertos y viceversa.

    \textbf{Mnemotecnia:}
    \begin{itemize}
        \item \textbf{Círculo relleno (●):} El valor \textbf{está incluido} ($\le$ o $\ge$)
        \item \textbf{Círculo abierto (○):} El valor \textbf{no está incluido} ($<$ o $>$)
    \end{itemize}

    \item \textbf{ERROR \#3: Confundir paréntesis y corchetes en notación de intervalos}

    \textbf{Regla nemotécnica:}
    \begin{itemize}
        \item \textbf{Paréntesis ( ):} ``abierto'' = no incluido
        \item \textbf{Corchete [ ]:} ``cerrado'' = incluido
        \item El infinito ($\infty$) \textbf{siempre} lleva paréntesis
    \end{itemize}

    \item \textbf{ERROR \#4: En desigualdades compuestas tipo ``o'', intentar combinarlas incorrectamente}

    \textbf{Ejemplo incorrecto:}
    \[x < -2 \text{ o } x \ge 3 \quad \Rightarrow \quad -2 < x \ge 3 \quad \textcolor{red}{\text{¡INCORRECTO!}}\]

    \textbf{Corrección:} Las desigualdades tipo ``o'' representan la \textbf{unión} de dos conjuntos disjuntos, no se pueden combinar en una sola expresión. La notación correcta es:
    \[(-\infty, -2) \cup [3, \infty)\]

    \item \textbf{ERROR \#5: Olvidar que algunas desigualdades no tienen solución}

    Como en el Ejercicio 10, donde se obtiene $-8 \ge 1$ (falso), el conjunto solución es $\varnothing$.
\end{enumerate}

\subsubsection*{Sugerencias de Enseñanza}

\begin{enumerate}
    \item \textbf{Enfatizar la verificación de soluciones}

    Anime a los estudiantes a verificar sus respuestas sustituyendo valores del conjunto solución en la desigualdad original. Esto refuerza la comprensión y detecta errores.

    \item \textbf{Uso de la recta numérica física}

    Considere usar una recta numérica grande en el pizarrón o en el suelo donde los estudiantes puedan pararse físicamente en diferentes puntos para visualizar las desigualdades.

    \item \textbf{Conexión con ecuaciones}

    Comience cada tema resolviendo una ecuación similar antes de pasar a la desigualdad. Por ejemplo:
    \begin{itemize}
        \item Ecuación: $-2x = 6 \Rightarrow x = -3$ (un punto)
        \item Desigualdad: $-2x > 6 \Rightarrow x < -3$ (un intervalo)
    \end{itemize}

    \item \textbf{Práctica progresiva}

    Orden sugerido de dificultad:
    \begin{enumerate}
        \item Desigualdades simples con coeficientes positivos
        \item Desigualdades con coeficientes negativos (enfatizar inversión)
        \item Desigualdades con fracciones
        \item Desigualdades compuestas tipo ``y''
        \item Desigualdades compuestas tipo ``o''
        \item Problemas de aplicación
    \end{enumerate}

    \item \textbf{Uso de color en el pizarrón}

    \begin{itemize}
        \item Use color rojo cuando se invierte una desigualdad
        \item Use azul para resaltar las soluciones finales
        \item Use verde para verificaciones correctas
    \end{itemize}
\end{enumerate}

\subsubsection*{Actividades de Clase Sugeridas}

\begin{enumerate}
    \item \textbf{Actividad 1: ¿Verdadero o Falso?}

    Presente una lista de valores y pida a los estudiantes que determinen si satisfacen una desigualdad dada. Ejemplo:
    \begin{itemize}
        \item Desigualdad: $2x - 5 < 7$
        \item ¿$x = 0$ es solución? (Sí, porque $-5 < 7$)
        \item ¿$x = 6$ es solución? (No, porque $7 \not< 7$)
        \item ¿$x = -10$ es solución? (Sí, porque $-25 < 7$)
    \end{itemize}

    \item \textbf{Actividad 2: Matching Game}

    Prepare tarjetas con desigualdades, notaciones de intervalos y gráficas. Los estudiantes deben emparejar las tres representaciones de la misma solución.

    \item \textbf{Actividad 3: Error Analysis}

    Presente soluciones incorrectas y pida a los estudiantes que identifiquen y corrijan los errores. Esto desarrolla pensamiento crítico.

    \item \textbf{Actividad 4: Problemas del Mundo Real}

    Pida a los estudiantes que creen sus propios problemas verbales que requieran desigualdades, luego intercambien con compañeros para resolver.
\end{enumerate}

\subsubsection*{Preguntas de Discusión}

\begin{enumerate}
    \item ¿Por qué multiplicar por un número negativo invierte la desigualdad? (Relacionar con la recta numérica y reflexión)

    \item ¿Cuál es la diferencia entre ``y'' e ``o'' en desigualdades compuestas? ¿Cómo se relaciona con intersección y unión de conjuntos?

    \item ¿Por qué el infinito siempre lleva paréntesis en notación de intervalos? (Porque no es un número alcanzable)

    \item ¿Cómo podemos verificar si nuestra solución es correcta? (Sustitución de valores de prueba)

    \item ¿En qué situaciones del mundo real usamos desigualdades en lugar de ecuaciones? (Presupuestos, límites de velocidad, restricciones de edad, etc.)
\end{enumerate}

\subsubsection*{Extensiones para Estudiantes Avanzados}

\begin{enumerate}
    \item Desigualdades con valor absoluto: $|x - 3| < 5$

    \item Sistemas de desigualdades en dos variables (representación gráfica en el plano)

    \item Desigualdades cuadráticas: $x^2 - 5x + 6 \le 0$

    \item Aplicaciones de optimización con restricciones
\end{enumerate}

\subsubsection*{Recursos Adicionales}

\begin{itemize}
    \item \textbf{Videos:} Khan Academy - Linear Inequalities
    \item \textbf{Software:} GeoGebra para visualizar desigualdades gráficamente
    \item \textbf{Práctica adicional:} IXL Math - Solve linear inequalities
    \item \textbf{Manipulativos:} Rectas numéricas físicas, fichas para marcar intervalos
\end{itemize}

\subsubsection*{Evaluación Formativa}

\textbf{Pregunta rápida de verificación (Exit Ticket):}

Resuelva: $-3x + 7 \le 1$

\textbf{Respuesta correcta:} $x \ge 2$ o $[2, \infty)$

Si los estudiantes responden $x \le 2$, necesitan refuerzo sobre la inversión de desigualdades.

\subsubsection*{Conexiones con Otros Temas}

\begin{itemize}
    \item \textbf{Lección anterior (Sistema de ecuaciones):} Las desigualdades lineales son la base para sistemas de desigualdades
    \item \textbf{Próxima lección:} Desigualdades con valor absoluto requieren comprender desigualdades compuestas
    \item \textbf{Aplicaciones futuras:} Programación lineal, optimización, cálculo (derivadas para encontrar máximos/mínimos)
\end{itemize}

\subsubsection*{Notas Específicas por Ejercicio}

\textbf{Ejercicios 1-3:} Buenos problemas para comenzar. Coeficientes positivos, sin complicaciones.

\textbf{Ejercicios 4, 6-8:} Requieren inversión de desigualdad. Monitoree cuidadosamente que los estudiantes inviertan correctamente.

\textbf{Ejercicio 5:} Práctica con fracciones. Algunos estudiantes pueden necesitar revisar operaciones con fracciones.

\textbf{Ejercicio 9:} Requiere reorganizar términos. Buena práctica de álgebra.

\textbf{Ejercicio 10:} No tiene solución ($\varnothing$). Asegúrese de que los estudiantes reconozcan declaraciones falsas.

\textbf{Ejercicios 11-13:} Desigualdades compuestas tipo ``y''. Enfatice que se trabaja con las tres partes simultáneamente.

\textbf{Ejercicios 14-15:} Desigualdades tipo ``o''. Enfatice la unión ($\cup$) y que las regiones están separadas.

\textbf{Ejercicio 16:} Problema verbal directo. Buena introducción al modelado.

\textbf{Ejercicio 17:} Problema más complejo que requiere plantear la expresión del puntaje. Excelente para pensamiento crítico. Note que la solución debe ser un entero entre 0 y 30.

\subsubsection*{Tiempo Estimado}

\begin{itemize}
    \item \textbf{Teoría y ejemplos:} 45-60 minutos
    \item \textbf{Práctica guiada:} 30-45 minutos
    \item \textbf{Práctica independiente:} 45-60 minutos
    \item \textbf{Total:} 2-3 sesiones de clase (50 minutos cada una)
\end{itemize}

\subsubsection*{Diferenciación}

\textbf{Para estudiantes que necesitan apoyo adicional:}
\begin{itemize}
    \item Proporcione rectas numéricas pre-dibujadas
    \item Use manipulativos físicos
    \item Comience con más ejemplos de coeficientes positivos
    \item Proporcione una tarjeta de referencia con las reglas
\end{itemize}

\textbf{Para estudiantes avanzados:}
\begin{itemize}
    \item Introduzca desigualdades con valor absoluto
    \item Pida que creen sus propios problemas
    \item Explore desigualdades en contextos más complejos
    \item Investigue sistemas de desigualdades
\end{itemize}

\vspace{1cm}

\hrule

\vspace{0.5cm}

\textbf{Nota final:} El concepto de invertir la desigualdad al multiplicar/dividir por negativos es contraintuitivo para muchos estudiantes. Dedique tiempo adicional a este concepto, use múltiples ejemplos, y verifique frecuentemente la comprensión. La visualización en la recta numérica es especialmente útil para desarrollar intuición.
