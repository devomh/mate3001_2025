%========================================
% LESSON CONTENT: Factorización y Evaluación de Polinomios
%========================================

\lesson{Factorización y Evaluación de Polinomios}

\subsectiontitle{Factorización de Polinomios}

La factorización es el proceso inverso de la multiplicación. Expresamos un polinomio como producto de sus factores.

% Factorization process flowchart
\begin{center}
\begin{tikzpicture}[
    scale=1.1,
    box/.style={rectangle, draw=black, thick, fill=blue!15, text width=3.5cm, text centered, minimum height=1.2cm, rounded corners=3pt},
    decision/.style={diamond, draw=black, thick, fill=yellow!20, text width=2.8cm, text centered, minimum height=1.5cm, aspect=2},
    arrow/.style={-{Stealth[length=4mm]}, ultra thick, blue!70!black}
]
    % Start node
    \node[box] (start) at (0,0) {\textbf{Expresión\\a factorizar}};
    
    % First decision
    \node[decision] (common) at (0,-3) {¿Hay factor\\común?};
    
    % Extract common factor branch
    \node[box] (extract) at (-4.5,-6) {\textbf{Extraer\\factor común}};
    
    % Second decision
    \node[decision] (terms) at (0,-6) {¿Cuántos\\términos?};
    
    % Final branches with better spacing
    \node[box] (two) at (-4,-9.5) {\textbf{2 términos:}\\Diferencia de\\cuadrados};
    \node[box] (three) at (0,-9.5) {\textbf{3 términos:}\\Trinomio\\cuadrático};
    \node[box] (four) at (4,-9.5) {\textbf{4+ términos:}\\Factorización\\por agrupación};
    
    % Arrows with better positioning
    \draw[arrow] (start) -- (common);
    \draw[arrow] (common) -- node[left, fill=white, inner sep=2pt] {\textbf{Sí}} (extract);
    \draw[arrow] (common) -- node[right, fill=white, inner sep=2pt] {\textbf{No}} (terms);
    \draw[arrow] (terms) -- (two);
    \draw[arrow] (terms) -- (three);
    \draw[arrow] (terms) -- (four);
\end{tikzpicture}
\end{center}

\textbf{Técnicas principales de factorización:}

\begin{enumerate}
\item \textbf{Factor común:} Extraer el mayor factor (numérico y literal) común a todos los términos.
\item \textbf{Agrupación:} Reagrupar términos para crear un factor común por pares o bloques.
\item \textbf{Trinomio cuadrático:} Para $ax^2 + bx + c$, usar el método AC cuando $a \neq 1$.
\item \textbf{Diferencia de cuadrados:} $a^2 - b^2 = (a + b)(a - b)$.
\item \textbf{Trinomio cuadrático perfecto:} $a^2 \pm 2ab + b^2 = (a \pm b)^2$.
\item \textbf{Suma y diferencia de cubos:} $a^3 \pm b^3 = (a \pm b)(a^2 \mp ab + b^2)$.
\end{enumerate}

%==============================
% TECNICA: FACTOR COMUN
%==============================
\subsectiontitle{Técnica: Factor común}

\textbf{Idea:} Todo producto distributivo $k\cdot(A + B + \cdots)$ puede “factorizarse hacia atrás”. Buscamos el \emph{máximo factor común} (MFC): el mayor número y las mayores potencias de variables comunes a todos los términos.

\textbf{Pasos:}
\begin{enumerate}
  \item Hallar el MFC numérico (máximo común divisor de coeficientes).
  \item Hallar el MFC literal (variables comunes con el menor exponente presente en todos los términos).
  \item Extraer el MFC y escribir el polinomio como producto: (MFC)$\times$(resto).
  \item Verificar multiplicando mentalmente.
\end{enumerate}

\begin{example}
\textbf{Ejemplo 1.} Factorizar $18x^3y^2 - 12x^2y + 6xy^3$.

\begin{enumerate}
  \item MFC numérico: $\gcd(18,12,6)=6$.
  \item MFC literal: $x^{\min(3,2,1)}=x$, $y^{\min(2,1,3)}=y$.
  \item Extraemos: $6xy(3x^2y - 2x + y^2)$.
  \item Verificación rápida: $6xy\cdot 3x^2y=18x^3y^2$, etc.
\end{enumerate}
\end{example}

\begin{example}
\textbf{Ejemplo 2.} Factorizar $-14a^2b + 21ab^2 - 7ab$.

\begin{enumerate}
  \item MFC numérico: $\gcd(14,21,7)=7$ y signo común conveniente $-7$ para simplificar signos.
  \item MFC literal: $a^{\min(2,1,1)}=a$, $b^{\min(1,2,1)}=b$.
  \item Extraemos: $-7ab(2a - 3b + 1)$.
\end{enumerate}
\end{example}

%==============================
% TECNICA: AGRUPACION
%==============================
\subsectiontitle{Técnica: Factorización por agrupación}

\textbf{Idea:} Reordenar y agrupar términos para crear un mismo factor binomial (o común) en cada grupo y luego factorizar ese factor común “de segundo nivel”.

\textbf{Pasos:}
\begin{enumerate}
  \item Intentar agrupar en pares (o bloques) que compartan un factor binomial.
  \item Factorizar cada grupo por su MFC.
  \item Verificar que aparece un factor común binomial. Factorizarlo.
  \item Si es necesario, reordenar para lograr el factor común.
\end{enumerate}

\begin{example}
\textbf{Ejemplo 1.} Factorizar $ax + ay + bx + by$.
\begin{enumerate}
  \item Agrupamos: $(ax + ay) + (bx + by)$.
  \item Por grupo: $a(x + y) + b(x + y)$.
  \item Factor común binomial: $(x + y)(a + b)$.
\end{enumerate}
\end{example}

\begin{example}
\textbf{Ejemplo 2.} Factorizar $3x^3 - 3x^2 - 12x + 12$.
\begin{enumerate}
  \item Agrupamos: $(3x^3 - 3x^2) + (-12x + 12)$.
  \item Por grupo: $3x^2(x - 1) - 12(x - 1)$.
  \item Factor común binomial: $(x - 1)(3x^2 - 12) = 3(x - 1)(x^2 - 4)$.
  \item Diferencia de cuadrados: $3(x - 1)(x - 2)(x + 2)$.
\end{enumerate}
\end{example}

%==============================
% TECNICA: TRINOMIO CUADRATICO (a=1 y a!=1)
%==============================
\subsectiontitle{Técnica: Trinomio cuadrático}

\textbf{Idea:} En $ax^2 + bx + c$, buscamos dos números que multipliquen $ac$ y sumen $b$. Si $a=1$, buscamos dos números que multipliquen $c$ y sumen $b$; si $a\neq 1$, usamos el \emph{método AC}.

\textbf{Pasos (caso $a=1$):}
\begin{enumerate}
  \item Hallar $m,n$ tales que $m+n=b$ y $mn=c$.
  \item Escribir $(x+m)(x+n)$.
\end{enumerate}

\textbf{Pasos (método AC, $a\neq 1$):}
\begin{enumerate}
  \item Calcular $AC=ac$ y hallar $m,n$ con $m+n=b$ y $mn=ac$.
  \item Reescribir $bx$ como $mx+nx$.
  \item Agrupar en dos binomios y factorizar por agrupación.
\end{enumerate}

\begin{example}
\textbf{Ejemplo 1 (a=1).} Factorizar $x^2 + 5x + 6$.
\begin{enumerate}
  \item Buscamos $m,n$ con $m+n=5$, $mn=6$. Sirven $2$ y $3$.
  \item Factorizamos: $(x+2)(x+3)$.
\end{enumerate}
\end{example}

\begin{example}
\textbf{Ejemplo 2 (método AC).} Factorizar $12x^2 - 7x - 10$.
\begin{enumerate}
  \item $ac=12\cdot(-10)=-120$. Buscamos $m,n$ con $m+n=-7$, $mn=-120$. Sirven $m=8$, $n=-15$.
  \item Reescribimos: $12x^2 + 8x - 15x - 10$.
  \item Agrupamos: $(12x^2+8x)+(-15x-10)=4x(3x+2)-5(3x+2)$.
  \item Factor común binomial: $(3x+2)(4x-5)$.
\end{enumerate}
\end{example}

%==============================
% TECNICA: DIFERENCIA DE CUADRADOS
%==============================
\subsectiontitle{Técnica: Diferencia de cuadrados}

\textbf{Idea:} Cualquier expresión $a^2-b^2$ se factoriza como $(a+b)(a-b)$. Es clave reconocer cuadrados perfectos (números y binomios).

\textbf{Pasos:}
\begin{enumerate}
  \item Identificar $a^2$ y $b^2$ (ambos cuadrados perfectos).
  \item Escribir $a$ y $b$ y formar $(a+b)(a-b)$.
  \item Si aparecen nuevas diferencias de cuadrados, repetir.
\end{enumerate}

\begin{example}
\textbf{Ejemplo 1.} Factorizar $4x^2-9$.
\begin{enumerate}
  \item $4x^2=(2x)^2$, $9=3^2$.
  \item $4x^2-9=(2x+3)(2x-3)$.
\end{enumerate}
\end{example}

\begin{example}
\textbf{Ejemplo 2.} Factorizar $x^4-16$.
\begin{enumerate}
  \item $x^4=(x^2)^2$, $16=4^2$.
  \item $x^4-16=(x^2+4)(x^2-4)$ y $x^2-4=(x+2)(x-2)$.
  \item Resultado: $(x^2+4)(x+2)(x-2)$.
\end{enumerate}
\end{example}

%==============================
% TECNICA: TRINOMIO CUADRATICO PERFECTO
%==============================
\subsectiontitle{Técnica: Trinomio cuadrático perfecto}

\textbf{Idea:} Reconocer $a^2\pm 2ab + b^2=(a\pm b)^2$ comparando el primer y último término con cuadrados perfectos y el término medio con $\pm 2ab$.

\textbf{Pasos:}
\begin{enumerate}
  \item Verificar que el primero y último término sean cuadrados perfectos: $a^2$ y $b^2$.
  \item Confirmar que el término medio sea $\pm 2ab$ (con el signo correspondiente).
  \item Escribir $(a\pm b)^2$.
\end{enumerate}

\begin{example}
\textbf{Ejemplo 1.} Factorizar $4x^2-12x+9$.
\begin{enumerate}
  \item $4x^2=(2x)^2$, $9=3^2$, término medio $-12x=-2(2x)(3)$.
  \item Entonces $(2x-3)^2$.
\end{enumerate}
\end{example}

\begin{example}
\textbf{Ejemplo 2.} Factorizar $9y^2+24y+16$.
\begin{enumerate}
  \item $9y^2=(3y)^2$, $16=4^2$, término medio $24y=2(3y)(4)$.
  \item Resultado: $(3y+4)^2$.
\end{enumerate}
\end{example}

%==============================
% TECNICA: SUMA Y DIFERENCIA DE CUBOS
%==============================
\subsectiontitle{Técnica: Suma y diferencia de cubos}

\textbf{Idea:} Usar las identidades
\[
 a^3+b^3=(a+b)(a^2-ab+b^2),\quad a^3-b^3=(a-b)(a^2+ab+b^2).
\]
Reconocer cubos perfectos (números y binomios) y aplicar la fórmula.

\textbf{Pasos:}
\begin{enumerate}
  \item Identificar $a^3$ y $b^3$.
  \item Escribir $a$ y $b$ y aplicar la identidad correspondiente.
  \item Simplificar si es posible.
\end{enumerate}

\begin{example}
\textbf{Ejemplo 1 (diferencia).} Factorizar $x^3-8$.
\begin{enumerate}
  \item $x^3=x^3$, $8=2^3$.
  \item $x^3-8=(x-2)(x^2+2x+4)$.
\end{enumerate}
\end{example}

\begin{example}
\textbf{Ejemplo 2 (suma).} Factorizar $8x^3+27$.
\begin{enumerate}
  \item $8x^3=(2x)^3$, $27=3^3$.
  \item $8x^3+27=(2x+3)(4x^2-6x+9)$.
\end{enumerate}
\end{example}

%==============================
% EJEMPLO INTEGRADO (CADENA DE TECNICAS)
%==============================
\begin{example}
\textbf{Ejemplo integrado.} Factorizar $2x^3-8x$.
\begin{enumerate}
  \item Factor común: $2x(x^2-4)$.
  \item Diferencia de cuadrados: $2x(x-2)(x+2)$.
  \item Verificación: al multiplicar se recupera $2x^3-8x$.
\end{enumerate}
\end{example}

\subsectiontitle{Evaluación de Polinomios}

\begin{definition}
Para evaluar un polinomio $P(x)$ en $x = a$, se sustituye $x$ por $a$.
\end{definition}

\begin{example}
Si $P(x) = 2x^3 - 3x^2 + x - 5$, entonces:
$$P(2) = 2(2)^3 - 3(2)^2 + 2 - 5 = 16 - 12 + 2 - 5 = 1$$
\end{example}

\begin{theorem}
\textbf{Teorema del Residuo:} Si un polinomio $P(x)$ se divide por $(x - a)$, entonces el residuo es $P(a)$.

\noindent \textbf{Teorema del Factor:} $(x - a)$ es un factor de $P(x)$ si y solo si $P(a) = 0$.
\end{theorem}

Estos teoremas son útiles para factorizar polinomios de grado alto y encontrar sus raíces.
