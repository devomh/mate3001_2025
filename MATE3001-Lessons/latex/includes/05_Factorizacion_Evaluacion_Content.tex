%========================================
% LESSON CONTENT: Factorización y Evaluación de Polinomios
%========================================

\lesson{Factorización y Evaluación de Polinomios}

\subsectiontitle{Factorización de Polinomios}

La factorización es el proceso inverso de la multiplicación. Expresamos un polinomio como producto de sus factores.

% Factorization process flowchart
\begin{center}
\begin{tikzpicture}[
    scale=1.1,
    box/.style={rectangle, draw=black, thick, fill=blue!15, text width=3.5cm, text centered, minimum height=1.2cm, rounded corners=3pt},
    decision/.style={diamond, draw=black, thick, fill=yellow!20, text width=2.8cm, text centered, minimum height=1.5cm, aspect=2},
    arrow/.style={-{Stealth[length=4mm]}, ultra thick, blue!70!black}
]
    % Start node
    \node[box] (start) at (0,0) {\textbf{Expresión\\a factorizar}};
    
    % First decision
    \node[decision] (common) at (0,-3) {¿Hay factor\\común?};
    
    % Extract common factor branch
    \node[box] (extract) at (-4.5,-6) {\textbf{Extraer\\factor común}};
    
    % Second decision
    \node[decision] (terms) at (0,-6) {¿Cuántos\\términos?};
    
    % Final branches with better spacing
    \node[box] (two) at (-3,-9.5) {\textbf{2 términos:}\\Diferencia de\\cuadrados};
    \node[box] (three) at (0,-9.5) {\textbf{3 términos:}\\Trinomio\\cuadrático};
    \node[box] (four) at (3,-9.5) {\textbf{4+ términos:}\\Factorización\\por agrupación};
    
    % Arrows with better positioning
    \draw[arrow] (start) -- (common);
    \draw[arrow] (common) -- node[left, fill=white, inner sep=2pt] {\textbf{Sí}} (extract);
    \draw[arrow] (common) -- node[right, fill=white, inner sep=2pt] {\textbf{No}} (terms);
    \draw[arrow] (terms) -- (two);
    \draw[arrow] (terms) -- (three);
    \draw[arrow] (terms) -- (four);
\end{tikzpicture}
\end{center}

\textbf{Técnicas principales de factorización:}

\begin{enumerate}
\item \textbf{Factor común:} Se extrae el mayor factor común de todos los términos.
\item \textbf{Agrupación:} Se agrupan términos para factorizar por partes.
\item \textbf{Trinomio cuadrático:} Para $ax^2 + bx + c$, se buscan dos números que multiplicados den $ac$ y sumados den $b$.
\item \textbf{Diferencia de cuadrados:} $a^2 - b^2 = (a + b)(a - b)$
\item \textbf{Trinomio cuadrático perfecto:} $a^2 \pm 2ab + b^2 = (a \pm b)^2$
\item \textbf{Suma y diferencia de cubos:} Usar las fórmulas correspondientes.
\end{enumerate}

\begin{example}
\textbf{Factor común:}
$$6x^3 + 9x^2 - 12x = 3x(2x^2 + 3x - 4)$$

\textbf{Agrupación:}
$$ax + ay + bx + by = a(x + y) + b(x + y) = (a + b)(x + y)$$

\textbf{Trinomio cuadrático:}
$$x^2 + 5x + 6 = (x + 2)(x + 3)$$

\textbf{Diferencia de cuadrados:}
$$4x^2 - 9 = (2x + 3)(2x - 3)$$

\textbf{Ejemplo completo:}
Factorizar: $2x^3 - 8x$

1. \textbf{Factor común:} $2x(x^2 - 4)$
2. \textbf{Diferencia de cuadrados:} $2x(x + 2)(x - 2)$

\textbf{Verificación:} $2x(x + 2)(x - 2) = 2x(x^2 - 4) = 2x^3 - 8x$ $\checkmark$
\end{example}

\subsectiontitle{Evaluación de Polinomios}

\begin{definition}
Para evaluar un polinomio $P(x)$ en $x = a$, se sustituye $x$ por $a$.
\end{definition}

\begin{example}
Si $P(x) = 2x^3 - 3x^2 + x - 5$, entonces:
$$P(2) = 2(2)^3 - 3(2)^2 + 2 - 5 = 16 - 12 + 2 - 5 = 1$$
\end{example}

\begin{theorem}
\textbf{Teorema del Residuo:} Si un polinomio $P(x)$ se divide por $(x - a)$, entonces el residuo es $P(a)$.

\noindent \textbf{Teorema del Factor:} $(x - a)$ es un factor de $P(x)$ si y solo si $P(a) = 0$.
\end{theorem}

Estos teoremas son útiles para factorizar polinomios de grado alto y encontrar sus raíces.