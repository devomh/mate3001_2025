%========================================
% EXERCISES: Factorización y Evaluación de Polinomios
%========================================

\section{Ejercicios}

\begin{exercise}
\problem Factorice completamente las siguientes expresiones:

\begin{exerciselist}
    \item $6x^2 + 12x$
    \item $x^2 + 7x + 12$
    \item $x^2 - 25$
    \item $4x^2 - 12x + 9$
    \item $x^3 - 8$
    \item $2x^3 - 8x$
    \item $x^4 - 16$
    \item $ac + ad + bc + bd$
\end{exerciselist}

\begin{solucion}
\begin{exerciselist}
    \item $6x^2 + 12x = 6x(x + 2)$
    
    \item $x^2 + 7x + 12 = (x + 3)(x + 4)$ [Buscamos números que sumen 7 y multipliquen 12: 3 y 4]
    
    \item $x^2 - 25 = x^2 - 5^2 = (x + 5)(x - 5)$ [Diferencia de cuadrados]
    
    \item $4x^2 - 12x + 9 = (2x)^2 - 2(2x)(3) + 3^2 = (2x - 3)^2$ [Trinomio cuadrático perfecto]
    
    \item $x^3 - 8 = x^3 - 2^3 = (x - 2)(x^2 + 2x + 4)$ [Diferencia de cubos]
    
    \item $2x^3 - 8x = 2x(x^2 - 4) = 2x(x + 2)(x - 2)$ [Factor común y diferencia de cuadrados]
    
    \item $x^4 - 16 = (x^2)^2 - 4^2 = (x^2 + 4)(x^2 - 4) = (x^2 + 4)(x + 2)(x - 2)$ [Diferencia de cuadrados aplicada dos veces]
    
    \item $ac + ad + bc + bd = a(c + d) + b(c + d) = (a + b)(c + d)$ [Factorización por agrupación]
\end{exerciselist}
\end{solucion}
\end{exercise}

\begin{exercise}
\problem Evalúe los siguientes polinomios en los valores dados:

\begin{exerciselist}
    \item $P(x) = 2x^3 - 3x^2 + x - 4$; evalúe $P(2)$
    \item $Q(x) = x^4 - 2x^2 + 1$; evalúe $Q(-1)$
    \item $R(x) = 3x^2 - 5x + 2$; evalúe $R(0)$ y $R(1)$
    \item $S(x) = x^3 + 2x^2 - 5x - 6$; evalúe $S(3)$ y $S(-2)$
\end{exerciselist}

\begin{solucion}
\begin{exerciselist}
    \item $P(2) = 2(2)^3 - 3(2)^2 + 2 - 4 = 2(8) - 3(4) + 2 - 4 = 16 - 12 + 2 - 4 = 2$
    
    \item $Q(-1) = (-1)^4 - 2(-1)^2 + 1 = 1 - 2(1) + 1 = 1 - 2 + 1 = 0$
    
    \item $R(0) = 3(0)^2 - 5(0) + 2 = 2$\\
    $R(1) = 3(1)^2 - 5(1) + 2 = 3 - 5 + 2 = 0$
    
    \item $S(3) = (3)^3 + 2(3)^2 - 5(3) - 6 = 27 + 18 - 15 - 6 = 24$\\
    $S(-2) = (-2)^3 + 2(-2)^2 - 5(-2) - 6 = -8 + 8 + 10 - 6 = 4$
\end{exerciselist}
\end{solucion}
\end{exercise}

\begin{exercise}
\problem Use el Teorema del Factor para determinar si el binomio dado es un factor del polinomio:

\begin{exerciselist}
    \item ¿Es $(x - 2)$ un factor de $x^3 - 3x^2 + 4x - 12$?
    \item ¿Es $(x + 1)$ un factor de $2x^3 + x^2 - 5x + 2$?
    \item ¿Es $(x - 3)$ un factor de $x^4 - 4x^3 + 6x^2 - 4x + 1$?
    \item ¿Es $(2x - 1)$ un factor de $2x^3 - 3x^2 + 1$?
\end{exerciselist}

\begin{solucion}
\begin{exerciselist}
    \item Para $(x - 2)$, evaluamos $P(2)$:\\
    $P(2) = 2^3 - 3(2^2) + 4(2) - 12 = 8 - 12 + 8 - 12 = -8 \neq 0$\\
    No es factor.
    
    \item Para $(x + 1)$, evaluamos $P(-1)$:\\
    $P(-1) = 2(-1)^3 + (-1)^2 - 5(-1) + 2 = -2 + 1 + 5 + 2 = 6 \neq 0$\\
    No es factor.
    
    \item Para $(x - 3)$, evaluamos $P(3)$:\\
    $P(3) = 3^4 - 4(3^3) + 6(3^2) - 4(3) + 1 = 81 - 108 + 54 - 12 + 1 = 16 \neq 0$\\
    No es factor.
    
    \item Para $(2x - 1)$, evaluamos $P(\frac{1}{2})$:\\
    $P(\frac{1}{2}) = 2(\frac{1}{2})^3 - 3(\frac{1}{2})^2 + 1 = 2(\frac{1}{8}) - 3(\frac{1}{4}) + 1 = \frac{1}{4} - \frac{3}{4} + 1 = \frac{1}{2} \neq 0$\\
    No es factor.
\end{exerciselist}
\end{solucion}
\end{exercise}

\begin{exercise}
\problem \textbf{Problemas de aplicación:}

\begin{exerciselist}
    \item Un rectángulo tiene largo $(2x + 3)$ y ancho $(x - 1)$. Encuentre una expresión para su área.
    
    \item El volumen de una caja rectangular es $V = x^3 + 6x^2 + 11x + 6$. Si las dimensiones son $(x + 1)$, $(x + 2)$, y $(x + 3)$, verifique que esta expresión es correcta.
    
    \item Un proyectil se lanza verticalmente hacia arriba. Su altura en metros después de $t$ segundos está dada por $h(t) = -5t^2 + 20t + 25$. ¿Cuál es la altura inicial? ¿Cuál es la altura después de 2 segundos?
    
    \item El costo total de producir $x$ artículos está dado por $C(x) = 2x^2 + 15x + 100$. ¿Cuál es el costo fijo? ¿Cuál es el costo de producir 10 artículos?
\end{exerciselist}

\begin{solucion}
\begin{exerciselist}
    \item Área $= (2x + 3)(x - 1) = 2x^2 - 2x + 3x - 3 = 2x^2 + x - 3$
    
    \item Verificación:\\
    $(x + 1)(x + 2)(x + 3)$\\
    $= (x + 1)[(x + 2)(x + 3)]$\\
    $= (x + 1)[x^2 + 5x + 6]$\\
    $= x^3 + 5x^2 + 6x + x^2 + 5x + 6$\\
    $= x^3 + 6x^2 + 11x + 6$ ✓
    
    \item Altura inicial: $h(0) = -5(0)^2 + 20(0) + 25 = 25$ metros\\
    Altura después de 2 segundos: $h(2) = -5(4) + 20(2) + 25 = -20 + 40 + 25 = 45$ metros
    
    \item Costo fijo: $C(0) = 2(0)^2 + 15(0) + 100 = 100$ (cuando $x = 0$)\\
    Costo de 10 artículos: $C(10) = 2(100) + 15(10) + 100 = 200 + 150 + 100 = 450$
\end{exerciselist}
\end{solucion}
\end{exercise}