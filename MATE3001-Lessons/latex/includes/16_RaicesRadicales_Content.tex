%========================================
% LESSON CONTENT: Raíces y Radicales
%========================================

\lesson{Raíces y Radicales}

\begin{warning}
Salvo aviso contrario, trabajamos en números reales. Para índices \textbf{pares} se exige radicando $\geq 0$ y, al simplificar, pueden aparecer valores absolutos (por ejemplo, $\sqrt{x^2}=|x|$).
\end{warning}

\subsectiontitle{Raíces y radicales --- Simplificación de expresiones con radicales}

\textbf{Objetivos:}
\begin{itemize}
    \item Reconocer partes de un radical: índice, signo radical y radicando.
    \item Determinar cuándo una raíz es real (índice par vs. impar).
    \item Simplificar extrayendo factores que sean potencias perfectas del índice.
\end{itemize}

\begin{definition}
\textbf{Raíz $n$-ésima:} $\sqrt[n]{a}$ es el número $b$ tal que $b^n=a$ (si existe en $\mathbb{R}$).

\textbf{Partes de un radical:}
\begin{itemize}
    \item \textbf{Índice}: El número $n$ en $\sqrt[n]{a}$ (si no aparece, se asume $n=2$)
    \item \textbf{Signo radical}: El símbolo $\sqrt{\phantom{x}}$
    \item \textbf{Radicando}: El número o expresión $a$ dentro del radical
\end{itemize}

\textbf{Propiedades importantes:}
\begin{itemize}
    \item Si $a=k^n$, entonces $\sqrt[n]{a}=k$ (potencia perfecta)
    \item Para índice par: $\sqrt[n]{a^n b}=|a|\,\sqrt[n]{b}$
    \item Para índice impar: $\sqrt[n]{a^n b}=a\,\sqrt[n]{b}$
\end{itemize}
\end{definition}

\textbf{Procedimiento de simplificación (índice general $n$):}
\begin{enumerate}
    \item Factorice el radicando como producto de potencias: $a=a_1^n\,a_2$.
    \item Extraiga $a_1$ fuera del radical; mantenga $a_2$ dentro.
    \item Para índices pares, use valor absoluto cuando corresponda.
\end{enumerate}

\begin{example}
\textbf{Ejemplo 1 (índice par):} Simplifique $\sqrt{72x^3y^4}$.

\solution
\begin{align*}
\sqrt{72x^3y^4} &= \sqrt{36\cdot2\cdot x^2\cdot x\cdot (y^2)^2}\\
&= \sqrt{36}\cdot\sqrt{x^2}\cdot\sqrt{y^4}\cdot\sqrt{2x}\\
&= 6\cdot |x|\cdot y^2\cdot\sqrt{2x}\\
&= 6|x|y^2\sqrt{2x}
\end{align*}

\textbf{Nota:} Si se asume $x\geq 0$, entonces la respuesta es $6xy^2\sqrt{2x}$.
\end{example}

\begin{example}
\textbf{Ejemplo 2 (índice impar):} Simplifique $\sqrt[3]{54a^5b^3}$.

\solution
\begin{align*}
\sqrt[3]{54a^5b^3} &= \sqrt[3]{27\cdot2}\cdot\sqrt[3]{a^3\cdot a^2}\cdot\sqrt[3]{b^3}\\
&= 3\cdot a\cdot b\cdot\sqrt[3]{2a^2}\\
&= 3ab\sqrt[3]{2a^2}
\end{align*}
\end{example}

\begin{example}
\textbf{Aplicación rápida (dominio):} Si $f(x)=\sqrt{x-3}$, determine el dominio.

\solution
Para que $\sqrt{x-3}$ sea real, necesitamos:
$$x-3\geq 0 \Rightarrow x\geq 3$$
Por lo tanto, el dominio es $[3,\infty)$.
\end{example}

\begin{warning}
\textbf{Errores comunes:}
\begin{itemize}
    \item Olvidar el valor absoluto al simplificar $\sqrt{x^2}=|x|$.
    \item Pretender que $\sqrt{a+b}=\sqrt{a}+\sqrt{b}$ (¡FALSO!).
    \item No simplificar completamente antes de proceder.
\end{itemize}
\end{warning}

\subsectiontitle{Multiplicación y simplificación de radicales}

\textbf{Objetivos:}
\begin{itemize}
    \item Multiplicar radicales con el mismo índice y simplificar el resultado.
\end{itemize}

\begin{theorem}
\textbf{Propiedad clave:}

Para $a,b\geq 0$ (si el índice es par):
$$\sqrt[n]{a}\cdot\sqrt[n]{b}=\sqrt[n]{ab}$$

Se debe simplificar el radicando y, si es posible, extraer factores perfectos.
\end{theorem}

\begin{example}
\textbf{Ejemplo 1:} Multiplique y simplifique: $(\sqrt{3x})(\sqrt{15x^3})$.

\solution
\begin{align*}
(\sqrt{3x})(\sqrt{15x^3}) &= \sqrt{3x\cdot 15x^3}\\
&= \sqrt{45x^4}\\
&= \sqrt{9\cdot 5\cdot x^4}\\
&= 3x^2\sqrt{5}
\end{align*}
\end{example}

\begin{example}
\textbf{Ejemplo 2:} Multiplique y simplifique: $\sqrt[3]{8a^2}\cdot\sqrt[3]{2a}$.

\solution
\begin{align*}
\sqrt[3]{8a^2}\cdot\sqrt[3]{2a} &= \sqrt[3]{16a^3}\\
&= \sqrt[3]{8\cdot 2\cdot a^3}\\
&= 2a\sqrt[3]{2}
\end{align*}
\end{example}

\textbf{Consejo:} Factorice antes de multiplicar para extraer más fácilmente potencias perfectas.

\subsectiontitle{División y racionalización de radicales}

\textbf{Objetivos:}
\begin{itemize}
    \item Dividir radicales con el mismo índice y racionalizar denominadores.
\end{itemize}

\begin{theorem}
\textbf{Propiedades:}

$$\frac{\sqrt[n]{a}}{\sqrt[n]{b}}=\sqrt[n]{\frac{a}{b}} \quad \text{con } b>0$$

\textbf{Para eliminar radicales en el denominador:}
\begin{itemize}
    \item \textbf{Denominador monomial} con $\sqrt{m}$: multiplicar por $\dfrac{\sqrt{m}}{\sqrt{m}}$
    \item \textbf{Denominador binomial} $a\pm b\sqrt{m}$: multiplicar por el conjugado $a\mp b\sqrt{m}$
\end{itemize}
\end{theorem}

\begin{example}
\textbf{Ejemplo 1 (conjugado):} Racionalice: $\dfrac{5}{4-\sqrt{2}}$.

\solution
\begin{align*}
\frac{5}{4-\sqrt{2}} &= \frac{5}{4-\sqrt{2}}\cdot\frac{4+\sqrt{2}}{4+\sqrt{2}}\\
&= \frac{5(4+\sqrt{2})}{(4)^2-(\sqrt{2})^2}\\
&= \frac{20+5\sqrt{2}}{16-2}\\
&= \frac{20+5\sqrt{2}}{14}\\
&= \frac{10}{7}+\frac{5\sqrt{2}}{14}
\end{align*}
\end{example}

\begin{example}
\textbf{Ejemplo 2 (monomial):} Racionalice: $\dfrac{3}{\sqrt{5}}$.

\solution
$$\frac{3}{\sqrt{5}}=\frac{3}{\sqrt{5}}\cdot\frac{\sqrt{5}}{\sqrt{5}}=\frac{3\sqrt{5}}{5}$$
\end{example}

\begin{example}
\textbf{Ejemplo 3 (simplificar antes de racionalizar):} Simplifique $\dfrac{2\sqrt{3}}{\sqrt{6}}$.

\solution
Método 1 (combinar radicales):
$$\frac{2\sqrt{3}}{\sqrt{6}}=2\sqrt{\frac{3}{6}}=2\sqrt{\frac{1}{2}}=2\cdot\frac{1}{\sqrt{2}}=\frac{2\sqrt{2}}{2}=\sqrt{2}$$

Método 2 (racionalizar primero):
$$\frac{2\sqrt{3}}{\sqrt{6}}\cdot\frac{\sqrt{6}}{\sqrt{6}}=\frac{2\sqrt{18}}{6}=\frac{2\cdot 3\sqrt{2}}{6}=\sqrt{2}$$
\end{example}

\begin{warning}
\textbf{Errores comunes:}
\begin{itemize}
    \item Racionalizar sin antes simplificar.
    \item Usar el conjugado incorrecto.
    \item Olvidar distribuir al multiplicar por el conjugado.
\end{itemize}
\end{warning}

\subsectiontitle{Suma y resta de radicales}

\textbf{Objetivos:}
\begin{itemize}
    \item Identificar radicales semejantes y combinarlos correctamente.
\end{itemize}

\begin{definition}
\textbf{Radicales semejantes:} Solo se suman/restan radicales con \textbf{mismo índice y mismo radicando} una vez simplificados. Actúan como ``términos semejantes''.

Por ejemplo: $3\sqrt{5}+2\sqrt{5}=5\sqrt{5}$ (similar a $3x+2x=5x$)
\end{definition}

\begin{example}
\textbf{Ejemplo 1:} Simplifique: $4\sqrt{20}-3\sqrt{5}+\sqrt{45}$.

\solution
Primero simplificamos cada radical:
\begin{align*}
\sqrt{20} &= \sqrt{4\cdot 5}=2\sqrt{5}\\
\sqrt{45} &= \sqrt{9\cdot 5}=3\sqrt{5}
\end{align*}

Entonces:
\begin{align*}
4\sqrt{20}-3\sqrt{5}+\sqrt{45} &= 4(2\sqrt{5})-3\sqrt{5}+3\sqrt{5}\\
&= 8\sqrt{5}-3\sqrt{5}+3\sqrt{5}\\
&= 8\sqrt{5}
\end{align*}
\end{example}

\begin{example}
\textbf{Ejemplo 2:} Simplifique: $3\sqrt{8}-\sqrt{18}+2\sqrt{2}$.

\solution
\begin{align*}
3\sqrt{8}-\sqrt{18}+2\sqrt{2} &= 3(2\sqrt{2})-3\sqrt{2}+2\sqrt{2}\\
&= 6\sqrt{2}-3\sqrt{2}+2\sqrt{2}\\
&= 5\sqrt{2}
\end{align*}
\end{example}

\begin{example}
\textbf{Ejemplo 3 (con variables):} Simplifique: $5x\sqrt{3y}-2\sqrt{27y}+\sqrt{12y}$.

\solution
\begin{align*}
5x\sqrt{3y}-2\sqrt{27y}+\sqrt{12y} &= 5x\sqrt{3y}-2(3\sqrt{3y})+2\sqrt{3y}\\
&= 5x\sqrt{3y}-6\sqrt{3y}+2\sqrt{3y}\\
&= (5x-4)\sqrt{3y}
\end{align*}
\end{example}

\begin{warning}
\textbf{Errores comunes:}
\begin{itemize}
    \item Intentar combinar radicales no semejantes.
    \item No simplificar cada radical antes de combinar.
\end{itemize}
\end{warning}

\subsectiontitle{Radicales y notación exponencial}

\textbf{Objetivos:}
\begin{itemize}
    \item Convertir entre notación radical y exponente racional y aplicar leyes de exponentes.
\end{itemize}

\begin{definition}
\textbf{Equivalencias fundamentales:}
\begin{itemize}
    \item $\sqrt[n]{a^m}=a^{\frac{m}{n}}$ (si $a\geq 0$ cuando $n$ es par)
    \item $a^{p/q}=\sqrt[q]{a^p}=(\sqrt[q]{a})^p$
    \item $a^{-r}=\dfrac{1}{a^r}$ para $a\neq 0$
\end{itemize}
\end{definition}

\begin{example}
\textbf{Ejemplo 1 (conversiones):}
\begin{itemize}
    \item Convertir: $\sqrt[5]{x^2}=x^{2/5}$
    \item Convertir: $y^{-3/4}=\dfrac{1}{y^{3/4}}=\dfrac{1}{\sqrt[4]{y^3}}$
\end{itemize}
\end{example}

\begin{example}
\textbf{Ejemplo 2 (operando con exponentes racionales):}

Simplifique: $\dfrac{x^{3/2}\,x^{1/2}}{x^{5/2}}$

\solution
\begin{align*}
\frac{x^{3/2}\,x^{1/2}}{x^{5/2}} &= x^{(3/2+1/2-5/2)}\\
&= x^{-1/2}\\
&= \frac{1}{x^{1/2}}\\
&= \frac{1}{\sqrt{x}}
\end{align*}
\end{example}

\begin{example}
\textbf{Ejemplo 3 (de radical a exponente y simplificación):}

Simplifique: $\sqrt[3]{a^5}$

\solution
\begin{align*}
\sqrt[3]{a^5} &= a^{5/3}\\
&= a^{3/3+2/3}\\
&= a^1\cdot a^{2/3}\\
&= a\sqrt[3]{a^2}
\end{align*}
\end{example}

\begin{warning}
\textbf{Errores comunes:}
\begin{itemize}
    \item Omitir el signo de fracción en el exponente ($a^{m/n}$).
    \item Olvidar que un exponente negativo ``manda al denominador''.
    \item Confundir $a^{m/n}$ con $\dfrac{a^m}{n}$.
\end{itemize}
\end{warning}

\subsectiontitle{Notas didácticas}

\begin{itemize}
    \item Mantenga visibles las restricciones de dominio para índices pares.
    \item Siempre simplifique radicales antes de sumar/restar o racionalizar.
    \item En problemas con variables, indique cuándo aparecen valores absolutos o especifique supuestos (por ejemplo, ``para $x\geq 0$'').
    \item Verifique sus respuestas sustituyendo valores numéricos simples cuando sea posible.
\end{itemize}
