%========================================
% DETAILED SOLUTIONS: Sistema de Dos Ecuaciones Lineales
%========================================

\subsection*{Notas para el Instructor}

Este documento contiene notas pedagógicas y errores comunes de los estudiantes. Las soluciones completas de los ejercicios se encuentran integradas en el documento principal cuando se activa el modo de soluciones.

%========================================
% Exercise 1 Notes
%========================================
\subsection*{Ejercicio 1: Resolver Sistemas por los Tres Métodos}

\textbf{Objetivo Pedagógico:} Que los estudiantes practiquen y comparen los tres métodos de solución.

\textbf{Errores Comunes:}

\begin{itemize}
    \item \textbf{Método Gráfico:}
    \begin{itemize}
        \item Trazar rectas incorrectamente por no usar suficientes puntos
        \item Leer incorrectamente las coordenadas del punto de intersección
        \item No verificar la solución sustituyéndola en ambas ecuaciones
    \end{itemize}

    \item \textbf{Método de Sustitución:}
    \begin{itemize}
        \item Errores algebraicos al despejar una variable
        \item Olvidar sustituir de vuelta para encontrar la segunda variable
        \item Sustituir en la ecuación incorrecta (debe ser en la ecuación original, no en una modificada)
    \end{itemize}

    \item \textbf{Método de Eliminación:}
    \begin{itemize}
        \item No multiplicar todos los términos de la ecuación por el mismo factor
        \item Sumar cuando se debe restar, o viceversa
        \item Olvidar ajustar los coeficientes para que sean opuestos
    \end{itemize}
\end{itemize}

\textbf{Sugerencias de Enseñanza:}

\begin{enumerate}
    \item Enfatice que los tres métodos deben dar la misma solución
    \item Muestre cuándo es más conveniente usar cada método:
    \begin{itemize}
        \item Gráfico: Cuando se necesita una visualización o una aproximación
        \item Sustitución: Cuando una variable ya está despejada o es fácil de despejar
        \item Eliminación: Cuando los coeficientes son múltiplos fáciles o ya son opuestos
    \end{itemize}
    \item Recuerde a los estudiantes SIEMPRE verificar su solución
\end{enumerate}

\textbf{Notas Específicas por Problema:}

\textbf{Problema 1.1:} $\begin{cases}2x - y = 4 \\ 3x + y = 6\end{cases}$
\begin{itemize}
    \item Este es un caso ideal para eliminación (coeficientes de $y$ ya son opuestos)
    \item Solución: $(2, 0)$ está en el eje $x$, lo cual facilita la verificación gráfica
\end{itemize}

\textbf{Problema 1.2:} $\begin{cases}x + y = 7 \\ 2x - 3y = -1\end{cases}$
\begin{itemize}
    \item La primera ecuación es fácil de despejar (buena para sustitución)
    \item Para eliminación, multiplicar la primera por 3 o la primera por 2
    \item Solución: $(4, 3)$ con números enteros positivos
\end{itemize}

\textbf{Problema 1.3:} $\begin{cases}2x + 5y = 15 \\ 4x + y = 21\end{cases}$
\begin{itemize}
    \item La segunda ecuación tiene coeficiente 1 en $y$ (fácil para sustitución)
    \item Para eliminación, multiplicar la segunda por $-5$
    \item Solución: $(5, 1)$
\end{itemize}

%========================================
% Exercise 2 Notes
%========================================
\subsection*{Ejercicio 2: Identificar el Número de Soluciones}

\textbf{Objetivo Pedagógico:} Desarrollar intuición para reconocer sistemas sin resolverlos completamente.

\textbf{Conceptos Clave:}

\begin{itemize}
    \item \textbf{Infinitas soluciones:} Las ecuaciones son equivalentes (una es múltiplo de la otra)
    \item \textbf{Sin solución:} Misma pendiente, diferentes intersecciones (paralelas)
    \item \textbf{Una solución:} Pendientes diferentes (se intersectan)
\end{itemize}

\textbf{Estrategia de Identificación:}

\begin{enumerate}
    \item Convertir ambas ecuaciones a la forma $y = mx + b$ (si es posible)
    \item Comparar las pendientes $m$:
    \begin{itemize}
        \item Si $m_1 \neq m_2$: Una solución
        \item Si $m_1 = m_2$ y $b_1 \neq b_2$: Sin solución
        \item Si $m_1 = m_2$ y $b_1 = b_2$: Infinitas soluciones
    \end{itemize}
    \item Alternativamente, verificar si una ecuación es múltiplo de la otra
\end{enumerate}

\textbf{Errores Comunes:}
\begin{itemize}
    \item Confundir "infinitas soluciones" con "sin solución"
    \item No reconocer ecuaciones equivalentes cuando están escritas de manera diferente
    \item Intentar resolver el sistema en lugar de analizar la estructura
\end{itemize}

%========================================
% Exercise 3 Notes
%========================================
\subsection*{Ejercicio 3: Problemas de Aplicación}

\textbf{Objetivo Pedagógico:} Conectar sistemas de ecuaciones con situaciones del mundo real.

\textbf{Proceso de 4 Pasos (Énfasis):}

\begin{enumerate}
    \item \textbf{Identificar variables:} Definir claramente qué representa cada variable
    \item \textbf{Expresar cantidades:} Traducir las palabras a expresiones matemáticas
    \item \textbf{Establecer sistema:} Escribir las ecuaciones basadas en las relaciones
    \item \textbf{Resolver e interpretar:} Resolver y expresar la respuesta en contexto
\end{enumerate}

\textbf{Errores Comunes:}

\begin{itemize}
    \item No definir las variables claramente desde el principio
    \item Confundir las unidades (centavos vs. dólares)
    \item No escribir la respuesta final como una oración completa
    \item Olvidar verificar que la solución tiene sentido en el contexto
\end{itemize}

\textbf{Notas Específicas por Problema:}

\textbf{Problema 3.1 (Suma y diferencia):}
\begin{itemize}
    \item Tipo clásico de problema
    \item Cuidado: "diferencia" implica el número mayor menos el menor
    \item Verificación: ¿Los números tienen sentido? ¿Son positivos?
\end{itemize}

\textbf{Problema 3.2 (Monedas):}
\begin{itemize}
    \item Recordar convertir dólares a centavos O centavos a dólares (consistencia)
    \item Dos ecuaciones: cantidad total y valor total
    \item Verificación: ¿El número de monedas es entero? ¿Es razonable?
\end{itemize}

\textbf{Problema 3.3 (Tienda):}
\begin{itemize}
    \item Similar al problema de monedas
    \item Enfatizar la estructura: cantidad total + valor total
    \item Este problema refuerza el patrón
\end{itemize}

\textbf{Problema 3.4 (Relación entre números):}
\begin{itemize}
    \item Traducir "tres veces el primer número menos el segundo" correctamente: $3x - y$
    \item No confundir con $3(x - y)$
\end{itemize}

\textbf{Problema 3.5 (Geometría - Rectángulo):}
\begin{itemize}
    \item Recordar la fórmula del perímetro: $P = 2l + 2a$
    \item "El largo es 6 metros más que el ancho": $l = a + 6$
    \item Conecta álgebra con geometría
\end{itemize}

%========================================
% Additional Teaching Strategies
%========================================
\subsection*{Estrategias Adicionales de Enseñanza}

\textbf{Para Estudiantes con Dificultades:}

\begin{enumerate}
    \item Comenzar siempre con el método gráfico para visualización
    \item Usar problemas con números enteros pequeños
    \item Proporcionar plantillas para el proceso de 4 pasos
    \item Enfatizar la verificación como paso obligatorio
\end{enumerate}

\textbf{Para Estudiantes Avanzados:}

\begin{enumerate}
    \item Presentar sistemas con fracciones o decimales
    \item Introducir sistemas con más de dos variables (3×3)
    \item Explorar casos especiales (rectas paralelas, coincidentes)
    \item Crear sus propios problemas de aplicación
\end{enumerate}

\textbf{Actividades Complementarias:}

\begin{itemize}
    \item Comparar la eficiencia de cada método con diferentes sistemas
    \item Usar software de graficación (Desmos, GeoGebra) para verificar soluciones
    \item Crear problemas de aplicación basados en situaciones de su vida diaria
    \item Trabajo en grupos: Cada estudiante resuelve por un método diferente
\end{itemize}

%========================================
% Assessment Rubric
%========================================
\subsection*{Rúbrica de Evaluación Sugerida}

\textbf{Para Resolución de Sistemas:}

\begin{itemize}
    \item Configuración correcta (identificar el sistema): 20\%
    \item Proceso algebraico correcto: 40\%
    \item Solución correcta: 30\%
    \item Verificación: 10\%
\end{itemize}

\textbf{Para Problemas de Aplicación:}

\begin{itemize}
    \item Definición de variables: 15\%
    \item Establecimiento del sistema: 25\%
    \item Solución algebraica correcta: 35\%
    \item Interpretación y respuesta en contexto: 20\%
    \item Verificación: 5\%
\end{itemize}
