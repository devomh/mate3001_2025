%========================================
% EXERCISES: Ecuaciones Cuadráticas
%========================================

\section{Ejercicios}

\begin{exercise}
\problem Identifique los coeficientes $a$, $b$ y $c$ de las siguientes ecuaciones cuadráticas. Si la ecuación no está en forma estándar, escríbala primero en forma estándar.

\begin{exerciselist}
    \item $x^2 + 8x + 15 = 0$
    \item $3x^2 - 7x + 2 = 0$
    \item $x^2 - 16 = 0$
    \item $5x^2 + 2x = 3$
    \item $-2x^2 = 8x - 10$
    \item $x(x + 5) = 6$
\end{exerciselist}

\begin{solucion}
\textbf{a)} $a = 1$, $b = 8$, $c = 15$

\textbf{b)} $a = 3$, $b = -7$, $c = 2$

\textbf{c)} $a = 1$, $b = 0$, $c = -16$

\textbf{d)} Forma estándar: $5x^2 + 2x - 3 = 0$. Por tanto: $a = 5$, $b = 2$, $c = -3$

\textbf{e)} Forma estándar: $-2x^2 - 8x + 10 = 0$ o $2x^2 + 8x - 10 = 0$. Por tanto: $a = -2$, $b = -8$, $c = 10$ (o $a = 2$, $b = 8$, $c = -10$)

\textbf{f)} Expandir: $x^2 + 5x = 6$. Forma estándar: $x^2 + 5x - 6 = 0$. Por tanto: $a = 1$, $b = 5$, $c = -6$
\end{solucion}
\end{exercise}

\begin{exercise}
\problem Resuelva las siguientes ecuaciones cuadráticas por \textbf{factorización}:

\begin{exerciselist}
    \item $x^2 + 9x + 20 = 0$
    \item $x^2 - x - 12 = 0$
    \item $x^2 + 2x - 15 = 0$
    \item $x^2 - 10x + 21 = 0$
    \item $x^2 + 6x + 9 = 0$
    \item $x^2 - 49 = 0$
    \item $4x^2 - 25 = 0$
    \item $x^2 + 8x = 0$
\end{exerciselist}

\begin{solucion}
\textbf{a)} $(x + 4)(x + 5) = 0 \Rightarrow x = -4$ o $x = -5$

\textbf{b)} $(x - 4)(x + 3) = 0 \Rightarrow x = 4$ o $x = -3$

\textbf{c)} $(x + 5)(x - 3) = 0 \Rightarrow x = -5$ o $x = 3$

\textbf{d)} $(x - 3)(x - 7) = 0 \Rightarrow x = 3$ o $x = 7$

\textbf{e)} $(x + 3)^2 = 0 \Rightarrow x = -3$ (raíz doble)

\textbf{f)} $(x + 7)(x - 7) = 0 \Rightarrow x = 7$ o $x = -7$

\textbf{g)} $(2x + 5)(2x - 5) = 0 \Rightarrow x = \frac{5}{2}$ o $x = -\frac{5}{2}$

\textbf{h)} $x(x + 8) = 0 \Rightarrow x = 0$ o $x = -8$
\end{solucion}
\end{exercise}

\begin{exercise}
\problem Resuelva las siguientes ecuaciones cuadráticas por factorización:

\begin{exerciselist}
    \item $2x^2 + 7x + 3 = 0$
    \item $3x^2 - 10x + 8 = 0$
    \item $5x^2 + 13x - 6 = 0$
    \item $4x^2 - 4x - 15 = 0$
    \item $6x^2 + 7x - 3 = 0$
    \item $2x^2 - 9x + 10 = 0$
\end{exerciselist}

\begin{solucion}
\textbf{a)} $ac = 6$, buscar números que sumen 7 y multipliquen 6: son 6 y 1
\begin{align*}
2x^2 + 6x + x + 3 &= 0\\
2x(x + 3) + 1(x + 3) &= 0\\
(x + 3)(2x + 1) &= 0
\end{align*}
$x = -3$ o $x = -\frac{1}{2}$

\textbf{b)} $ac = 24$, buscar números que sumen $-10$ y multipliquen 24: son $-6$ y $-4$
\begin{align*}
3x^2 - 6x - 4x + 8 &= 0\\
3x(x - 2) - 4(x - 2) &= 0\\
(x - 2)(3x - 4) &= 0
\end{align*}
$x = 2$ o $x = \frac{4}{3}$

\textbf{c)} $ac = -30$, buscar números que sumen 13 y multipliquen $-30$: son 15 y $-2$
\begin{align*}
5x^2 + 15x - 2x - 6 &= 0\\
5x(x + 3) - 2(x + 3) &= 0\\
(x + 3)(5x - 2) &= 0
\end{align*}
$x = -3$ o $x = \frac{2}{5}$

\textbf{d)} $ac = -60$, buscar números que sumen $-4$ y multipliquen $-60$: son 6 y $-10$
\begin{align*}
4x^2 + 6x - 10x - 15 &= 0\\
2x(2x + 3) - 5(2x + 3) &= 0\\
(2x + 3)(2x - 5) &= 0
\end{align*}
$x = -\frac{3}{2}$ o $x = \frac{5}{2}$

\textbf{e)} $ac = -18$, buscar números que sumen 7 y multipliquen $-18$: son 9 y $-2$
\begin{align*}
6x^2 + 9x - 2x - 3 &= 0\\
3x(2x + 3) - 1(2x + 3) &= 0\\
(2x + 3)(3x - 1) &= 0
\end{align*}
$x = -\frac{3}{2}$ o $x = \frac{1}{3}$

\textbf{f)} $ac = 20$, buscar números que sumen $-9$ y multipliquen 20: son $-5$ y $-4$
\begin{align*}
2x^2 - 5x - 4x + 10 &= 0\\
x(2x - 5) - 2(2x - 5) &= 0\\
(2x - 5)(x - 2) &= 0
\end{align*}
$x = \frac{5}{2}$ o $x = 2$
\end{solucion}
\end{exercise}

\begin{exercise}
\problem Resuelva las siguientes ecuaciones cuadráticas \textbf{completando el cuadrado}:

\begin{exerciselist}
    \item $x^2 + 4x - 5 = 0$
    \item $x^2 - 6x + 8 = 0$
    \item $x^2 + 10x + 21 = 0$
    \item $x^2 - 2x - 8 = 0$
    \item $2x^2 + 12x + 10 = 0$
    \item $3x^2 - 6x - 9 = 0$
\end{exerciselist}

\begin{solucion}
\textbf{a)} $x^2 + 4x = 5 \Rightarrow x^2 + 4x + 4 = 5 + 4 \Rightarrow (x + 2)^2 = 9 \Rightarrow x + 2 = \pm 3$

$x = -2 + 3 = 1$ o $x = -2 - 3 = -5$

\textbf{b)} $x^2 - 6x = -8 \Rightarrow x^2 - 6x + 9 = -8 + 9 \Rightarrow (x - 3)^2 = 1 \Rightarrow x - 3 = \pm 1$

$x = 3 + 1 = 4$ o $x = 3 - 1 = 2$

\textbf{c)} $x^2 + 10x = -21 \Rightarrow x^2 + 10x + 25 = -21 + 25 \Rightarrow (x + 5)^2 = 4 \Rightarrow x + 5 = \pm 2$

$x = -5 + 2 = -3$ o $x = -5 - 2 = -7$

\textbf{d)} $x^2 - 2x = 8 \Rightarrow x^2 - 2x + 1 = 8 + 1 \Rightarrow (x - 1)^2 = 9 \Rightarrow x - 1 = \pm 3$

$x = 1 + 3 = 4$ o $x = 1 - 3 = -2$

\textbf{e)} Dividir por 2: $x^2 + 6x + 5 = 0$

$x^2 + 6x = -5 \Rightarrow x^2 + 6x + 9 = -5 + 9 \Rightarrow (x + 3)^2 = 4 \Rightarrow x + 3 = \pm 2$

$x = -3 + 2 = -1$ o $x = -3 - 2 = -5$

\textbf{f)} Dividir por 3: $x^2 - 2x - 3 = 0$

$x^2 - 2x = 3 \Rightarrow x^2 - 2x + 1 = 3 + 1 \Rightarrow (x - 1)^2 = 4 \Rightarrow x - 1 = \pm 2$

$x = 1 + 2 = 3$ o $x = 1 - 2 = -1$
\end{solucion}
\end{exercise}

\begin{exercise}
\problem Resuelva las siguientes ecuaciones cuadráticas usando la \textbf{fórmula cuadrática}:

\begin{exerciselist}
    \item $x^2 + 3x - 10 = 0$
    \item $x^2 - 7x + 12 = 0$
    \item $2x^2 + x - 6 = 0$
    \item $3x^2 - 5x - 2 = 0$
    \item $x^2 + 4x + 1 = 0$
    \item $2x^2 - 6x + 1 = 0$
    \item $x^2 - 2x - 5 = 0$
    \item $4x^2 + 4x - 3 = 0$
\end{exerciselist}

\begin{solucion}
\textbf{a)} $a = 1, b = 3, c = -10$
$$x = \frac{-3 \pm \sqrt{9 + 40}}{2} = \frac{-3 \pm \sqrt{49}}{2} = \frac{-3 \pm 7}{2}$$
$x = 2$ o $x = -5$

\textbf{b)} $a = 1, b = -7, c = 12$
$$x = \frac{7 \pm \sqrt{49 - 48}}{2} = \frac{7 \pm 1}{2}$$
$x = 4$ o $x = 3$

\textbf{c)} $a = 2, b = 1, c = -6$
$$x = \frac{-1 \pm \sqrt{1 + 48}}{4} = \frac{-1 \pm 7}{4}$$
$x = \frac{3}{2}$ o $x = -2$

\textbf{d)} $a = 3, b = -5, c = -2$
$$x = \frac{5 \pm \sqrt{25 + 24}}{6} = \frac{5 \pm 7}{6}$$
$x = 2$ o $x = -\frac{1}{3}$

\textbf{e)} $a = 1, b = 4, c = 1$
$$x = \frac{-4 \pm \sqrt{16 - 4}}{2} = \frac{-4 \pm \sqrt{12}}{2} = \frac{-4 \pm 2\sqrt{3}}{2} = -2 \pm \sqrt{3}$$

\textbf{f)} $a = 2, b = -6, c = 1$
$$x = \frac{6 \pm \sqrt{36 - 8}}{4} = \frac{6 \pm \sqrt{28}}{4} = \frac{6 \pm 2\sqrt{7}}{4} = \frac{3 \pm \sqrt{7}}{2}$$

\textbf{g)} $a = 1, b = -2, c = -5$
$$x = \frac{2 \pm \sqrt{4 + 20}}{2} = \frac{2 \pm \sqrt{24}}{2} = \frac{2 \pm 2\sqrt{6}}{2} = 1 \pm \sqrt{6}$$

\textbf{h)} $a = 4, b = 4, c = -3$
$$x = \frac{-4 \pm \sqrt{16 + 48}}{8} = \frac{-4 \pm 8}{8}$$
$x = \frac{1}{2}$ o $x = -\frac{3}{2}$
\end{solucion}
\end{exercise}

\begin{exercise}
\problem Calcule el discriminante de cada ecuación cuadrática y determine la naturaleza de sus raíces (sin resolverlas):

\begin{exerciselist}
    \item $x^2 + 5x + 6 = 0$
    \item $x^2 - 8x + 16 = 0$
    \item $x^2 + 3x + 5 = 0$
    \item $2x^2 - 7x + 3 = 0$
    \item $9x^2 + 6x + 1 = 0$
    \item $x^2 - x + 1 = 0$
\end{exerciselist}

\begin{solucion}
\textbf{a)} $D = 25 - 24 = 1 > 0$. Dos raíces reales distintas.

\textbf{b)} $D = 64 - 64 = 0$. Una raíz real doble.

\textbf{c)} $D = 9 - 20 = -11 < 0$. No hay raíces reales.

\textbf{d)} $D = 49 - 24 = 25 > 0$. Dos raíces reales distintas.

\textbf{e)} $D = 36 - 36 = 0$. Una raíz real doble.

\textbf{f)} $D = 1 - 4 = -3 < 0$. No hay raíces reales.
\end{solucion}
\end{exercise}

\begin{exercise}
\problem Resuelva las siguientes ecuaciones que requieren primero escribirlas en forma estándar:

\begin{exerciselist}
    \item $x(x + 5) = -6$
    \item $2x(x - 3) = 8$
    \item $(x + 2)(x - 3) = 6$
    \item $x^2 = 4(x - 3)$
    \item $(x + 1)^2 = 9$
    \item $(2x - 1)^2 = 25$
\end{exerciselist}

\begin{solucion}
\textbf{a)} $x^2 + 5x + 6 = 0 \Rightarrow (x + 2)(x + 3) = 0$

$x = -2$ o $x = -3$

\textbf{b)} $2x^2 - 6x = 8 \Rightarrow x^2 - 3x - 4 = 0 \Rightarrow (x - 4)(x + 1) = 0$

$x = 4$ o $x = -1$

\textbf{c)} $x^2 - x - 6 = 6 \Rightarrow x^2 - x - 12 = 0 \Rightarrow (x - 4)(x + 3) = 0$

$x = 4$ o $x = -3$

\textbf{d)} $x^2 = 4x - 12 \Rightarrow x^2 - 4x + 12 = 0$

$D = 16 - 48 = -32 < 0$. No hay soluciones reales.

\textbf{e)} $x + 1 = \pm 3$

$x = 2$ o $x = -4$

\textbf{f)} $2x - 1 = \pm 5$

$2x = 1 + 5 = 6 \Rightarrow x = 3$ o $2x = 1 - 5 = -4 \Rightarrow x = -2$
\end{solucion}
\end{exercise}

\begin{exercise}
\problem \textbf{Problemas de aplicación:}

\problem Un rectángulo tiene un largo que es 4 cm mayor que su ancho. Si el área del rectángulo es 60 cm$^2$, encuentre las dimensiones del rectángulo.

\begin{solucion}
Sea $x$ = ancho. Entonces $x + 4$ = largo.

Área: $x(x + 4) = 60$
$$x^2 + 4x = 60$$
$$x^2 + 4x - 60 = 0$$
$$(x + 10)(x - 6) = 0$$

$x = 6$ (descartamos $x = -10$ por ser negativo)

\textbf{Respuesta:} Ancho = 6 cm, Largo = 10 cm
\end{solucion}

\problem La suma de dos números es 12 y su producto es 35. Encuentre los números.

\begin{solucion}
Sea $x$ = primer número. Entonces $12 - x$ = segundo número.

Producto: $x(12 - x) = 35$
$$12x - x^2 = 35$$
$$x^2 - 12x + 35 = 0$$
$$(x - 5)(x - 7) = 0$$

$x = 5$ o $x = 7$

\textbf{Respuesta:} Los números son 5 y 7
\end{solucion}

\problem Un número positivo es 3 menor que otro número positivo. Si la suma de sus cuadrados es 89, encuentre los números.

\begin{solucion}
Sea $x$ = número mayor. Entonces $x - 3$ = número menor.

$$x^2 + (x - 3)^2 = 89$$
$$x^2 + x^2 - 6x + 9 = 89$$
$$2x^2 - 6x - 80 = 0$$
$$x^2 - 3x - 40 = 0$$
$$(x - 8)(x + 5) = 0$$

$x = 8$ (descartamos $x = -5$ por requerir número positivo)

\textbf{Respuesta:} Los números son 8 y 5
\end{solucion}

\problem El perímetro de un rectángulo es 28 cm y su área es 48 cm$^2$. Encuentre las dimensiones del rectángulo.

\begin{solucion}
Sea $x$ = ancho y $y$ = largo.

Perímetro: $2x + 2y = 28 \Rightarrow x + y = 14 \Rightarrow y = 14 - x$

Área: $xy = 48$

Sustituyendo:
$$x(14 - x) = 48$$
$$14x - x^2 = 48$$
$$x^2 - 14x + 48 = 0$$
$$(x - 6)(x - 8) = 0$$

$x = 6$ o $x = 8$

\textbf{Respuesta:} Las dimensiones son 6 cm × 8 cm
\end{solucion}

\problem Un objeto se lanza verticalmente hacia arriba desde el suelo con una velocidad inicial de 48 pies/s. Su altura $h$ (en pies) después de $t$ segundos está dada por $h = -16t^2 + 48t$. ¿En qué momento alcanza una altura de 32 pies?

\begin{solucion}
Sustituimos $h = 32$:
$$32 = -16t^2 + 48t$$
$$-16t^2 + 48t - 32 = 0$$

Dividimos por $-16$:
$$t^2 - 3t + 2 = 0$$
$$(t - 1)(t - 2) = 0$$

\textbf{Respuesta:} En $t = 1$ segundo (subiendo) y $t = 2$ segundos (bajando)
\end{solucion}

\problem Un granjero tiene 100 metros de cerca para encerrar un área rectangular. Si el área encerrada es 600 m$^2$, encuentre las dimensiones del rectángulo.

\begin{solucion}
Sea $x$ = ancho. Perímetro: $2x + 2y = 100 \Rightarrow x + y = 50 \Rightarrow y = 50 - x$

Área: $x(50 - x) = 600$
$$50x - x^2 = 600$$
$$x^2 - 50x + 600 = 0$$

Usando la fórmula cuadrática:
$$x = \frac{50 \pm \sqrt{2500 - 2400}}{2} = \frac{50 \pm 10}{2}$$

$x = 30$ o $x = 20$

\textbf{Respuesta:} Las dimensiones son 20 m × 30 m
\end{solucion}
\end{exercise}

\begin{exercise}
\problem \textbf{Problemas desafiantes:}

\begin{exerciselist}
    \item Resuelva: $\frac{x}{x-2} + \frac{x-1}{x} = 2$

    \item Resuelva: $(x + 1)^2 = 3(x + 1)$

    \item Encuentre el valor de $k$ para que la ecuación $x^2 + kx + 9 = 0$ tenga una raíz doble.

    \item Encuentre el valor de $k$ para que la ecuación $kx^2 - 6x + 2 = 0$ tenga dos raíces reales distintas.
\end{exerciselist}

\begin{solucion}
\textbf{a)} Multiplicamos por $x(x-2)$:
$$x^2 + (x-1)(x-2) = 2x(x-2)$$
$$x^2 + x^2 - 3x + 2 = 2x^2 - 4x$$
$$2x^2 - 3x + 2 = 2x^2 - 4x$$
$$x = 2$$

Pero $x = 2$ hace el denominador cero, por tanto no hay solución válida.

\textbf{b)} $(x + 1)^2 - 3(x + 1) = 0$
$$(x + 1)[(x + 1) - 3] = 0$$
$$(x + 1)(x - 2) = 0$$

$x = -1$ o $x = 2$

\textbf{c)} Para una raíz doble, $D = 0$:
$$k^2 - 4(1)(9) = 0$$
$$k^2 = 36$$
$$k = \pm 6$$

\textbf{d)} Para dos raíces reales distintas, $D > 0$:
$$(-6)^2 - 4(k)(2) > 0$$
$$36 - 8k > 0$$
$$k < 4.5$$

También se requiere $k \neq 0$ (para que sea cuadrática).
\end{solucion}
\end{exercise}
