%========================================
% EXERCISES: Problemas Verbales
%========================================

\section{Ejercicios}

\begin{exercise}
\problem Traduzca las siguientes frases a expresiones algebraicas:

\begin{exerciselist}
    \item Cinco más que el triple de un número
    \item La diferencia entre un número y su cuadrado
    \item El cociente de un número y 8
    \item Dos números consecutivos pares
    \item Tres enteros consecutivos impares
    \item La mitad de la suma de dos números
\end{exerciselist}

\begin{solucion}
\textbf{a)} $3x + 5$

\textbf{b)} $x - x^2$ o $x^2 - x$ (dependiendo del orden)

\textbf{c)} $\dfrac{x}{8}$

\textbf{d)} $n, n+2$ (donde $n$ es par)

\textbf{e)} $n, n+2, n+4$ (donde $n$ es impar)

\textbf{f)} $\dfrac{x + y}{2}$
\end{solucion}
\end{exercise}

\begin{exercise}
\problem Resuelva los siguientes problemas de números:

\begin{exerciselist}
    \item Halle dos números cuya suma es 65 y su diferencia es 15.
    \item El producto de dos enteros consecutivos es 132. Halle los enteros.
    \item La suma de tres enteros consecutivos es 75. Halle los números.
    \item Un número es 4 más que otro número. La suma de sus cuadrados es 106. Halle los números.
\end{exerciselist}

\begin{solucion}
\textbf{a)} Los números son 40 y 25.

\textbf{b)} Los enteros son 11 y 12, o $-12$ y $-11$.

\textbf{c)} Los números son 24, 25, y 26.

\textbf{d)} Los números son 5 y 9, o $-9$ y $-5$.
\end{solucion}
\end{exercise}

\begin{exercise}
\problem Resuelva los siguientes problemas con edades:

\begin{exerciselist}
    \item La edad de Ana es el doble de la edad de su hermano. En 5 años, la suma de sus edades será 40. ¿Cuántos años tienen ahora?
    \item Pedro tiene 3 años más que María. Hace 5 años, la edad de Pedro era el doble de la edad de María. ¿Cuántos años tienen ahora?
\end{exerciselist}

\begin{solucion}
\textbf{a)} Ana tiene 20 años y su hermano tiene 10 años.

\textbf{b)} Pedro tiene 13 años y María tiene 10 años.
\end{solucion}
\end{exercise}

\begin{exercise}
\problem Resuelva los siguientes problemas de aplicación:

\begin{exerciselist}
    \item Un empleado gana \$12 por hora más un bono de \$50 por semana. ¿Cuántas horas debe trabajar para ganar \$530 en una semana?
    \item Una tienda ofrece un descuento del 15\% en todas las compras. Si después del descuento, un artículo cuesta \$68, ¿cuál era el precio original?
    \item Un tren viaja a 60 mph. ¿Cuánto tiempo tardará en recorrer 270 millas?
\end{exerciselist}

\begin{solucion}
\textbf{a)} Debe trabajar 40 horas.

\textbf{b)} El precio original era \$80.

\textbf{c)} Tardará 4.5 horas (o 4 horas y 30 minutos).
\end{solucion}
\end{exercise}

\begin{exercise}
\problem Resuelva los siguientes problemas de interés simple:

\begin{exerciselist}
    \item Se invierten \$5000 a un interés simple de 6\% anual. ¿Cuánto interés se gana en 3 años?
    \item ¿Qué cantidad debe invertirse a un interés simple de 4.5\% anual para ganar \$225 en 2 años?
    \item Se prestan \$8000 a un interés simple de 7.5\% anual. ¿Cuánto dinero se debe devolver después de 18 meses?
\end{exerciselist}

\begin{solucion}
\textbf{a)} El interés ganado es \$900.

\textbf{b)} Debe invertirse \$2500.

\textbf{c)} Se debe devolver \$8900.
\end{solucion}
\end{exercise}

\begin{exercise}
\problem Resuelva los siguientes problemas de interés compuesto:

\begin{exerciselist}
    \item Se invierten \$4000 a un interés de 5\% anual compuesto semestralmente. ¿Cuál es la cantidad acumulada después de 10 años?
    \item ¿Cuánto dinero debe invertirse a un interés de 6\% anual compuesto mensualmente para tener \$10,000 después de 5 años?
    \item Compare el interés ganado en \$2000 invertidos al 8\% anual por 3 años usando: (i) interés simple, (ii) interés compuesto trimestral.
\end{exerciselist}

\begin{solucion}
\textbf{a)} La cantidad acumulada es aproximadamente \$6533.28.

\textbf{b)} Debe invertirse aproximadamente \$7407.94.

\textbf{c)} Interés simple: \$480. Interés compuesto trimestral: aproximadamente \$537.49. Diferencia: \$57.49 más con interés compuesto.
\end{solucion}
\end{exercise}
