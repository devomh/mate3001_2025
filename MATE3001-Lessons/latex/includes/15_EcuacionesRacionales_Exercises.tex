%========================================
% EXERCISES: Ecuaciones Racionales
%========================================

\subsection{Ejercicios}

%========================================
% QUICK PRACTICE
%========================================
\subsection*{Práctica Rápida}

\begin{exercise}
Resuelva las siguientes ecuaciones racionales. Indique las restricciones y verifique sus respuestas.

\problem \(\displaystyle \frac{1}{x+1} + \frac{1}{x-1} = 1\)

\begin{solucion}
\textbf{Restricciones:} \(x \neq \pm 1\)

\textbf{MCD:} \((x+1)(x-1)\)

Multiplicando:
\[(x-1) + (x+1) = (x+1)(x-1)\]
\[2x = x^2 - 1\]
\[0 = x^2 - 2x - 1\]

Por la fórmula cuadrática:
\[x = \frac{2 \pm \sqrt{4 + 4}}{2} = \frac{2 \pm \sqrt{8}}{2} = \frac{2 \pm 2\sqrt{2}}{2} = 1 \pm \sqrt{2}\]

\textbf{Respuesta:} \(x = 1 + \sqrt{2}\) o \(x = 1 - \sqrt{2}\) (ambas válidas, \(x \neq \pm 1\))
\end{solucion}

\problem \(\displaystyle \frac{x}{x-2} = 2\)

\begin{solucion}
\textbf{Restricción:} \(x \neq 2\)

\textbf{MCD:} \(x-2\)

Multiplicando:
\[x = 2(x-2)\]
\[x = 2x - 4\]
\[4 = x\]

Verificación: \(\dfrac{4}{4-2} = \dfrac{4}{2} = 2\) \(\checkmark\)

\textbf{Respuesta:} \(x = 4\)
\end{solucion}

\problem \(\displaystyle \frac{x+1}{x} = \frac{3}{x-2}\)

\begin{solucion}
\textbf{Restricciones:} \(x \neq 0\) y \(x \neq 2\)

\textbf{MCD:} \(x(x-2)\)

Multiplicando:
\[(x+1)(x-2) = 3x\]
\[x^2 - 2x + x - 2 = 3x\]
\[x^2 - x - 2 = 3x\]
\[x^2 - 4x - 2 = 0\]

Por la fórmula cuadrática:
\[x = \frac{4 \pm \sqrt{16 + 8}}{2} = \frac{4 \pm \sqrt{24}}{2} = \frac{4 \pm 2\sqrt{6}}{2} = 2 \pm \sqrt{6}\]

\textbf{Respuesta:} \(x = 2 + \sqrt{6}\) o \(x = 2 - \sqrt{6}\) (ambas válidas)
\end{solucion}

\problem \(\displaystyle \frac{1}{x-3} - \frac{1}{x+3} = \frac{1}{3}\)

\begin{solucion}
\textbf{Restricciones:} \(x \neq \pm 3\)

\textbf{MCD:} \(3(x-3)(x+3)\)

Multiplicando:
\[3(x+3) - 3(x-3) = (x-3)(x+3)\]
\[3x + 9 - 3x + 9 = x^2 - 9\]
\[18 = x^2 - 9\]
\[27 = x^2\]
\[x = \pm\sqrt{27} = \pm 3\sqrt{3}\]

\textbf{Respuesta:} \(x = 3\sqrt{3}\) o \(x = -3\sqrt{3}\) (ambas válidas, \(x \neq \pm 3\))
\end{solucion}
\end{exercise}

%========================================
% PROPOSED EXERCISES
%========================================
\subsection*{Ejercicios Propuestos}

\begin{exercise}
Resuelva las siguientes ecuaciones racionales. Para cada una recuerde listar restricciones, identificar MCD, resolver la ecuación y verificar soluciones.
\problem \(\displaystyle \frac{3}{x-2} + \frac{4}{x+3} = 1\)

\begin{solucion}
\textbf{Restricciones:} \(x \neq 2\) y \(x \neq -3\)

\textbf{MCD:} \((x-2)(x+3)\)

\[3(x+3) + 4(x-2) = (x-2)(x+3)\]
\[3x + 9 + 4x - 8 = x^2 + x - 6\]
\[7x + 1 = x^2 + x - 6\]
\[0 = x^2 - 6x - 7\]
\[0 = (x-7)(x+1)\]

\textbf{Respuesta:} \(x = 7\) o \(x = -1\)
\end{solucion}

\vspace{0.25cm}
\problem \(\displaystyle \frac{x}{x^2-4} - \frac{2}{x+2} = 0\)

\begin{solucion}
\textbf{Restricciones:} \(x \neq \pm 2\) (ya que \(x^2 - 4 = (x-2)(x+2)\))

\textbf{MCD:} \((x-2)(x+2)\)

\[\frac{x}{(x-2)(x+2)} - \frac{2}{x+2} = 0\]
\[x - 2(x-2) = 0\]
\[x - 2x + 4 = 0\]
\[-x + 4 = 0\]
\[x = 4\]

\textbf{Respuesta:} \(x = 4\)
\end{solucion}

\vspace{0.25cm}
\problem \(\displaystyle \frac{1}{x} + \frac{2}{x^2} - \frac{3}{x^3} = 0\)

\begin{solucion}
\textbf{Restricción:} \(x \neq 0\)

\textbf{MCD:} \(x^3\)

\[x^2 + 2x - 3 = 0\]
\[(x+3)(x-1) = 0\]

\textbf{Respuesta:} \(x = -3\) o \(x = 1\)
\end{solucion}

\vspace{0.25cm}
\problem \(\displaystyle \frac{2x}{x^2-1} + \frac{3}{x-1} - \frac{1}{x+1} = 0\)

\begin{solucion}
\textbf{Restricciones:} \(x \neq \pm 1\) (ya que \(x^2 - 1 = (x-1)(x+1)\))

\textbf{MCD:} \((x-1)(x+1)\)

\[\frac{2x}{(x-1)(x+1)} + \frac{3}{x-1} - \frac{1}{x+1} = 0\]
\[2x + 3(x+1) - (x-1) = 0\]
\[2x + 3x + 3 - x + 1 = 0\]
\[4x + 4 = 0\]
\[x = -1\]

¡SOLUCIÓN EXTRAÑA! La solución \(x = -1\) viola la restricción.

\textbf{Respuesta:} No hay solución.
\end{solucion}

\vspace{0.25cm}
\problem \(\displaystyle \frac{x+1}{x^2+5x+6} - \frac{x-1}{x^2+4x+3} = \frac{1}{x+3}\)

\begin{solucion}
\textbf{Factorización:}
\begin{itemize}
    \item \(x^2 + 5x + 6 = (x+2)(x+3)\)
    \item \(x^2 + 4x + 3 = (x+1)(x+3)\)
\end{itemize}

\textbf{Restricciones:} \(x \neq -1, -2, -3\)

\textbf{MCD:} \((x+1)(x+2)(x+3)\)

\[\frac{x+1}{(x+2)(x+3)} - \frac{x-1}{(x+1)(x+3)} = \frac{1}{x+3}\]

Multiplicando por el MCD:
\[(x+1)^2 - (x-1)(x+2) = (x+1)(x+2)\]
\[x^2 + 2x + 1 - (x^2 + x - 2) = x^2 + 3x + 2\]
\[x^2 + 2x + 1 - x^2 - x + 2 = x^2 + 3x + 2\]
\[x + 3 = x^2 + 3x + 2\]
\[0 = x^2 + 2x - 1\]

Por la fórmula cuadrática:
\[x = \frac{-2 \pm \sqrt{4 + 4}}{2} = \frac{-2 \pm 2\sqrt{2}}{2} = -1 \pm \sqrt{2}\]

Verificar que \(x = -1 + \sqrt{2} \approx 0.41\) y \(x = -1 - \sqrt{2} \approx -2.41\) no violan las restricciones.

\textbf{Respuesta:} \(x = -1 + \sqrt{2}\) o \(x = -1 - \sqrt{2}\)
\end{solucion}
\end{exercise}

