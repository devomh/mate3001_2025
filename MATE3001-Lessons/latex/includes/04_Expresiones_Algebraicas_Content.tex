%========================================
% LESSON CONTENT: Expresiones Algebraicas y Polinomios
%========================================

\lesson{Expresiones Algebraicas y Polinomios}

\subsectiontitle{Expresiones Algebraicas}

\begin{definition}
Una \textbf{expresión algebraica} es una combinación de variables, números, y operaciones aritméticas (suma, resta, multiplicación, división y exponenciación).
\end{definition}

\begin{example}
Ejemplos de expresiones algebraicas:
\begin{itemize}
\item $2x + 3y - 5$
\item $\frac{x^2 - 4}{x + 2}$
\item $3a^2b - 7ab + 2b^2$
\item $\sqrt{x + 1} + 2x^3$
\end{itemize}
\end{example}

\subsectiontitle{Componentes de una Expresión Algebraica}

% Visual breakdown of algebraic expression components
\begin{center}
\begin{tikzpicture}[scale=1.1]
    % Main expression
    \node at (0,3) {\Large $3x^2 - 5xy + 7y^2 - 2$};
    
    % Term indicators with different colors
    \draw[red, thick, -{Stealth}] (-2.2,2.5) to[bend right=20] (-5.2,2);
    \node[red] at (-5.2,1.5) {\textbf{Término 1:} $3x^2$};
    
    \draw[blue, thick, -{Stealth}] (-0.7,2.5) to[bend right=20] (-0.7,2);
    \node[blue] at (-0.7,1.5) {\textbf{Término 2:} $-5xy$};
    
    \draw[green!70!black, thick, -{Stealth}] (0.8,2.5) to[bend left=20] (0.8,2);
    \node[green!70!black] at (0.8,1.5) {\textbf{Término 3:} $7y^2$};
    
    \draw[purple, thick, -{Stealth}] (2,2.5) to[bend left=20] (5,2);
    \node[purple] at (5,1.5) {\textbf{Término 4:} $-2$};
    
    % Component breakdown
    \node at (-3,0.5) {\textbf{Coeficientes:} $3, -5, 7, -2$};
    \node at (0,0.5) {\textbf{Variables:} $x, y$};
    \node at (3,0.5) {\textbf{Exponentes:} $2, 1, 2, 0$};
\end{tikzpicture}
\end{center}

\begin{definition}
\textbf{Términos:} Las partes de una expresión algebraica separadas por los signos $+$ o $-$.

\textbf{Coeficientes:} Los números que multiplican a las variables.

\textbf{Variables:} Las letras que representan números desconocidos o que pueden variar.

\textbf{Exponentes:} Los números que indican cuántas veces se multiplica la variable por sí misma.
\end{definition}

\subsectiontitle{Dominio de una Expresión Algebraica}

\begin{definition}
El \textbf{dominio} de una expresión algebraica es el conjunto de todos los valores reales que pueden asignarse a las variables para que la expresión esté definida.
\end{definition}

\textbf{Restricciones comunes:}
\begin{enumerate}
\item \textbf{División por cero:} El denominador no puede ser cero
\item \textbf{Raíces pares de números negativos:} No están definidas en los reales
\item \textbf{Logaritmos de números negativos o cero:} No están definidos en los reales
\end{enumerate}

\begin{example}
\textbf{Ejemplo 1:} $f(x) = \frac{2x + 1}{x - 3}$

El denominador $x - 3 = 0$ cuando $x = 3$

\textbf{Dominio:} $\{x \in \mathbb{R} \mid x \neq 3\}$ o $(-\infty, 3) \cup (3, \infty)$

\textbf{Ejemplo 2:} $g(x) = \frac{x^2 - 4}{x^2 - 9}$

El denominador $x^2 - 9 = 0$ cuando $x^2 = 9$, es decir, $x = \pm 3$

\textbf{Dominio:} $\{x \in \mathbb{R} \mid x \neq 3, x \neq -3\}$ o $(-\infty, -3) \cup (-3, 3) \cup (3, \infty)$

\textbf{Ejemplo 3:} $h(x) = \sqrt{x - 2}$

Para que la raíz cuadrada esté definida: $x - 2 \geq 0$, por tanto $x \geq 2$

\textbf{Dominio:} $\{x \in \mathbb{R} \mid x \geq 2\}$ o $[2, \infty)$
\end{example}

\subsectiontitle{Polinomios}

\begin{definition}
Un \textbf{polinomio} en la variable $x$ es una expresión algebraica de la forma:
$$P(x) = a_n x^n + a_{n-1} x^{n-1} + \cdots + a_1 x + a_0$$
donde:
\begin{itemize}
\item $n$ es un entero no negativo
\item $a_n, a_{n-1}, \ldots, a_1, a_0$ son números reales llamados \textbf{coeficientes}
\item $a_n \neq 0$ (coeficiente principal)
\end{itemize}
\end{definition}

\textbf{Terminología de polinomios:}
\begin{itemize}
\item \textbf{Grado del polinomio:} El mayor exponente de la variable
\item \textbf{Término principal:} El término con el mayor exponente
\item \textbf{Coeficiente principal:} El coeficiente del término principal
\item \textbf{Término constante:} El término sin variable (exponente cero)
\end{itemize}

\begin{example}
\textbf{Ejemplo 1:} $P(x) = 3x^4 - 2x^3 + x - 7$
\begin{itemize}
\item Grado: 4
\item Término principal: $3x^4$
\item Coeficiente principal: 3
\item Término constante: $-7$
\end{itemize}

\textbf{Ejemplo 2:} $Q(x) = -5x^2 + 8$
\begin{itemize}
\item Grado: 2
\item Término principal: $-5x^2$
\item Coeficiente principal: $-5$
\item Término constante: 8
\end{itemize}
\end{example}

\subsectiontitle{Clasificación de Polinomios}

% Polynomial degree visualization (temporarily disabled for compilation)
% \begin{center}
% \begin{tikzpicture}[scale=0.7]
%     % Linear function
%     \begin{scope}[xshift=0cm]
%         \draw[->] (-2,0) -- (2,0) node[right] {$x$};
%         \draw[->] (0,-1) -- (0,3) node[above] {$y$};
%         \draw[domain=-1.5:1.5, smooth, variable=\x, blue, thick] plot (\x, {\x + 1});
%         \node at (0,-1.8) {\textbf{Lineal (grado 1)}};
%         \node at (0,-2.2) {$y = x + 1$};
%     \end{scope}
%     
%     % Quadratic function
%     \begin{scope}[xshift=5cm]
%         \draw[->] (-2,0) -- (2,0) node[right] {$x$};
%         \draw[->] (0,-1) -- (0,3) node[above] {$y$};
%         \draw[domain=-1.5:1.5, smooth, variable=\x, red, thick] plot (\x, {\x*\x + 0.5});
%         \node at (0,-1.8) {\textbf{Cuadrático (grado 2)}};
%         \node at (0,-2.2) {$y = x^2$};
%     \end{scope}
%     
%     % Cubic function
%     \begin{scope}[xshift=10cm]
%         \draw[->] (-2,0) -- (2,0) node[right] {$x$};
%         \draw[->] (0,-2) -- (0,3) node[above] {$y$};
%         \draw[domain=-1.3:1.3, smooth, variable=\x, green!70!black, thick] plot (\x, {\x*\x*\x + 1});
%         \node at (0,-2.8) {\textbf{Cúbico (grado 3)}};
%         \node at (0,-3.2) {$y = x^3$};
%     \end{scope}
% \end{tikzpicture}
% \end{center}

\textbf{Comparación visual de diferentes grados de polinomios:}
\begin{itemize}
    \item \textbf{Grado 1 (Lineal):} $y = x + 1$ - Forma una línea recta
    \item \textbf{Grado 2 (Cuadrático):} $y = x^2$ - Forma una parábola
    \item \textbf{Grado 3 (Cúbico):} $y = x^3$ - Forma una curva en S
\end{itemize}

\textbf{Clasificación por grado:}

\begin{center}
\begin{tabular}{|c|c|c|c|}
\hline
\textbf{Grado} & \textbf{Nombre} & \textbf{Forma General} & \textbf{Ejemplo} \\
\hline
0 & Constante & $a_0$ & $5$ \\
\hline
1 & Lineal & $a_1x + a_0$ & $2x - 3$ \\
\hline
2 & Cuadrático & $a_2x^2 + a_1x + a_0$ & $x^2 - 4x + 4$ \\
\hline
3 & Cúbico & $a_3x^3 + a_2x^2 + a_1x + a_0$ & $2x^3 + x^2 - x + 1$ \\
\hline
4 & Cuártico & $a_4x^4 + a_3x^3 + a_2x^2 + a_1x + a_0$ & $x^4 - 16$ \\
\hline
\end{tabular}
\end{center}

\textbf{Clasificación por número de términos:}

\begin{center}
\begin{tabular}{|c|c|c|}
\hline
\textbf{Número de Términos} & \textbf{Nombre} & \textbf{Ejemplo} \\
\hline
1 & Monomio & $3x^2$ \\
\hline
2 & Binomio & $2x + 5$ \\
\hline
3 & Trinomio & $x^2 - 3x + 2$ \\
\hline
Más de 3 & Polinomio & $2x^4 - x^3 + 3x^2 - x + 1$ \\
\hline
\end{tabular}
\end{center}

\subsectiontitle{Operaciones con Polinomios}

\textbf{Suma y Resta de Polinomios:}

Para sumar o restar polinomios, se combinan los \textbf{términos semejantes} (términos con la misma variable elevada a la misma potencia).

\begin{example}
\begin{align}
P(x) &= 3x^3 - 2x^2 + x - 4 \\
Q(x) &= x^3 + 5x^2 - 3x + 1
\end{align}

\textbf{Suma:}
$$P(x) + Q(x) = (3x^3 + x^3) + (-2x^2 + 5x^2) + (x - 3x) + (-4 + 1)$$
$$= 4x^3 + 3x^2 - 2x - 3$$

\textbf{Resta:}
$$P(x) - Q(x) = (3x^3 - x^3) + (-2x^2 - 5x^2) + (x - (-3x)) + (-4 - 1)$$
$$= 2x^3 - 7x^2 + 4x - 5$$
\end{example}

\subsectiontitle{Leyes de los Exponentes}

Las leyes de los exponentes son fundamentales para el trabajo con polinomios:

\begin{theorem}
Para $a, b \neq 0$ y $m, n$ enteros:

\begin{center}
\begin{tabular}{|c|c|c|}
\hline
\textbf{Ley} & \textbf{Fórmula} & \textbf{Ejemplo} \\
\hline
Producto & $a^m \cdot a^n = a^{m+n}$ & $x^3 \cdot x^2 = x^5$ \\
\hline
Cociente & $\frac{a^m}{a^n} = a^{m-n}$ & $\frac{x^5}{x^2} = x^3$ \\
\hline
Potencia de potencia & $(a^m)^n = a^{mn}$ & $(x^2)^3 = x^6$ \\
\hline
Potencia de producto & $(ab)^n = a^n b^n$ & $(2x)^3 = 8x^3$ \\
\hline
Potencia de cociente & $\left(\frac{a}{b}\right)^n = \frac{a^n}{b^n}$ & $\left(\frac{x}{2}\right)^2 = \frac{x^2}{4}$ \\
\hline
Exponente cero & $a^0 = 1$ & $5^0 = 1, x^0 = 1$ \\
\hline
Exponente negativo & $a^{-n} = \frac{1}{a^n}$ & $x^{-2} = \frac{1}{x^2}$ \\
\hline
\end{tabular}
\end{center}
\end{theorem}

\begin{example}
\textbf{Ejemplo 1:} Simplificar $(2x^3y^2)^4$
$$= 2^4 \cdot (x^3)^4 \cdot (y^2)^4 = 16x^{12}y^8$$

\textbf{Ejemplo 2:} Simplificar $\frac{15x^7y^3}{3x^2y^5}$
$$= \frac{15}{3} \cdot \frac{x^7}{x^2} \cdot \frac{y^3}{y^5} = 5x^5y^{-2} = \frac{5x^5}{y^2}$$

\textbf{Ejemplo 3:} Simplificar $(x^{-2}y^3)^{-1}$
$$= x^{(-2)(-1)} \cdot y^{3(-1)} = x^2y^{-3} = \frac{x^2}{y^3}$$
\end{example}

\subsectiontitle{Multiplicación de Polinomios}

\textbf{Monomio por Monomio:}
Para multiplicar monomios, se multiplican los coeficientes y se suman los exponentes de variables iguales.

\begin{example}
$(3x^2y)(4x^3y^2) = 12x^5y^3$
\end{example}

\textbf{Monomio por Polinomio:}
Se aplica la propiedad distributiva: cada término del polinomio se multiplica por el monomio.

\begin{example}
$$2x^2(3x^3 - 4x + 5) = 6x^5 - 8x^3 + 10x^2$$
\end{example}

\textbf{Binomio por Binomio:}
Se multiplica cada término del primer binomio por cada término del segundo binomio.

\begin{example}
$$(2x + 3)(x - 4) = 2x \cdot x + 2x \cdot (-4) + 3 \cdot x + 3 \cdot (-4)$$
$$= 2x^2 - 8x + 3x - 12 = 2x^2 - 5x - 12$$
\end{example}

\subsectiontitle{Método FOIL}

% FOIL method visual guide
\begin{center}
\begin{tikzpicture}[scale=1.2]
    % The expression (a+b)(c+d)
    \node at (0,2) {\huge $(a + b)(c + d)$};
    
    % FOIL arrows and labels
    \draw[red, thick, -{Stealth}] (-0.8,1.7) to[bend left=30] (-0.3,1.7);
    \node[red] at (-0.55, 1.3) {\textbf{F}irst: $ac$};
    
    \draw[blue, thick, -{Stealth}] (-0.8,1.7) to[bend left=50] (0.3,1.7);
    \node[blue] at (-0.2, 0.8) {\textbf{O}uter: $ad$};
    
    \draw[green!70!black, thick, -{Stealth}] (-0.3,1.7) to[bend right=50] (0.3,1.7);
    \node[green!70!black] at (0.2, 0.8) {\textbf{I}nner: $bc$};
    
    \draw[purple, thick, -{Stealth}] (-0.3,1.7) to[bend right=30] (0.3,1.7);
    \node[purple] at (0.55, 1.3) {\textbf{L}ast: $bd$};
    
    % Result
    \node at (0,0) {\textbf{Resultado:} $ac + ad + bc + bd$};
\end{tikzpicture}
\end{center}

Para $(a + b)(c + d)$:
\begin{itemize}
\item \textbf{F}irst: $a \cdot c$
\item \textbf{O}uter: $a \cdot d$  
\item \textbf{I}nner: $b \cdot c$
\item \textbf{L}ast: $b \cdot d$
\end{itemize}

\textbf{Resultado:} $ac + ad + bc + bd$

\subsectiontitle{Productos Notables}

\begin{theorem}
\textbf{Cuadrado de un Binomio:}
$$(a + b)^2 = a^2 + 2ab + b^2$$
$$(a - b)^2 = a^2 - 2ab + b^2$$
\end{theorem}

\begin{example}
\begin{itemize}
\item $(x + 3)^2 = x^2 + 6x + 9$
\item $(2x - 5)^2 = 4x^2 - 20x + 25$
\end{itemize}
\end{example}

\begin{theorem}
\textbf{Diferencia de Cuadrados:}
$$(a + b)(a - b) = a^2 - b^2$$
\end{theorem}

\begin{example}
\begin{itemize}
\item $(x + 4)(x - 4) = x^2 - 16$
\item $(3x + 2y)(3x - 2y) = 9x^2 - 4y^2$
\end{itemize}
\end{example}

\begin{theorem}
\textbf{Cubo de un Binomio:}
$$(a + b)^3 = a^3 + 3a^2b + 3ab^2 + b^3$$
$$(a - b)^3 = a^3 - 3a^2b + 3ab^2 - b^3$$
\end{theorem}

\begin{example}
$(x + 2)^3 = x^3 + 6x^2 + 12x + 8$
\end{example}

\begin{theorem}
\textbf{Suma y Diferencia de Cubos:}
$$a^3 + b^3 = (a + b)(a^2 - ab + b^2)$$
$$a^3 - b^3 = (a - b)(a^2 + ab + b^2)$$
\end{theorem}

\begin{example}
\begin{itemize}
\item $x^3 + 8 = (x + 2)(x^2 - 2x + 4)$
\item $8x^3 - 27 = (2x - 3)(4x^2 + 6x + 9)$
\end{itemize}
\end{example}

\subsectiontitle{Factorización de Polinomios}

La factorización es el proceso inverso de la multiplicación. Expresamos un polinomio como producto de sus factores.

% Factorization process flowchart
\begin{center}
\begin{tikzpicture}[
    box/.style={rectangle, draw, fill=blue!10, text width=3cm, text centered},
    decision/.style={diamond, draw, fill=yellow!10, text width=2cm, text centered},
    arrow/.style={-{Stealth[length=3mm]}, thick}
]
    \node[box] (start) at (0,0) {Expresión a factorizar};
    \node[decision] (common) at (0,-2) {¿Hay factor común?};
    \node[box] (extract) at (-3,-4) {Extraer factor común};
    \node[decision] (terms) at (0,-4) {¿Cuántos términos?};
    \node[box] (two) at (-2,-6) {2 términos:\\ Diferencia cuadrados};
    \node[box] (three) at (0,-6) {3 términos:\\ Trinomio};
    \node[box] (four) at (2,-6) {4+ términos:\\ Agrupación};
    
    \draw[arrow] (start) -- (common);
    \draw[arrow] (common) -- node[left] {Sí} (extract);
    \draw[arrow] (common) -- node[right] {No} (terms);
    \draw[arrow] (terms) -- (two);
    \draw[arrow] (terms) -- (three);
    \draw[arrow] (terms) -- (four);
\end{tikzpicture}
\end{center}

\textbf{Técnicas principales de factorización:}

\begin{enumerate}
\item \textbf{Factor común:} Se extrae el mayor factor común de todos los términos.
\item \textbf{Agrupación:} Se agrupan términos para factorizar por partes.
\item \textbf{Trinomio cuadrático:} Para $ax^2 + bx + c$, se buscan dos números que multiplicados den $ac$ y sumados den $b$.
\item \textbf{Diferencia de cuadrados:} $a^2 - b^2 = (a + b)(a - b)$
\item \textbf{Trinomio cuadrático perfecto:} $a^2 \pm 2ab + b^2 = (a \pm b)^2$
\item \textbf{Suma y diferencia de cubos:} Usar las fórmulas correspondientes.
\end{enumerate}

\begin{example}
\textbf{Factor común:}
$$6x^3 + 9x^2 - 12x = 3x(2x^2 + 3x - 4)$$

\textbf{Agrupación:}
$$ax + ay + bx + by = a(x + y) + b(x + y) = (a + b)(x + y)$$

\textbf{Trinomio cuadrático:}
$$x^2 + 5x + 6 = (x + 2)(x + 3)$$

\textbf{Diferencia de cuadrados:}
$$4x^2 - 9 = (2x + 3)(2x - 3)$$

\textbf{Ejemplo completo:}
Factorizar: $2x^3 - 8x$

1. \textbf{Factor común:} $2x(x^2 - 4)$
2. \textbf{Diferencia de cuadrados:} $2x(x + 2)(x - 2)$

\textbf{Verificación:} $2x(x + 2)(x - 2) = 2x(x^2 - 4) = 2x^3 - 8x$ $\checkmark$
\end{example}

\subsectiontitle{Evaluación de Polinomios}

\begin{definition}
Para evaluar un polinomio $P(x)$ en $x = a$, se sustituye $x$ por $a$.
\end{definition}

\begin{example}
Si $P(x) = 2x^3 - 3x^2 + x - 5$, entonces:
$$P(2) = 2(2)^3 - 3(2)^2 + 2 - 5 = 16 - 12 + 2 - 5 = 1$$
\end{example}

\begin{theorem}
\textbf{Teorema del Residuo:} Si un polinomio $P(x)$ se divide por $(x - a)$, entonces el residuo es $P(a)$.

\textbf{Teorema del Factor:} $(x - a)$ es un factor de $P(x)$ si y solo si $P(a) = 0$.
\end{theorem}

Estos teoremas son útiles para factorizar polinomios de grado alto y encontrar sus raíces.