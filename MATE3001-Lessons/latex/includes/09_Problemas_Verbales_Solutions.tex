%========================================
% DETAILED SOLUTIONS: Problemas Verbales
%========================================

\subsection*{Soluciones Detalladas}

%========================================
% EJERCICIO 1: Traducción de Frases
%========================================

\subsubsection*{Ejercicio 1}
\textbf{Problema:} Traduzca las siguientes frases a expresiones algebraicas.

\medskip

\textbf{a)} Cinco más que el triple de un número

\textbf{Solución Detallada:}
\begin{itemize}
    \item Sea $x$ el número desconocido
    \item El triple de un número: $3x$
    \item Cinco más que esa cantidad: $3x + 5$
\end{itemize}

\textbf{Respuesta:} $3x + 5$

\medskip

\textbf{b)} La diferencia entre un número y su cuadrado

\textbf{Solución Detallada:}
\begin{itemize}
    \item Sea $x$ el número desconocido
    \item El cuadrado del número: $x^2$
    \item La diferencia puede interpretarse de dos formas:
    \begin{itemize}
        \item Un número menos su cuadrado: $x - x^2$
        \item Su cuadrado menos el número: $x^2 - x$
    \end{itemize}
\end{itemize}

\textbf{Respuesta:} $x - x^2$ o $x^2 - x$ (ambas son válidas)

\medskip

\textbf{c)} El cociente de un número y 8

\textbf{Solución Detallada:}
\begin{itemize}
    \item Sea $x$ el número desconocido
    \item El cociente indica división
    \item Un número dividido por 8: $\dfrac{x}{8}$
\end{itemize}

\textbf{Respuesta:} $\dfrac{x}{8}$

\medskip

\textbf{d)} Dos números consecutivos pares

\textbf{Solución Detallada:}
\begin{itemize}
    \item Los números pares difieren por 2 unidades
    \item Si $n$ es un número par, el siguiente número par es $n + 2$
    \item Ejemplos: 2 y 4, o 10 y 12, o $-6$ y $-4$
\end{itemize}

\textbf{Respuesta:} $n, n+2$ (donde $n$ es par)

\medskip

\textbf{e)} Tres enteros consecutivos impares

\textbf{Solución Detallada:}
\begin{itemize}
    \item Los números impares difieren por 2 unidades
    \item Si $n$ es un número impar, los siguientes impares son $n+2$ y $n+4$
    \item Ejemplos: 1, 3, 5 o 11, 13, 15
\end{itemize}

\textbf{Respuesta:} $n, n+2, n+4$ (donde $n$ es impar)

\medskip

\textbf{f)} La mitad de la suma de dos números

\textbf{Solución Detallada:}
\begin{itemize}
    \item Sean $x$ y $y$ los dos números
    \item La suma de los dos números: $x + y$
    \item La mitad de esa suma: $\dfrac{x + y}{2}$
\end{itemize}

\textbf{Respuesta:} $\dfrac{x + y}{2}$

\hrulefill

%========================================
% EJERCICIO 2: Problemas de Números
%========================================

\subsubsection*{Ejercicio 2}
\textbf{Problema:} Resuelva los siguientes problemas de números.

\medskip

\textbf{a)} Halle dos números cuya suma es 65 y su diferencia es 15.

\textbf{Solución Detallada:}
\begin{itemize}
    \item Sea $x$ el número mayor y $y$ el número menor
    \item Sistema de ecuaciones:
    \begin{align*}
    x + y &= 65 \\
    x - y &= 15
    \end{align*}
    \item Sumando las dos ecuaciones:
    \begin{align*}
    (x + y) + (x - y) &= 65 + 15 \\
    2x &= 80 \\
    x &= 40
    \end{align*}
    \item Sustituyendo en la primera ecuación:
    \begin{align*}
    40 + y &= 65 \\
    y &= 25
    \end{align*}
    \item \textbf{Verificación:}
    \begin{itemize}
        \item Suma: $40 + 25 = 65$ \checkmark
        \item Diferencia: $40 - 25 = 15$ \checkmark
    \end{itemize}
\end{itemize}

\textbf{Respuesta:} Los números son 40 y 25.

\medskip

\textbf{b)} El producto de dos enteros consecutivos es 132. Halle los enteros.

\textbf{Solución Detallada:}
\begin{itemize}
    \item Sean $n$ y $n+1$ los dos enteros consecutivos
    \item Ecuación: $n(n+1) = 132$
    \item Desarrollando:
    \begin{align*}
    n^2 + n &= 132 \\
    n^2 + n - 132 &= 0
    \end{align*}
    \item Factorizando:
    \begin{align*}
    (n + 12)(n - 11) &= 0
    \end{align*}
    \item Soluciones:
    \begin{align*}
    n + 12 &= 0 \quad \Rightarrow \quad n = -12 \\
    n - 11 &= 0 \quad \Rightarrow \quad n = 11
    \end{align*}
    \item Si $n = 11$, los enteros son 11 y 12
    \item Si $n = -12$, los enteros son $-12$ y $-11$
    \item \textbf{Verificación:}
    \begin{itemize}
        \item $11 \times 12 = 132$ \checkmark
        \item $(-12) \times (-11) = 132$ \checkmark
    \end{itemize}
\end{itemize}

\textbf{Respuesta:} Los enteros son 11 y 12, o $-12$ y $-11$.

\medskip

\textbf{c)} La suma de tres enteros consecutivos es 75. Halle los números.

\textbf{Solución Detallada:}
\begin{itemize}
    \item Sean $n$, $n+1$, y $n+2$ los tres enteros consecutivos
    \item Ecuación: $n + (n+1) + (n+2) = 75$
    \item Simplificando:
    \begin{align*}
    3n + 3 &= 75 \\
    3n &= 72 \\
    n &= 24
    \end{align*}
    \item Los tres números son: $24$, $25$, y $26$
    \item \textbf{Verificación:} $24 + 25 + 26 = 75$ \checkmark
\end{itemize}

\textbf{Respuesta:} Los números son 24, 25, y 26.

\medskip

\textbf{d)} Un número es 4 más que otro número. La suma de sus cuadrados es 106. Halle los números.

\textbf{Solución Detallada:}
\begin{itemize}
    \item Sea $x$ un número y $x + 4$ el otro número
    \item Ecuación: $x^2 + (x+4)^2 = 106$
    \item Expandiendo:
    \begin{align*}
    x^2 + x^2 + 8x + 16 &= 106 \\
    2x^2 + 8x + 16 &= 106 \\
    2x^2 + 8x - 90 &= 0 \\
    x^2 + 4x - 45 &= 0
    \end{align*}
    \item Factorizando:
    \begin{align*}
    (x + 9)(x - 5) &= 0
    \end{align*}
    \item Soluciones: $x = -9$ o $x = 5$
    \item Si $x = 5$, los números son 5 y 9
    \item Si $x = -9$, los números son $-9$ y $-5$
    \item \textbf{Verificación:}
    \begin{itemize}
        \item $5^2 + 9^2 = 25 + 81 = 106$ \checkmark
        \item $(-9)^2 + (-5)^2 = 81 + 25 = 106$ \checkmark
    \end{itemize}
\end{itemize}

\textbf{Respuesta:} Los números son 5 y 9, o $-9$ y $-5$.

\hrulefill

%========================================
% EJERCICIO 3: Problemas con Edades
%========================================

\subsubsection*{Ejercicio 3}
\textbf{Problema:} Resuelva los siguientes problemas con edades.

\medskip

\textbf{a)} La edad de Ana es el doble de la edad de su hermano. En 5 años, la suma de sus edades será 40. ¿Cuántos años tienen ahora?

\textbf{Solución Detallada:}
\begin{itemize}
    \item Sea $h$ = edad actual del hermano
    \item Edad actual de Ana: $2h$
    \item En 5 años:
    \begin{itemize}
        \item Hermano tendrá: $h + 5$
        \item Ana tendrá: $2h + 5$
    \end{itemize}
    \item Ecuación: $(h + 5) + (2h + 5) = 40$
    \item Resolviendo:
    \begin{align*}
    3h + 10 &= 40 \\
    3h &= 30 \\
    h &= 10
    \end{align*}
    \item Edad actual del hermano: $10$ años
    \item Edad actual de Ana: $2(10) = 20$ años
    \item \textbf{Verificación:} En 5 años: $15 + 25 = 40$ \checkmark
\end{itemize}

\textbf{Respuesta:} Ana tiene 20 años y su hermano tiene 10 años.

\medskip

\textbf{b)} Pedro tiene 3 años más que María. Hace 5 años, la edad de Pedro era el doble de la edad de María. ¿Cuántos años tienen ahora?

\textbf{Solución Detallada:}
\begin{itemize}
    \item Sea $m$ = edad actual de María
    \item Edad actual de Pedro: $m + 3$
    \item Hace 5 años:
    \begin{itemize}
        \item María tenía: $m - 5$
        \item Pedro tenía: $(m + 3) - 5 = m - 2$
    \end{itemize}
    \item Ecuación: $m - 2 = 2(m - 5)$
    \item Resolviendo:
    \begin{align*}
    m - 2 &= 2m - 10 \\
    -2 + 10 &= 2m - m \\
    8 &= m
    \end{align*}
    \item Edad actual de María: $8$ años
    \item Edad actual de Pedro: $8 + 3 = 11$ años
    \item \textbf{Verificación:} Hace 5 años: María tenía 3, Pedro tenía 6, y $6 = 2(3)$ \checkmark
\end{itemize}

\textbf{Respuesta:} Pedro tiene 11 años y María tiene 8 años.

\medskip

\textit{Nota: La respuesta difiere de la proporcionada en el ejercicio (Pedro 13, María 10). Verificación: Si Pedro tiene 13 y María 10, hace 5 años Pedro tenía 8 y María 5, pero $8 \neq 2(5) = 10$. La solución correcta es Pedro 11 y María 8.}

\hrulefill

%========================================
% EJERCICIO 4: Problemas de Aplicación
%========================================

\subsubsection*{Ejercicio 4}
\textbf{Problema:} Resuelva los siguientes problemas de aplicación.

\medskip

\textbf{a)} Un empleado gana \$12 por hora más un bono de \$50 por semana. ¿Cuántas horas debe trabajar para ganar \$530 en una semana?

\textbf{Solución Detallada:}
\begin{itemize}
    \item Sea $h$ = número de horas trabajadas
    \item Ingreso total = (pago por hora)(horas) + bono
    \item Ecuación: $12h + 50 = 530$
    \item Resolviendo:
    \begin{align*}
    12h + 50 &= 530 \\
    12h &= 530 - 50 \\
    12h &= 480 \\
    h &= 40
    \end{align*}
    \item \textbf{Verificación:} $12(40) + 50 = 480 + 50 = 530$ \checkmark
\end{itemize}

\textbf{Respuesta:} Debe trabajar 40 horas.

\medskip

\textbf{b)} Una tienda ofrece un descuento del 15\% en todas las compras. Si después del descuento, un artículo cuesta \$68, ¿cuál era el precio original?

\textbf{Solución Detallada:}
\begin{itemize}
    \item Sea $p$ = precio original
    \item Descuento = $0.15p$
    \item Precio después del descuento = $p - 0.15p = 0.85p$
    \item Ecuación: $0.85p = 68$
    \item Resolviendo:
    \begin{align*}
    p &= \frac{68}{0.85} \\
    p &= 80
    \end{align*}
    \item \textbf{Verificación:} $80 - 0.15(80) = 80 - 12 = 68$ \checkmark
\end{itemize}

\textbf{Respuesta:} El precio original era \$80.

\medskip

\textbf{c)} Un tren viaja a 60 mph. ¿Cuánto tiempo tardará en recorrer 270 millas?

\textbf{Solución Detallada:}
\begin{itemize}
    \item Usamos la fórmula: distancia = velocidad $\times$ tiempo
    \item $d = vt$
    \item Datos: $d = 270$ millas, $v = 60$ mph
    \item Ecuación: $270 = 60t$
    \item Resolviendo:
    \begin{align*}
    t &= \frac{270}{60} \\
    t &= 4.5 \text{ horas}
    \end{align*}
    \item $4.5$ horas = $4$ horas y $30$ minutos
\end{itemize}

\textbf{Respuesta:} Tardará 4.5 horas (o 4 horas y 30 minutos).

\hrulefill

%========================================
% EJERCICIO 5: Problemas de Interés Simple
%========================================

\subsubsection*{Ejercicio 5}
\textbf{Problema:} Resuelva los siguientes problemas de interés simple.

\medskip

\textbf{a)} Se invierten \$5000 a un interés simple de 6\% anual. ¿Cuánto interés se gana en 3 años?

\textbf{Solución Detallada:}
\begin{itemize}
    \item Usamos la fórmula de interés simple: $I = Prt$
    \item Datos:
    \begin{itemize}
        \item $P = 5000$ (principal)
        \item $r = 0.06$ (tasa de interés en decimal)
        \item $t = 3$ (tiempo en años)
    \end{itemize}
    \item Calculando:
    \begin{align*}
    I &= Prt \\
    I &= (5000)(0.06)(3) \\
    I &= 900
    \end{align*}
\end{itemize}

\textbf{Respuesta:} El interés ganado es \$900.

\medskip

\textbf{b)} ¿Qué cantidad debe invertirse a un interés simple de 4.5\% anual para ganar \$225 en 2 años?

\textbf{Solución Detallada:}
\begin{itemize}
    \item Usamos la fórmula de interés simple: $I = Prt$
    \item Datos conocidos:
    \begin{itemize}
        \item $I = 225$ (interés ganado)
        \item $r = 0.045$ (tasa de interés)
        \item $t = 2$ (tiempo en años)
    \end{itemize}
    \item Despejamos $P$:
    \begin{align*}
    I &= Prt \\
    225 &= P(0.045)(2) \\
    225 &= 0.09P \\
    P &= \frac{225}{0.09} \\
    P &= 2500
    \end{align*}
    \item \textbf{Verificación:} $I = (2500)(0.045)(2) = 225$ \checkmark
\end{itemize}

\textbf{Respuesta:} Debe invertirse \$2500.

\medskip

\textbf{c)} Se prestan \$8000 a un interés simple de 7.5\% anual. ¿Cuánto dinero se debe devolver después de 18 meses?

\textbf{Solución Detallada:}
\begin{itemize}
    \item Primero calculamos el interés con $I = Prt$
    \item Datos:
    \begin{itemize}
        \item $P = 8000$ (principal)
        \item $r = 0.075$ (tasa de interés)
        \item $t = \dfrac{18}{12} = 1.5$ años (18 meses = 1.5 años)
    \end{itemize}
    \item Calculando el interés:
    \begin{align*}
    I &= Prt \\
    I &= (8000)(0.075)(1.5) \\
    I &= 900
    \end{align*}
    \item La cantidad total a devolver es:
    \begin{align*}
    A &= P + I \\
    A &= 8000 + 900 \\
    A &= 8900
    \end{align*}
\end{itemize}

\textbf{Respuesta:} Se debe devolver \$8900.

\hrulefill

%========================================
% EJERCICIO 6: Problemas de Interés Compuesto
%========================================

\subsubsection*{Ejercicio 6}
\textbf{Problema:} Resuelva los siguientes problemas de interés compuesto.

\medskip

\textbf{a)} Se invierten \$4000 a un interés de 5\% anual compuesto semestralmente. ¿Cuál es la cantidad acumulada después de 10 años?

\textbf{Solución Detallada:}
\begin{itemize}
    \item Usamos la fórmula: $A = P\left(1 + \dfrac{r}{m}\right)^{mt}$
    \item Datos:
    \begin{itemize}
        \item $P = 4000$ (principal)
        \item $r = 0.05$ (tasa de interés)
        \item $m = 2$ (semestralmente)
        \item $t = 10$ (años)
    \end{itemize}
    \item Calculando:
    \begin{align*}
    A &= 4000\left(1 + \frac{0.05}{2}\right)^{2 \times 10} \\
    A &= 4000(1.025)^{20} \\
    A &= 4000(1.6386) \\
    A &\approx 6554.52
    \end{align*}
\end{itemize}

\textbf{Respuesta:} La cantidad acumulada es aproximadamente \$6554.52.

\textit{Nota: Hay una pequeña diferencia con la respuesta proporcionada (\$6533.28) debido a redondeo en los cálculos intermedios.}

\medskip

\textbf{b)} ¿Cuánto dinero debe invertirse a un interés de 6\% anual compuesto mensualmente para tener \$10,000 después de 5 años?

\textbf{Solución Detallada:}
\begin{itemize}
    \item Usamos la fórmula: $A = P\left(1 + \dfrac{r}{m}\right)^{mt}$
    \item Datos conocidos:
    \begin{itemize}
        \item $A = 10000$ (cantidad deseada)
        \item $r = 0.06$ (tasa de interés)
        \item $m = 12$ (mensualmente)
        \item $t = 5$ (años)
    \end{itemize}
    \item Despejamos $P$:
    \begin{align*}
    10000 &= P\left(1 + \frac{0.06}{12}\right)^{12 \times 5} \\
    10000 &= P(1.005)^{60} \\
    10000 &= P(1.3489) \\
    P &= \frac{10000}{1.3489} \\
    P &\approx 7412.91
    \end{align*}
\end{itemize}

\textbf{Respuesta:} Debe invertirse aproximadamente \$7412.91.

\textit{Nota: Hay una pequeña diferencia con la respuesta proporcionada (\$7407.94) debido a redondeo.}

\medskip

\textbf{c)} Compare el interés ganado en \$2000 invertidos al 8\% anual por 3 años usando: (i) interés simple, (ii) interés compuesto trimestral.

\textbf{Solución Detallada:}

\textbf{(i) Interés Simple:}
\begin{itemize}
    \item $I = Prt = (2000)(0.08)(3) = 480$
    \item Interés ganado: \$480
\end{itemize}

\textbf{(ii) Interés Compuesto Trimestral:}
\begin{itemize}
    \item $A = P\left(1 + \dfrac{r}{m}\right)^{mt}$
    \item Datos: $P = 2000$, $r = 0.08$, $m = 4$, $t = 3$
    \item Calculando:
    \begin{align*}
    A &= 2000\left(1 + \frac{0.08}{4}\right)^{4 \times 3} \\
    A &= 2000(1.02)^{12} \\
    A &= 2000(1.2682) \\
    A &\approx 2536.48
    \end{align*}
    \item Interés ganado: $I = A - P = 2536.48 - 2000 = 536.48$
\end{itemize}

\textbf{Comparación:}
\begin{itemize}
    \item Diferencia: $536.48 - 480 = 56.48$
    \item Con interés compuesto se gana \$56.48 más que con interés simple
\end{itemize}

\textbf{Respuesta:} Interés simple: \$480. Interés compuesto trimestral: aproximadamente \$536.48. Diferencia: \$56.48 más con interés compuesto.

\textit{Nota: Hay una pequeña diferencia con la respuesta proporcionada (\$537.49) debido a redondeo en los cálculos.}

\hrulefill

%========================================
% Resumen de Estrategias
%========================================

\subsection*{Resumen de Estrategias para Resolver Problemas}

\begin{enumerate}
    \item \textbf{Leer cuidadosamente:} Identifique qué información se le da y qué se le pide encontrar.

    \item \textbf{Definir variables:} Asigne una letra para representar cada cantidad desconocida.

    \item \textbf{Traducir a ecuaciones:} Convierta las relaciones verbales en ecuaciones algebraicas.

    \item \textbf{Resolver sistemáticamente:} Use las técnicas algebraicas apropiadas.

    \item \textbf{Verificar:} Sustituya sus respuestas en el problema original para confirmar que son correctas.

    \item \textbf{Contestar con unidades:} Incluya las unidades apropiadas en su respuesta final.
\end{enumerate}

\vspace{1cm}

\textbf{Fórmulas Importantes:}

\begin{itemize}
    \item \textbf{Interés Simple:} $I = Prt$ donde $I$ es el interés, $P$ es el principal, $r$ es la tasa (decimal), y $t$ es el tiempo en años.

    \item \textbf{Cantidad Acumulada (Simple):} $A = P + I = P + Prt = P(1 + rt)$

    \item \textbf{Interés Compuesto:} $A = P\left(1 + \dfrac{r}{m}\right)^{mt}$ donde $m$ es el número de veces que se compone por año.

    \item \textbf{Promedio:} $\text{Promedio} = \dfrac{\text{Suma de valores}}{\text{Número de valores}}$

    \item \textbf{Distancia:} $d = vt$ donde $d$ es distancia, $v$ es velocidad, y $t$ es tiempo.
\end{itemize}
