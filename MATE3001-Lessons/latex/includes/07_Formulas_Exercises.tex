%========================================
% EXERCISES: Fórmulas
%========================================

\section{Ejercicios}

\begin{exercise}
\problem Evalúe la fórmula $I = Prt$ usando los valores dados:

\begin{exerciselist}
    \item $P = 1000$, $r = 0.05$, $t = 3$
    \item $P = 2500$, $r = 0.04$, $t = 2$
\end{exerciselist}

\begin{solucion}
\textbf{a)} $I = 1000 \cdot 0.05 \cdot 3 = 150$

\textbf{b)} $I = 2500 \cdot 0.04 \cdot 2 = 200$
\end{solucion}

\problem Evalúe la fórmula $A = P + Prt$ usando los valores dados:

\begin{exerciselist}
    \item $P = 500$, $r = 0.08$, $t = 2$
    \item $P = 1500$, $r = 0.05$, $t = 3$
\end{exerciselist}

\begin{solucion}
\textbf{a)} $A = 500 + 500 \cdot 0.08 \cdot 2 = 500 + 80 = 580$

\textbf{b)} $A = 1500 + 1500 \cdot 0.05 \cdot 3 = 1500 + 225 = 1725$
\end{solucion}

\problem Evalúe la fórmula $C = \frac{5}{9}(F-32)$ para los valores de $F$ dados:

\begin{exerciselist}
    \item $F = 32°$
    \item $F = 212°$
\end{exerciselist}

\begin{solucion}
\textbf{a)} $C = \frac{5}{9}(32-32) = \frac{5}{9}(0) = 0°$

\textbf{b)} $C = \frac{5}{9}(212-32) = \frac{5}{9}(180) = 100°$
\end{solucion}
\end{exercise}

\begin{exercise}
\problem Encuentre el valor de la variable indicada usando la fórmula $I = Prt$:

\begin{exerciselist}
    \item Encuentre $P$ si $I = 240$, $r = 0.06$, $t = 4$
    \item Encuentre $r$ si $I = 180$, $P = 1500$, $t = 2$
\end{exerciselist}

\begin{solucion}
\textbf{a)} $240 = P \cdot 0.06 \cdot 4 \Rightarrow P = \frac{240}{0.24} = 1000$

\textbf{b)} $180 = 1500 \cdot r \cdot 2 \Rightarrow r = \frac{180}{3000} = 0.06$
\end{solucion}

\problem Encuentre el valor de la variable indicada usando la fórmula $A = P + Prt$:

\begin{exerciselist}
    \item Encuentre $P$ si $A = 1320$, $r = 0.08$, $t = 2$
    \item Encuentre $t$ si $A = 1150$, $P = 1000$, $r = 0.05$
\end{exerciselist}

\begin{solucion}
\textbf{a)} $1320 = P + P \cdot 0.08 \cdot 2 = P(1 + 0.16) = 1.16P$
$\Rightarrow P = \frac{1320}{1.16} = 1137.93$

\textbf{b)} $1150 = 1000 + 1000 \cdot 0.05 \cdot t = 1000(1 + 0.05t)$
$\Rightarrow 1.15 = 1 + 0.05t \Rightarrow t = \frac{0.15}{0.05} = 3$
\end{solucion}
\end{exercise}

\begin{exercise}
\problem Despeje las siguientes fórmulas para la variable indicada:

\begin{exerciselist}
    \item $V = lwh$ para $l$ (volumen de un prisma rectangular)
    \item $P = 2l + 2w$ para $w$ (perímetro de un rectángulo)
    \item $A = \frac{1}{2}bh$ para $h$ (área de un triángulo)
    \item $S = 2\pi r^2 + 2\pi rh$ para $h$ (superficie de un cilindro)
\end{exerciselist}

\begin{solucion}
\textbf{a)} $l = \frac{V}{wh}$

\textbf{b)} $P = 2l + 2w \Rightarrow P - 2l = 2w \Rightarrow w = \frac{P - 2l}{2}$

\textbf{c)} $h = \frac{2A}{b}$

\textbf{d)} $S = 2\pi r^2 + 2\pi rh \Rightarrow S - 2\pi r^2 = 2\pi rh \Rightarrow h = \frac{S - 2\pi r^2}{2\pi r}$
\end{solucion}

\problem Despeje las siguientes expresiones algebraicas para la variable $x$ (requieren factorización):

\begin{exerciselist}
    \item $ax + b = c$ para $x$ (donde $a \neq 0$)
    \item $ax + bx = c$ para $x$
    \item $ax + b = cx + d$ para $x$ (donde $a \neq c$)
    \item $a(x + b) = c(x + d)$ para $x$ (donde $a \neq c$)
    \item $ax + ay = bx + by$ para $x$ (donde $a \neq b$)
    \item $a(x + y) + b(x + z) = c$ para $x$
    \item $\frac{x}{a} + \frac{x}{b} = c$ para $x$ (donde $a, b \neq 0$)
\end{exerciselist}

\begin{solucion}
\textbf{a)} $ax + b = c \Rightarrow ax = c - b \Rightarrow x = \frac{c - b}{a}$

\textbf{b)} $ax + bx = c \Rightarrow x(a + b) = c \Rightarrow x = \frac{c}{a + b}$

\textbf{c)} $ax + b = cx + d \Rightarrow ax - cx = d - b \Rightarrow x(a - c) = d - b \Rightarrow x = \frac{d - b}{a - c}$

\textbf{d)} $a(x + b) = c(x + d) \Rightarrow ax + ab = cx + cd \Rightarrow ax - cx = cd - ab \Rightarrow x(a - c) = cd - ab \Rightarrow x = \frac{cd - ab}{a - c}$

\textbf{e)} $ax + ay = bx + by \Rightarrow ax - bx = by - ay \Rightarrow x(a - b) = y(b - a) = -y(a - b) \Rightarrow x = -y$

\textbf{f)} $a(x + y) + b(x + z) = c \Rightarrow ax + ay + bx + bz = c \Rightarrow x(a + b) = c - ay - bz \Rightarrow x = \frac{c - ay - bz}{a + b}$

\textbf{g)} $\frac{x}{a} + \frac{x}{b} = c \Rightarrow x\left(\frac{1}{a} + \frac{1}{b}\right) = c \Rightarrow x\left(\frac{b + a}{ab}\right) = c \Rightarrow x = \frac{abc}{a + b}$
\end{solucion}
\end{exercise}

\begin{exercise}
\problem \textbf{Problemas de despeje aplicado:}

\begin{exerciselist}
    \item Un rectángulo tiene área $A = 48$ y ancho $w = 6$. Use la fórmula $A = lw$ para encontrar el largo $l$.

    \item Un tanque cilíndrico tiene volumen $V = 500\pi$ y radio $r = 5$. Use la fórmula $V = \pi r^2 h$ para encontrar la altura $h$.

    \item La ecuación del movimiento uniforme es $d = vt$, donde $d$ es distancia, $v$ es velocidad y $t$ es tiempo. Si un auto recorre $240$ km en $3$ horas, ¿cuál es su velocidad?
\end{exerciselist}

\begin{solucion}
\textbf{a)} $48 = l \cdot 6 \Rightarrow l = \frac{48}{6} = 8$

\textbf{b)} $500\pi = \pi (5)^2 h \Rightarrow 500\pi = 25\pi h \Rightarrow h = \frac{500\pi}{25\pi} = 20$

\textbf{c)} $240 = v \cdot 3 \Rightarrow v = \frac{240}{3} = 80$ km/h
\end{solucion}
\end{exercise}