\documentclass[12pt]{article}

%========================================
% PACKAGES AND CONFIGURATION
%========================================
\usepackage[utf8]{inputenc}
\usepackage[spanish]{babel}              % Spanish language support
\decimalpoint                                % Force decimal point instead of comma
\usepackage[margin=1in]{geometry}
\setlength{\headheight}{15pt}
\usepackage{amsmath, amsthm, amssymb}
\usepackage{mdframed}
\usepackage{xcolor}
\usepackage{enumitem}
\usepackage{fancyhdr}
\usepackage{graphicx}
\usepackage{tikz}                        % For LaTeX-generated diagrams
\usetikzlibrary{arrows.meta}           % For arrow styles
\usetikzlibrary{shapes}                % For diamond and other shapes
\usetikzlibrary{decorations.pathreplacing} % For braces and decorations
\usepackage{comment}                     % For conditional content exclusion

% Fix Spanish babel conflicts with TikZ
\usetikzlibrary{babel}

%========================================
% COURSE CUSTOMIZATION SECTION
%========================================
% MODIFY THESE FOR EACH COURSE:
\newcommand{\coursecode}{MATE 3001}      % Course code
\newcommand{\coursename}{Matemática Elemental}     % Course name
\newcommand{\institution}{UPR-Humacao}   % Institution name
\newcommand{\lessontitle}{Ecuaciones Cuadráticas}    % Will be overridden per lesson

%========================================
% COLOR SCHEME DEFINITIONS
%========================================
\definecolor{defcolor}{RGB}{240,248,255}     % Light blue for definitions
\definecolor{examplecolor}{RGB}{245,255,245} % Light green for examples
\definecolor{exercisecolor}{RGB}{255,248,240} % Light orange for exercises
\definecolor{theoremcolor}{RGB}{255,250,240}  % Light orange for theorems

%========================================
% CUSTOM ENVIRONMENTS
%========================================
% Definition Environment (Blue)
\newmdenv[
    backgroundcolor=defcolor,
    linecolor=blue!50,
    linewidth=2pt,
    leftmargin=10pt,
    rightmargin=10pt,
    innertopmargin=10pt,
    innerbottommargin=10pt,
    frametitle={\textbf{Definición}},
    frametitlealignment=\raggedright
]{definition}

% Example Environment (Green)
\newmdenv[
    backgroundcolor=examplecolor,
    linecolor=green!50,
    linewidth=2pt,
    leftmargin=10pt,
    rightmargin=10pt,
    innertopmargin=10pt,
    innerbottommargin=10pt,
    frametitle={\textbf{Ejemplo}},
    frametitlealignment=\raggedright
]{example}

% Exercise Environment (Orange)
\newcounter{exercise}[section]
\newcounter{problem}[exercise]
\newmdenv[
    backgroundcolor=exercisecolor,
    linecolor=orange!50,
    linewidth=2pt,
    leftmargin=10pt,
    rightmargin=10pt,
    innertopmargin=10pt,
    innerbottommargin=10pt,
    frametitle={\stepcounter{exercise}\textbf{Ejercicio \theexercise}},
    frametitlealignment=\raggedright
]{exercise}

% Theorem Environment (Orange variant)
\newmdenv[
    backgroundcolor=theoremcolor,
    linecolor=orange!50,
    linewidth=2pt,
    leftmargin=10pt,
    rightmargin=10pt,
    innertopmargin=10pt,
    innerbottommargin=10pt
]{theorem}

%========================================
% HEADER AND FOOTER CONFIGURATION
%========================================
\pagestyle{fancy}
\fancyhf{}
\rhead{\coursecode\ - \coursename}
\lhead{\lessontitle}
\cfoot{\thepage}

%========================================
% CUSTOM COMMANDS
%========================================
\newcommand{\lesson}[1]{\renewcommand{\lessontitle}{#1}\section{#1}}
\newcommand{\subsectiontitle}[1]{\subsection{#1}}

% Exercise numbering commands
\newcommand{\problem}{\stepcounter{problem}\textbf{\theproblem.} }
\newcommand{\solution}{\textbf{Solución:} }

% Custom environment for exercise lists
\newenvironment{exerciselist}
    {\begin{enumerate}[label=\textbf{\alph*.}]}
    {\end{enumerate}}

% Solution environment with conditional display
\newif\ifshowsolutions
% \showsolutionstrue  % Uncomment for instructor version
\showsolutionsfalse   % Default: student version

\ifshowsolutions
    \newenvironment{solucion}[1][Solución]
      {\par\medskip\noindent\textbf{#1:}\par\nopagebreak}
      {\par\medskip}
\else
    \excludecomment{solucion}
\fi

%========================================
% GRAPHICS CONFIGURATION
%========================================
\graphicspath{{../images/}{../images/shared/}{../images/diagrams/}{../images/08_ecuaciones_cuadraticas/}}

%========================================
% DOCUMENT CONTENT
%========================================
\begin{document}

% Title page
\title{\lessontitle}
\author{\coursecode\ - \coursename}
\date{}
\maketitle

% Set section counter to lesson number
\setcounter{section}{7}

% Modular content inclusion
%========================================
% LESSON CONTENT: Ecuaciones Cuadráticas
%========================================

\lesson{Ecuaciones Cuadráticas}

%========================================
% SECTION 8.1: Definición y Forma Estándar
%========================================
\subsectiontitle{Definición y Forma Estándar}

\begin{definition}
Una \textbf{ecuación cuadrática} es una ecuación que puede escribirse en la forma:
$$ax^2 + bx + c = 0$$
donde $a$, $b$ y $c$ son números reales con $a \neq 0$.

Esta forma se conoce como la \textbf{forma estándar} de una ecuación cuadrática.
\end{definition}

% Visual representation of quadratic equation components
\begin{center}
\begin{tikzpicture}[scale=1.2]
    % Central equation box
    \node[draw, fill=blue!20, minimum width=6cm, minimum height=1.8cm, rounded corners] (eq) at (0,0)
        {\Huge $ax^2 + bx + c = 0$};

    % Coefficient labels with color coding
    \node[draw, fill=red!20, rounded corners] (a) at (-4.5,2.5) {$a$: Coeficiente cuadrático};
    \node[draw, fill=green!20, rounded corners] (b) at (0,2.5) {$b$: Coeficiente lineal};
    \node[draw, fill=orange!20, rounded corners] (c) at (4.5,2.5) {$c$: Término constante};

    % Arrows pointing to equation
    \draw[->, thick, red] (a) -- (-2,0.8);
    \draw[->, thick, green!70!black] (b) -- (0,0.8);
    \draw[->, thick, orange] (c) -- (1.4,0.8);

    % Important note
    \node[draw, fill=yellow!20, thick] at (0,-1.8) {Condición: $a \neq 0$};
\end{tikzpicture}
\end{center}

\textbf{Identificación de coeficientes:}

Para identificar correctamente los coeficientes $a$, $b$ y $c$, es esencial escribir la ecuación en forma estándar primero.

\begin{example}
\textbf{Ejemplo 1:} Identifique los coeficientes $a$, $b$ y $c$ en las siguientes ecuaciones cuadráticas:

\textbf{a)} $x^2 + 5x + 6 = 0$

\textbf{Solución:} Ya está en forma estándar: $a = 1$, $b = 5$, $c = 6$

\textbf{b)} $2x^2 - 7x = 4$

\textbf{Solución:} Primero escribimos en forma estándar:
\begin{align}
2x^2 - 7x &= 4\\
2x^2 - 7x - 4 &= 0
\end{align}
Por tanto: $a = 2$, $b = -7$, $c = -4$

\textbf{c)} $3x^2 = 12$

\textbf{Solución:} Forma estándar:
$$3x^2 - 12 = 0$$
Por tanto: $a = 3$, $b = 0$, $c = -12$

\textbf{d)} $-x^2 + 4x + 1 = 0$

\textbf{Solución:} Ya está en forma estándar: $a = -1$, $b = 4$, $c = 1$
\end{example}

%========================================
% SECTION 8.2: Método de Factorización
%========================================
\subsectiontitle{Método de Factorización}

El primer método para resolver ecuaciones cuadráticas se basa en el principio fundamental del producto cero.

\begin{theorem}
\textbf{Principio del Producto Cero (Zero Product Property):}

Si $p$ y $q$ son expresiones algebraicas, entonces:
$$pq = 0 \quad \text{si y solo si} \quad p = 0 \quad \text{o bien} \quad q = 0$$
\end{theorem}

\textbf{Procedimiento para resolver por factorización:}

\begin{enumerate}
\item Escribir la ecuación en forma estándar: $ax^2 + bx + c = 0$
\item Factorizar completamente el lado izquierdo de la ecuación
\item Aplicar el principio del producto cero: igualar cada factor a cero
\item Resolver las ecuaciones lineales resultantes
\item Verificar las soluciones en la ecuación original
\end{enumerate}

% Visual flowchart for factorization method
\begin{center}
\begin{tikzpicture}[
    stepstyle/.style={rectangle, draw=blue!60, fill=blue!10, thick, minimum width=4.5cm, text centered, minimum height=1cm, rounded corners},
    arrowstyle/.style={->, thick}
]
    \node[stepstyle] (step1) at (0,0) {Forma estándar: $ax^2 + bx + c = 0$};
    \node[stepstyle] (step2) at (0,-2) {Factorizar: $(px + m)(qx + n) = 0$};
    \node[stepstyle] (step3) at (0,-4) {Igualar cada factor a cero};
    \node[stepstyle, fill=green!10] (step4) at (0,-6) {$px + m = 0$ \quad o \quad $qx + n = 0$};
    \node[stepstyle, fill=green!20] (step5) at (0,-8) {Soluciones: $x = -\frac{m}{p}$ \quad o \quad $x = -\frac{n}{q}$};

    \draw[arrowstyle] (step1) -- (step2);
    \draw[arrowstyle] (step2) -- (step3);
    \draw[arrowstyle] (step3) -- (step4);
    \draw[arrowstyle] (step4) -- (step5);
\end{tikzpicture}
\end{center}

\begin{example}
\textbf{Ejemplo 2:} Resuelva $x^2 + 7x + 12 = 0$ por factorización.

\textbf{Solución:}

\textbf{Paso 1:} La ecuación ya está en forma estándar.

\textbf{Paso 2:} Factorizamos el trinomio. Buscamos dos números que multipliquen 12 y sumen 7: estos son 3 y 4.
$$x^2 + 7x + 12 = (x + 3)(x + 4) = 0$$

\textbf{Paso 3:} Aplicamos el principio del producto cero:
$$x + 3 = 0 \quad \text{o bien} \quad x + 4 = 0$$

\textbf{Paso 4:} Resolvemos cada ecuación:
\begin{align}
x + 3 &= 0 \quad \Rightarrow \quad x = -3\\
x + 4 &= 0 \quad \Rightarrow \quad x = -4
\end{align}

\textbf{Respuesta:} Las soluciones son $x = -3$ y $x = -4$.

\textbf{Verificación:} Para $x = -3$: $(-3)^2 + 7(-3) + 12 = 9 - 21 + 12 = 0$ $\checkmark$
\end{example}

\newpage

\begin{example}
\textbf{Ejemplo 3:} Resuelva $6x^2 + x - 12 = 0$ por factorización.

\textbf{Solución:}

Usamos el método AC: $ac = 6(-12) = -72$. Buscamos dos números que multipliquen $-72$ y sumen $1$: estos son $9$ y $-8$.

\begin{align}
6x^2 + x - 12 &= 0\\
6x^2 + 9x - 8x - 12 &= 0\\
3x(2x + 3) - 4(2x + 3) &= 0\\
(2x + 3)(3x - 4) &= 0
\end{align}

Aplicamos el principio del producto cero:
\begin{align}
2x + 3 &= 0 \quad \Rightarrow \quad x = -\frac{3}{2}\\
3x - 4 &= 0 \quad \Rightarrow \quad x = \frac{4}{3}
\end{align}

\textbf{Respuesta:} $x = -\frac{3}{2}$ o $x = \frac{4}{3}$
\end{example}

\begin{example}
\textbf{Ejemplo 4:} Resuelva $x^2 - 9 = 0$ por factorización.

\textbf{Solución:}

Reconocemos una diferencia de cuadrados: $x^2 - 9 = x^2 - 3^2$

\begin{align}
x^2 - 9 &= 0\\
(x + 3)(x - 3) &= 0
\end{align}

Aplicamos el principio del producto cero:
\begin{align}
x + 3 &= 0 \quad \Rightarrow \quad x = -3\\
x - 3 &= 0 \quad \Rightarrow \quad x = 3
\end{align}

\textbf{Respuesta:} $x = -3$ o $x = 3$
\end{example}

\newpage
%========================================
% SECTION 8.3: Método de Completar el Cuadrado
%========================================
\subsectiontitle{Método de Completar el Cuadrado}

El método de completar el cuadrado transforma una ecuación cuadrática en la forma $(x + p)^2 = q$, que se puede resolver fácilmente tomando la raíz cuadrada de ambos lados.

\begin{definition}
\textbf{Completar el cuadrado} es una técnica algebraica que convierte una expresión cuadrática $x^2 + bx$ en un trinomio cuadrado perfecto sumando el término $\left(\frac{b}{2}\right)^2$.
\end{definition}

\textbf{Fórmula clave:}
$$x^2 + bx + \left(\frac{b}{2}\right)^2 = \left(x + \frac{b}{2}\right)^2$$

% Geometric visualization of completing the square
\begin{center}
\begin{tikzpicture}[scale=0.9]
    % Original square x^2
    \draw[fill=blue!30, thick] (0,0) rectangle (2,2);
    \node at (1,1) {$x^2$};
    \node at (1,-0.4) {$x$};
    \node at (-0.4,1) {$x$};

    % Added rectangles for bx/2
    \draw[fill=green!30, thick] (2,0) rectangle (3,2);
    \node at (2.5,1) {$\frac{b}{2}x$};
    \node at (2.5,-0.4) {$\frac{b}{2}$};

    \draw[fill=green!30, thick] (0,2) rectangle (2,3);
    \node at (1,2.5) {$\frac{b}{2}x$};
    \node at (-0.4,2.5) {$\frac{b}{2}$};

    % Missing square (b/2)^2
    \draw[fill=red!30, thick] (2,2) rectangle (3,3);
    \node at (2.5,2.5) {$\left(\frac{b}{2}\right)^2$};

    % Bracket showing complete square
    \draw[decorate, decoration={brace, amplitude=8pt, mirror}, thick] (3.2,0) -- (3.2,3);
    \node at (4.5,1.5) {$\left(x + \frac{b}{2}\right)^2$};

    \node at (1.5,-1.5) {\textbf{Interpretación Geométrica}};
\end{tikzpicture}
\end{center}

\textbf{Procedimiento para completar el cuadrado:}

\begin{enumerate}
\item Si $a \neq 1$, dividir toda la ecuación por $a$
\item Mover el término constante al lado derecho
\item Calcular $\left(\frac{b}{2}\right)^2$ y sumar a ambos lados
\item Factorizar el lado izquierdo como un cuadrado perfecto
\item Tomar la raíz cuadrada de ambos lados (recordar el $\pm$)
\item Resolver para $x$
\end{enumerate}

\begin{example}
\textbf{Ejemplo 5:} Resuelva $x^2 + 6x + 5 = 0$ completando el cuadrado.

\textbf{Solución:}

\textbf{Paso 1:} El coeficiente de $x^2$ es 1, continuamos.

\textbf{Paso 2:} Movemos el término constante:
$$x^2 + 6x = -5$$

\textbf{Paso 3:} Calculamos $\left(\frac{6}{2}\right)^2 = 9$ y sumamos a ambos lados:
$$x^2 + 6x + 9 = -5 + 9$$

\textbf{Paso 4:} Factorizamos el lado izquierdo:
$$(x + 3)^2 = 4$$

\textbf{Paso 5:} Tomamos raíz cuadrada:
$$x + 3 = \pm 2$$

\textbf{Paso 6:} Resolvemos:
\begin{align}
x + 3 &= 2 \quad \Rightarrow \quad x = -1\\
x + 3 &= -2 \quad \Rightarrow \quad x = -5
\end{align}

\textbf{Respuesta:} $x = -1$ o $x = -5$
\end{example}

\begin{example}
\textbf{Ejemplo 6:} Resuelva $2x^2 - 8x + 2 = 0$ completando el cuadrado.

\textbf{Solución:}

\textbf{Paso 1:} Dividimos por 2:
$$x^2 - 4x + 1 = 0$$

\textbf{Paso 2:} Movemos el término constante:
$$x^2 - 4x = -1$$

\textbf{Paso 3:} Calculamos $\left(\frac{-4}{2}\right)^2 = 4$ y sumamos a ambos lados:
$$x^2 - 4x + 4 = -1 + 4$$

\textbf{Paso 4:} Factorizamos:
$$(x - 2)^2 = 3$$

\textbf{Paso 5:} Tomamos raíz cuadrada:
$$x - 2 = \pm\sqrt{3}$$

\textbf{Paso 6:} Resolvemos:
$$x = 2 \pm \sqrt{3}$$

\textbf{Respuesta:} $x = 2 + \sqrt{3}$ o $x = 2 - \sqrt{3}$
\end{example}

%========================================
% SECTION 8.4: Fórmula Cuadrática
%========================================
\subsectiontitle{Fórmula Cuadrática}

La fórmula cuadrática es el método más general para resolver ecuaciones cuadráticas. Funciona para cualquier ecuación cuadrática, sin importar si es factorizable o no.

\begin{theorem}
\textbf{Fórmula Cuadrática:}

Las soluciones de la ecuación cuadrática $ax^2 + bx + c = 0$ (con $a \neq 0$) están dadas por:
$$\boxed{x = \frac{-b \pm \sqrt{b^2 - 4ac}}{2a}}$$
\end{theorem}

% Visual breakdown of the quadratic formula
\begin{center}
\begin{tikzpicture}[scale=1]
    % Formula at top
    \node at (0,3.5) {\Huge $x = \dfrac{-b \pm \sqrt{b^2 - 4ac}}{2a}$};

    % Component boxes - reordered: left (-b), middle (2a), right (discriminant)
    \node[draw, fill=blue!20, rounded corners, text width=2.8cm, align=center] (comp1) at (-4.5,0)
        {$-b$\\Opuesto del\\coeficiente lineal};

    \node[draw, fill=red!20, rounded corners, text width=2.8cm, align=center] (comp3) at (0,0)
        {$2a$\\Doble del\\coef. cuadrático};

    \node[draw, fill=green!20, rounded corners, text width=3.2cm, align=center] (comp2) at (4.5,0)
        {$\sqrt{b^2 - 4ac}$\\Raíz del\\discriminante};

    % Arrows pointing to specific parts of the formula
    \draw[->, thick, blue] (comp1.north) -- (-2,3);
    \draw[->, thick, red] (comp3.north) -- (0.6,2.2);
    \draw[->, thick, green!70!black] (comp2.north) -- (2.5,3);

    % Plus-minus symbol note
    \node[draw, fill=yellow!20, rounded corners] at (0,-2) {El símbolo $\pm$ genera dos soluciones};
\end{tikzpicture}
\end{center}

\textbf{Procedimiento para usar la fórmula cuadrática:}

\begin{enumerate}
\item Escribir la ecuación en forma estándar: $ax^2 + bx + c = 0$
\item Identificar los valores de $a$, $b$ y $c$
\item Sustituir en la fórmula cuadrática
\item Simplificar la expresión bajo el radical
\item Calcular ambas soluciones usando $+$ y $-$
\item Simplificar las respuestas
\end{enumerate}

\newpage

\begin{example}
\textbf{Ejemplo 7:} Resuelva $x^2 - 5x + 6 = 0$ usando la fórmula cuadrática.

\textbf{Solución:}

\textbf{Paso 1:} La ecuación está en forma estándar.

\textbf{Paso 2:} Identificamos: $a = 1$, $b = -5$, $c = 6$

\textbf{Paso 3:} Sustituimos en la fórmula:
$$x = \frac{-(-5) \pm \sqrt{(-5)^2 - 4(1)(6)}}{2(1)}$$

\textbf{Paso 4:} Simplificamos:
\begin{align}
x &= \frac{5 \pm \sqrt{25 - 24}}{2}\\
x &= \frac{5 \pm \sqrt{1}}{2}\\
x &= \frac{5 \pm 1}{2}
\end{align}

\textbf{Paso 5:} Calculamos ambas soluciones:
\begin{align}
x_1 &= \frac{5 + 1}{2} = \frac{6}{2} = 3\\
x_2 &= \frac{5 - 1}{2} = \frac{4}{2} = 2
\end{align}

\textbf{Respuesta:} $x = 3$ o $x = 2$
\end{example}

\begin{example}
\textbf{Ejemplo 8:} Resuelva $2x^2 + 3x - 5 = 0$ usando la fórmula cuadrática.

\textbf{Solución:}

Identificamos: $a = 2$, $b = 3$, $c = -5$

\begin{align}
x &= \frac{-3 \pm \sqrt{3^2 - 4(2)(-5)}}{2(2)}\\
x &= \frac{-3 \pm \sqrt{9 + 40}}{4}\\
x &= \frac{-3 \pm \sqrt{49}}{4}\\
x &= \frac{-3 \pm 7}{4}
\end{align}

Calculamos:
\begin{align}
x_1 &= \frac{-3 + 7}{4} = \frac{4}{4} = 1\\
x_2 &= \frac{-3 - 7}{4} = \frac{-10}{4} = -\frac{5}{2}
\end{align}

\textbf{Respuesta:} $x = 1$ o $x = -\frac{5}{2}$
\end{example}

\begin{example}
\textbf{Ejemplo 9:} Resuelva $x^2 + 2x - 4 = 0$ usando la fórmula cuadrática.

\textbf{Solución:}

Identificamos: $a = 1$, $b = 2$, $c = -4$

\begin{align}
x &= \frac{-2 \pm \sqrt{2^2 - 4(1)(-4)}}{2(1)}\\
x &= \frac{-2 \pm \sqrt{4 + 16}}{2}\\
x &= \frac{-2 \pm \sqrt{20}}{2}\\
x &= \frac{-2 \pm 2\sqrt{5}}{2}\\
x &= \frac{2(-1 \pm \sqrt{5})}{2}\\
x &= -1 \pm \sqrt{5}
\end{align}

\textbf{Respuesta:} $x = -1 + \sqrt{5}$ o $x = -1 - \sqrt{5}$
\end{example}

\newpage
%========================================
% SECTION 8.5: El Discriminante
%========================================
\subsectiontitle{El Discriminante}

El discriminante es la expresión que aparece bajo el radical en la fórmula cuadrática. Nos permite determinar la naturaleza de las soluciones sin necesidad de resolverlas completamente.

\begin{definition}
Para una ecuación cuadrática $ax^2 + bx + c = 0$, el \textbf{discriminante} se define como:
$$D = b^2 - 4ac$$
\end{definition}

\begin{theorem}
\textbf{Naturaleza de las raíces según el discriminante:}

Sea $D = b^2 - 4ac$ el discriminante de la ecuación $ax^2 + bx + c = 0$. Entonces:

\begin{itemize}
\item Si $D > 0$: La ecuación tiene \textbf{dos raíces reales distintas}
\item Si $D = 0$: La ecuación tiene \textbf{una raíz real doble} (raíz repetida)
\item Si $D < 0$: La ecuación \textbf{no tiene raíces reales}
\end{itemize}
\end{theorem}

% Table showing discriminant cases
\begin{center}
\begin{tabular}{|c|c|c|c|}
\hline
\textbf{Discriminante} & \textbf{Valor de $D$} & \textbf{Naturaleza} & \textbf{Ejemplo} \\
\hline
$D > 0$ & Positivo & Dos raíces reales distintas & $x^2 - 5x + 6 = 0$ \\
& & & $D = 25 - 24 = 1$ \\
\hline
$D = 0$ & Cero & Una raíz real doble & $x^2 - 4x + 4 = 0$ \\
& & & $D = 16 - 16 = 0$ \\
\hline
$D < 0$ & Negativo & No hay raíces reales & $x^2 + x + 1 = 0$ \\
& & & $D = 1 - 4 = -3$ \\
\hline
\end{tabular}
\end{center}

% Visual representation with parabolas
\begin{center}
\begin{tikzpicture}[scale=0.8]
    % D > 0 case
    \begin{scope}[xshift=0cm]
        \draw[->] (-2,0) -- (2,0) node[right] {$x$};
        \draw[->] (0,-1.5) -- (0,2.5) node[above] {$y$};
        \draw[domain=-1.8:1.3, smooth, variable=\x, blue, thick] plot ({\x}, {(\x-1)*(\x+1.5)+0.2});
        \filldraw[red] (-1.5,0) circle (2pt);
        \filldraw[red] (1,0) circle (2pt);
        \node at (0,-2.2) {\textbf{$D > 0$}};
        \node at (0,-2.8) {Dos raíces};
    \end{scope}

    % D = 0 case
    \begin{scope}[xshift=5.5cm]
        \draw[->] (-2,0) -- (2,0) node[right] {$x$};
        \draw[->] (0,-1.5) -- (0,2.5) node[above] {$y$};
        \draw[domain=-2:2, smooth, variable=\x, green!60!black, thick] plot ({\x}, {(\x)^2});
        \filldraw[red] (0,0) circle (2pt);
        \node at (0,-2.2) {\textbf{$D = 0$}};
        \node at (0,-2.8) {Una raíz doble};
    \end{scope}

    % D < 0 case
    \begin{scope}[xshift=11cm]
        \draw[->] (-2,0) -- (2,0) node[right] {$x$};
        \draw[->] (0,-1.5) -- (0,2.5) node[above] {$y$};
        \draw[domain=-2:2, smooth, variable=\x, red!70!black, thick] plot ({\x}, {(\x)^2+0.5});
        \node at (0,-2.2) {\textbf{$D < 0$}};
        \node at (0,-2.8) {Sin raíces reales};
    \end{scope}
\end{tikzpicture}
\end{center}

\begin{example}
\textbf{Ejemplo 10:} Sin resolver la ecuación, determine la naturaleza de las raíces:

\textbf{a)} $x^2 - 6x + 9 = 0$

\textbf{Solución:} $a = 1$, $b = -6$, $c = 9$
\begin{align}
D &= b^2 - 4ac\\
D &= (-6)^2 - 4(1)(9)\\
D &= 36 - 36 = 0
\end{align}
Como $D = 0$, la ecuación tiene una raíz real doble.

\textbf{b)} $2x^2 + 5x - 3 = 0$

\textbf{Solución:} $a = 2$, $b = 5$, $c = -3$
\begin{align}
D &= 5^2 - 4(2)(-3)\\
D &= 25 + 24 = 49 > 0
\end{align}
Como $D > 0$, la ecuación tiene dos raíces reales distintas.

\textbf{c)} $x^2 + 2x + 5 = 0$

\textbf{Solución:} $a = 1$, $b = 2$, $c = 5$
\begin{align}
D &= 2^2 - 4(1)(5)\\
D &= 4 - 20 = -16 < 0
\end{align}
Como $D < 0$, la ecuación no tiene raíces reales.
\end{example}

\newpage
%========================================
% SECTION 8.6: Estrategia de Solución
%========================================
\subsectiontitle{Estrategia de Solución y Comparación de Métodos}

Al resolver una ecuación cuadrática, es importante elegir el método más apropiado según la forma de la ecuación.

% Decision tree flowchart
\begin{center}
\begin{tikzpicture}[
    node distance=2cm,
    decision/.style={diamond, draw, fill=yellow!20, text width=3.5cm, text centered, inner sep=2pt, aspect=2},
    block/.style={rectangle, draw, fill=blue!20, text width=3.5cm, text centered, rounded corners, minimum height=1cm},
    result/.style={rectangle, draw, fill=green!30, text width=3cm, text centered, rounded corners},
    arrow/.style={->, thick}
]
    % Start
    \node[block, fill=purple!20] (start) {Ecuación Cuadrática\\$ax^2+bx+c=0$};

    % First decision
    \node[decision, below of=start, yshift=-1cm] (easy) {¿Es fácil de\\factorizar?};

    % Factorization path
    \node[result, below of=easy, yshift=-1.5cm, xshift=-4cm] (factor) {\textbf{Usar Factorización}\\Rápido y directo};

    % Second decision
    \node[decision, below of=easy, yshift=-1.5cm, xshift=4cm] (choose) {¿Qué método\\prefieres?};

    % Formula path
    \node[result, below of=choose, yshift=-1.5cm, xshift=-2.5cm] (formula) {\textbf{Fórmula Cuadrática}\\Siempre funciona};

    % Completing square path
    \node[result, below of=choose, yshift=-1.5cm, xshift=2.5cm] (complete) {\textbf{Completar Cuadrado}\\Útil teóricamente};

    % Arrows
    \draw[arrow] (start) -- (easy);
    \draw[arrow] (easy) -- node[left, near start] {Sí} (factor);
    \draw[arrow] (easy) -- node[above, near start] {No} (choose);
    \draw[arrow] (choose) -- (formula);
    \draw[arrow] (choose) -- (complete);
\end{tikzpicture}
\end{center}

\textbf{Comparación de métodos:}

\begin{center}
\begin{tabular}{|l|c|c|l|}
\hline
\textbf{Método} & \textbf{Velocidad} & \textbf{Aplicabilidad} & \textbf{Cuándo usar} \\
\hline
Factorización & Rápida & Limitada & Cuando se ve factorización obvia \\
\hline
Fórmula Cuadrática & Media & Universal & Cuando no es fácil factorizar \\
\hline
Completar Cuadrado & Lenta & Universal & Para derivar fórmula, análisis \\
\hline
\end{tabular}
\end{center}

\textbf{Recomendaciones generales:}

\begin{enumerate}
\item \textbf{Intenta factorizar primero} si ves patrones obvios (trinomios simples, diferencia de cuadrados)
\item \textbf{Usa la fórmula cuadrática} si la factorización no es evidente
\item \textbf{Calcula el discriminante} primero si solo necesitas saber la naturaleza de las raíces
\item \textbf{Completa el cuadrado} principalmente para propósitos teóricos o cuando se requiere específicamente
\end{enumerate}

%========================================
% SECTION 8.7: Aplicaciones
%========================================
\subsectiontitle{Aplicaciones de Ecuaciones Cuadráticas}

Las ecuaciones cuadráticas aparecen frecuentemente en problemas del mundo real.

\begin{example}
\textbf{Ejemplo 11 (Geometría):} Un rectángulo tiene un largo que es 3 cm mayor que su ancho. Si el área del rectángulo es 40 cm$^2$, encuentre las dimensiones del rectángulo.

\textbf{Solución:}

Sea $x$ = ancho del rectángulo (en cm)

Entonces: $x + 3$ = largo del rectángulo

El área está dada por:
$$\text{Área} = \text{ancho} \times \text{largo}$$
$$40 = x(x + 3)$$
$$40 = x^2 + 3x$$
$$x^2 + 3x - 40 = 0$$

Factorizamos:
$$(x + 8)(x - 5) = 0$$

Soluciones: $x = -8$ o $x = 5$

Como $x$ representa una longitud, debe ser positiva: $x = 5$ cm

\textbf{Respuesta:} Ancho = 5 cm, Largo = 8 cm
\end{example}

\begin{example}
\textbf{Ejemplo 12 (Números):} Encuentre dos números consecutivos cuyo producto sea 132.

\textbf{Solución:}

Sea $x$ = primer número

Entonces: $x + 1$ = número consecutivo

El producto es:
$$x(x + 1) = 132$$
$$x^2 + x = 132$$
$$x^2 + x - 132 = 0$$

Factorizamos:
$$(x + 12)(x - 11) = 0$$

Soluciones: $x = -12$ o $x = 11$

\textbf{Respuesta:} Los números son 11 y 12, o $-12$ y $-11$
\end{example}

\begin{example}
\textbf{Ejemplo 13 (Movimiento):} Un objeto se lanza verticalmente hacia arriba con una velocidad inicial de 64 pies/s. Su altura $h$ (en pies) después de $t$ segundos está dada por:
$$h = -16t^2 + 64t$$

¿En qué tiempo(s) el objeto estará a una altura de 48 pies?

\textbf{Solución:}

Sustituimos $h = 48$:
$$48 = -16t^2 + 64t$$
$$-16t^2 + 64t - 48 = 0$$

Dividimos por $-16$:
$$t^2 - 4t + 3 = 0$$

Factorizamos:
$$(t - 1)(t - 3) = 0$$

\textbf{Respuesta:} El objeto estará a 48 pies en $t = 1$ segundo (subiendo) y $t = 3$ segundos (bajando)
\end{example}

\textbf{Nota final:} Las ecuaciones cuadráticas son fundamentales en muchas áreas de las matemáticas y sus aplicaciones. Dominar los diferentes métodos de solución y saber cuándo usar cada uno es esencial para el éxito en cursos más avanzados.

%========================================
% EXERCISES: Ecuaciones Cuadráticas
%========================================

\section{Ejercicios}

\begin{exercise}
\problem Identifique los coeficientes $a$, $b$ y $c$ de las siguientes ecuaciones cuadráticas. Si la ecuación no está en forma estándar, escríbala primero en forma estándar.

\begin{exerciselist}
    \item $x^2 + 8x + 15 = 0$
    \item $3x^2 - 7x + 2 = 0$
    \item $x^2 - 16 = 0$
    \item $5x^2 + 2x = 3$
    \item $-2x^2 = 8x - 10$
    \item $x(x + 5) = 6$
\end{exerciselist}

\begin{solucion}
\textbf{a)} $a = 1$, $b = 8$, $c = 15$

\textbf{b)} $a = 3$, $b = -7$, $c = 2$

\textbf{c)} $a = 1$, $b = 0$, $c = -16$

\textbf{d)} Forma estándar: $5x^2 + 2x - 3 = 0$. Por tanto: $a = 5$, $b = 2$, $c = -3$

\textbf{e)} Forma estándar: $-2x^2 - 8x + 10 = 0$ o $2x^2 + 8x - 10 = 0$. Por tanto: $a = -2$, $b = -8$, $c = 10$ (o $a = 2$, $b = 8$, $c = -10$)

\textbf{f)} Expandir: $x^2 + 5x = 6$. Forma estándar: $x^2 + 5x - 6 = 0$. Por tanto: $a = 1$, $b = 5$, $c = -6$
\end{solucion}
\end{exercise}

\begin{exercise}
\problem Resuelva las siguientes ecuaciones cuadráticas por \textbf{factorización}:

\begin{exerciselist}
    \item $x^2 + 9x + 20 = 0$
    \item $x^2 - x - 12 = 0$
    \item $x^2 + 2x - 15 = 0$
    \item $x^2 - 10x + 21 = 0$
    \item $x^2 + 6x + 9 = 0$
    \item $x^2 - 49 = 0$
    \item $4x^2 - 25 = 0$
    \item $x^2 + 8x = 0$
\end{exerciselist}

\begin{solucion}
\textbf{a)} $(x + 4)(x + 5) = 0 \Rightarrow x = -4$ o $x = -5$

\textbf{b)} $(x - 4)(x + 3) = 0 \Rightarrow x = 4$ o $x = -3$

\textbf{c)} $(x + 5)(x - 3) = 0 \Rightarrow x = -5$ o $x = 3$

\textbf{d)} $(x - 3)(x - 7) = 0 \Rightarrow x = 3$ o $x = 7$

\textbf{e)} $(x + 3)^2 = 0 \Rightarrow x = -3$ (raíz doble)

\textbf{f)} $(x + 7)(x - 7) = 0 \Rightarrow x = 7$ o $x = -7$

\textbf{g)} $(2x + 5)(2x - 5) = 0 \Rightarrow x = \frac{5}{2}$ o $x = -\frac{5}{2}$

\textbf{h)} $x(x + 8) = 0 \Rightarrow x = 0$ o $x = -8$
\end{solucion}
\end{exercise}

\begin{exercise}
\problem Resuelva las siguientes ecuaciones cuadráticas por factorización:

\begin{exerciselist}
    \item $2x^2 + 7x + 3 = 0$
    \item $3x^2 - 10x + 8 = 0$
    \item $5x^2 + 13x - 6 = 0$
    \item $4x^2 - 4x - 15 = 0$
    \item $6x^2 + 7x - 3 = 0$
    \item $2x^2 - 9x + 10 = 0$
\end{exerciselist}

\begin{solucion}
\textbf{a)} $ac = 6$, buscar números que sumen 7 y multipliquen 6: son 6 y 1
\begin{align*}
2x^2 + 6x + x + 3 &= 0\\
2x(x + 3) + 1(x + 3) &= 0\\
(x + 3)(2x + 1) &= 0
\end{align*}
$x = -3$ o $x = -\frac{1}{2}$

\textbf{b)} $ac = 24$, buscar números que sumen $-10$ y multipliquen 24: son $-6$ y $-4$
\begin{align*}
3x^2 - 6x - 4x + 8 &= 0\\
3x(x - 2) - 4(x - 2) &= 0\\
(x - 2)(3x - 4) &= 0
\end{align*}
$x = 2$ o $x = \frac{4}{3}$

\textbf{c)} $ac = -30$, buscar números que sumen 13 y multipliquen $-30$: son 15 y $-2$
\begin{align*}
5x^2 + 15x - 2x - 6 &= 0\\
5x(x + 3) - 2(x + 3) &= 0\\
(x + 3)(5x - 2) &= 0
\end{align*}
$x = -3$ o $x = \frac{2}{5}$

\textbf{d)} $ac = -60$, buscar números que sumen $-4$ y multipliquen $-60$: son 6 y $-10$
\begin{align*}
4x^2 + 6x - 10x - 15 &= 0\\
2x(2x + 3) - 5(2x + 3) &= 0\\
(2x + 3)(2x - 5) &= 0
\end{align*}
$x = -\frac{3}{2}$ o $x = \frac{5}{2}$

\textbf{e)} $ac = -18$, buscar números que sumen 7 y multipliquen $-18$: son 9 y $-2$
\begin{align*}
6x^2 + 9x - 2x - 3 &= 0\\
3x(2x + 3) - 1(2x + 3) &= 0\\
(2x + 3)(3x - 1) &= 0
\end{align*}
$x = -\frac{3}{2}$ o $x = \frac{1}{3}$

\textbf{f)} $ac = 20$, buscar números que sumen $-9$ y multipliquen 20: son $-5$ y $-4$
\begin{align*}
2x^2 - 5x - 4x + 10 &= 0\\
x(2x - 5) - 2(2x - 5) &= 0\\
(2x - 5)(x - 2) &= 0
\end{align*}
$x = \frac{5}{2}$ o $x = 2$
\end{solucion}
\end{exercise}

\begin{exercise}
\problem Resuelva las siguientes ecuaciones cuadráticas \textbf{completando el cuadrado}:

\begin{exerciselist}
    \item $x^2 + 4x - 5 = 0$
    \item $x^2 - 6x + 8 = 0$
    \item $x^2 + 10x + 21 = 0$
    \item $x^2 - 2x - 8 = 0$
    \item $2x^2 + 12x + 10 = 0$
    \item $3x^2 - 6x - 9 = 0$
\end{exerciselist}

\begin{solucion}
\textbf{a)} $x^2 + 4x = 5 \Rightarrow x^2 + 4x + 4 = 5 + 4 \Rightarrow (x + 2)^2 = 9 \Rightarrow x + 2 = \pm 3$

$x = -2 + 3 = 1$ o $x = -2 - 3 = -5$

\textbf{b)} $x^2 - 6x = -8 \Rightarrow x^2 - 6x + 9 = -8 + 9 \Rightarrow (x - 3)^2 = 1 \Rightarrow x - 3 = \pm 1$

$x = 3 + 1 = 4$ o $x = 3 - 1 = 2$

\textbf{c)} $x^2 + 10x = -21 \Rightarrow x^2 + 10x + 25 = -21 + 25 \Rightarrow (x + 5)^2 = 4 \Rightarrow x + 5 = \pm 2$

$x = -5 + 2 = -3$ o $x = -5 - 2 = -7$

\textbf{d)} $x^2 - 2x = 8 \Rightarrow x^2 - 2x + 1 = 8 + 1 \Rightarrow (x - 1)^2 = 9 \Rightarrow x - 1 = \pm 3$

$x = 1 + 3 = 4$ o $x = 1 - 3 = -2$

\textbf{e)} Dividir por 2: $x^2 + 6x + 5 = 0$

$x^2 + 6x = -5 \Rightarrow x^2 + 6x + 9 = -5 + 9 \Rightarrow (x + 3)^2 = 4 \Rightarrow x + 3 = \pm 2$

$x = -3 + 2 = -1$ o $x = -3 - 2 = -5$

\textbf{f)} Dividir por 3: $x^2 - 2x - 3 = 0$

$x^2 - 2x = 3 \Rightarrow x^2 - 2x + 1 = 3 + 1 \Rightarrow (x - 1)^2 = 4 \Rightarrow x - 1 = \pm 2$

$x = 1 + 2 = 3$ o $x = 1 - 2 = -1$
\end{solucion}
\end{exercise}

\begin{exercise}
\problem Resuelva las siguientes ecuaciones cuadráticas usando la \textbf{fórmula cuadrática}:

\begin{exerciselist}
    \item $x^2 + 3x - 10 = 0$
    \item $x^2 - 7x + 12 = 0$
    \item $2x^2 + x - 6 = 0$
    \item $3x^2 - 5x - 2 = 0$
    \item $x^2 + 4x + 1 = 0$
    \item $2x^2 - 6x + 1 = 0$
    \item $x^2 - 2x - 5 = 0$
    \item $4x^2 + 4x - 3 = 0$
\end{exerciselist}

\begin{solucion}
\textbf{a)} $a = 1, b = 3, c = -10$
$$x = \frac{-3 \pm \sqrt{9 + 40}}{2} = \frac{-3 \pm \sqrt{49}}{2} = \frac{-3 \pm 7}{2}$$
$x = 2$ o $x = -5$

\textbf{b)} $a = 1, b = -7, c = 12$
$$x = \frac{7 \pm \sqrt{49 - 48}}{2} = \frac{7 \pm 1}{2}$$
$x = 4$ o $x = 3$

\textbf{c)} $a = 2, b = 1, c = -6$
$$x = \frac{-1 \pm \sqrt{1 + 48}}{4} = \frac{-1 \pm 7}{4}$$
$x = \frac{3}{2}$ o $x = -2$

\textbf{d)} $a = 3, b = -5, c = -2$
$$x = \frac{5 \pm \sqrt{25 + 24}}{6} = \frac{5 \pm 7}{6}$$
$x = 2$ o $x = -\frac{1}{3}$

\textbf{e)} $a = 1, b = 4, c = 1$
$$x = \frac{-4 \pm \sqrt{16 - 4}}{2} = \frac{-4 \pm \sqrt{12}}{2} = \frac{-4 \pm 2\sqrt{3}}{2} = -2 \pm \sqrt{3}$$

\textbf{f)} $a = 2, b = -6, c = 1$
$$x = \frac{6 \pm \sqrt{36 - 8}}{4} = \frac{6 \pm \sqrt{28}}{4} = \frac{6 \pm 2\sqrt{7}}{4} = \frac{3 \pm \sqrt{7}}{2}$$

\textbf{g)} $a = 1, b = -2, c = -5$
$$x = \frac{2 \pm \sqrt{4 + 20}}{2} = \frac{2 \pm \sqrt{24}}{2} = \frac{2 \pm 2\sqrt{6}}{2} = 1 \pm \sqrt{6}$$

\textbf{h)} $a = 4, b = 4, c = -3$
$$x = \frac{-4 \pm \sqrt{16 + 48}}{8} = \frac{-4 \pm 8}{8}$$
$x = \frac{1}{2}$ o $x = -\frac{3}{2}$
\end{solucion}
\end{exercise}

\begin{exercise}
\problem Calcule el discriminante de cada ecuación cuadrática y determine la naturaleza de sus raíces (sin resolverlas):

\begin{exerciselist}
    \item $x^2 + 5x + 6 = 0$
    \item $x^2 - 8x + 16 = 0$
    \item $x^2 + 3x + 5 = 0$
    \item $2x^2 - 7x + 3 = 0$
    \item $9x^2 + 6x + 1 = 0$
    \item $x^2 - x + 1 = 0$
\end{exerciselist}

\begin{solucion}
\textbf{a)} $D = 25 - 24 = 1 > 0$. Dos raíces reales distintas.

\textbf{b)} $D = 64 - 64 = 0$. Una raíz real doble.

\textbf{c)} $D = 9 - 20 = -11 < 0$. No hay raíces reales.

\textbf{d)} $D = 49 - 24 = 25 > 0$. Dos raíces reales distintas.

\textbf{e)} $D = 36 - 36 = 0$. Una raíz real doble.

\textbf{f)} $D = 1 - 4 = -3 < 0$. No hay raíces reales.
\end{solucion}
\end{exercise}

\begin{exercise}
\problem Resuelva las siguientes ecuaciones que requieren primero escribirlas en forma estándar:

\begin{exerciselist}
    \item $x(x + 5) = -6$
    \item $2x(x - 3) = 8$
    \item $(x + 2)(x - 3) = 6$
    \item $x^2 = 4(x - 3)$
    \item $(x + 1)^2 = 9$
    \item $(2x - 1)^2 = 25$
\end{exerciselist}

\begin{solucion}
\textbf{a)} $x^2 + 5x + 6 = 0 \Rightarrow (x + 2)(x + 3) = 0$

$x = -2$ o $x = -3$

\textbf{b)} $2x^2 - 6x = 8 \Rightarrow x^2 - 3x - 4 = 0 \Rightarrow (x - 4)(x + 1) = 0$

$x = 4$ o $x = -1$

\textbf{c)} $x^2 - x - 6 = 6 \Rightarrow x^2 - x - 12 = 0 \Rightarrow (x - 4)(x + 3) = 0$

$x = 4$ o $x = -3$

\textbf{d)} $x^2 = 4x - 12 \Rightarrow x^2 - 4x + 12 = 0$

$D = 16 - 48 = -32 < 0$. No hay soluciones reales.

\textbf{e)} $x + 1 = \pm 3$

$x = 2$ o $x = -4$

\textbf{f)} $2x - 1 = \pm 5$

$2x = 1 + 5 = 6 \Rightarrow x = 3$ o $2x = 1 - 5 = -4 \Rightarrow x = -2$
\end{solucion}
\end{exercise}

\begin{exercise}
\problem \textbf{Problemas de aplicación:}

\problem Un rectángulo tiene un largo que es 4 cm mayor que su ancho. Si el área del rectángulo es 60 cm$^2$, encuentre las dimensiones del rectángulo.

\begin{solucion}
Sea $x$ = ancho. Entonces $x + 4$ = largo.

Área: $x(x + 4) = 60$
$$x^2 + 4x = 60$$
$$x^2 + 4x - 60 = 0$$
$$(x + 10)(x - 6) = 0$$

$x = 6$ (descartamos $x = -10$ por ser negativo)

\textbf{Respuesta:} Ancho = 6 cm, Largo = 10 cm
\end{solucion}

\problem La suma de dos números es 12 y su producto es 35. Encuentre los números.

\begin{solucion}
Sea $x$ = primer número. Entonces $12 - x$ = segundo número.

Producto: $x(12 - x) = 35$
$$12x - x^2 = 35$$
$$x^2 - 12x + 35 = 0$$
$$(x - 5)(x - 7) = 0$$

$x = 5$ o $x = 7$

\textbf{Respuesta:} Los números son 5 y 7
\end{solucion}

\problem Un número positivo es 3 menor que otro número positivo. Si la suma de sus cuadrados es 89, encuentre los números.

\begin{solucion}
Sea $x$ = número mayor. Entonces $x - 3$ = número menor.

$$x^2 + (x - 3)^2 = 89$$
$$x^2 + x^2 - 6x + 9 = 89$$
$$2x^2 - 6x - 80 = 0$$
$$x^2 - 3x - 40 = 0$$
$$(x - 8)(x + 5) = 0$$

$x = 8$ (descartamos $x = -5$ por requerir número positivo)

\textbf{Respuesta:} Los números son 8 y 5
\end{solucion}

\problem El perímetro de un rectángulo es 28 cm y su área es 48 cm$^2$. Encuentre las dimensiones del rectángulo.

\begin{solucion}
Sea $x$ = ancho y $y$ = largo.

Perímetro: $2x + 2y = 28 \Rightarrow x + y = 14 \Rightarrow y = 14 - x$

Área: $xy = 48$

Sustituyendo:
$$x(14 - x) = 48$$
$$14x - x^2 = 48$$
$$x^2 - 14x + 48 = 0$$
$$(x - 6)(x - 8) = 0$$

$x = 6$ o $x = 8$

\textbf{Respuesta:} Las dimensiones son 6 cm × 8 cm
\end{solucion}

\problem Un objeto se lanza verticalmente hacia arriba desde el suelo con una velocidad inicial de 48 pies/s. Su altura $h$ (en pies) después de $t$ segundos está dada por $h = -16t^2 + 48t$. ¿En qué momento alcanza una altura de 32 pies?

\begin{solucion}
Sustituimos $h = 32$:
$$32 = -16t^2 + 48t$$
$$-16t^2 + 48t - 32 = 0$$

Dividimos por $-16$:
$$t^2 - 3t + 2 = 0$$
$$(t - 1)(t - 2) = 0$$

\textbf{Respuesta:} En $t = 1$ segundo (subiendo) y $t = 2$ segundos (bajando)
\end{solucion}

\problem Un granjero tiene 100 metros de cerca para encerrar un área rectangular. Si el área encerrada es 600 m$^2$, encuentre las dimensiones del rectángulo.

\begin{solucion}
Sea $x$ = ancho. Perímetro: $2x + 2y = 100 \Rightarrow x + y = 50 \Rightarrow y = 50 - x$

Área: $x(50 - x) = 600$
$$50x - x^2 = 600$$
$$x^2 - 50x + 600 = 0$$

Usando la fórmula cuadrática:
$$x = \frac{50 \pm \sqrt{2500 - 2400}}{2} = \frac{50 \pm 10}{2}$$

$x = 30$ o $x = 20$

\textbf{Respuesta:} Las dimensiones son 20 m × 30 m
\end{solucion}
\end{exercise}

\begin{exercise}
\problem \textbf{Problemas desafiantes:}

\begin{exerciselist}
    \item Resuelva: $\frac{x}{x-2} + \frac{x-1}{x} = 2$

    \item Resuelva: $(x + 1)^2 = 3(x + 1)$

    \item Encuentre el valor de $k$ para que la ecuación $x^2 + kx + 9 = 0$ tenga una raíz doble.

    \item Encuentre el valor de $k$ para que la ecuación $kx^2 - 6x + 2 = 0$ tenga dos raíces reales distintas.
\end{exerciselist}

\begin{solucion}
\textbf{a)} Multiplicamos por $x(x-2)$:
$$x^2 + (x-1)(x-2) = 2x(x-2)$$
$$x^2 + x^2 - 3x + 2 = 2x^2 - 4x$$
$$2x^2 - 3x + 2 = 2x^2 - 4x$$
$$x = 2$$

Pero $x = 2$ hace el denominador cero, por tanto no hay solución válida.

\textbf{b)} $(x + 1)^2 - 3(x + 1) = 0$
$$(x + 1)[(x + 1) - 3] = 0$$
$$(x + 1)(x - 2) = 0$$

$x = -1$ o $x = 2$

\textbf{c)} Para una raíz doble, $D = 0$:
$$k^2 - 4(1)(9) = 0$$
$$k^2 = 36$$
$$k = \pm 6$$

\textbf{d)} Para dos raíces reales distintas, $D > 0$:
$$(-6)^2 - 4(k)(2) > 0$$
$$36 - 8k > 0$$
$$k < 4.5$$

También se requiere $k \neq 0$ (para que sea cuadrática).
\end{solucion}
\end{exercise}


% Conditional solution inclusion
\ifshowsolutions
    \newpage
    \section*{Soluciones}
    %========================================
% DETAILED SOLUTIONS: Ecuaciones Cuadráticas
%========================================

\subsection*{Ejercicio 1: Identificación de Coeficientes}

\textbf{Problema 1a:} $x^2 + 8x + 15 = 0$

La ecuación ya está en forma estándar. Identificamos directamente:
\begin{itemize}
\item $a = 1$ (coeficiente de $x^2$)
\item $b = 8$ (coeficiente de $x$)
\item $c = 15$ (término constante)
\end{itemize}

\textbf{Problema 1b:} $3x^2 - 7x + 2 = 0$

Ya está en forma estándar:
\begin{itemize}
\item $a = 3$
\item $b = -7$ (notar el signo negativo)
\item $c = 2$
\end{itemize}

\textbf{Problema 1c:} $x^2 - 16 = 0$

Esta ecuación no tiene término lineal. Ya está en forma estándar:
\begin{itemize}
\item $a = 1$
\item $b = 0$ (el coeficiente de $x$ es cero)
\item $c = -16$
\end{itemize}

\textbf{Problema 1d:} $5x^2 + 2x = 3$

Primero escribimos en forma estándar restando 3 de ambos lados:
$$5x^2 + 2x - 3 = 0$$
Por tanto:
\begin{itemize}
\item $a = 5$
\item $b = 2$
\item $c = -3$
\end{itemize}

\textbf{Problema 1e:} $-2x^2 = 8x - 10$

Movemos todos los términos al lado izquierdo:
\begin{align}
-2x^2 - 8x + 10 &= 0
\end{align}
Por tanto: $a = -2$, $b = -8$, $c = 10$

Alternativamente, podríamos multiplicar por $-1$:
$$2x^2 + 8x - 10 = 0$$
Dando: $a = 2$, $b = 8$, $c = -10$ (ambas formas son correctas)

\textbf{Problema 1f:} $x(x + 5) = 6$

Primero expandimos el producto:
\begin{align}
x^2 + 5x &= 6\\
x^2 + 5x - 6 &= 0
\end{align}
Por tanto: $a = 1$, $b = 5$, $c = -6$

%========================================

\subsection*{Ejercicio 2: Factorización Simple}

\textbf{Problema 2a:} $x^2 + 9x + 20 = 0$

Buscamos dos números que sumen 9 y multipliquen 20. Probamos factores de 20:
\begin{itemize}
\item $1 \times 20 = 20$, pero $1 + 20 = 21$ ✗
\item $2 \times 10 = 20$, pero $2 + 10 = 12$ ✗
\item $4 \times 5 = 20$, y $4 + 5 = 9$ $\checkmark$
\end{itemize}

Factorizamos:
$$(x + 4)(x + 5) = 0$$

Aplicamos el principio del producto cero:
\begin{align}
x + 4 &= 0 \quad \Rightarrow \quad x = -4\\
x + 5 &= 0 \quad \Rightarrow \quad x = -5
\end{align}

\textbf{Verificación para $x = -4$:}
$$(-4)^2 + 9(-4) + 20 = 16 - 36 + 20 = 0$$ $\checkmark$

\textbf{Respuesta:} $x = -4$ o $x = -5$

\textbf{Problema 2b:} $x^2 - x - 12 = 0$

Buscamos dos números que sumen $-1$ y multipliquen $-12$:
\begin{itemize}
\item $3 \times (-4) = -12$, y $3 + (-4) = -1$ $\checkmark$
\end{itemize}

Factorizamos:
$$(x + 3)(x - 4) = 0$$

Soluciones:
$$x = -3 \quad \text{o} \quad x = 4$$

\textbf{Problema 2e:} $x^2 + 6x + 9 = 0$

Reconocemos un trinomio cuadrado perfecto:
$$x^2 + 6x + 9 = (x + 3)^2 = 0$$

Esto da una raíz doble:
$$(x + 3)^2 = 0 \quad \Rightarrow \quad x + 3 = 0 \quad \Rightarrow \quad x = -3$$

\textbf{Nota:} Cuando hay una raíz doble, la parábola toca el eje $x$ en un solo punto (el vértice).

\textbf{Problema 2f:} $x^2 - 49 = 0$

Esta es una diferencia de cuadrados: $x^2 - 7^2$

Factorizamos usando la fórmula $a^2 - b^2 = (a+b)(a-b)$:
$$(x + 7)(x - 7) = 0$$

Soluciones:
$$x = -7 \quad \text{o} \quad x = 7$$

\textbf{Problema 2g:} $4x^2 - 25 = 0$

Otra diferencia de cuadrados: $(2x)^2 - 5^2$

Factorizamos:
$$(2x + 5)(2x - 5) = 0$$

Resolvemos:
\begin{align}
2x + 5 &= 0 \quad \Rightarrow \quad x = -\frac{5}{2}\\
2x - 5 &= 0 \quad \Rightarrow \quad x = \frac{5}{2}
\end{align}

\textbf{Problema 2h:} $x^2 + 8x = 0$

Factorizamos el término común $x$:
$$x(x + 8) = 0$$

Soluciones:
$$x = 0 \quad \text{o} \quad x = -8$$

\textbf{Nota importante:} No dividir por $x$. Siempre factorizar para no perder la solución $x = 0$.

%========================================

\subsection*{Ejercicio 3: Método AC}

\textbf{Problema 3a:} $2x^2 + 7x + 3 = 0$

\textbf{Paso 1:} Calculamos $ac = 2 \times 3 = 6$

\textbf{Paso 2:} Buscamos dos números que sumen $b = 7$ y multipliquen $ac = 6$:
\begin{itemize}
\item Factores de 6: 1 y 6
\item Verificamos: $1 + 6 = 7$ $\checkmark$ y $1 \times 6 = 6$ $\checkmark$
\end{itemize}

\textbf{Paso 3:} Reescribimos $7x$ como $6x + x$:
$$2x^2 + 6x + x + 3 = 0$$

\textbf{Paso 4:} Agrupamos y factorizamos:
\begin{align}
(2x^2 + 6x) + (x + 3) &= 0\\
2x(x + 3) + 1(x + 3) &= 0\\
(x + 3)(2x + 1) &= 0
\end{align}

\textbf{Paso 5:} Resolvemos:
\begin{align}
x + 3 &= 0 \quad \Rightarrow \quad x = -3\\
2x + 1 &= 0 \quad \Rightarrow \quad x = -\frac{1}{2}
\end{align}

\textbf{Problema 3b:} $3x^2 - 10x + 8 = 0$

$ac = 3 \times 8 = 24$

Buscamos dos números que sumen $-10$ y multipliquen $24$:
\begin{itemize}
\item $-6 \times (-4) = 24$ y $-6 + (-4) = -10$ $\checkmark$
\end{itemize}

Reescribimos y factorizamos:
\begin{align}
3x^2 - 6x - 4x + 8 &= 0\\
3x(x - 2) - 4(x - 2) &= 0\\
(x - 2)(3x - 4) &= 0
\end{align}

Soluciones: $x = 2$ o $x = \frac{4}{3}$

\textbf{Problema 3d:} $4x^2 - 4x - 15 = 0$

$ac = 4 \times (-15) = -60$

Buscamos dos números que sumen $-4$ y multipliquen $-60$:
\begin{itemize}
\item $6 \times (-10) = -60$ y $6 + (-10) = -4$ $\checkmark$
\end{itemize}

\begin{align}
4x^2 + 6x - 10x - 15 &= 0\\
2x(2x + 3) - 5(2x + 3) &= 0\\
(2x + 3)(2x - 5) &= 0
\end{align}

Soluciones: $x = -\frac{3}{2}$ o $x = \frac{5}{2}$

%========================================

\subsection*{Ejercicio 4: Completar el Cuadrado}

\textbf{Problema 4a:} $x^2 + 4x - 5 = 0$

\textbf{Paso 1:} Movemos el término constante:
$$x^2 + 4x = 5$$

\textbf{Paso 2:} Calculamos $\left(\frac{b}{2}\right)^2 = \left(\frac{4}{2}\right)^2 = 4$

\textbf{Paso 3:} Sumamos 4 a ambos lados:
$$x^2 + 4x + 4 = 5 + 4$$

\textbf{Paso 4:} Factorizamos el lado izquierdo:
$$(x + 2)^2 = 9$$

\textbf{Paso 5:} Tomamos raíz cuadrada:
$$x + 2 = \pm 3$$

\textbf{Paso 6:} Resolvemos:
\begin{align}
x + 2 &= 3 \quad \Rightarrow \quad x = 1\\
x + 2 &= -3 \quad \Rightarrow \quad x = -5
\end{align}

\textbf{Respuesta:} $x = 1$ o $x = -5$

\textbf{Problema 4e:} $2x^2 + 12x + 10 = 0$

\textbf{Paso 1:} Dividimos por 2 para que $a = 1$:
$$x^2 + 6x + 5 = 0$$

\textbf{Paso 2:} Movemos el término constante:
$$x^2 + 6x = -5$$

\textbf{Paso 3:} Calculamos $\left(\frac{6}{2}\right)^2 = 9$ y sumamos:
$$x^2 + 6x + 9 = -5 + 9 = 4$$

\textbf{Paso 4:} Factorizamos:
$$(x + 3)^2 = 4$$

\textbf{Paso 5:} Tomamos raíz cuadrada:
$$x + 3 = \pm 2$$

\textbf{Paso 6:} Resolvemos:
$$x = -3 + 2 = -1 \quad \text{o} \quad x = -3 - 2 = -5$$

%========================================

\subsection*{Ejercicio 5: Fórmula Cuadrática}

\textbf{Problema 5a:} $x^2 + 3x - 10 = 0$

Identificamos: $a = 1$, $b = 3$, $c = -10$

Aplicamos la fórmula:
$$x = \frac{-b \pm \sqrt{b^2 - 4ac}}{2a}$$

Sustituimos:
\begin{align}
x &= \frac{-3 \pm \sqrt{3^2 - 4(1)(-10)}}{2(1)}\\
x &= \frac{-3 \pm \sqrt{9 + 40}}{2}\\
x &= \frac{-3 \pm \sqrt{49}}{2}\\
x &= \frac{-3 \pm 7}{2}
\end{align}

Calculamos ambas soluciones:
\begin{align}
x_1 &= \frac{-3 + 7}{2} = \frac{4}{2} = 2\\
x_2 &= \frac{-3 - 7}{2} = \frac{-10}{2} = -5
\end{align}

\textbf{Verificación para $x = 2$:}
$$2^2 + 3(2) - 10 = 4 + 6 - 10 = 0$$ $\checkmark$

\textbf{Problema 5e:} $x^2 + 4x + 1 = 0$

Identificamos: $a = 1$, $b = 4$, $c = 1$

\begin{align}
x &= \frac{-4 \pm \sqrt{16 - 4}}{2}\\
x &= \frac{-4 \pm \sqrt{12}}{2}\\
x &= \frac{-4 \pm 2\sqrt{3}}{2}\\
x &= \frac{2(-2 \pm \sqrt{3})}{2}\\
x &= -2 \pm \sqrt{3}
\end{align}

\textbf{Respuesta:} $x = -2 + \sqrt{3} \approx -0.268$ o $x = -2 - \sqrt{3} \approx -3.732$

\textbf{Nota:} Estas son soluciones irracionales exactas. Es importante simplificar los radicales completamente.

\textbf{Problema 5g:} $x^2 - 2x - 5 = 0$

Identificamos: $a = 1$, $b = -2$, $c = -5$

\begin{align}
x &= \frac{-(-2) \pm \sqrt{(-2)^2 - 4(1)(-5)}}{2(1)}\\
x &= \frac{2 \pm \sqrt{4 + 20}}{2}\\
x &= \frac{2 \pm \sqrt{24}}{2}\\
x &= \frac{2 \pm 2\sqrt{6}}{2}\\
x &= 1 \pm \sqrt{6}
\end{align}

%========================================

\subsection*{Ejercicio 6: El Discriminante}

\textbf{Problema 6a:} $x^2 + 5x + 6 = 0$

Calculamos el discriminante:
$$D = b^2 - 4ac = 5^2 - 4(1)(6) = 25 - 24 = 1$$

Como $D = 1 > 0$, la ecuación tiene \textbf{dos raíces reales distintas}.

Podemos confirmar factorizando: $(x+2)(x+3) = 0$, dando $x = -2$ y $x = -3$.

\textbf{Problema 6b:} $x^2 - 8x + 16 = 0$

$$D = (-8)^2 - 4(1)(16) = 64 - 64 = 0$$

Como $D = 0$, la ecuación tiene \textbf{una raíz real doble}.

Confirmación: $x^2 - 8x + 16 = (x-4)^2 = 0$, dando $x = 4$ (doble).

\textbf{Problema 6c:} $x^2 + 3x + 5 = 0$

$$D = 3^2 - 4(1)(5) = 9 - 20 = -11$$

Como $D = -11 < 0$, la ecuación \textbf{no tiene raíces reales}.

La parábola $y = x^2 + 3x + 5$ no interseca el eje $x$.

\textbf{Problema 6e:} $9x^2 + 6x + 1 = 0$

$$D = 6^2 - 4(9)(1) = 36 - 36 = 0$$

Raíz doble. Confirmación: $9x^2 + 6x + 1 = (3x+1)^2 = 0$, dando $x = -\frac{1}{3}$ (doble).

%========================================

\subsection*{Ejercicio 7: Forma Estándar}

\textbf{Problema 7c:} $(x + 2)(x - 3) = 6$

\textbf{Error común:} No igualar cada factor a 6. Primero debemos expandir.

\textbf{Paso 1:} Expandir:
$$x^2 - 3x + 2x - 6 = 6$$
$$x^2 - x - 6 = 6$$

\textbf{Paso 2:} Forma estándar:
$$x^2 - x - 12 = 0$$

\textbf{Paso 3:} Factorizar:
$$(x - 4)(x + 3) = 0$$

Soluciones: $x = 4$ o $x = -3$

\textbf{Problema 7d:} $x^2 = 4(x - 3)$

\textbf{Paso 1:} Expandir:
$$x^2 = 4x - 12$$

\textbf{Paso 2:} Forma estándar:
$$x^2 - 4x + 12 = 0$$

\textbf{Paso 3:} Calcular discriminante:
$$D = (-4)^2 - 4(1)(12) = 16 - 48 = -32 < 0$$

\textbf{Conclusión:} No hay soluciones reales.

%========================================

\subsection*{Ejercicio 8: Aplicaciones - Soluciones Detalladas}

\textbf{Problema 8.1:} Rectángulo con largo 4 cm mayor que ancho, área 60 cm$^2$

\textbf{Paso 1 - Definir variable:}
\begin{itemize}
\item Sea $x$ = ancho del rectángulo (cm)
\item Entonces $x + 4$ = largo del rectángulo (cm)
\end{itemize}

\textbf{Paso 2 - Escribir ecuación:}
$$\text{Área} = \text{largo} \times \text{ancho}$$
$$60 = (x+4) \cdot x$$
$$60 = x^2 + 4x$$

\textbf{Paso 3 - Forma estándar:}
$$x^2 + 4x - 60 = 0$$

\textbf{Paso 4 - Factorizar:}

Buscamos dos números que sumen 4 y multipliquen $-60$: son 10 y $-6$
$$(x + 10)(x - 6) = 0$$

\textbf{Paso 5 - Resolver:}
$$x = -10 \quad \text{o} \quad x = 6$$

\textbf{Paso 6 - Interpretar:}

Como $x$ representa una longitud física, debe ser positiva. Descartamos $x = -10$.

Por tanto: $x = 6$ cm (ancho) y $x + 4 = 10$ cm (largo)

\textbf{Verificación:}
$$\text{Área} = 6 \times 10 = 60 \text{ cm}^2$$ $\checkmark$

\textbf{Problema 8.3:} Dos números positivos con diferencia 3, suma de cuadrados 89

\textbf{Paso 1 - Variables:}
\begin{itemize}
\item $x$ = número mayor
\item $x - 3$ = número menor
\end{itemize}

\textbf{Paso 2 - Ecuación:}
$$x^2 + (x-3)^2 = 89$$

\textbf{Paso 3 - Expandir:}
$$x^2 + x^2 - 6x + 9 = 89$$
$$2x^2 - 6x + 9 = 89$$

\textbf{Paso 4 - Forma estándar:}
$$2x^2 - 6x - 80 = 0$$

Dividir por 2:
$$x^2 - 3x - 40 = 0$$

\textbf{Paso 5 - Factorizar:}

Buscamos dos números que sumen $-3$ y multipliquen $-40$: son 5 y $-8$
$$(x + 5)(x - 8) = 0$$

\textbf{Paso 6 - Soluciones:}
$$x = -5 \quad \text{o} \quad x = 8$$

Como necesitamos números positivos, $x = 8$.

\textbf{Respuesta:} Los números son 8 y 5

\textbf{Verificación:}
$$8^2 + 5^2 = 64 + 25 = 89$$ $\checkmark$

\textbf{Problema 8.5:} Objeto lanzado verticalmente, $h = -16t^2 + 48t$, altura 32 pies

\textbf{Interpretación física:}

El objeto alcanza 32 pies dos veces: una vez subiendo y otra vez bajando.

\textbf{Paso 1 - Sustituir:}
$$32 = -16t^2 + 48t$$

\textbf{Paso 2 - Forma estándar:}
$$-16t^2 + 48t - 32 = 0$$

\textbf{Paso 3 - Simplificar (dividir por $-16$):}
$$t^2 - 3t + 2 = 0$$

\textbf{Paso 4 - Factorizar:}
$$(t - 1)(t - 2) = 0$$

\textbf{Paso 5 - Soluciones:}
$$t = 1 \text{ segundo} \quad \text{o} \quad t = 2 \text{ segundos}$$

\textbf{Interpretación:}
\begin{itemize}
\item En $t = 1$ s: el objeto está a 32 pies \textbf{subiendo}
\item En $t = 2$ s: el objeto está a 32 pies \textbf{bajando}
\end{itemize}

El objeto alcanza su altura máxima en $t = 1.5$ s (punto medio).

%========================================

\subsection*{Ejercicio 9: Problemas Desafiantes}

\textbf{Problema 9c:} Encuentre $k$ para que $x^2 + kx + 9 = 0$ tenga una raíz doble

\textbf{Concepto:} Una ecuación tiene raíz doble cuando su discriminante es cero.

\textbf{Solución:}

Para $ax^2 + bx + c = 0$: $a = 1$, $b = k$, $c = 9$

Discriminante:
$$D = b^2 - 4ac = k^2 - 4(1)(9) = k^2 - 36$$

Para raíz doble, $D = 0$:
$$k^2 - 36 = 0$$
$$k^2 = 36$$
$$k = \pm 6$$

\textbf{Verificación con $k = 6$:}
$$x^2 + 6x + 9 = (x+3)^2 = 0$$ $\checkmark$ (raíz doble en $x = -3$)

\textbf{Verificación con $k = -6$:}
$$x^2 - 6x + 9 = (x-3)^2 = 0$$ $\checkmark$ (raíz doble en $x = 3$)

\textbf{Respuesta:} $k = 6$ o $k = -6$

\textbf{Problema 9d:} Encuentre $k$ para que $kx^2 - 6x + 2 = 0$ tenga dos raíces reales distintas

\textbf{Concepto:} Dos raíces reales distintas requieren $D > 0$ y $a \neq 0$.

\textbf{Solución:}

Para $kx^2 - 6x + 2 = 0$: $a = k$, $b = -6$, $c = 2$

Discriminante:
$$D = (-6)^2 - 4(k)(2) = 36 - 8k$$

Para dos raíces reales distintas:
$$D > 0$$
$$36 - 8k > 0$$
$$36 > 8k$$
$$k < \frac{36}{8} = 4.5$$

Además, para que sea ecuación cuadrática: $k \neq 0$

\textbf{Respuesta:} $k < 4.5$ y $k \neq 0$

En notación de intervalos: $k \in (-\infty, 0) \cup (0, 4.5)$

\fi

\end{document}
