\documentclass[12pt]{article}

%========================================
% PACKAGES AND CONFIGURATION
%========================================
\usepackage[utf8]{inputenc}
\usepackage[spanish]{babel}              % Spanish language support
\decimalpoint                                % Force decimal point instead of comma
\usepackage[margin=1in]{geometry}
\setlength{\headheight}{15pt}
\usepackage{amsmath, amsthm, amssymb}
\usepackage{mdframed}
\usepackage{xcolor}
\usepackage{enumitem}
\usepackage{fancyhdr}
\usepackage{graphicx}
\usepackage{tikz}                        % For LaTeX-generated diagrams
\usetikzlibrary{arrows.meta}           % For arrow styles
\usetikzlibrary{shapes}                % For diamond and other shapes
\usetikzlibrary{decorations.pathreplacing} % For braces and decorations
\usetikzlibrary{calc}                  % For coordinate calculations
\usepackage{comment}                     % For conditional content exclusion

% Fix Spanish babel conflicts with TikZ
\usetikzlibrary{babel}

%========================================
% COURSE CUSTOMIZATION SECTION
%========================================
% MODIFY THESE FOR EACH COURSE:
\newcommand{\coursecode}{MATE 3001}      % Course code
\newcommand{\coursename}{Matemática Elemental}     % Course name
\newcommand{\institution}{UPR-Humacao}   % Institution name
\newcommand{\lessontitle}{Sistema de Dos Ecuaciones Lineales}    % Will be overridden per lesson

%========================================
% COLOR SCHEME DEFINITIONS
%========================================
\definecolor{defcolor}{RGB}{240,248,255}     % Light blue for definitions
\definecolor{examplecolor}{RGB}{245,255,245} % Light green for examples
\definecolor{exercisecolor}{RGB}{255,248,240} % Light orange for exercises
\definecolor{theoremcolor}{RGB}{255,250,240}  % Light orange for theorems

%========================================
% CUSTOM ENVIRONMENTS
%========================================
% Definition Environment (Blue)
\newmdenv[
    backgroundcolor=defcolor,
    linecolor=blue!50,
    linewidth=2pt,
    leftmargin=10pt,
    rightmargin=10pt,
    innertopmargin=10pt,
    innerbottommargin=10pt,
    frametitle={\textbf{Definición}},
    frametitlealignment=\raggedright
]{definition}

% Example Environment (Green)
\newmdenv[
    backgroundcolor=examplecolor,
    linecolor=green!50,
    linewidth=2pt,
    leftmargin=10pt,
    rightmargin=10pt,
    innertopmargin=10pt,
    innerbottommargin=10pt,
    frametitle={\textbf{Ejemplo}},
    frametitlealignment=\raggedright
]{example}

% Exercise Environment (Orange)
\newcounter{exercise}[section]
\newcounter{problem}[exercise]
\newmdenv[
    backgroundcolor=exercisecolor,
    linecolor=orange!50,
    linewidth=2pt,
    leftmargin=10pt,
    rightmargin=10pt,
    innertopmargin=10pt,
    innerbottommargin=10pt,
    frametitle={\stepcounter{exercise}\textbf{Ejercicio \theexercise}},
    frametitlealignment=\raggedright
]{exercise}

% Theorem Environment (Orange variant)
\newmdenv[
    backgroundcolor=theoremcolor,
    linecolor=orange!50,
    linewidth=2pt,
    leftmargin=10pt,
    rightmargin=10pt,
    innertopmargin=10pt,
    innerbottommargin=10pt
]{theorem}

%========================================
% HEADER AND FOOTER CONFIGURATION
%========================================
\pagestyle{fancy}
\fancyhf{}
\rhead{\coursecode\ - \coursename}
\lhead{\lessontitle}
\cfoot{\thepage}

%========================================
% CUSTOM COMMANDS
%========================================
\newcommand{\lesson}[1]{\renewcommand{\lessontitle}{#1}\section{#1}}
\newcommand{\subsectiontitle}[1]{\subsection{#1}}

% Exercise numbering commands
\newcommand{\problem}{\stepcounter{problem}\textbf{\theproblem.} }
\newcommand{\solution}{\textbf{Solución:} }

% Custom environment for exercise lists
\newenvironment{exerciselist}
    {\begin{enumerate}[label=\textbf{\alph*.}]}
    {\end{enumerate}}

% Solution environment with conditional display
\newif\ifshowsolutions
% \showsolutionstrue  % Uncomment for instructor version
\showsolutionsfalse   % Default: student version

\ifshowsolutions
    \newenvironment{solucion}[1][Solución]
      {\par\medskip\noindent\textbf{#1:}\par\nopagebreak}
      {\par\medskip}
\else
    \excludecomment{solucion}
\fi

%========================================
% GRAPHICS CONFIGURATION
%========================================
\graphicspath{{../images/}{../images/shared/}{../images/diagrams/}{../images/12_sistema_dos_ecuaciones/}}

%========================================
% DOCUMENT CONTENT
%========================================
\begin{document}

% Title page
\title{\lessontitle}
\author{\coursecode\ - \coursename}
\date{}
\maketitle

% Set section counter to lesson number
\setcounter{section}{11}

% Modular content inclusion
%========================================
% LESSON CONTENT: Sistema de Dos Ecuaciones Lineales
%========================================

\lesson{Sistema de Dos Ecuaciones Lineales}

%========================================
% SECTION 12.1: Introducción
%========================================
\subsectiontitle{Introducción a los Sistemas de Ecuaciones Lineales}

\begin{definition}
\textbf{Sistema de Ecuaciones Lineales:}

Un \textbf{sistema de ecuaciones} es un conjunto de ecuaciones con las mismas incógnitas.

En un sistema lineal, cada ecuación es una ecuación lineal.

Una \textbf{solución} de un sistema es una asignación de valores para las incógnitas que hace verdadera cada una de las ecuaciones del sistema.

\textbf{Resolver un sistema} significa hallar todas las soluciones del sistema.
\end{definition}

\textbf{Forma General de un Sistema de Dos Ecuaciones Lineales con Dos Variables:}

$$\begin{cases}
a_1x + b_1y = c_1 \\
a_2x + b_2y = c_2
\end{cases}$$

donde $a_1$, $b_1$, $c_1$, $a_2$, $b_2$, $c_2$ son constantes y $x$, $y$ son las variables.

\begin{example}
\textbf{Sistema de dos ecuaciones lineales}

Considere el sistema:

$$\begin{cases}
2x - y = 5 \\
x + 4y = 7
\end{cases}$$

Este es un sistema de 2 ecuaciones lineales con 2 variables.

\textbf{Verificar que $(3, 1)$ es una solución:}

Para $x = 3$ y $y = 1$:

\textbf{Primera ecuación:}
$$2(3) - 1 = 6 - 1 = 5 \quad \checkmark$$

\textbf{Segunda ecuación:}
$$3 + 4(1) = 3 + 4 = 7 \quad \checkmark$$

Como ambas ecuaciones son verdaderas, $(3, 1)$ \textbf{es una solución} del sistema.
\end{example}

\textbf{Observación importante:} Para que un par ordenado sea solución del sistema, debe satisfacer \textbf{todas} las ecuaciones simultáneamente.

\newpage
%========================================
% SECTION 12.2: Método Gráfico
%========================================
\subsectiontitle{Método Gráfico}

El método gráfico consiste en representar cada ecuación como una recta en el plano de coordenadas y encontrar el punto (o puntos) donde las rectas se intersectan.

\begin{theorem}
\textbf{Método Gráfico para Resolver Sistemas:}

\begin{enumerate}
    \item \textbf{Graficar cada ecuación:} Trace la gráfica de cada ecuación lineal en el mismo sistema de coordenadas.
    \item \textbf{Hallar los puntos de intersección:} Las soluciones del sistema son las coordenadas $(x, y)$ de los puntos donde las rectas se intersectan.
\end{enumerate}

\textbf{Nota:} Si las rectas no se intersectan (son paralelas), el sistema no tiene solución. Si las rectas coinciden (son la misma recta), el sistema tiene infinitas soluciones.
\end{theorem}

\begin{example}
\textbf{Resolver gráficamente el sistema}

$$\begin{cases}
2x - y = 4 \\
3x + y = 6
\end{cases}$$

\textbf{Paso 1:} Encontrar las intersecciones con los ejes para cada ecuación.

\textbf{Para $2x - y = 4$:}
\begin{itemize}
    \item Intersección con eje $x$ ($y = 0$): $2x = 4 \implies x = 2$. Punto: $(2, 0)$
    \item Intersección con eje $y$ ($x = 0$): $-y = 4 \implies y = -4$. Punto: $(0, -4)$
\end{itemize}

\textbf{Para $3x + y = 6$:}
\begin{itemize}
    \item Intersección con eje $x$ ($y = 0$): $3x = 6 \implies x = 2$. Punto: $(2, 0)$
    \item Intersección con eje $y$ ($x = 0$): $y = 6$. Punto: $(0, 6)$
\end{itemize}

\textbf{Paso 2:} Graficar ambas rectas.

\begin{center}
\begin{tikzpicture}[scale=0.8]
    % Grid
    \draw[gray!20, very thin] (-1,-5) grid (4,7);

    % Axes
    \draw[->, thick] (-1,0) -- (4,0) node[right] {$x$};
    \draw[->, thick] (0,-5) -- (0,7) node[above] {$y$};

    % Axis labels
    \foreach \x in {1,2,3}
        \draw (\x,0.1) -- (\x,-0.1) node[below, font=\small] {\x};
    \foreach \y in {-4,-2,2,4,6}
        \draw (0.1,\y) -- (-0.1,\y) node[left, font=\small] {\y};

    % First line: 2x - y = 4  (or y = 2x - 4)
    \draw[blue, very thick, domain=-0.5:4] plot (\x, {2*\x - 4});
    \node[blue, above right] at (3,2) {$2x - y = 4$};

    % Second line: 3x + y = 6  (or y = -3x + 6)
    \draw[red, very thick, domain=-0.2:2.5] plot (\x, {-3*\x + 6});
    \node[red, above left] at (0.5,4.5) {$3x + y = 6$};

    % Intersection point
    \filldraw[green!70!black] (2,0) circle (4pt);
    \node[green!70!black, below right] at (2,0) {\textbf{Solución: $(2, 0)$}};
\end{tikzpicture}
\end{center}

\textbf{Solución:} El punto de intersección es $(2, 0)$.

\textbf{Verificación:}
\begin{itemize}
    \item $2(2) - 0 = 4 - 0 = 4$ \quad $\checkmark$
    \item $3(2) + 0 = 6 + 0 = 6$ \quad $\checkmark$
\end{itemize}
\end{example}

\newpage
%========================================
% SECTION 12.3: Método de Sustitución
%========================================
\subsectiontitle{Método de Sustitución}

El método de sustitución es un proceso algebraico que permite resolver sistemas de ecuaciones sin necesidad de graficar.

\begin{theorem}
\textbf{Método de Sustitución:}

\begin{enumerate}
    \item \textbf{Despejar una incógnita:} Escoja una ecuación y despeje una incógnita en términos de la otra incógnita.

    \item \textbf{Sustituir:} Sustituya la expresión hallada en el paso anterior en la otra ecuación para obtener una ecuación con una sola incógnita y, a continuación, despeje esa incógnita.

    \item \textbf{Sustituir a la inversa:} En la expresión hallada en el primer paso, sustituya el valor hallado para despejar la incógnita restante.

    \item \textbf{Verificar:} Compruebe la solución sustituyendo los valores en ambas ecuaciones originales.
\end{enumerate}
\end{theorem}

\begin{example}
\textbf{Resolver por sustitución el sistema}

$$\begin{cases}
2x - y = 4 \\
3x + y = 6
\end{cases}$$

\textbf{Paso 1: Despejar una incógnita}

Despejamos $y$ de la primera ecuación:
\begin{align*}
2x - y &= 4 \\
-y &= 4 - 2x \\
y &= 2x - 4
\end{align*}

\textbf{Paso 2: Sustituir}

Sustituimos $y = 2x - 4$ en la segunda ecuación:
\begin{align*}
3x + y &= 6 \\
3x + (2x - 4) &= 6 \\
5x - 4 &= 6 \\
5x &= 10 \\
x &= 2
\end{align*}

\textbf{Paso 3: Sustituir a la inversa}

Sustituimos $x = 2$ en la expresión $y = 2x - 4$:
\begin{align*}
y &= 2(2) - 4 \\
y &= 4 - 4 \\
y &= 0
\end{align*}

\textbf{Paso 4: Verificar}

Verificamos la solución $(2, 0)$ en ambas ecuaciones:
\begin{itemize}
    \item Primera ecuación: $2(2) - 0 = 4$ \quad $\checkmark$
    \item Segunda ecuación: $3(2) + 0 = 6$ \quad $\checkmark$
\end{itemize}

\textbf{Solución del sistema:} $(x, y) = (2, 0)$
\end{example}

\newpage
%========================================
% SECTION 12.4: Método de Eliminación
%========================================
\subsectiontitle{Método de Eliminación (o Método de Suma y Resta)}

El método de eliminación consiste en sumar o restar las ecuaciones para eliminar una de las variables.

\begin{theorem}
\textbf{Método de Eliminación:}

\begin{enumerate}
    \item \textbf{Ajustar los coeficientes:} Multiplique una o más de las ecuaciones por números apropiados, de modo que el coeficiente de una incógnita de una ecuación sea el \textbf{opuesto} (negativo) de su coeficiente en la otra ecuación.

    \item \textbf{Sumar las ecuaciones:} Sume las dos ecuaciones para eliminar una incógnita y, a continuación, despeje la incógnita restante.

    \item \textbf{Sustituir a la inversa:} En una de las ecuaciones originales, sustituya el valor hallado y despeje la incógnita restante.

    \item \textbf{Verificar:} Compruebe la solución sustituyendo los valores en ambas ecuaciones originales.
\end{enumerate}
\end{theorem}

\begin{example}
\textbf{Resolver por eliminación el sistema}

$$\begin{cases}
2x - y = 4 \\
3x + y = 6
\end{cases}$$

\textbf{Paso 1: Ajustar los coeficientes}

Observamos que los coeficientes de $y$ son $-1$ y $+1$, que ya son opuestos. No necesitamos multiplicar por ningún número.

\textbf{Paso 2: Sumar las ecuaciones}

Sumamos ambas ecuaciones:
\begin{align*}
(2x - y) + (3x + y) &= 4 + 6 \\
2x - y + 3x + y &= 10 \\
5x + 0y &= 10 \\
5x &= 10 \\
x &= 2
\end{align*}

\textbf{Paso 3: Sustituir a la inversa}

Sustituimos $x = 2$ en la primera ecuación original:
\begin{align*}
2(2) - y &= 4 \\
4 - y &= 4 \\
-y &= 0 \\
y &= 0
\end{align*}

\textbf{Paso 4: Verificar}

Verificamos la solución $(2, 0)$ en ambas ecuaciones:
\begin{itemize}
    \item Primera ecuación: $2(2) - 0 = 4$ \quad $\checkmark$
    \item Segunda ecuación: $3(2) + 0 = 6$ \quad $\checkmark$
\end{itemize}

\textbf{Solución del sistema:} $(x, y) = (2, 0)$
\end{example}

\newpage

\begin{example}
\textbf{Resolver por eliminación cuando se requiere ajustar coeficientes}

$$\begin{cases}
3x + 2y = 8 \\
5x - 3y = 1
\end{cases}$$

\textbf{Paso 1: Ajustar los coeficientes}

Decidimos eliminar $y$. Los coeficientes de $y$ son 2 y $-3$.

Multiplicamos la primera ecuación por 3 y la segunda por 2:

\textbf{Primera ecuación} $\times 3$:
$$3(3x + 2y) = 3(8) \implies 9x + 6y = 24$$

\textbf{Segunda ecuación} $\times 2$:
$$2(5x - 3y) = 2(1) \implies 10x - 6y = 2$$

Ahora los coeficientes de $y$ son $+6$ y $-6$ (opuestos).

\textbf{Paso 2: Sumar las ecuaciones}

\begin{align*}
(9x + 6y) + (10x - 6y) &= 24 + 2 \\
19x + 0y &= 26 \\
19x &= 26 \\
x &= \frac{26}{19}
\end{align*}

\textbf{Paso 3: Sustituir a la inversa}

Sustituimos $x = \frac{26}{19}$ en la primera ecuación original:
\begin{align*}
3\left(\frac{26}{19}\right) + 2y &= 8 \\
\frac{78}{19} + 2y &= 8 \\
2y &= 8 - \frac{78}{19} \\
2y &= \frac{152 - 78}{19} \\
2y &= \frac{74}{19} \\
y &= \frac{37}{19}
\end{align*}

\textbf{Solución del sistema:} $(x, y) = \left(\dfrac{26}{19}, \dfrac{37}{19}\right)$
\end{example}

\newpage
%========================================
% SECTION 12.5: Número de Soluciones
%========================================
\subsectiontitle{Número de Soluciones de un Sistema Lineal}

\begin{theorem}
\textbf{Clasificación de Sistemas de Ecuaciones Lineales:}

Para un sistema de dos ecuaciones lineales con dos incógnitas, \textbf{exactamente una} de las siguientes afirmaciones es verdadera:

\begin{enumerate}
    \item \textbf{El sistema tiene exactamente una solución} (las rectas se intersectan en un punto).
    \item \textbf{El sistema no tiene solución} (las rectas son paralelas).
    \item \textbf{El sistema tiene un número infinito de soluciones} (las rectas coinciden).
\end{enumerate}
\end{theorem}

\textbf{Representación Gráfica de los Tres Casos:}

\begin{center}
\begin{tikzpicture}[scale=0.9]
    % Case 1: One solution (intersecting lines)
    \begin{scope}[xshift=0cm]
        \draw[gray!20, very thin] (-1,-1) grid (3,3);
        \draw[->, thick] (-1,0) -- (3,0) node[right, font=\small] {$x$};
        \draw[->, thick] (0,-1) -- (0,3) node[above, font=\small] {$y$};

        \draw[blue, very thick] (-0.5,0.5) -- (2.5,2.5);
        \draw[red, very thick] (-0.5,2.5) -- (2.5,0.5);

        \filldraw[green!70!black] (1,1.5) circle (3pt);

        \node[below, align=center] at (1,-1.5) {\textbf{Una Solución} \\ Rectas que se intersectan};
    \end{scope}

    % Case 2: No solution (parallel lines)
    \begin{scope}[xshift=6cm]
        \draw[gray!20, very thin] (-1,-1) grid (3,3);
        \draw[->, thick] (-1,0) -- (3,0) node[right, font=\small] {$x$};
        \draw[->, thick] (0,-1) -- (0,3) node[above, font=\small] {$y$};

        \draw[blue, very thick] (-0.5,0.5) -- (2.5,2.5);
        \draw[red, very thick] (-0.5,-0.5) -- (2.5,1.5);

        \node[below, align=center] at (1,-1.5) {\textbf{Sin Solución} \\ Rectas paralelas};
    \end{scope}

    % Case 3: Infinite solutions (coincident lines)
    \begin{scope}[xshift=12cm]
        \draw[gray!20, very thin] (-1,-1) grid (3,3);
        \draw[->, thick] (-1,0) -- (3,0) node[right, font=\small] {$x$};
        \draw[->, thick] (0,-1) -- (0,3) node[above, font=\small] {$y$};

        \draw[blue, very thick] (-0.5,0.5) -- (2.5,2.5);
        \draw[red, very thick, dashed] (-0.5,0.5) -- (2.5,2.5);

        \node[below, align=center] at (1,-1.5) {\textbf{Infinitas Soluciones} \\ Rectas coincidentes};
    \end{scope}
\end{tikzpicture}
\end{center}

\begin{example}
\textbf{Identificar el número de soluciones}

\textbf{a) Sistema con una solución:}
$$\begin{cases}
x + y = 5 \\
x - y = 1
\end{cases}$$

Las rectas tienen diferentes pendientes y se intersectan en un punto. \textbf{Una solución:} $(3, 2)$.

\textbf{b) Sistema sin solución:}
$$\begin{cases}
2x + y = 4 \\
2x + y = 7
\end{cases}$$

Ambas ecuaciones tienen la misma forma ($2x + y =$) pero diferentes constantes (4 y 7). Las rectas son paralelas. \textbf{Sin solución.}

\textbf{c) Sistema con infinitas soluciones:}
$$\begin{cases}
2x + 4y = 8 \\
x + 2y = 4
\end{cases}$$

Si multiplicamos la segunda ecuación por 2, obtenemos la primera ecuación. Las rectas coinciden. \textbf{Infinitas soluciones.}
\end{example}

\newpage
%========================================
% SECTION 12.6: Modelado con Sistemas Lineales
%========================================
\subsectiontitle{Modelado con Sistemas Lineales}

Los sistemas de ecuaciones lineales son herramientas poderosas para resolver problemas del mundo real.

\begin{theorem}
\textbf{Proceso para Resolver Problemas con Sistemas de Ecuaciones:}

\begin{enumerate}
    \item \textbf{Identificar las variables:} Identifique las cantidades que el problema pide hallar. Introduzca notación para las variables (por ejemplo, $x$ e $y$).

    \item \textbf{Expresar todas las cantidades desconocidas en términos de las variables:} Exprese todas las cantidades mencionadas en el problema en términos de las variables definidas.

    \item \textbf{Establecer un sistema de ecuaciones:} Encuentre los datos cruciales del problema que den las relaciones entre las expresiones y establezca un sistema de ecuaciones (un modelo) que exprese estas relaciones.

    \item \textbf{Resolver el sistema e interpretar los resultados:} Resuelva el sistema usando cualquiera de los métodos aprendidos, verifique sus soluciones y dé su respuesta final como una frase que conteste la pregunta planteada en el problema.
\end{enumerate}
\end{theorem}

\begin{example}
\textbf{Problema de suma y diferencia}

Encuentre dos números cuya suma es 34 y cuya diferencia es 10.

\textbf{Paso 1: Identificar las variables}

Sea $x$ el número mayor y $y$ el número menor.

\textbf{Paso 2: Expresar las cantidades}

\begin{itemize}
    \item La suma de los números es 34: $x + y = 34$
    \item La diferencia de los números es 10: $x - y = 10$
\end{itemize}

\textbf{Paso 3: Establecer el sistema}

$$\begin{cases}
x + y = 34 \\
x - y = 10
\end{cases}$$

\textbf{Paso 4: Resolver el sistema}

Usando el método de eliminación (sumamos las ecuaciones):
\begin{align*}
(x + y) + (x - y) &= 34 + 10 \\
2x &= 44 \\
x &= 22
\end{align*}

Sustituimos en la primera ecuación:
\begin{align*}
22 + y &= 34 \\
y &= 12
\end{align*}

\textbf{Verificación:}
\begin{itemize}
    \item Suma: $22 + 12 = 34$ \quad $\checkmark$
    \item Diferencia: $22 - 12 = 10$ \quad $\checkmark$
\end{itemize}

\textbf{Respuesta:} Los dos números son 22 y 12.
\end{example}

\begin{example}
\textbf{Problema de monedas}

Un hombre tiene 14 monedas en su bolsillo, todas las cuales son de 10 o de 25 centavos. Si el valor total de su cambio es \$2.75, ¿cuántas monedas de 10 centavos y cuántas de 25 centavos tiene?

\textbf{Paso 1: Identificar las variables}

Sea:
\begin{itemize}
    \item $x$ = número de monedas de 10 centavos
    \item $y$ = número de monedas de 25 centavos
\end{itemize}

\textbf{Paso 2: Expresar las cantidades}

\begin{itemize}
    \item Total de monedas: $x + y = 14$
    \item Valor total (en centavos): $10x + 25y = 275$
\end{itemize}

\textbf{Paso 3: Establecer el sistema}

$$\begin{cases}
x + y = 14 \\
10x + 25y = 275
\end{cases}$$

\textbf{Paso 4: Resolver el sistema}

Despejamos $x$ de la primera ecuación:
$$x = 14 - y$$

Sustituimos en la segunda ecuación:
\begin{align*}
10(14 - y) + 25y &= 275 \\
140 - 10y + 25y &= 275 \\
140 + 15y &= 275 \\
15y &= 135 \\
y &= 9
\end{align*}

Sustituimos para hallar $x$:
$$x = 14 - 9 = 5$$

\textbf{Verificación:}
\begin{itemize}
    \item Total de monedas: $5 + 9 = 14$ \quad $\checkmark$
    \item Valor total: $10(5) + 25(9) = 50 + 225 = 275$ centavos = \$2.75 \quad $\checkmark$
\end{itemize}

\textbf{Respuesta:} El hombre tiene 5 monedas de 10 centavos y 9 monedas de 25 centavos.
\end{example}

%========================================
% EXERCISES: Sistema de Dos Ecuaciones Lineales
%========================================

\section{Ejercicios}

%========================================
% Exercise 1: Solving Systems by All Three Methods
%========================================
\begin{exercise}
\textbf{Resolver Sistemas por los Tres Métodos}

Resuelva cada sistema de ecuaciones usando:
\begin{itemize}
    \item Método Gráfico
    \item Método de Sustitución
    \item Método de Eliminación
\end{itemize}

\problem $$\begin{cases}2x - y = 4 \\ 3x + y = 6\end{cases}$$

\begin{solucion}
\textbf{Método Gráfico:}

Para $2x - y = 4$: Intersecciones $(2, 0)$ y $(0, -4)$

Para $3x + y = 6$: Intersecciones $(2, 0)$ y $(0, 6)$

(Se grafican ambas rectas y se identifica su intersección)

Solución gráfica: $(2, 0)$

\vspace{0.3cm}

\textbf{Método de Sustitución:}

Despejando $y$ de la primera ecuación: $y = 2x - 4$

Reemplazando en la segunda ecuación:
$$3x + (2x - 4) = 6$$
$$5x = 10$$
$$x = 2$$

Sustituir en el despeje: $y = 2(2) - 4 = 0$

Solución: $(x,y) = (2,0)$

\vspace{0.3cm}

\textbf{Método por Eliminación:}

Los coeficientes de $y$ en las ecuaciones ya son opuestos.

Sumar ambas ecuaciones:
$$\begin{aligned}
(2x - y) + (3x + y) &= 4 + 6 \\
5x + 0y &= 10 \\
x &= 2
\end{aligned}$$

Sustituir en la primera ecuación: $2(2) - y = 4 \implies y = 0$

Solución: $(x,y) = (2,0)$
\end{solucion}

\problem $$\begin{cases}x + y = 7 \\ 2x - 3y = -1\end{cases}$$

\begin{solucion}
\textbf{Método Gráfico:}

Para $x + y = 7$: Intersecciones $(7, 0)$ y $(0, 7)$

Para $2x - 3y = -1$: Intersecciones $(-\frac{1}{2}, 0)$ y $(0, \frac{1}{3})$

Solución gráfica: $(4, 3)$

\vspace{0.3cm}

\textbf{Método de Sustitución:}

Despejando $x$ de la primera ecuación: $x = 7 - y$

Reemplazando en la segunda ecuación:
$$2(7 - y) - 3y = -1$$
$$14 - 2y - 3y = -1$$
$$-5y = -15$$
$$y = 3$$

Sustituir: $x = 7 - 3 = 4$

Solución: $(x,y) = (4,3)$

\vspace{0.3cm}

\textbf{Método por Eliminación:}

Multiplicar la primera ecuación por $-2$:
$$-2(x + y) = -2(7) \implies -2x - 2y = -14$$

Sumar con la segunda ecuación:
$$\begin{aligned}
(-2x - 2y) + (2x - 3y) &= -14 + (-1) \\
-5y &= -15 \\
y &= 3
\end{aligned}$$

Sustituir en la primera ecuación: $x + 3 = 7 \implies x = 4$

Solución: $(x,y) = (4,3)$
\end{solucion}

\problem $$\begin{cases}2x + 5y = 15 \\ 4x + y = 21\end{cases}$$

\begin{solucion}
\textbf{Método Gráfico:}

Para $2x + 5y = 15$: Intersecciones $(\frac{15}{2}, 0)$ y $(0, 3)$

Para $4x + y = 21$: Intersecciones $(\frac{21}{4}, 0)$ y $(0, 21)$

Solución gráfica: $(5, 1)$

\vspace{0.3cm}

\textbf{Método de Sustitución:}

Despejando $y$ de la segunda ecuación: $y = 21 - 4x$

Reemplazando en la primera ecuación:
$$2x + 5(21 - 4x) = 15$$
$$2x + 105 - 20x = 15$$
$$-18x = -90$$
$$x = 5$$

Sustituir: $y = 21 - 4(5) = 1$

Solución: $(x,y) = (5,1)$

\vspace{0.3cm}

\textbf{Método por Eliminación:}

Multiplicar la segunda ecuación por $-5$:
$$-5(4x + y) = -5(21) \implies -20x - 5y = -105$$

Sumar con la primera ecuación:
$$\begin{aligned}
(2x + 5y) + (-20x - 5y) &= 15 + (-105) \\
-18x &= -90 \\
x &= 5
\end{aligned}$$

Sustituir en la segunda ecuación: $4(5) + y = 21 \implies y = 1$

Solución: $(x,y) = (5,1)$
\end{solucion}
\end{exercise}

%========================================
% Exercise 2: Identifying Number of Solutions
%========================================
\begin{exercise}
\textbf{Identificar el Número de Soluciones}

Para cada sistema, determine (sin resolverlo) si tiene una solución, ninguna solución o infinitas soluciones. Explique su razonamiento.

\problem $$\begin{cases}x + 2y = 5 \\ 2x + 4y = 10\end{cases}$$

\begin{solucion}
La segunda ecuación es $2$ veces la primera ecuación: $2(x + 2y) = 2(5)$.

Las rectas \textbf{coinciden}.

\textbf{Infinitas soluciones}
\end{solucion}

\problem $$\begin{cases}3x - y = 4 \\ 3x - y = 7\end{cases}$$

\begin{solucion}
Ambas ecuaciones tienen la misma forma ($3x - y =$) pero diferentes constantes (4 y 7).

Las rectas son \textbf{paralelas}.

\textbf{Sin solución}
\end{solucion}

\problem $$\begin{cases}x + y = 8 \\ x - y = 2\end{cases}$$

\begin{solucion}
Las rectas tienen diferentes pendientes (pendientes: $-1$ y $1$).

Las rectas se \textbf{intersectan} en un punto.

\textbf{Una solución}
\end{solucion}

\problem $$\begin{cases}2x + 3y = 6 \\ 4x + 6y = 12\end{cases}$$

\begin{solucion}
La segunda ecuación es $2$ veces la primera ecuación.

Las rectas \textbf{coinciden}.

\textbf{Infinitas soluciones}
\end{solucion}

\problem $$\begin{cases}y = 2x + 1 \\ y = 2x - 3\end{cases}$$

\begin{solucion}
Ambas ecuaciones tienen la misma pendiente ($m = 2$) pero diferentes intersecciones con el eje $y$ (1 y $-3$).

Las rectas son \textbf{paralelas}.

\textbf{Sin solución}
\end{solucion}

\problem $$\begin{cases}5x - 2y = 10 \\ x + 3y = 6\end{cases}$$

\begin{solucion}
Las rectas tienen diferentes pendientes (pendientes: $\frac{5}{2}$ y $-\frac{1}{3}$).

Las rectas se \textbf{intersectan} en un punto.

\textbf{Una solución}
\end{solucion}
\end{exercise}

%========================================
% Exercise 3: Word Problems
%========================================
\begin{exercise}
\textbf{Problemas de Aplicación}

Resuelva los siguientes problemas usando sistemas de ecuaciones lineales. Siga el proceso de 4 pasos:
\begin{enumerate}
    \item Identificar las variables
    \item Expresar las cantidades en términos de las variables
    \item Establecer el sistema de ecuaciones
    \item Resolver e interpretar los resultados
\end{enumerate}

\problem Encuentre dos números cuya suma es 34 y cuya diferencia es 10.

\begin{solucion}
\textbf{Paso 1:} Sea $x$ el número mayor y $y$ el número menor.

\textbf{Paso 2:}
\begin{itemize}
    \item Suma: $x + y = 34$
    \item Diferencia: $x - y = 10$
\end{itemize}

\textbf{Paso 3:} Sistema:
$$\begin{cases}
x + y = 34 \\
x - y = 10
\end{cases}$$

\textbf{Paso 4:} Sumando ambas ecuaciones:
$$2x = 44 \implies x = 22$$

Sustituyendo: $22 + y = 34 \implies y = 12$

\textbf{Respuesta:} Los dos números son 22 y 12.
\end{solucion}

\problem Un hombre tiene 14 monedas en su bolsillo, todas las cuales son de 10 o de 25 centavos. Si el valor total de su cambio es \$2.75, ¿cuántas monedas de 10 centavos y cuántas de 25 centavos tiene?

\begin{solucion}
\textbf{Paso 1:} Sea $x$ = número de monedas de 10 centavos, $y$ = número de monedas de 25 centavos.

\textbf{Paso 2:}
\begin{itemize}
    \item Total de monedas: $x + y = 14$
    \item Valor total (en centavos): $10x + 25y = 275$
\end{itemize}

\textbf{Paso 3:} Sistema:
$$\begin{cases}
x + y = 14 \\
10x + 25y = 275
\end{cases}$$

\textbf{Paso 4:} De la primera ecuación: $x = 14 - y$

Sustituyendo en la segunda:
$$10(14 - y) + 25y = 275$$
$$140 - 10y + 25y = 275$$
$$15y = 135$$
$$y = 9$$

Entonces: $x = 14 - 9 = 5$

\textbf{Respuesta:} Tiene 5 monedas de 10 centavos y 9 monedas de 25 centavos.
\end{solucion}

\problem Una tienda vende camisetas a \$15 cada una y pantalones a \$25 cada uno. Si en un día se vendieron 45 artículos por un total de \$825, ¿cuántas camisetas y cuántos pantalones se vendieron?

\begin{solucion}
\textbf{Paso 1:} Sea $x$ = número de camisetas, $y$ = número de pantalones.

\textbf{Paso 2:}
\begin{itemize}
    \item Total de artículos: $x + y = 45$
    \item Valor total: $15x + 25y = 825$
\end{itemize}

\textbf{Paso 3:} Sistema:
$$\begin{cases}
x + y = 45 \\
15x + 25y = 825
\end{cases}$$

\textbf{Paso 4:} De la primera ecuación: $x = 45 - y$

Sustituyendo:
$$15(45 - y) + 25y = 825$$
$$675 - 15y + 25y = 825$$
$$10y = 150$$
$$y = 15$$

Entonces: $x = 45 - 15 = 30$

\textbf{Respuesta:} Se vendieron 30 camisetas y 15 pantalones.
\end{solucion}

\problem La suma de dos números es 50. Tres veces el primer número menos el segundo número es igual a 10. Encuentre los dos números.

\begin{solucion}
\textbf{Paso 1:} Sea $x$ el primer número y $y$ el segundo número.

\textbf{Paso 2:}
\begin{itemize}
    \item Suma: $x + y = 50$
    \item Relación: $3x - y = 10$
\end{itemize}

\textbf{Paso 3:} Sistema:
$$\begin{cases}
x + y = 50 \\
3x - y = 10
\end{cases}$$

\textbf{Paso 4:} Sumando ambas ecuaciones:
$$4x = 60 \implies x = 15$$

Sustituyendo: $15 + y = 50 \implies y = 35$

\textbf{Respuesta:} Los dos números son 15 y 35.
\end{solucion}

\problem Un rectángulo tiene un perímetro de 60 metros. El largo es 6 metros más que el ancho. Encuentre las dimensiones del rectángulo.

\begin{solucion}
\textbf{Paso 1:} Sea $l$ = largo y $a$ = ancho.

\textbf{Paso 2:}
\begin{itemize}
    \item Perímetro: $2l + 2a = 60$
    \item Relación: $l = a + 6$
\end{itemize}

\textbf{Paso 3:} Sistema:
$$\begin{cases}
2l + 2a = 60 \\
l = a + 6
\end{cases}$$

\textbf{Paso 4:} Sustituyendo la segunda ecuación en la primera:
$$2(a + 6) + 2a = 60$$
$$2a + 12 + 2a = 60$$
$$4a = 48$$
$$a = 12$$

Entonces: $l = 12 + 6 = 18$

\textbf{Respuesta:} El rectángulo tiene 18 metros de largo y 12 metros de ancho.
\end{solucion}
\end{exercise}


% Conditional solution inclusion
\ifshowsolutions
    \newpage
    \section*{Soluciones}
    %========================================
% DETAILED SOLUTIONS: Sistema de Dos Ecuaciones Lineales
%========================================

\subsection*{Notas para el Instructor}

Este documento contiene notas pedagógicas y errores comunes de los estudiantes. Las soluciones completas de los ejercicios se encuentran integradas en el documento principal cuando se activa el modo de soluciones.

%========================================
% Exercise 1 Notes
%========================================
\subsection*{Ejercicio 1: Resolver Sistemas por los Tres Métodos}

\textbf{Objetivo Pedagógico:} Que los estudiantes practiquen y comparen los tres métodos de solución.

\textbf{Errores Comunes:}

\begin{itemize}
    \item \textbf{Método Gráfico:}
    \begin{itemize}
        \item Trazar rectas incorrectamente por no usar suficientes puntos
        \item Leer incorrectamente las coordenadas del punto de intersección
        \item No verificar la solución sustituyéndola en ambas ecuaciones
    \end{itemize}

    \item \textbf{Método de Sustitución:}
    \begin{itemize}
        \item Errores algebraicos al despejar una variable
        \item Olvidar sustituir de vuelta para encontrar la segunda variable
        \item Sustituir en la ecuación incorrecta (debe ser en la ecuación original, no en una modificada)
    \end{itemize}

    \item \textbf{Método de Eliminación:}
    \begin{itemize}
        \item No multiplicar todos los términos de la ecuación por el mismo factor
        \item Sumar cuando se debe restar, o viceversa
        \item Olvidar ajustar los coeficientes para que sean opuestos
    \end{itemize}
\end{itemize}

\textbf{Sugerencias de Enseñanza:}

\begin{enumerate}
    \item Enfatice que los tres métodos deben dar la misma solución
    \item Muestre cuándo es más conveniente usar cada método:
    \begin{itemize}
        \item Gráfico: Cuando se necesita una visualización o una aproximación
        \item Sustitución: Cuando una variable ya está despejada o es fácil de despejar
        \item Eliminación: Cuando los coeficientes son múltiplos fáciles o ya son opuestos
    \end{itemize}
    \item Recuerde a los estudiantes SIEMPRE verificar su solución
\end{enumerate}

\textbf{Notas Específicas por Problema:}

\textbf{Problema 1.1:} $\begin{cases}2x - y = 4 \\ 3x + y = 6\end{cases}$
\begin{itemize}
    \item Este es un caso ideal para eliminación (coeficientes de $y$ ya son opuestos)
    \item Solución: $(2, 0)$ está en el eje $x$, lo cual facilita la verificación gráfica
\end{itemize}

\textbf{Problema 1.2:} $\begin{cases}x + y = 7 \\ 2x - 3y = -1\end{cases}$
\begin{itemize}
    \item La primera ecuación es fácil de despejar (buena para sustitución)
    \item Para eliminación, multiplicar la primera por 3 o la primera por 2
    \item Solución: $(4, 3)$ con números enteros positivos
\end{itemize}

\textbf{Problema 1.3:} $\begin{cases}2x + 5y = 15 \\ 4x + y = 21\end{cases}$
\begin{itemize}
    \item La segunda ecuación tiene coeficiente 1 en $y$ (fácil para sustitución)
    \item Para eliminación, multiplicar la segunda por $-5$
    \item Solución: $(5, 1)$
\end{itemize}

%========================================
% Exercise 2 Notes
%========================================
\subsection*{Ejercicio 2: Identificar el Número de Soluciones}

\textbf{Objetivo Pedagógico:} Desarrollar intuición para reconocer sistemas sin resolverlos completamente.

\textbf{Conceptos Clave:}

\begin{itemize}
    \item \textbf{Infinitas soluciones:} Las ecuaciones son equivalentes (una es múltiplo de la otra)
    \item \textbf{Sin solución:} Misma pendiente, diferentes intersecciones (paralelas)
    \item \textbf{Una solución:} Pendientes diferentes (se intersectan)
\end{itemize}

\textbf{Estrategia de Identificación:}

\begin{enumerate}
    \item Convertir ambas ecuaciones a la forma $y = mx + b$ (si es posible)
    \item Comparar las pendientes $m$:
    \begin{itemize}
        \item Si $m_1 \neq m_2$: Una solución
        \item Si $m_1 = m_2$ y $b_1 \neq b_2$: Sin solución
        \item Si $m_1 = m_2$ y $b_1 = b_2$: Infinitas soluciones
    \end{itemize}
    \item Alternativamente, verificar si una ecuación es múltiplo de la otra
\end{enumerate}

\textbf{Errores Comunes:}
\begin{itemize}
    \item Confundir "infinitas soluciones" con "sin solución"
    \item No reconocer ecuaciones equivalentes cuando están escritas de manera diferente
    \item Intentar resolver el sistema en lugar de analizar la estructura
\end{itemize}

%========================================
% Exercise 3 Notes
%========================================
\subsection*{Ejercicio 3: Problemas de Aplicación}

\textbf{Objetivo Pedagógico:} Conectar sistemas de ecuaciones con situaciones del mundo real.

\textbf{Proceso de 4 Pasos (Énfasis):}

\begin{enumerate}
    \item \textbf{Identificar variables:} Definir claramente qué representa cada variable
    \item \textbf{Expresar cantidades:} Traducir las palabras a expresiones matemáticas
    \item \textbf{Establecer sistema:} Escribir las ecuaciones basadas en las relaciones
    \item \textbf{Resolver e interpretar:} Resolver y expresar la respuesta en contexto
\end{enumerate}

\textbf{Errores Comunes:}

\begin{itemize}
    \item No definir las variables claramente desde el principio
    \item Confundir las unidades (centavos vs. dólares)
    \item No escribir la respuesta final como una oración completa
    \item Olvidar verificar que la solución tiene sentido en el contexto
\end{itemize}

\textbf{Notas Específicas por Problema:}

\textbf{Problema 3.1 (Suma y diferencia):}
\begin{itemize}
    \item Tipo clásico de problema
    \item Cuidado: "diferencia" implica el número mayor menos el menor
    \item Verificación: ¿Los números tienen sentido? ¿Son positivos?
\end{itemize}

\textbf{Problema 3.2 (Monedas):}
\begin{itemize}
    \item Recordar convertir dólares a centavos O centavos a dólares (consistencia)
    \item Dos ecuaciones: cantidad total y valor total
    \item Verificación: ¿El número de monedas es entero? ¿Es razonable?
\end{itemize}

\textbf{Problema 3.3 (Tienda):}
\begin{itemize}
    \item Similar al problema de monedas
    \item Enfatizar la estructura: cantidad total + valor total
    \item Este problema refuerza el patrón
\end{itemize}

\textbf{Problema 3.4 (Relación entre números):}
\begin{itemize}
    \item Traducir "tres veces el primer número menos el segundo" correctamente: $3x - y$
    \item No confundir con $3(x - y)$
\end{itemize}

\textbf{Problema 3.5 (Geometría - Rectángulo):}
\begin{itemize}
    \item Recordar la fórmula del perímetro: $P = 2l + 2a$
    \item "El largo es 6 metros más que el ancho": $l = a + 6$
    \item Conecta álgebra con geometría
\end{itemize}

%========================================
% Additional Teaching Strategies
%========================================
\subsection*{Estrategias Adicionales de Enseñanza}

\textbf{Para Estudiantes con Dificultades:}

\begin{enumerate}
    \item Comenzar siempre con el método gráfico para visualización
    \item Usar problemas con números enteros pequeños
    \item Proporcionar plantillas para el proceso de 4 pasos
    \item Enfatizar la verificación como paso obligatorio
\end{enumerate}

\textbf{Para Estudiantes Avanzados:}

\begin{enumerate}
    \item Presentar sistemas con fracciones o decimales
    \item Introducir sistemas con más de dos variables (3×3)
    \item Explorar casos especiales (rectas paralelas, coincidentes)
    \item Crear sus propios problemas de aplicación
\end{enumerate}

\textbf{Actividades Complementarias:}

\begin{itemize}
    \item Comparar la eficiencia de cada método con diferentes sistemas
    \item Usar software de graficación (Desmos, GeoGebra) para verificar soluciones
    \item Crear problemas de aplicación basados en situaciones de su vida diaria
    \item Trabajo en grupos: Cada estudiante resuelve por un método diferente
\end{itemize}

%========================================
% Assessment Rubric
%========================================
\subsection*{Rúbrica de Evaluación Sugerida}

\textbf{Para Resolución de Sistemas:}

\begin{itemize}
    \item Configuración correcta (identificar el sistema): 20\%
    \item Proceso algebraico correcto: 40\%
    \item Solución correcta: 30\%
    \item Verificación: 10\%
\end{itemize}

\textbf{Para Problemas de Aplicación:}

\begin{itemize}
    \item Definición de variables: 15\%
    \item Establecimiento del sistema: 25\%
    \item Solución algebraica correcta: 35\%
    \item Interpretación y respuesta en contexto: 20\%
    \item Verificación: 5\%
\end{itemize}

\fi

\end{document}
